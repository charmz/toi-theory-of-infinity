\documentclass{article}
\usepackage[utf8]{inputenc}
\begin{document}

Note:

Images of this file set were scanned
at 120% of their original page size.



| es

Farce pot mangucam “Ure fiir Abi

Ning Perera dente pene

inter’

Orber, Buchdue Grove gungid

Tradidt,, Arstora mine

Saber nose nett ey
egrets

Ar Teaco eK une’
te

law de

Conalg

Johannes Kepler, Mysterium cosmographicum. Copper engraving
from the first edition (Tiibingen, 1596).


JOHANNES KEPLER °¢ Mysterium Cosmographicum
The Secret of the Universe


Lis
ar.

Copyright © 1981 by Abaris Books, Inc.
International Standard Book Number 0-913870-64-1
Library of Congress Card Number 77-86245

First published 1981 by Abaris Books, Inc.

24 West 40th Street, New York, New York 10018
Printed in the United States of America

_A William J. Prendergast Production

This book is sold subject to the condition that no portion
shall be reproduced in any form or by any means, and that
it shall not, by way of trade, be lent, resold, hired out, or
otherwise disposed of without the publisher’s consent, in
any form of binding or cover other than that in which it is
published.


Preface ... Bo
Apparatus Criticus
Translator’s Notes .
Introduction... .
Mysterium Cosmographicum
Title page of the first edition (facsimile) .
Facsimile and translation of the second edition .
Dedicatory Epistle (second edition).
Original Dedication.........
Original Preface to the Reader
Chapter I. The reasoning with which the hypotheses of Copernicus
agree, and exposition of the hypotheses of Copernicus .
Chapter I. Outlines of the primary derivation......
Chapter III. That these five solids are classified into two types; and that
the Earth has been correctly located .
Chapter IV. Why should three bodies go roun¢
ing two inside it? .......
Chapter V. That the cube is the first of the solids, and between the
highest planets
Chapter VI. That the pyramid is between Jupiter and Mars...
Chapter VII. On the order of the secondaries and their properties
Chapter VIII. That the octahedron is between Venus and Mercury .
Chapter 1X. The distribution of the solids among the planets; the at-
tribution of their properties; the derivation from the solids of the
mutual kinship of the planets
Chapter X. On the origin of the noble numbers .
Chapter XI. On the arrangement of the solids, and the origin of the
zodiac ..
Chapter XII. The division of the zodiac, and the aspect
Chapter XIII. On calculating the spheres which are inscribed in the
solids, and which circumscribe them
Chapter XIV. Primary aim of the Book, and astronomical proof that
the five solids are between the spheres ........0 0+ 000cceeecscceeenneees 154
Chapter XV. Correction of the distances and variation of the equa-
tions
Chapter XVI. A particular comment on the Moon, and on the material
of the solids and spheres
Chapter XVII. Another comment on Mercury.
Chapter XVIII. On the disagreement between the equations derived
from the solids and those of Copernicus in general, and on the precision
of astronomy .



Eas


Chapter XIX. On the remaining disagreement in the case of particular
individual planets
Chapter XX. What the ratio of the motions to the orbits is
Chapter XXI. What is to be inferred from the deficiency . :
Chapter XXII. Why a planet moves uniformly about the center of the

OCQUANE 00 cece cece cee ta seers neenaeenraaeeeeeeresensetecee 214
Chapter XXIII. On the astronomical beginning and end of the universe
and the Platonic Year . . . .220

Conclusion of the Book
Translation of the Annotations to the Plates
Commentary Notes .
Appendix. Ptolemaic and Copernican representations use

Maestlin ...
Bibliography .
Index


Although Johannes Kepler is universally esteemed as one of the major
scientists of the seventeenth century and one of the greatest astronomers
who has ever lived, his reputation has not been paralleled by a series of
translations of his works into languages other than German. Until now,
not one of Kepler’s major publications, in which his discoveries were
disclosed, has appeared in a complete English version, nor are any of
these writings available in French or Italian translations.! These works in-
clude —in addition to Kepler’s Mysterium cosmographicum (1596; ed. 2,
1621)—the Ad Vitellionem paralipomena or Astronomiae pars optica
(1604), the Astronomia nova (1609), the Harmonice mundi (1619), the
Epitome astronomiae Copernicanae (1618-1620-1621), and the Tabulae
Rudolphinae (1627). Until not very long ago, only a few short selections
had been published in English? of which the longest was Kepler’s
response to Galileo’s Sidereal messenger, taken from the introduction to
Kepler’s Dioptrice (1611).?

The first translations of any usable length were published in 1952 as
part of the series of “Great Books of the Western World”: Book 4 and
Book 5 of the Epitome of Copernican astronomy and Book 5 of the Har-
mony of the world.* Then, in 1965, Edward Rosen brought out his
English version of Kepler’s Conversation with Galileo’s sidereal
messenger’ and John Lear presented Kepler’s dream,® so that at last a
work of Kepler’s was completely translated and published in its entirety.”
The next year, 1966, saw an English version of Kepler’s little New Year’s
essay of 1611 on the snowflake® and in 1967 Rosen produced his English
translation of Kepler’s Somnium or Dream, the third time this work had
been rendered in an English version.? At last there had been made
available to English and American readers a complete work written by
Kepler on an astronomical subject. In 1979 J.V. Field published an
English version of Kepler’s account of star polyhedra.1°

Now, thanks to the intellectual labors of Alistair Duncan and Eric
Aiton, we have an annotated translation of the Mysterium cosmo-
graphicum, Kepler’s first full-dress essay in astronomy. May we hope that
this same team will next produce a translation of the Harmony of the
world, in which Kepler announced the third or harmonic law of planetary
motion. How useful it would be if scholars and students could also have
available an English version of Kepler’s revolutionary New astronomy, in
which Kepler disclosed the law of equal areas and the elliptical orbits of
the planets!

Kepler’s Mysterium cosmographicum of 1596 marked his public ap-
pearance as a major astronomer. It is a notable book for the student of
Kepler’s thought not only because it is his first significant astronomical
work, but also because it was revised by Kepler for a new edition (1621), in
which he introduced a series of annotations containing second thoughts
Preface

on the topics discussed and relating his ideas to the exciting new develop-
ments made in astronomy between the two editions, of which the most
remarkable had been the disclosures made by the use of the astronomical
telescope.!! A particularly valuable feature of the present edition is the in-
clusion of these later notes.??

Kepler’s Mysterium cosmographicum usually appears in the literature
of the history of astronomy primarily because it contains the first
astronomical “law” that Kepler discovered, a geometric relation among
the average distances of the planets from the sun in the Copernican
system. This “law” stated that the planetary orbits lie in a set of six imag-
ined nested spheres, separated by the five regular solids, so placed that
each solid circumscribed the immediate inner sphere and was circum-
scribed by the next outer sphere.!3 The discovery of this “law” is often
considered as a curiosity or as an aberration of Kepler’s youth, but readers
of The secret of the universe will see that Kepler displayed in this
discovery the same traits of logical reasoning, geometric skill, and com-
putational ability that characterize his later writings such as The new
astronomy. Far from rejecting this “law” as an extravagance of his youth,
Kepler continually reiterated his faith in its truth and importance. This is
clear not only from the notes Kepler added to the second edition in 1621
(and which are given in English translation in the present edition), but
also from the fact that he did not ever revise or retract the statement at the
beginning of this book, in which he declared that its purpose was to
demonstrate that when God created the universe and “determined the
order of the heavenly bodies” or planets, He had in mind “the five regular
bodies which have enjoyed such great distinction from the time of
Pythagoras and Plato down to our own days.” In the Harmonice mundi
(1619), he reiterated his faith in what he had shown “in my Mysterium
cosmographicum, published twenty-two years ago, that the number of
planets, or spheres, surrounding the sun had been fixed by the all-wise
creator as a function of the five regular solids on which Euclid wrote a
book many centuries ago.”!4 He even included a diagram indicating the
relation of the planetary orbital spheres to the regular solids. This scheme
also appears at great length in full display in Book 4 of Kepler’s Epitome
of Copernican astronomy.

This continuing belief in the planetary “law” of geometric solids ex-
plains why Kepler was so disturbed when he heard that Galileo had
discovered some new “planets.” Since there are only five regular solids,
Kepler’s “law” provides for only six spheres, corresponding to six possible
orbits and six planets. There is no place for any additional planets and, in
fact, this was an argument that Kepler had used to justify belief in the
Copernican system with its six planets (Mercury, Venus, earth, Mars,
Jupiter, Saturn) as opposed to the Ptolemaic system with its seven planets
(moon, Mercury, Venus, sun, Mars, Jupiter, Saturn).'’ Eventually, when
Kepler got hold of a copy of Galileo’s book, he discovered that the new

Preface

“planets” (or wandering astronomical bodies) were secondary planets or
satellites and so did not disturb his system. He could thus continue to
maintain his belief that God had had in mind the five regular solids when
He had designed and created the solar system.

What may most commend this book to astronomers is that in it Kepler
sets forth a goal, method, and program, which he was to follow suc-
cessfully throughout the rest of his astronomical career. As he says in the
beginning of the Mysterium cosmographicum, there were three things for
which he chiefly sought the cause: the number and size of the planetary
orbits and the motions of the planets in those orbits. In the Harmonice
mundi, announcing the third law of planetary motion, Kepler could refer
to “that which I prophesied two-and-twenty years ago, as soon as I
discovered the five solids among the celestial orbits.” As Eric Aiton points
out below, almost all “the astronomical books written by Kepler (notably
the Astronomia nova and the Harmonice mundi) are concerned with the
further development and completion of themes that were introduced in
the Mysterium cosmographicum.”

I. Bernard Cohen


1. On Kepler’s reputation and the lack of availability of his writings in English
versions, see my “Kepler’s century: prelude to Newton’s,” Vistas in Astronomy,
1975, vol. 18, pp. 3-36, esp. pp. 34-35. On editions and translations, see Max
Caspar: Bibliographia Kepleriana (Munich: C.H. Beck’sche Verlagsbuch-
handlung, 1936; revised and updated by Martha List, Munich, 1968). See addi-
tionally, the bibliographical supplement to the account of Kepler by Owen Gin-
gerich in the Dictionary of Scientific Biography, 1973, vol. 7, pp. 289-312, and
Martha List’s “ ‘Bibliographia Kepleriana’ 1967-1975,” Vistas in Astronomy, 1975,
vol. 18, pp. 955-1010.

2. Notably John H. Walden’s translation of a portion of the Harmonice mundi
on pp. 30-40 of Harlow Shapley and Helen Z. Howarth (eds.): A source book in
astronomy (New York: McGraw-Hill Book Company, 1929). This same volume
also includes a two-page extract on the reconciliation of the texts of Scripture with
the Copernican doctrine of the mobility of the earth, taken from Thomas
Salusbury’s translation of a portion of the Astronomia nova, originally published
in 1661.

Additionally, in 1951, Carola Baumgardt made available in English a number of
Kepler's letters; see her Johannes Kepler: Life and letters, with an introduction by
Albert Einstein (New York: Philosophical Library, 1951).

3. The Sidereal messenger of Galileo Galilei and a part of the preface to Kepler's
Dioptrics containing the original account of Galileo’s astronomical discoveries, a
translation with introduction and notes by E.S. Carlos (London, Oxford, Cam-
bridge: Rivington’s, 1880; facsimile reprint, London: Dawsons of Pall Mall, 1959).

4. Robert Maynard Hutchins (ed. in chief): Great books of the western world,
vol. 15, Ptolemy, Copernicus, Kepler (Chicago, London, Toronto: Encyclopaedia
Britannica, 1952). The translations from Kepler had been completed in 1939 by
Charles Glenn Wallis.
Preface

5. Kepler's conversation with Galileo’s sidereal messenger, first complete
translation, with an introduction and notes, by Edward Rosen (New York, Lon-
don: Johnson Reprint Corporation, 1965—The Sources of Science, No. 5).

6. John Lear: Kepler’s dream: with the full text and notes of Somnium, sive
astronomia lunaris, Joannis Kepleri, translated by Patricia Frueh Kirkwood
(Berkeley, Los Angeles: University of California Press, 1965).

7. An earlier translation, made by Joseph Keith Lane, was submitted in June
1947 “in partial fulfillment of the requirements for the degree of Master of Arts in
the Faculty of Philosophy, Columbia University.”

8. Johannes Kepler: The six-cornered snowflake, translated by Colin Hardie,
with the Latin text on facing pages, and essays by B.J. Mason and L.L. Whyte (Ox-
ford: at the Clarendon Press, 1966).

9. Kepler's Somnium, the dream, or posthumous work on lunar astronomy,
translated with a commentary by Edward Rosen (Madison, Milwaukee, London:
The University of Wisconsin Press, 1967). For the earlier translations see notes 6 &
7 supra.

10. See J.V. Field’s “Kepler’s star polyhedra,” Vistas in Astronomy, 1979, vol.
23, pp. 109-141.

It should be noted that Alexandre Koyré’s La révolution astronomique (Paris:
Hermann, 1961), translated as The astronomical revolution (London: Methuen;
Ithaca: Cornell University Press; P: Hermann, 1973), contains such a wealth of
extracts as to constitute a veritable Keplerian anthology; these were rendered into
French by Koyré and given in English versions made by the translator of the
volume, R.E.W. Maddison.

11. The first publication of the astronomical revelations of the telescope was
made in Galileo’s Sidereus nuncius (1610), for which see notes 3 & 5 supra.

12. Kepler’s Somnium (see notes 6 & 9 supra) also contains a series of notes; these
are very extensive and by far outweigh the short text.

13. The spheres are purely imaginary since Tycho’s studies of the motion of a
comet through the solar system had shown conclusively that real spheres (say of
crystal) could not exist. Kepler says explicitly in the Mysterium cosmographicum
that the spheres are not supposed to be real or physical, citing Tycho’s work.

14. Harmonice mundi, Book 5, ch. 3; see Koyré’s Astronomical revolution (cited
in note 10 supra), pp. 330-331.

15, See I.B. Cohen: “Perfect numbers in the Copernican system: Rheticus and
Huygens,” Studia Copernicana, 1978, vol. 16 (“Science and History: Studies in
honor of Edward Rosen”), pp. 419-425.


The Latin text is a facsimile of the second edition (Frankfurt, 1621). The two edi-
tions (Tibingen, 1596 and Frankfurt, 1621) are designated by T and F respective-
ly. Misprints and textual variations are printed in roman, editorial explanations in
italic. In general, misprints in T have only been indicated when these also occur in
F. However, the variant readings in 7 enable the reader to recognize those
misprints which originate in F. Again only variants in words are indicated, so that
differences in spelling or in abbreviated forms are not shown. The number
references to Kepler’s own notes were of course added in F but their absence in T
has not been indicated in each individual case.
The following are a few examples of the notation used.

42) repererit /

This means that, in line 42 of the relevant page (here p. 54), there is a mis-

print, the correct word being repererit.

1) T: Verum hoc pacto neque

This means that, on line 1 (p. 64), in place of Verum neque, T has Verum

hoc pacto neque.

Notae Auctoris in F only. ;

This means that the collection of notes appended to the chapter is to be

found only in F.

The following abbreviations are used by Kepler in the text.

Latin Greek
efor a «9 for po
& for et ¢ for or
q; for que 6 for ov

for que

sc. for scilicet
n, for enim

a, é, U for am, em, um
ét for etiam

Footnotes to Latin text, showing misprints, variants, etc. Numbers in brackets {] are pages in this
edition

[36] 1) This epigram (Latin translation only) (72) 1) sit
was printed on the title page in T. 5) maximae

[38] Epistola Dedicatoria in F only. 15) fol. 26
24) Astronomiae 176] 19) T: apparet, quod

{40} 32) aliis 21) T: demonstravit, et quod ex ¢0 nulla

[44] 17) lugentibus sit causa, simile

[48] LK. added at the end of the poem in T. [80] 35) angulo TGV

[50] Novae in F only, [82] 3) Planetarum. (14) Nam

[52] 24) atqui 7: atqui [84] Notae Auctoris in F only.

[54] 42) repererit {92} 5/6) ouo, qua 7: uo, qua

[56] Novae Auctoris in F only. (98] 2/3) (ut ipse Rhetico dicere solitus est)

[60] 35) distinctis
[62] 1) 7: Praefatio Ad Lectorum

26) T: &e 95

31) intermedi
[64] 1) T: Verum hoc pacto neque
[68] 7) Dodecaedron 7: Dodecaedron
170] Notae Auctoris in F only.

18) fol. 437 & 438

34) fol. 145

T: (ut ipse Rhetico dicere solitus est)
4) credidit ea T: credidit ea

[100] 4) T: Annotatio in Caput Secundum
35) constituitur
38) constituitur
40) sane alijs

[102] Notae Auctoris in F only.

{110} 29) 7: etiam basium non

(112) 34) multifacia 7: multifacia

[114] Notae Auctoris in F only.

[118] Notae Auctoris in F only.
Apparatus Criticus

[120] _Notae Auctoris in F only.

{122} 38) Dodecaedron 7: Dodecaedron
39) Dodecaedri

[124] 374) 7: igitur per medium, decem
lateribus, talem describit viam,
transeunte
‘Notae Auctoris in F only.

[126] 37) lateribus

[134] 30) repererit

[136] 20) patet, (32) quare
‘Notae Auctoris in F only.
39) in terna

{140} 42) Harmonicarum

(142) 7) uni
13) ut in Harmon. |i
22) rectanguli
48/49) per sectionem

[144] 21) scripta
25) Trisdiapason cum epidiapente
4) repudiat

33) T: autem ex HI
39) cubi habeat



[152] 5) 7: em Octaedron
10) exsecto
23/25) Breviter...NM 36 in F only.
‘Notae Auctoris in F only.
(154) 43/44) repleatur
[156] 4) ob oculos 7: ob oculos
4/6) Parentheses in F only.
25) T: terreni, et lunam comprehendens
1158] Notae Auctoris in F only.
7) diversos
40) eorum
‘Notae Auctoris in F only.
[168] Notae Auctoris in F only.
8) solis
19) spissitudinis
19/20) estimae


(170) 37) impediunt
[174] 15) T: Porro, varietas
26/27) emexeinua T: emexedotua
‘Notae Auctoris in F only.
37/38) aliorsum
{180} 6) post hane 7: post hane
(182) 7) raw
(184) 8) anni 1551 7: anni 1551
39) possemus
[186] Notae Auctoris in F only.
15/16) Geometricas
29) considerarent
[188] 13) specie 7: specie
[190] 10) ab ipsis
18) inferiorum theorijs 7: inferiorum
theorijs
[192] 1) sollicite
[194] Norae Auctoris in F only.
10) ante 17. annos
200} 11) Venerium
[202] 23) 7: est ambo theoremata
‘Notae Auctoris in F only.
42) delapsam
(204) 23) 5261 dat distantiam Martis
[206] 30) lucens
44) attestantur
[208] 5) orbes
28/29) iugulare
38) confessionem

41) falso
[214] 20) inter medias
(216) 12) tardus
[218] Notae Auctoris in F only.
[222] 8) boreo,
M1) erit Pin Se 7: eit R ine
21) extrema
Notae Auctoris in F only.
42) Astronomics
(224) 14) si ineffabiles


Kepler writes sound Renaissance Latin, which is on the whole correct Ciceronian
Latin with the exception of a few constructions not found in the best classical
prose. For instance, although he sometimes uses the accusative and infinitive for
short indirect statements, he generally uses quod with the subjunctive, particular-
ly for longer indirect statements. This construction is not found in formal
classical Latin but later became normal.

Similarly, although he has a taste for slightly unusual and colorful words and
phrases, Kepler’s vocabulary is essentially that of classical Latin. There are occa-
sional exceptions. For instance, in note 37 to chapter 12 in the second edition
(page 146) he seems to use causor to mean “cause” rather than in its normal post-
Augustan sense of “give as a pretext”; and like Copernicus he occasionally uses
ipse simply as a definite article, especially with indeclinable nouns or with
mathematical expressions used as nouns, where there is no other means but the
declension of ipse to indicate the case, as in the phrase ipsius AC in note 4 to
chapter 22 (page 218).

Like many other authors of his time, Kepler frequently uses Greek words,
usually but not always in Greek script, where there is no obvious reason for not
using the equivalent Latin word. For example, he uses the Greek for the title of
Copernicus’s De revolutionibus more than once (for instance on page 182),
although there is no apparent reason for not using the Latin title, as he does
elsewhere, and although the book was, after all, written in Latin. Caspar (KGW
1:430) explains that this particular Greek phrase is derived from Rheticus’s
Ephemerides (Leipzig, 1550). Originally no doubt authors used Greek words and
phrases in this way because a knowledge of Greek was a recent acquisition of
Renaissance scholars and its use seemed to add a touch of distinction to their
styles; or possibly they may just have been showing off their knowledge.
However, by Kepler’s time the habit had become so common that Greek words
might be used almost without thinking. The effect is rather like the use of occa-
sional French words or phrases in English. For the sake of clarity, however, a
French equivalent for a Greek word has been used in only one or two cases in this
translation, where the meaning is obvious; and the English translation has usually
been given.

Kepler is inclined to slight inaccuracies in the classical references with which he
embellishes his prose. For instance, in chapter 18 (page 178) he misquotes
Horace, Epistles, 1, i, 32 as “Est aliqua prodire tenus” instead of “Est quadam
prodire tenus.” The misquoted version scans correctly, and is also found in De
cometa anni 1607 (KGW 8:139). However, such slips occur only in allusions
which do not affect the sense of what Kepler is saying, and probably show merely
that he did not always have the means of verifying such references at hand.

The following words need a particular comment.

Artifex. This word has been translated “practitioner.” Kepler refers to Tycho
Brahe as summus Artifex (“the leading practitioner”) in the notes on the original
dedication added to the second edition (page 58). It seems to mean a practicing
astronomer who actually makes observations of the positions of stars and calcula
tions based on a given cosmographical model. In the first paragraph of chapter 18
(page 176) he distinguishes them from the cosmographers or physicists, who are
presumably those who construct cosmographical models of spheres and so forth
without quantitative data. Clearly Kepler counts himself not in this class, but as a
practitioner. In the second paragraph of chapter 18 he seems to distinguish practi-
tioners also from astronomers, who are presumably those who use the results of

Translator’s Notes

the practitioners in practice. However, in chapter 19, pages 188-90, astronomers
and practitioners seem to be the same.

Corpus. Sometimes, as in chapter 2 (page 94), Kepler uses corpus to mean “mat-
ter,” though not quite in the full Aristotelean sense. Elsewhere he uses it to mean
a body such as a star. However, where he uses the word to refer to the five perfect
solids it has been translated as “s¢ ” as it would sound very odd in English to
speak of “the five perfect bodie:

Demonstro and demonstratio. These words, which mean literally “show” or
“demonstrate” and “demonstration,” are sometimes used in that general sense,
sometimes to mean “prove” or “proof,” and once to mean the geometrical con-
struction for particular figures, referring presumably to the demonstration that
the construction does yield the figure concerned (in note 16 on chapter 12, page
142). Often, however, they mean the process of inferring a particular set of data,
or formula for calculation, from the system of circles or other figures assumed as
the hypothesis for the principles governing the phenomena concerned. This pro-
cess is called in English “deriving” a formula or set of data, and the words
“derive” and “derivation” have therefore been used to translate demonstro and
demonstratio in such cases, where “prove” or “proof” would not represent thera
correctly. Copernicus habitually uses those words in this sense.

Exorno. When Kepler writes of a particular planet or group of planets being
associated with a geometrical construction of spheres or circles which will account
for its observed motions, he sometimes refers to it as being “fitted out,” using the
word exorno in its primary sense of equipping, fitting out, or furnishing.
However, the word also carries with it something of the secondary but common
sense of embellishing or decorating. It has therefore been paraphrased in this
translation as “display,” which is intended to convey the connotations of exorno
rather than its literal meaning.

Mundus. In Kepler’s time, as well as before him and long after, the universe was
generally supposed to be bounded by the sphere of the fixed stars, and there was
therefore little ambiguity in using mundus as if it were synonymous with univer-
sum, the universe. However, properly speaking, mundus in such writers as Coper-
nicus and Kepler refers only to that part of the universe which includes the sun,
moon, earth, and other planets, and is bounded by the visible stars. It is difficult
to avoid translating the word as “universe,” as has been done here; but no answer
is implied to the question whether there is anything outside the fixed stars.

Orbis. This word can mean either a sphere, such as the hollow spheres which are
supposed to make up the mundus, or the circle, passing through the thickness of
the sphere, on which the planets move. This ambiguity has been pointed out by
Edward Rosen in Three Copernican Treatises, (nd edition [New York, 1959],
pp. 13-21). Birkenmajer recommended therefore that orbis should be translated
into Polish by the word krag, which had a similar ambiguity, though he did not
follow his own recommendation (see Copernicus, Opera Omnia, Vol. 11 [War-
saw, 1975], p. 356). However, there is no suitable word in English which has such
a convenient ambiguity. Where Kepler clearly means a two-dimensional circle, as
in chapter | (Plate I) where he is referring to circles drawn in the diagram to repre-
sent the orbits, orbis has been translated as “orbit.” Elsewhere orbis must refer to
the solid sphere, as in the title of chapter 13 where Kepler discusses the spheres in-
scribed within the five perfect solids; and in those cases the word has been
translated “sphere,” though the same English term has been used where Kepler
uses sphaera. Nevertheless, there are many instances where orbis is used with little
indication, or no certain indication, of which sense is intended. In those instances
“orbit” or “sphere” has been chosen according to which seemed the most probable.

Translator’s Notes

Prosthaphaeresis. This is a Greek word, usually written by Kepler in Latin script,
meaning a correction to be added to or subtracted from a mean position in calcu-
lating a true position. However, as the word prosthaphaeresis is used in the
history of mathematics to mean a quite different process, the word has been
translated here as “equation,” representing the equivalent Latin term aequatio.

Proportio and ratio. In English the word “ratio” generally refers to the relation-
ship between a pair of numbers, and “proportion” to the relationship between the
various dimensions of a figure or between a series of more than two numbers.
This distinction has generally been maintained in the present translation.
However, the distinction between the two corresponding words in Kepler’s Latin
is not the same; and indeed it has often seemed appropriate to translate proportio
as “ratio” and ratio as “proportion,” though not invariably.

Quantitas. In some cases Kepler uses quantitas to mean merely “amount” or
“quantity” in a general sense. However, in chapter 2 and occasionally elsewhere
he uses the word in a special technical sense to mean whatever is capable of being
represented quantitatively. Thus on page 94 he writes “We see three kinds of
quantity in the universe: the shape, number, and extension of objects.” The
English word “quantity” has been used to translate quantitas in this sense,
although its normal meaning is not quite as wide as Kepler’s use of the Latin,
since there is no exact English equivalent. Kepler also sometimes uses quantum as
a noun equivalent to quantitas in either sense.

Species. Kepler sometimes uses species in its ordinary senses to mean “ap-
pearance” or “kind.” However, in the particular phrase species immateriata,
which he uses for instance in note 4 to chapter 16 and in note 3 to chapter 20, in
the second edition, he clearly means something akin to the Neoplatonic sense of
an emanation flowing from God. Species was occasionally used in classical Latin
to mean “vision,” as in Lucretius, De rerum natura 1V:236, 242, and V:707, 724.
However, it was also used to mean a very fine simulacrum of itself which was
given off by a visible body, and by the effect of which on the eye it was seen. For
the Epicureans, of course, such a simulacrum would be material; but a species
might also be immaterial. In the later Middle Ages the word was used in this sense
by Robert Grosseteste and Roger Bacon, under the influence of Neoplatonic phi-
losophy, to mean a power propagated by a body, of which light was only one ex-
ample. (See A.C. Crombie, Robert Grosseteste and the Origins of Experimental
Science, 2nd ed. Oxford, 1962 (1953), pp. 104-16 and 144-47). The phrase species
immateriata has therefore been translated here as “immaterial emanation.”
Vale. The Romans put the names of both the writer of a letter and the person to
whom it was addressed at the beginning of the letter, and at the end of it wrote
Vale or Valete (literally “farewell”). This pattern was followed in Renaissance
Latin. However, since the modern custom is to put the writer’s name at the end of
a letter, Vale has here been translated simply as “J. Kepler.”

blank page


Johannes Kepler was born in Weil der Stadt on 27 December 1571* and
six years later moved with his parents to nearby Leonberg. In 1589, after
attending the Cloister Schools in Adelberg and Maulbronn, he entered the
theological college of the University of Tiibingen, at that time one of the
leading Lutheran centers of higher education. For the first two years
Kepler studied in the faculty of arts, taking his M.A. in 1591, and then
entered on his theological studies with the intention of becoming a pastor
in the Lutheran Church. A few months before he was due to complete his
theological studies in 1594, there occurred an event which proved to be a
decisive turning point in Kepler’s life. Georg Stadius, the mathematics
teacher at the Protestant School in Graz, had died, and in response toa re-
quest from the school authorities, the theological faculty in Tibingen
recommended Kepler for the post (KGW 19, 3).! Although Kepler had
already begun to question some Lutheran doctrines—and indeed would
have encountered great difficulties if he had become a Lutheran pastor —
this lack of orthodoxy could not have been the reason why the Tiibingen
theologians sent him to Graz, for at this time, as Kepler himself relates,?
he had, on account of his youth, kept his theological doubts to himself.
There is no reason to suppose that the Tiibingen theologians had any
other motive than the desire to recommend the candidate with the best
qualifications, who would do credit to the University. In accepting the
post, Kepler expressed his wish to be allowed later to return to his
theological studies (KGW 13, 9-11).

Kepler had been introduced to the Copernican system by his teacher
Michael Maestlin at the University of Tiibingen. On the basis of
Maestlin’s lectures and his own reflections, he gradually compiled a list of
superiorities of Copernicus over Ptolemy from the mathematical point of
view (KGW 1, 9). At the time of these early studies, Kepler had evidently
not read the Narratio prima, for he remarked later that Rheticus, who had
made the comparison briefly and clearly in his Narratio prima, could have
saved him the trouble of compiling the list himself. In Graz, Kepler made
use of both the Narratio prima of Rheticus and the De revolutionibus of
Copernicus, of which he possessed his own copy (KGW 13, 45).*

Kepler was attracted to the Copernican system because, in his view, this
system alone provided the reasons for things which in others provoked
astonishment. In effect, each of the motions attributed to the earth by
Copernicus explained some irregularity or apparent coincidence in the
motions of the other planets (KGW 1, 17-18). Whereas Copernicus,
however, had recognized the wonderful arrangement of the world @
posteriori from the observations, Kepler claimed that this could have been
proved a priori from the idea of creation. According to his own account,
the decisive insight that led him to discover, as he thought, God’s plan for
the construction of the universe in the five regular polyhedra or Platonic

Introduction

solids came to him on 19 July 1595 during the course of a lecture to his
class on the great conjunctions of Jupiter and Saturn. (Cf. KGW 13, 28).
The pattern of these conjunctions suggested to Kepler’s receptive mind
that the distances of Jupiter and Saturn might be approximated by the
radii of the circumscribed and inscribed circles of an equilateral triangle.
Reflecting that this figure was the first regular polygon, he tried to repre-
sent the distances of the other planets by means of a sequence of such
polygons, inscribing a square in the circle of Jupiter, a circle in this square
(to represent the orbit of Mars), a regular pentagon in this circle, and so
on. But he found that this scheme failed to represent the distances of the
planets in general, and he had to recognize that, in any case, it could not
explain the restriction of the number of planets to six. He then reflected
that two-dimensional figures were inappropriate to explain the arrange-
ment of solid planets. Clearly a finite set of three-dimensional figures was
needed, and this brought to mind the five Platonic solids.

On 2 August 1595 (0.s.) Kepler communicated to Maestlin the first
results of his attempt to deduce the distances of the planets a priori,
remarking that nothing is fashioned by God without a plan (KGW 13, 27),
but making no mention of the polyhedral hypothesis. In a second letter,
written on 14 September 1595, he gave a brief account of the polyhedral
hypothesis and also of his attempt to explain the relation between the
distances and periodic times of the planets. While the polyhedral
hypothesis provided the reasons for the number, order and magnitudes of
the planetary spheres, in order to explain the motions Kepler postulated
an anima movens in the sun, whose efficacy (vigor motus) weakened with
distance from the sun, in the same way that the intensity of light weakened
with distance from the source. At this time, however, he believed that the
weakening depended on the distance according to a relation involving the
sine function. Kepler concluded his letter by asking Maestlin for his opin-
ion concerning these ideas.

Without waiting for a reply to his last two letters, Kepler wrote again to
Maestlin on 3 October 1595, giving the first full account of the polyhedral
hypothesis and asking whether Maestlin could recommend Georg Grup-
penbach in Tiibingen as a suitable printer to be entrusted with the work
(KGW 13, 39). Besides a detailed description of the polyhedral hypothesis
and his view concerning the principles underlying the construction of the
world, this letter also contains a clarification of his explanation of the ef-
fect of the anima movens in the sun. Having abandoned the formula i n
volving the sine function, Kepler now supposed that this force, like the in-
tensity of light (which he described as spreading out in a circle, not a
sphere), weakened in proportion to the distance from the source (KGW
13, 38). Here we see the beginnings of the physical theory that led Kepler
to the discovery of the area law and the elliptical orbits of the planets.
Later he discovered the inverse-square law for the intensity of light

Introduction

emanating from a source. This fundamental law of photometry is first
stated in his Ad Vitellionem paralipomena (KGW 2, 22).

Describing his plans for the book, Kepler explained to Maestlin that, at
the beginning, he intended to introduce some theses to show that the
Copernican hypothesis was not opposed to Scripture (KGW 13, 34). Con-
cerning the principles underlying the construction of the world, he main-
tained that these were to be sought not in the idea of pure numbers but in
the concept of geometrical relations. For the properties of pure numbers
(except those of the Trinity, which was God himself), Kepler regarded as
accidental, whereas the properties of geometrical relations, such as those
associated with the regular polyhedra and the musical ratios, he held to be
grounded in nature (KGW 13, 35). Fundamental among geometrical rela-
tions, in Kepler’s view, was the distinction between the curved and the
straight, by means of which Nicholas of Cusa (who is not mentioned in
the letter) had expressed the relation of God to the creation. Towards the
end of the letter, Kepler declared, “I wished to be a theologian; for a long
time I was troubled, but now see how God is also praised through my
work in astronomy” (KGW 13, 40). Evidently, he had abandoned the idea
of becoming a Lutheran pastor, having found his true vocation in
astronomy, From this time, he regarded himself as a priest of the Book of
Nature.*

Towards the end of January 1596 Kepler was given two months leave
from his post in Graz in order to visit his ailing grandfathers. He took the
opportunity afforded by this visit to consult Maestlin in person and ar-
range with Gruppenbach for the publication of his work. Kepler did not
in fact return to Graz until August, though most of this time was spent in
Stuttgart at the court of the Duke of Wiirttemberg, negotiating the con-
struction of a model of his new system in the form of a “Kredenzbecher,”
which the Duke had authorized (KGW 13, 50-54 and 74-75). On his
return, the authorities in Graz accepted Kepler’s explanations for his ex-
tended absence and granted his request for the payment of his salary in re-
spect of this period (KGW 13, 94 and KGW 19, 11-12).

Among the problems that Kepler put to Maestlin was the following.
Whereas Copernicus had taken the center of the earth’s orbit as his
reference point (in Kepler’s view, so as not to confuse the reader by de-
parting too much from Ptolemy), a valid test of the polyhedral hypothesis
needed a comparison with the distances of the planets from the true sun.
Maestlin calculated these distances for Kepler, after he had first com-
puted the dimensions of the Copernican planetary representations anew
from the Prutenic tables. The new dimensions are appended to a letter of
27 February 1596 (KGW 13, 56-65), which contains Maestlin’s first writ-
ten comments on Kepler’s discovery, and also printed as an appendix to
the Mysterium cosmographicum (KGW 1, 132-145).5 The distances from
the sun and the table of apogees and aphelions, together with illustrative
diagrams prepared by Maestlin, appear only in the Mysterium cosmo-
Introduction

Bible (KGW 13, 203). The substance of the omitted chapter was later
published in the introduction to the Astronomia nova.

Writing to Maestlin on 9 April 1597, soon after the publication of the
Mysterium cosmographicum, Kepler expressed relief that the defenders
of Scripture had raised no objections against his book (KGW 13, 113). Six
months later, Maestlin informed him that some theologians were not
pleased with the book and that Hafenreffer, in the course of a sermon,
had declared that God did not hang up the sun in the middle of the
universe like a lantern in the middle of a room (KGW 13, 151). Maestlin
added, however, that the critics were inhibited from open hostility by
Kepler’s dedication of the key diagram to the Duke (see frontispiece). The
Duke’s attitude was no doubt influenced decisively by Maestlin’s state-
ment (in his letter of 12 March 1596) that, while the ancient hypotheses
were easier to understand, and for that reason were taught to beginners,
nevertheless all practitioners (artifices) agreed with the demonstrations of
Copernicus (KGW 13, 68). Thus the Duke had Maestlin’s authoritative
confirmation for the statement of Kepler himself (in his letter of 29
February 1596), that all the famous astronomers (berhiimbte astronomi)
of their time followed Copernicus rather than Ptolemy and Alfonso
(KGW 13, 66). Maestlin’s more carefully worded statement, which em-
phasizes a concern with the technical aspects of Copernican astronomy,
would truthfully include Tycho Brahe, who regarded the system of
Copernicus as mathematically admirable, although not in accord with
physical principles. In particular, he followed Copernicus in rejecting
Ptolemy’s equant. Concerning Hafenreffer, Kepler expressed the view to
Maestlin that he was really a secret Copernican, whose advice to treat this
system as a mathematical hypothesis was prompted simply by his desire to
avoid dissension in the Lutheran Church (KGW 13, 231).

Although Maestlin himself was committed to the Copernican system
and in sympathy with Kepler’s a priori reasons, such as the polyhedral
hypothesis, he was critical of Kepler’s speculations concerning the anima
movens in the sun, suggesting that this idea would be the ruin of
astronomy (KGW 13, 111). For Maestlin, the distinction was between
mathematical and physical hypotheses. It is curious that, even after the
clear success of physical reasoning in Kepler’s Astronomia nova, Maestlin
explicitly rejected phsyical astronomy in a letter of 21 September 1616

On 13 December 1597 Kepler sent a presentation copy of his Mysterium
cosmographicum to Tycho Brahe in Denmark (KGW 13, 154-155). When.
this reached Tycho in March of the following year in Wandsbek, he
replied to Kepler in friendly terms, inviting him to make a visit that would
allow personal discussion. Acknowledging that, without doubt, God had
a harmonius plan for the creation, Tycho suggested that a better test of
the polyhedral hypothesis would be possible, if the true values of the ec-
centricities, which he had sought to determine over a number of years,


Introduction

were substituted for those used by Kepler (KGW 13, 197-200). It was for
the purpose of obtaining these values of the eccentricities, in order to
make such a test, that Kepler visited Tycho in Prague early in 1600 (KGW
14, 128). Tycho’s interest was attracted by Kepler’s ability rather than by
the polyhedral hypothesis, for in a letter to Maestlin, written soon after
his invitation to Kepler, he made clear his view that progress in astronomy
could not be expected from a priori deductions but only from more ac-
curate observations (KGW 13, 204-205).

In his preface to the reader, at the beginning of the Mysterium cosmo-
graphicum, Kepler explains that there were above all three things whose
causes he sought; namely, the number, magnitudes and motions of the
planetary spheres. From the beginning, as is evident from his cor-
respondence with Maestlin, Kepler envisaged two types of causes, ex-
emplified by the polyhedral hypothesis and the anima movens that he
postulated in the sun. The first may be described as a final cause, for it
reflects God’s purpose to create the most beautiful and perfect world,
while the second has the character of an efficient cause. In thus combining
final and efficient causes Kepler was in fact following Plato. For in the
Timaeus (46D-E), Plato emphasizes that, in explaining the origins of in-
dividual things, both mechanical causes and divine purposes must be con-
sidered, and moreover, if we wish to attain a true scientific explanation
satisfying to the human reason, we must be primarily concerned with the
causes that lie outside the material in the realm of the spiritual. Aesthetic
principles, such as those of beauty and perfection, will serve as guides in
the search for a@ priori causes; for Kepler claims, quoting Cicero’s transla-
tion of the Timaeus, that it was not possible for the perfect architect to
create anything other than the most beautiful (KGW 1, 23-24. Cf.
Timaeus 30A).

The general idea of the world as the visible image of God, which we find
at the end of the Timaeus (92C), is in keeping with many passages of the
Bible (e.g. Romans 1, v. 20) and came to be transformed by Christian
writers into the concept of the Book of Nature. In his Compendium
theologiae, the Tiibingen theologian Jakob Heerbrand described this con-
cept as embracing “the whole universe, the world and everything that is in
it,” and he also took the beauty of the universe as the basis of his first
argument for the existence of God. These ideas are so closely paralleled in
Kepler’s thought that a direct influence seems likely.”

At the beginning of chapter two, where he outlined his principal thesis,
Kepler raised the question why God had first created material bodies. The
key to the solution of this problem he found in the comparison of God
with the “curved” and of created nature with the “straight,” which had
been made by Nicholas of Cusa and others (KGW 1, 23). Kepler saw the
harmony between the things at rest, in the order sun, sphere of fixed stars
and intervening space, as a symbol of that between the three Persons of
the Trinity (KGW 1, 9 and 23). In seeking a similar harmony for the things

Introduction

in motion, namely the planetary spheres, he was led to speculate on the
divine cosmological intention. In Kepler’s view, God intended that we
should discover the plan of creation by sharing in his thoughts (KGW 13,
309). First, it seemed to Kepler that such a useful idea as the distinction
between the curved and the straight could not have arisen by accident, but
must have been contrived in the beginning by God, according to his
decrees. Then, in order that the world should be the best and most
beautiful and reveal his image, Kepler supposed, God had created
magnitudes and designed quantities whose nature was locked in the
distinction between the curved and the straight, and to bring these quan-
tities into being, he created bodies before all other things (KGW 1, 24). As
the eye is for colors and the ear for sounds, Kepler wrote to Maestlin on 9
April 1597, so is the mind or intellect for the knowledge of quantity
(KGW 13, 113. Cf. Timaeus 46D-47E).* Seeking to comprehend God’s
thoughts through human thoughts, however, was like trying to reach the
curved through the straight, so that, in Kepler’s view, certainty was im-
possible. Consequently, his a priori reasons were only probable and need-
ed to be tested against observations (KGW 1, 24 and 71).

Kepler first ordered the solids by comparing the differences between the
radii of their circumscribed and inscribed spheres with the intervals be-
tween the planets (KGW 1, 27). Then, in chapters three to eight, he gave
the a priori reasons for the order thus indicated by the data. This provides
an example of Kepler’s methodological principle that hypotheses must be
“built upon and confirmed by observations” (KGW 14, 412). Of special
interest in his ordering of the solids is the position of the earth. In Kepler’s
view, the regular solids fall naturally into two classes. The first class com-
prised the cube, tetrahedron and dodecahedron, possessing faces of dif-
ferent shapes and vertices common to three faces, while the second con-
sisted of the octahedron and icosahedron, possessing faces of the same
shape and vertices common to four or five faces (KGW 1, 29). As the
abode of man, the earth occupied a privileged place between the two
classes (a kind of geocentrism of importance), and had also been pro-
vided, unlike the other planets, with a satellite of similar nature (KGW 1,

Before introducing the test of the polyhedral hypothesis against the em-
pirical data, Kepler interpolated four chapters on astrological and har-
monic questions. Although he knew of the existence of Ptolemy’s ‘Har-
monica and the commentary by Porphyry, he had not yet read them.
Moreover, his ideas on astrology had probably not yet completely
crystallized, so that it is perhaps not surprising that he later became
dissatisfied with these chapters. In the second edition, chapter nine, on
the astrological properties of the planets, is described as a digression
(KGW 8, 59), while chapter eleven, on the origin of the zodiac, is dis-
missed as meaningless (KGW 8, 62), and in chapter ten, the primary
source of the numeri numerati is transferred from the regular polyhedra

Introduction

to the division of the circle into plane polygons (KGW 8, 60). Chapter
twelve, in which Kepler discussed the division of the zodiac and the
astrological aspects, is especially interesting for the introduction of the
idea of a correlation between the properties of the regular solids and the
aspects on the one hand, and the musical harmonies on the other (KGW 1,
39-43). At the beginning of this chapter, which is heavily annotated in the
second edition, Kepler remarks that many were of the opinion that the
division of the zodiac into twelve signs was arbitrary, a view he himself
vigorously defends in De stella nova (KGW 1, 168-172). Indeed, as he
points out in a letter to Herwart (KGW 15, 453), the only part of tradi-
tional astrology he had retained in this work was that relating to the
aspects. These derived their efficacy from their grounding in the
geometrical structure underlying the natural world. Kepler’s definitive ac-
count of the efficacy of the aspects and their relation to the musical har-
monies is given in the Harmonice mundi (Book 4, chapters 6 and 7).

Kepler’s first test of the polyhedral hypothesis, described in chapter
fourteen, compares the ratio of the least distance of each planet and the
greatest distance of the one immediately below, as predicted by the
hypothesis, with the corresponding ratio of the distances given by Coper-
nicus (KGW 1, 98). In this test, the distances are taken from the center of
the earth’s orbit, except for the pairs Mars-earth and earth-Venus, where
they are taken from the sun, so as to give the earth’s sphere a thickness
equal to the eccentricity of the earth’s orbit.

In chapter fifteen we reach the core of the Mysterium cosmographi-
cum, for this contains Maestlin’s diagrams and tables together with
Kepler’s test of the polyhedral hypothesis using the distances from the sun
calculated by Maestlin. To some extent, the test is marred by major errors
in Maestlin’s calculation of the distances of Venus and Saturn, and as an
added confusion, the distances of Mercury on which Kepler had based his
calculation (given to him by Maestlin in his letter of 11 April 1596) had
been revised by Maestlin (presumably without Kepler’s knowledge) during
the printing of the Mysterium cosmographicum.

To calculate the distances of the planets acccording to the polyhedral
hypothesis, Kepler started with the earth’s sphere, taking for the radii of
the inner and outer surfaces the least and greatest distances of the earth
from the sun. The outer surface of the earth’s sphere was then taken as
the inscribed sphere of the dodecahedron, whose circumscribed sphere
became the inner surface of the sphere of Mars. The radius of the outer
sphere of Mars (that is, the theoretical greatest distance of Mars) was then.
calculated from the known radius of the inner sphere, using the Coper-
nican value of the eccentricity. This process was continued upwards to
Saturn and downwards to Mercury. In the case of Mercury, however,
Kepler found that a better fit was obtained by taking the circle in the
octahedron-square instead of the inscribed sphere as the outer bound of
the orbit. He adduced a priori reasons to justify this exception from the

Introduction

general rule. For example, Mercury was Special among the planets in hav-
ing an eccentric whose radius varied according to the position of the earth
in relation to the apsides (De revolutionibus, Book 5, chapter 27), while
the octahedron was similarly exceptional among the regular polyhedra in
having the possibility of an unobstructed circular path outside the in-
scribed sphere (KGW 1, 58). In the second edition, he explained that the
reason for the peculiarity of Mercury did not, after all, lie in the octa-
hedron (KGW 8, 97. Cf. KGW 7, 435).

Kepler constructed two versions of the polyhedral hypothesis. In the
first, the earth’s sphere was given a thickness equal to the eccentricity of
the earth’s orbit, while in the second, the thickness of the earth’s sphere
was increased to include the moon’s orbit. As he had no a priori reasons
for preferring either version, Kepler expressed his willingness to choose
whichever gave the better fit. If the first version were chosen, no difficulty
would arise in relation to the moon’s orbit cutting the solids, because
these, as Kepler emphasized, were not material (KGW 1, 55). It was the
apparent connection of the earth and the moon that led Kepler to his
earliest speculations on gravity. Inclining to the Platonic view, described
in the Timaeus (63C-E), according to which bodies of the same nature are
drawn together, Kepler explained that the moon, being of the same nature
as the earth (an idea he attributed to Maestlin), follows or is drawn
wherever the earth goes (KGW 1, 55).

Besides computing tables of distances of the planets according to the
polyhedral hypothesis, Kepler also computed tables of angles which bring
out clearly the correspondence with the Copernican data. For Venus and
Mercury, the sine of the angle is taken to be the greatest distance of the
planet from the sun when the mean distance of the earth from the sun is
taken as a unit, so that the angle represents approximately the maximum
elongation of the planet from the sun. For Mars, Jupiter and Saturn, the
sine of the angle is taken to be the mean distance of the earth from the sun
when the greatest distance of the planet is taken as a unit, so that the angle
represents approximately the prosthaphaeresis in the apogee (KGW 1,
54). Tables I and II show, respectively, the distances and angles as given
by Kepler and (in parentheses) the corrected values, calculated from the
same Copernican data.

Although the errors in Kepler’s values prevented him from making a
choice between the two versions, the corrected angles show clearly that the
hypothesis in which the moon’s orbit is included in the earth’s sphere
gives the better fit and is indeed remarkably close to the Copernican data.
Kepler himself pointed out that the differences in the angles did not ex-
ceed the margin of error in the longitudes of the planets calculated from
the Prutenic tables (KGW 1, 65). Again, the small discrepancies could
arise from errors in the eccentricities. For Kepler, lacking a knowledge of
the a priori reasons of the eccentricities and their differences, had to use
the Copernican values, which were known to be unreliable (KGW 1, 60).


Introduction


Greatest and least distances of the planets from the sun, taking the mean distance of
the earth = 1000.

Polyhedral Polyhedral
According to Copernicus Hypothesis Hypothesis
(moon inc.)
Saturn 9987 (9727) 10599 (10011) 11304 (10588)
Jupiter $492 (5492) SIL (5109) 5451 (5403)
Mars 1649 (1648) 1551 (1550) 1658 (1639)
Earth 1042 (1042) 1042 (1042) 1102 (1102)
Venus 741 (721) 761 (762) 714 (714)
Mercury 489 (481) 306 (535) 474 (502)
Polyhedral Polyhedral
Hypothesis Diff. Copernicus ‘Diff. Hypothesis
(moon inc.)
Saturn 5°25" -20 5°45 —4r 3° 4
Jupiter 10° 17" =12 10°29 —6 10°23"
qe17) (+48) (10°. 29) (+11) __(10° 409
Mars 40° 9 42°47 37°22 +30 37°52"
Venus 49° 36 41°45 47°51 —2°18 45°33"
Mercury? 30° 23" 41° 4 29°19 =1e Y 28° 18"


The tables show Kepler’s values and (in parentheses) the correct values calculated
from the Copernican data used by Kepler.

These a priori reasons he eventually located in the ideas of musical har-
mony described in the Harmonice mundi, a work he started to plan in
1599, while the search for more accurate empirical values of the eccen-
tricities, as we have already remarked, led him to Tycho Brahe.

Having completed his proof that the a priori reasons of the distances of
the planets in the Copernican system were to be found in the five regular
solids, Kepler turned his attention from final causes to efficient causes,
seeking a confirmation of the distances in the effect of the moving soul
(anima motrix) in the sun on the motions of the planets (KGW 1, 68).

Introduction

Thus, in chapter 20, he introduced the concept of the solar force (weaken-
ing in proportion to the distance from the sun) and the theory of the mo-
tions of the planets that he had described to Maestlin in the letter of 3
October 1595.


Interpolation of the polyhedra so as to obtain the best fit with the distances
predicted by the theory of the motions and with the Copernican data.

Copernicus Motions Polyhedra
max circum-radius
mean distance mean distance in-radius
min

Saturn 9164 9163
Cube 9163
Jupiter 5246 5261
5000a
Tetrahedron 3000a
1648b
Mars 1520 1440
Dodecahedron 1393¢
102d
1042 ig 11024
Earth 1000 non 1000 1000
958e 898
Icosahedron 9586
762f
74th
Venus 719 7626
Octahedron 7alh
429g
Mercury 360 429g

In chapter twenty-one, Kepler attempted to bring the two causes, final
and efficient, together in a comparison with the Copernican data. The
results are shown in Table III. Kepler’s values of the Copernican
distances, in the first column, have been retained without correction, but
the arrangement of the third column has been changed to clarify Kepler’s
intention to interpolate the polyhedra so as to obtain the best fit (KGW 8,


Introduction

109 and 117). The mean distances calculated from the motions, given in
the second column, are free of arithmetical errors. Kepler regarded these
distances as more reliable than the Copernican data. The cube is found to
fit between the mean distances (based on the motions) of Saturn and
Jupiter, while the method of fitting the other solids is indicated by the let-
ters marking the starting and finishing points in each case.

Kepler’s preference for the distances based on the motions, and his
application, in chapter twenty-two, of the theory of the moving soul in the
sun to explain the Ptolemaic equant (and other representations used by
Copernicus) already point the way to the achievements of the Astronomia
nova, where physical reasoning (in the form of a search for efficient
causes) was to play a decisive role in the discovery of the first and second
laws of planetary motion.?°

Almost all the astronomical books written by Kepler (notably the
Astronomia nova and the Harmonice mundi) are concerned with the fur-
ther development and completion of themes that were introduced in the
Mysterium cosmographicum. The ideas of this work did not constitute
just a passing fancy of youth but rather the seeds from which Kepler’s
mature astronomy grew. When a new edition was called for, he decided
against changing the text itself, for a complete revision would have re-
quired the inclusion of all the main ideas of his other books (KGW 8, 10).
Instead, he simply added explanatory notes and references to his
definitive accounts of various topics given elsewhere, especially in the
Harmonice mundi and the Epitome astronomiae copernicanae. Kepler’s
correspondence gives no clues concerning the composition of these notes;
the only reference to them is contained in a letter to Bernegger of 11
August 1621, where Kepler remarks that Gottfried Tampach (the Frank-
furt publisher) was preparing a new printing of the Mysterium cosmo-
graphicum with his notes (KGW 18, 75). These notes were probably writ-
ten hurriedly —no attempt was made to correct the arithmetical errors of
the first edition—shortly before the book was published in Frankfurt in

Before Kepler was born, the French humanist Pierre de la Ramée
(Ramus) had called for a reform of astronomy by the rejection of hy-
potheses — that is, mathematical fictions such as epicycles having no basis
in nature —and he expressed the hope that one of the celebrated schools of
Germany would provide the philosopher and mathematician capable of
constructing this astronomy without hypotheses." Writing to Maestlin at
the beginning of October 1597, Kepler claimed that he (and Copernicus
also) had answered the challenge of Ramus, for he supposed that Ramus
had proposed only the rejection of fictitious hypotheses and not those
that were natural and true (KGW 13, 141. Cf. 165).?? Kepler returned to
this theme in the Astronomia nova, where he presented his claim on the
verso of the title page, and again, in the preface to the Tabulae
Rudolphinae, he mentioned among the causes for the long delay in

Introduction

publication, “the transfer of the whole of astronomy from fictitious
circles to natural causes.” Traditional astronomy had sought to “save the
appearances presented by the planets,”!? using mathematical hypotheses
of the kind condemned by Ramus. In place of this, Kepler substituted a
concept of astronomy as a science which sought to describe and explain
physical reality in terms of both final (aesthetic) and efficient
(mechanical) causes, by the invention of hypotheses based upon and con-
firmed by observations.’ From our vantage point we can see that, when
Kepler made the discovery forming the basis of the Mysterium cosmo-
graphicum, he had not just “discovered something new in astronomy,” as
Martin Crusius noted in his diary, but a new way of doing astronomy,
which may be seen (at least in part) as a return to the authentic teaching of
Plato in the Timaeus (in the sense of explanation in terms of both final
and efficient causes), thereby effecting a revolution in method which has
earned him the title of founder of modern astronomy.


*This introduction was published in an earlier version as an essay dedicated to
Bernhard Sticker (on his seventieth birthday), leader of the International Sym-
posium held in Weil der Stadt in 1971 to commemorate the quatercentenary of
Kepler’s birth. E. Aiton, Johannes Kepler and the ‘Mysterium cosmographicum,’
Sudhoffs Archiv, 61 (1977), 173-194.

1, KGW = Johannes Kepler, Gesammelte Werke, edited by Walther von Dyck,
Max Caspar, Franz Hammer and Martha List, Munich, 1937—.

2. Johannes Kepler, Se/bstzeugnisse, edited by Franz Hammer and translated by
Esther Hammer, Stuttgart-Bad Cannstatt, 1971, p. 63.

3. A facsimile reprint of Kepler’s copy of De revolutionibus, with introduction
by Johannes Miiller, has been published by Johnson Reprint Corporation, New
York and London, 1965. There are two new English translations: Copernicus, On
the revolutions of the heavenly spheres, translated by A. M. Duncan, London,
Vancouver and New York, 1976; Copernicus, On the revolutions, translated by Ed.
ward Rosen, Warsaw and London, 1978.

4. For Kepler’s own account of his theological development, see Johannes
Kepler, Selbstzeugnisse (see note 2 above), pp. 61-65. See also Jiirgen Hiibner,
Naturwissenschaft als Lobpreis des Schopfers, in Internationales Kepler-
Symposium Weil der Stadt 1971, edited by Fritz Krafft, Karl Meyer and Bernhard
Sticker, Hildesheim, 1973, pp. 335-356, and Martha List, Kepler und die Gegen-
reformation, in Kepler Festschrift 1971, edited by E. Preuss, Regensburg, 1971,
pp. 45-63. On Kepler’s theology, see Jurgen Hiibner, Die Theologie Johannes
Keplers zwischen Orthodoxie und Naturwissenschaft, Tiibingen, 1975.

5. There is an English translation by A. Grafton in Symposium on Copernicus,
Philadelphia, 1973 (= Proceedings of the American Philosophical Society, 117


Introduction

6. Kepler und Tiibingen (Tiibingen Kataloge Nummer 13), published by the
Kulturamt der Stadt Tiibingen, 1971, p. 29.

7. Internationales Kepler-Symposium (see note 4), pp. 338-340.

8. According to Rheticus, the function of the human mind was to understand
harmony and number. E. Rosen, Three Copernican treatises, New York, 1971,

° 9. The value 29° 19’ given in the middle column is inconsistent with the value
taken by Kepler for the distance of Mercury from the sun, namely 29’ 19” (with the
mean distance of the earth as 1°). The correct value is 29° 15’.

10. See C. Wilson, Kepler’s derivation of the elliptical path, Isis, 59 (1968), 5-25
and E. J. Aiton, Kepler’s second law of planetary motion, Isis, 60 (1969), 75-90.

11. See R. Hooykaas, Humanisme, science et réforme: Pierre de la Ramée,
Leiden, 1968, p. 67.

12. See E. Aiton, Johannes Kepler and the astronomy without hypotheses,
Japanese studies in the history of science, 14 (1975), 49-71.

13. Following a misinterpretation of Simplicius in his commentary on Aristotle’s
De caelo, this concept of astronomical method has been mistakenly attributed to
Plato. According to two recent analyses of this problem, it would seem that the idea
of saving the appearances originated. either with the Stoics of the time of
Posidonius or with Eudoxus. For the arguments relating the idea to Posidonius, see
Fritz Krafft, Physikalische Realitaét oder mathematische Hypothese? Philosophia
naturalis, 14 (1973), 243-275. Cf. Internationales Kepler-Symposium (see note 4),
pp. 64-66. On the attribution to Eudoxus, see Jiirgen Mittelstrass, Die Rettung der
Phdnomene. Ursprung und Geschichte eines antiken Forschungsprinzips, Berlin,
1962. Cf. Jiirgen Mittelstrass, Neuzeit und Aufkldrung, Berlin, 1970, pp. 250-263.
See also E. J. Aiton, Celestial spheres and circles, History of Science (on press).

14. On Kepler’s methodology see J. Mittelstrass, Wissenschaftsliche Elemente
der keplerschen Astronomie; R.S. Westmen, Kepler’s theory of hypothesis and the
realist dilemma; G. Buchdahl, Methodological aspects of Kepler’s theory of refrac-
tion. In Internationales Kepler-Symposium (see note 4). These papers have been
reprinted (that of Mittelstrass in English translation) in Studies in history and
Philosophy of science, 3 (1972), 203-298. See also J. L. Russell, Kepler and scien-
tific method, Vistas in astronomy, 18 (1975), 733-745.

[Title page of First Edition]

Prodromus




colorum numeti,magnitudinis, motnumque pe-
riodicorum genuinis & pro-
prijs,

regulariacorro1a Geometiica,

blank page


bergico, Muftrium Styria prohincia-
linin Mathemstico.

Quotide morior,fateorque:(ed inter Olympi
Dum tenet afliduas me mea cura vias:

Non pedibus terram contingo: fed ante Tonantem
Neétare,diuina pafcor & ambrofia.

RHETICI, de Libris Renolutionnm, atg, admirandis de numero, <> -
dine,e> diftantys Sphararium Mundi hypothefibus excellerit:/finit Ma-
thematicr , toting, Aftronomis Reftanrators D. NIC ULAL

wp Soe
g
TvpINnGs
Excudebat Gcorgius Gruppenbachius,

Prodromus
continens


bium cxleftium: deque caufisceelorum numeri, magni-
tudinis, motuumque periodicorum ge-
nuinis & proptiis,

Demonflratum per quinque regularia corpora Geometrica.

Libellus primum Tabingz in lucem datus Anno Chrifti
a

14s Ulsflrium Styria Prouincialinm Mathematico.

Nuncvero poft annos 25.ab eodem authore recognitus , & Notis notabilifimis
pareim emendatus, partim explicacus, partim conficmatus : denig; omnibus (uis
membris collatus ad alia cognati argumenti opera,quz Author exillotem-
pore fub duorum Impp. Rudolphi & Macthizaufpiciis; etiamg in
Uluftr. Ord. Auftriz Supr-Anifanz clientela
diuerfis locis edidic.

PotifSirnums ad ikuftrandas occafiones Oper is,Harmonice Mundi, diFiseinf-
que progre{fuum in materia cr methodo.

AdditacRerudita NARRATIO M. Grorert loacaimr RuETICI de
Libris Reuolutionum , atqueadrairandis de numero, ordine, & diftantiis Sphasa
rum Mundi hypothefibus,excellentiffimi Mathematici,totiufque Aftrononia Rec
fauratoris D. Nrcotar Corernicn


Einfdem LOAN IS KE PL ERE profuo Opere Harmonices Mundi APotocta aduer=
fas Demonftrationems Analyticam Cl. VD. Rokertide Fludibus, Mea
dici Oxonienfis.

Cum Priuilegio Cxfareo ad anaos XV.

Boe

Recufus Typis Enasmt Kempreri, fumptibus


[Title page of Second Edition]

Forerunner of the Cosmological Essays, which contains


On the Marvelous Proportion of the Celestial Spheres, and on
the true and particular causes of the number, size, and periodic
motions of the heavens,

Established by means of the five regular Geometric solids.

A little book first brought into the light of day at Tubingen in the Year of Christ
by
Master Johannes Kepler of Wiirttemberg, at that time
Mathematician of the Illustrious Districts of Styria;

Now after 25 years revised by the same author, partly emended,
partly explained, and partly confirmed by most remarkable
notes, and lastly compared in all its parts with other works hav-
ing a similar argument, which the author since that time has
published in various places under the auspices of the two
Emperors Rudolph and Matthias, and also under the patronage
of the Illustrious Orders of Austria over the Enns.

Especially to illustrate the relevance of the work entitled Harmonice Mundi,
and of its advances in matter and method.

In addition, the learned Narratio of Master George Joachim Rheticus, on the

Books of the Revolutions, and the wonderful hypotheses on the number, order

and distances of the Spheres of the Universe of the most excellent Mathemati-
cian and Restorer of the whole of Astronomy, Dr. Nicolaus Copernicus.

Also, the same Johannes Kepler’s Defense for his Work Harmonice Mundi,
against the Analytical Description of the famous Dr. Robert Fludde,
Physician of Oxford.

With Imperial Privilege for Fifteen Years.

Frankfurt,
Printed at the Press of Erasmus Kempfer, at the expense of
Godefried Tampach.
In the Year 1621.
Epigramma Polemaoadferiptum.
Epigram ascribed to Ptolemy

OF 3 Sarde dye x9} Ed uepOr. dm’ Sra dspuaw I know that Iam mortal and ephemeral.
Madi munivad dpoidpipus trina, But when I search for the close-knit en-
Ovx is’ PaYpade mei pains, dha map’ ads compassing convolutions of the stars,
Zhai horepeQiG- aiparapay duCpoains. my feet no longer touch the earth, but in
the presence of Zeus himself I take my
LaTIN« fill of ambrosia which the gods

produce.*

Quotidie morior, fateorque::fedinter Olympi
‘Dumtenet afiduas me meacuravias:

Nonpedibus terram contingo: fed ante Tonanters
Neitare,dinind pafcor G ambrofia.


Eristota DebIcaToria

sviisvs; ILtvsrrisvs, Geverosis, LL. Ba-
ronibus; Nobilibus,Screnuis,Equeftris Ordinis, DD. Pro-
uincialibus vniuerfis Splendidiffimi Ducatus Sty-
ri; Dominis meis gratiofif-
fimis.

EVERENDISSIME Princeps ; Admodum Reue-
rendt, Wuftres , Generoft, Nobiles , Strenut, Domunt
gratiofifims, Annus hiceft vicefimusquintus , ex quo
\ bellum ego, |prafentem »Myflerium Cofmographicum
A indigetarums Magefratibus illins temports, deveftre
; communitatis bonoratifiimo corpore Lefts infcriptum
SOS inter homines vulgaui. Etfi vero tunc oppide inuent
eram , primumque hoc Affronomice profefronss tyrocininm edebam: fuccelfius
tamen ipfi' confecutorum temporum elata voce teftantur,nullum admirabilius,
nulluin felicius, aullum ferlicet in materiadigmori pofitum effe vnquam a quo-
quamtyrocinium. Nonenimbaberidebet tllud nudum ingenij mei commen-
tum(abfit burns reiialtantia d mets ,admiratio a leétorss fenfibus,dum fapien-
tia creatricts tangimus Pfalterium heptachordum) quandoquidem, non fecus,
acfidiftatum mibifiuffet adcalamum, oraculum calitus delapfum , itaomnia
vulgati libelli capita pracipua,ex verifima flatim (quod folent opera Dei ma~
nifefta) fuerunt aguita ab iatelligentibus: éx per hos viginti quinque annos
mibitelam pertexenti reflaurationis Aftronomice (ca ptam a Tychone Brahe é
Nobilitate Danica celebratiCimo Aflronome) facem nonvnam, pretulerunt:
denique quicquid fere lbrorum Aftronomicorum ex illo tempore edidi,idad
conum aliquod vr 2cipuorum capitum , hoc libello propoftorum , referri, i potutt,
cuins aut illuflrationem aut integrationem contineret 5 non equidem amore
mearum inuentionum , abfit iterum hacinfania; fed quiarebus ip/is,¢m obfer=
wo uation







Most revered Prince; greatly revered, illustrious, eminent, noble, ener-
getic, most dear lords. This is the twenty-fifth year since I made public
among men the present little book, entitled The Secret of the Universe,
and dedicated to the elected magistrates of that time of the most
honorable corporation of your community. Although indeed I was very
much a young man, and was producing it as the first apprentice piece of
my vocation to astronomy, yet its successes in the times which have
followed bear witness at the tops of their voices that no apprentice piece
has ever been more remarkable, more successful, or of course carried out
on worthier material by anyone. For it should not be considered as a
mere contrivance of my own intellect (may there be no boasting of this
affair in my feelings, nor wonder in the reader’s, while we touch the
seven-stringed Psaltery of the Creative Wisdom) since, just as if it had
been dictated to my pen, an oracle fallen from heaven, every chapter of
the little book was recognized at once, by those who understood it, as im-
portant and quite true (as the manifest works of God usually are).
Throughout these twenty-five years, while I have been weaving the fabric
of the reform of astronomy (started by Tycho Brahe of the Danish
nobility, the very celebrated astronomer), they have carried a torch
before me more than once. And, finally, almost every book on
astronomy which I have published since that time could be referred to
one or another of the important chapters set out in this little book, and
would contain either an illustration or a completion of it. I say this not
out of love of my own discoveries —again may there be no such madness
in me—but because from the subject itself, and from the observations of
Tycho Brahe, which deserve complete trust, I have thoroughly learnt that

uationibus Tychonis Brabei fide omni digniffimis edoftusfui, nullam alia
inueniri poffe via ad perfectionem Aftronomia,certitudinemque calculi ,nul-
lam ad conftituendam feientiam huius feu partis Metaphyfice de caclo, few
Phyfice coeleftis; quam que hoc libello vel expreffe preferipta, veltimidss fal-
temopinionibus, 8 rudi Minerua adumbrataeffet. Teftes fifte illiccommen-
taria Martis anno 16.09. ¢dita, quaque adbuc domi premo commentariade
motibus caterorum Planetarum, bic vero Harmonices Mundi libros V.an-
no 1619. vulgatos, (x Epitomes Aftronomia ibrum 1V.anno1620. abfo-
lutum: tefles tot numero leffores , qui, ex quo nati funt opera diffa, iamab
annit bene multss exemplariaflagitant , dudum diftratta,, huius primi mes li»

bellis ut ex quo tam multa'vident deriuatatheoremata,

Cum sgiturinflarent amici now Librarij tantum, fedetiam Phslofophia

periti, vt, fecundam editionem adornarem: officij quidem mei putawi , non diu-
tins repugnare; de modo tamen editions aliquantulum contradixi. Erant
enim, qui confilerent , bellum emendarem , augerem, perficerem : morem
Seiler caterorum Authorum , quem tenent in excolendis libres propriss em ipfé
obferuarem. Mibicontrafic-vsfum,, nec perfici libellum poffe, nifttranferiprss
in illum plerifque meorum operum , qua per hos ‘vigintiquingue annos edidt,
peneintegris nec hoc tam tempus amplins effe librum alquem boc titulo, poft
editos alsos, veluti de nouo publicandi : demaue lbellum ipfum propter ficcef-
femadmirabilem, pro meo non reputandum, quem arbitratu meomutem, au-
geam've s quin potius intereffe letoris vt intelligat , a quibus initiss , quosf-
que perdutta a mefuerint contemplationes Mundane. Vincentibusergora-
tionibus iftts , formam editions talem elegt , que folet obferuarsin libris alienss
recudendss ; vbi nihil mutamus, que vero locaemendatione egent, aut expli-~
catione,aut integratione, ea commentarits adiunamus, differents typoexara-
"88. Seruinit hacformag religions co breustati, vt errores quidem demen-
tis mea tenebris ortos , interfperfofque materia deoperibus Dei perfettiffinns,
ipfé coarguerem ingenue, expungeremque: qua'vero capita lsbelli , acie mentis
srretorta,, in lumen illud operum disinorum ineffabile direfta , clare percepif-
fem; autvbiviam quidem reftam ingreffius , nimium tamen propere fubfhiti{~
femeafecernerem, 6 quibus als operum meorum locis ad, ifcopum tandem pers
uenerim,letori fignificarem.

Vrigueur libellum inhac alrera editione , etiam quoad ipfam dedicatio-
nem , relinquerem intattum , ut tpfum etiam veftsbulum resBonderet opufeu-
loreliquo: viderts, opinor, Proceres Reuerendifirmi, Generofifimi,aliter mi«
hinon faciendui fuiffe , quin etiam hanceditionem ad| [primos patronos ,quos
in fequenti dedicatione fum alloquutus, aut, fi gut ex hoc tempore rebus humd-

mh

Dedicatory Epistle

no other way can be found to the perfection of astronomy and accuracy
in its calculations, no other way to establish knowledge of this
metaphysical aspect of the heaven, or heavenly physics, than what had
been written already in this little book either expressly, or at least in timid
conjectures, and in a rough and ready way. I cite as witnesses on the
former point the Commentaries on Mars published in the year 1609, and
the Commentaries on the motions of the other planets which up till now I
have kept to myself, and in the latter case the five books of the Harmony
of the Universe made public in the year 1619 and Book IV of the Epitome
of Astronomy, completed in the year 1620, and I count as witnesses so
many readers who have, for very many years since the time when they
obtained the works mentioned, been demanding copies, long since scat-
tered, of this my first little book, as they see so many theorems derived
from it.

Then since my friends, not only booksellers, but also those versed in
philosophy, were pressing me to prepare a second edition, I did indeed
think it my duty not to object any longer; yet I disagreed with them a lit-
tle over the character of the edition. For there were some who advised me
to emend, enlarge and complete the book; that is to say, that I should
myself adopt the custom of other authors, which they observe in refining
their own books. It seemed to me on the contrary that I could not com-
plete the book, except by transcribing into it several of my works, which
I have published during these twenty-five years, almost in their entirety;
that this was no longer the time for putting out a book with this title,
after I had published others, as if it were new; and lastly that the little
book itself, on account of its remarkable success, should not be thought
of as my own, to alter or enlarge at will, but that it was rather of interest
for the reader to understand from what beginnings, and to what point
my studies of the universe have been brought. This reasoning won, then,
and I chose the form of edition which is usually adopted in reprinting
other people’s books, in which we change nothing. Those places which
need emendation, or explanation, or completion, we reinforce with com-
mentaries, set in different type. This form assisted both religion and
brevity, so that I could frankly refute and expunge the errors which had
sprung indeed from the darkness of my mind, and were scattered among
material on the most perfect works of God; I could distinguish those
chapters in the book which had not been deflected by the action of my
mind but which I had perceived clearly because they were turned towards
the unutterable light of the divine works, and those where, although I
kad set off on the right path, I had yet stopped too soon; and I could in-
dicate to the reader the other places in my works in which J have at last
reached the goal.

Then in order to leave the little book intact in this second edition, even
including the actual dedication, so that it should serve as an anteroom
for the remainder of my little work, you see, I believe, most revered,
most eminent nobles, that I could do no other than submit this edition
also with a new dedication to my first patrons, whom I addressed in the

Eprsrora Depicaroria

nis exemptifant? ad eorum filios, att fucceffores 5 (quorum nonnullos interea
Terrarum Orbis Monarche ,virtutem remunerats ,adfummum dignitatis
culmen enexerunt)denique ad hoc idem corpus communitatis honoratifimum,
cuins fspenduis fiffisltus , olsm ibelum conferipfi , nona dedicatione remit»
terem,

Nec lenia mibi hoc agitanti prabuit intitamenta , inde Styrie moders
na,hinc prousnciarum circumiacentiumrefpettus. Mincnamque multos ?no-
bilitate-videbam , quimevelandiuere docentem, velcommuni menfa aut
contubernio meoufi, me propius cognouerunt , exque €0 tempore beneuolens
tiam apatribus in fe deriuatam erga me conferuant , quibufque pollent copissy
demonftrant , dignitatis 6 gratia Cafaree fruttum per beneficentiamexigen=
tes: nec defunt ex Eccleffaflicorum numero, qui non minus , quam anteceffores
Sei, & artes Mathematicas 6 me cultorem amant, meque ad fernuifendos,, fe
turba conguieniffent »de propinquoféeuocaturos nuntiarunt. Dignum igia
tur erat meain utrofque gratitudine, ‘vt quibus, poffer mutuss officiis tantos
fantores percolerem,amplinfque demererifinderem.

Hincvero ex parte Auflrig ,panidam imbellemque Aftronomiam cir
cumftantiaperscula, terrores, calamitates, arumng fubindeadmonent, decira
cumppiciendss auxilit. Tranfiuitillaanno 1600. ¢Styriain Bohemiam ut
que fab Auftriace domus cmbraprimasradices egerat adem fubilla (max
turefceret. Ibi varie iaflata a tempeflatibus bellorum , taminteftinorum,
quam externorum, tandem poft exceffum "Rydolphi Imperatoris anno 1612.
conflanti, domus Aufiriace fiudio »recurritin Auftriam: cubi-vtinam quam
benmigne excepta cr fota, tam impenfagenerofarum mentium occupatione(now
minus atque a me ews inflauratore) percol: potuiffer. Verum,eheu, quantis fe=
fe mutuo bonis exuunt mortales mifers , per feabiem contentionum turpi (i=
mam? Quam profunda, fic meritos, obruit ignorantiafats? Quamlamentaa
bikconfiko  \gnemdum fugimus,medios incurrimus ignes?

Vtinamveroetiamnunc, poft confequutam rerum Auftriacarum conuers
fronem, locus fuperfit ills Platonss oraculo ; qui, cam Grecialongo & ciui=

libelloarderet vndique,malifque vexaretur omnibus,quz ciuilebel-
lum comitari folent, confulcus fuper Problemate Deliaco; quafito
pratextu, ad fuggerenda populis confilia faluraria; ita demumcran-
quillamex Apollinisfententia Graciam futuratn refpondit: fi fe ad
Geometriam ceteraq; philofophica ftudia Greciconuertiffent: quia
hac ftudia animos abambitione & reliquis cupiditatibus, ex quibus
bella &catera mala exiftunt, adamorem pacis & moderationem if
omnibus rebus adducerent.

2 ):( 3 Vtinam


Dedicatory Epistle

dedication which follows, or, if any since that time have been removed
from human affairs, to their sons, or their successors (some of whom the
monarchs of this earthly sphere have meanwhile raised up, as the reward
of their excellence, to the loftiest peak of honor), and finally to that same
most honorable corporation of your community, by whose stipend I was
supported when I wrote this little book long ago.

Also, when I was engaged on it, considerable incentives were supplied
to me by the regard of modern Styria on the one hand, and the surround-
ing provinces on the other. For in the former I saw many from the nobili-
ty who either listened to my teaching, or made closer acquaintance with
me through sharing the same table or dwelling, and since that time have
maintained the generosity passed on to them by their fathers, and have
shown it with all the resources at their command, claiming imperial
honor and gratitude as the result of their kindness. There has also been
no lack of churchmen, who are no less fond than their predecessors of
the mathematical arts and of myself as fostering them, and have stated
that they would invite me to visit them, if the disorders had abated, at
close quarters. It was therefore fitting in view of my debt of gratitude to
both of these that I should honor such generous patrons as far as I was
able with reciprocal courtesies, and strive to deserve more.

In the latter, however, on the Austrian side, timorous and unwarlike
astronomy is warned by the conditions, dangers, terrors, disasters, and
troubles to look round for assistance. She crossed in the year 1600 from
Styria into Bohemia, so that just as she had put her first roots under the
shelter of the Austrian house she might also grow to maturity under it.
After being tossed to and fro there by the tempests of both civil and
foreign wars, in the end after the death of the Emperor Rudolph in the
year 1612, with unceasing zeal for the Austrian house, she returned to
Austria. Would that she could have been honored there with the devoted
attention of eminent minds (no less than by myself, who restored her) as
much as she was accepted and favored with goodwill. Yet, alas, of what
great goods do miserable mortals despoil one another, by their shameful
itching for quarrels. How profound an ignorance of their fate over-
whelms them, as they have deserved. With what deplorable perverseness
do we rush into the midst of the flames, in fleeing from the fire.

Would that even now indeed there may still, after the reversal of
Austrian affairs which followed, be a place for Plato’s oracular saying.
For when Greece was on fire on all sides with a long civil war, and was
troubled with all the evils which usually accompany civil war, he was
consulted about a Delian Riddle, and was seeking a pretext for sug-
gesting salutary advice to the peoples. At length he replied that, accord-
ing to Apollo’s opinion Greece would be peaceful if the Greeks turned to
geometry and other philosophical studies, as these studies would lead
their spirits from ambition and other forms of greed, out of which wars
and other evils arise, to the love of peace and to moderation in all things.


Veinam denique iam fuppreffis armis tantum detur induciarum amife-
vtts,-ve-viris bonss'vacet , fimsle quippiam Ciceroniant ilins confilij commini-
fei: qui, enerfa Republica "Romana, cum effet vix confolabilis dolor, in
tanta omnium rerum amiflione & defperatione tecuperandi, poft-
uamilliarti, cuiftuducrar, nihil effeloci, nequcin Curia, nequein
Prosvidit: emnem fuam curam atque operamad Philofophiamcon-
tulit; monens Sulpitium fuum, in ifdem verfarirebus, qua, ctiamft
minus prodeflent , animum tamen afollicitudine abducerent ; aque
moleftiis leuarent.

Quibus cots fi Dews annuat , non equidem indignas bomine Chrifliano
voluptates, arumnarum folatia, Mathematice meavelex aflronomicts exer-
citi, velex contemplatione diuinorum operum ,exqne Har monice Mundi
(fatal lla occupatione, in durifvimis exath biennij d:ffonantuss) proponere pa~
ratafempererit. At quia in ide(tincepta hac occupatio Aftronomica, vt per-
fictatur: quid rgitur boc Aufiria ftatu calamitofifoime potius agat ,quamut

refidia, quilus ipfaindizet, ad opera inser homines vulganda:  adquenomen
Rudolphi » Tabules perpecuss a lerendum; pudore cobibitane ab afi velin~
bentibus omnia petat ; potins inde corroget , quorfum clades ifte , quorfum 1pr0~
digiorum calefiium expiationes horribih{iime non pertigerunt : denique ad
priflinos ‘patronos, ad. quos dimidso Vialam anno 1612. appropinquauerat,
reliquoetiam dimidioexcurrat? E Styria quondam ,vti dixi,ad Braheum,
ideft,ad Opus Tabularum Rudolphinarum maturandum, profecius eft libel
lus ifte, me latore : quid infolens , quid adeo alienum a priftino inflitutoveftro,
Proceres , quid denique non gratum Ferdinando Imperatori Auguflo ,Rudol-
phipoft Matthiam fuccefort, feceritas 5 firenertentem nunc libellum, veterem
clientem veftrum, de rebus tnterea geftis,audiatts,fi Tabularum Opus laborio
fum @ folkcitum, fi delicias humani generts, fi Rudolphs lmperatorts Nomen
bono: refqueymodica Lberalitate; ‘promouendos sfufeipratss; Shane vetufiffimam
Mathematicarum difeiplinarum clientelam don-us Auftriaca, ne hoc quidem
grauifsino motuconcuffa sintercedentevefira fuccenturiata prousdentia, dim
mittat, exterifve cedat?

Hic igitur dedications huins repetite feopss eflo,, quem fiarveftra,Proces
ves, magnificentia fuero confequutus id omen mihi maximumerit , fore, Ut,
priufquam ego Rudolphinas in lucem proferam » colophone hoc reflaurationt
Aftronomica impofito : reftauratus fub Ferdinando 11. poft annos ab exceffie
Ferdinand primi minus fexaginta »prouinciarum Auftriacarum,antigquusille
quinarins, repreffis bellss ciuslibus, © pace rerum optimaredufta,denno prifti=
num in nitorem efflorefeat , quod omen , angoribns obmala iprafentia non lea

miter


Dedicatory Epistle

Lastly, may arms now be abandoned and enough respite from miseries
be granted for good men to have leisure to compose such advice as
Cicero gave. When the Roman republic had been overthrown, “as his
sorrow was scarcely consolable at such complete loss of everything and
despair of recovery, when he saw that there was no place either in the
Senate house or in the lawcourt, for the art which had been his study, he
devoted.his whole attention and effort to philosophy, advising his friend
Sulpicius to occupy himself with the same subject, as although it would
be less profitable, yet it would divert the spirit from anxiety, and relieve
it of troubles.”

If God were to consent to these wishes, my mathematics would always
be ready to propose, either from astronomical exercises, or from the con-
templation of the works of God, or from the harmony of the universe
(that destined occupation during the harsh discords of the past two
years), pleasures certainly not unworthy of a Christian man, as consola-
tions for his troubles. But because this astronomical occupation was
undertaken with the intention of completing it, what in the present
calamitous state of Austria should she rather do, than, restrained by
decency, seek all the assistance which she needs, to make public her
works among men, and to claim the name of Rudolph for her perpetual
tables, not from those who are afflicted or in need, but rather entreat
them from quarters to which those misfortunes, those horrible expiations
of heavenly portents, have not penetrated, and lastly hasten over the re-
maining half of her journey to her original patrons, to whom she had
already approached halfway in the year 1612? As I have said, this little
book borne by me had already set out long ago from Styria to Brahe,
that is, to expedite the work of the Rudolphine Tables. What would be
unprecedented, what so foreign to your original undertaking, noble sirs,
and lastly what unwelcome to the Emperor Ferdinand Augustus, the next
successor of Rudolph after Matthias, if you were to listen to the things
which the little book now returning, of which you used to be the patrons,
has to say about what has been achieved in the meantime; if you were to
undertake the promotion with modest liberality of the anxious and
laborious work of the Tables, of the delight of mankind, and of the
honor and repute of the Emperor Rudolph; and if the Austrian house,
unshaken even by this most grievous disturbance, at the intercession of
your own provision as replacement, were to part with its ancient
patronage of the mathematical disciplines, or yield it to strangers?

Then let this be the goal of this renewed dedication; and if by your
magnanimity, noble sirs, I achieve it, that will be a most important omen
to me, that before I bring the Rudolphine Tables into the light of day,
with the addition of this finishing touch to the restoration of astronomy,
the ancient fivefold confederation of the Austrian provinces, restored
under Ferdinand the Second less than sixty years after the decease of Fer-
dinand the First, with the suppression of civil wars, and the return of the
best of all blessings, peace, will finally blossom forth into her original
splendor. May that omen, though considerably impaired by anxieties on

Errsrota Depicaroria

witer quaffatum, Devs Orr. Max. miferatione Ecelefie, Fils fz
neredempta, propitins firmet ram fuam , tandem a nobis aucrfam in Eettes
Ecclefiamvaftantes conuertat, Imperium Férdinandi 11. Imperatorts Au: u-
Sf, extin€iss trarum incentiuss , falutari Clementia aura mitigatum prof? eret,
quaratione € Styria, fortuna mea prima incunabula, cumque slla € vos Re-
uerendifimi Generofifvimique Proceres 5 fab alis Aquile tutra-vulture limita-
neo, rerumaqueomnium copia locupletés, in annos innumeros, perduretis: qui-
bus debita cum veneratione mecommendo. Valete. Dabam Francofurti

Reu™ & Gen™ Mag? V*
Deditiffimus Cliens

Tohannes Keplerus, olim Styria Pro-
cerum,poftImpp. Caff. Rudolphi
&Matthia,|.m. Ordd.q; Auftrie
Supr-Anifang Mathematicus,

Dedicatory Epistle

account of the present evils, be favorably confirmed by God the most ex-
cellent and greatest, in compassion for the Church, redeemed by the
blood of his Son; may his anger at length be averted from us, and turned
against the nations which are laying waste the Church; may the empire of
the Emperor Ferdinand II Augustus, all incitements to anger being quen-
ched, prosper in the mildness of the health-giving breeze of clemency;
and by the same token may both Styria, the first cradle of my fortune,
and with her you also, most revered and eminent nobles, continue for
countless years, safe under the wings of the Eagle from the vulture on her
borders, and rich in abundance of all things. To you with due respect I
commend myself.

Frankfurt, 20/30 June, in the year 1621.

“Your Most Revered and Eminent Magnanimity’s

Most Devoted Adherent,

Johannes Kepler,

formerly Mathematician to the Nobles of Styria,
and later to the Emperors Rudolph and Matthias
and to the Orders of Austria

over the Enns.


Vip mundus,qua caufa Deo, ratioque creandi, The nature of the universe, God’s motive and plan for creating

5 A it, God’s source for the numbers, the law for such a great mass

D i,que tantz regula mol, ne 5 ae A 7

Vande co numer bq _ ZA Oil, the reason why there are six orbits, the spaces which fall between
Quid faciat {ex circuitus,quo qualibet orbe

: all the spheres, the cause of the great gap separating Jupiter and
Incerualla cadant,cur tanto lupiter & Mars, Mars, though they are not in the first spheres—here Pythagoras

Orbibushaud primis,interftinguantur hiacu: oe all hed you by we figures. hors ly he ie ahases by

_ . . . this example that we can be born again after two thousan years
Hicce Pythagoras docet omnia quingue figuris. of error, until the appearance of Copernicus, in virtue of this
Scilicet exemplo docuit,nos pofle rena ci, name, a better explorer of the universe. But hold back no longer
Bis millecrratis,dum fit Copernicusannis, from the fruits found within these rinds.

Hoc,melior Mundi {peculator,nominis. Acta
Glandibusinuentas soli poftponere fruges,





Ne Oi,


In Titulumlibri Notz Audtoris,

Ropromvs.] Poitquam ad Philofophieftudium acceft, anno atarieri
Anno Chrifliss89.verfabantur in manibusinsentutisexercitationes exoterich
5 luli C.Scaligeri:cuins ego libri occafione capi fucceffine varia comminifirdeva=
A rie quftionsbusve de Caled Anime de Genin de Elements, delgnisnatara,
Ay by de fontium origine,de flux & refluxu marie, defiguracontinentium terrarum,

22 interfafarumque marininc> fimilia.V erum cum inuentioifta propertionis Or-
biurn caceftium mibivideretur eximi;non expettandum mibi fumratus,donec omnes nature par
tes peruagarer , nec hoc intientumeobiter enulgandumn , conieEtumincumulum quaftionum cetera
ruin,leut quadam probabilitate vtentium. Quin potius ab buiusinuenti editione initium differta>
tion mcaruan facereplacuit: aufieque fim i omnibusreliquisquaftionibua finilem Perare fues
coffin: fed fraflra, Calumenimn , principinmn operum Dei, longe preftantiorem ornatum habet,
quarn réligica minuta vila, Ttaque Prodromus quidem egreginsfait : Epidromus vere,qualens
ego tunc propoferarn,mulluselt fecutus quia in reliquisqusftionibus nequaquam mibieque fatifas
cicbam, Lector tamen opera mea Aftronomica , & inprimislibros Harmonicorumpro genuine or
proprio epidronio haberepoterit buinlibeli,quiaeadem vtringue via curritur queque tuncimpes
dita fatiserat, facta nuncelttritifiana, G que tunc breuisnec ad feopum pertingens; ile Cr conti
auatur in Harmonicis , Cr currus circa metam agitur. Talis fuit Prodromus,nauigatioprima Ames
rici Vefpucifseales Epidromi nauigationes hodierna ansiuain Americam.

Myftcrium Cofmographicum.] Extant apud Germanos Cofmographie,Munfleri as
Uiorumque, vbi de toro quidem mundo partibufque cotleflibus fit initium,fed breuibusilla paginis db=
{foluuntuc,pracipua verolibrimolescompledtitur deferiptionesregionurm G> vrbiumn.ltaque vulgua
Cofinographie pro Geogr apbie dictione vtitur : impofuitquevoxifia, A mundolicet dedulte,

nis liars, toque qui Catalogos ibrorum conferibunt vt libellum meum inter Geographia
ca réferrent. Myflerism autem pre Arcano vfurpaui,c>protali venditans
inuentum hoe: qutippein nullius Philofophilitrotalia
vnquamlegerattte

ao6(oltom




Notes of the author on the title of the book.

Forerunner.) After I came to the study of Philosophy, in my eighteenth year, the year of
Christ 1589, the Exercitationes Exotericae of Julius C. Scaliger were passing through the
hands of the younger generation; and taking the opportunity offered by that book I began
to devise various views on various enquiries, such as on the heaven, on souls, on characters,
on the elements, on the nature of fire, on the origin of springs, on the ebb and flow of the
sea, on the shape of the continents of the Earth, and the seas that flow between them, and
the like. Yet since the discovery of the proportion of the heavenly spheres seemed to me
outstanding, I thought I should not wait until I could traverse all the parts of Nature, and
that this discovery should not be published incidentally, thrown onto a pile of other in-
quiries, achieving but a slight probability. I decided rather to make the publication of this
discovery the starting point of my dissertations, and dared to hope for a similar success in
all the remaining inquiries; but in vain. For the heaven, the chief of the works of God, is
much more notably embellished than the rest, which are paltry and mean. So the fore-
runner was indeed excellent; but no successor, of the kind which I had then intended,
followed it, because in the rest of the inquiries I did not achieve anything which gave me
equal satisfaction. However the reader will be able to have my astronomical works, and
especially the books of the Harmonice, as the authentic and appropriate successor of this
little book; because the same course is run in both cases; what was then rather obscure has
now been made easily accessible; and not only is what was then brief and short of the goal
now continued in the Harmonice but the chariot is rounding the turning point.? The
forerunner was like the first voyage of Amerigo Vespucci; the successors are like today’s an-
nual voyages to America.

The Secret of the Universe. There exist in Germany cosmographies by Munster and
others, in which indeed the beginning is about the whole universe and the heavenly regions,
but they are finished off in a few pages. The main bulk of the book, however, comprises
descriptions of territories and cities. Thus the word cosmography is commonly used to
mean geography; and that title, though it is drawn from the universe, has induced
bookshops and those who compose catalogues of books, to include my little book under
geography. Nevertheless I have taken the mystery as a secret, and marketed this discovery
as such: and indeed I had never read anything of the sort in any philosopher's book.

2 Depvicatia ANnTIQvA

SG ee eee cae

do Friderico, Libero Baroni ab Herberftein, in Neuperg & Guetten-
haag,Domino in Lancowviz,Camerario & Dapifero Carinthiz
hareditario, Cxzfarex Maieftati & ferenitlimo Ar-
chiduci Auftriz Ferdinandoaconfiliis,
Capitaneo Prouinciz
Styriz:

oO

Quingque-viris Ordinatiis, Virisampliffimis, Dominismeisclementibus .
& beneficis,falutem & mea fernizia.

Von ante (1) feztem menfes promifiopus dottorum teflimsnio
pulchrum,ey iucundum,lonacque preferendum annuts prognofti-
cis: tandem aliquando Corona veftrefifto, AmpliffiniVirtsOpus,
[A inqnam,exigua mole , labore modica , wateria undiquaque mira-
Coy bili. Nam fine qutsantiqnitatem fpectets (2) tentata fuit antebis
milleannes & Pythag.ra: fine nouttatem , primum nuncameinter
homines vulgatur, Placet moles? Nibil est hoc vniuerfo mundo maius neque am-
plius. Defideratur dignitastNihilprectofius,nibilpulchrins hoc lucidsffimoDei tem-
plo. Lubct fecreti quid cognofcere? Nihil est aut, shitin Yersm natura occultius; So~
Gira bac ia renon ommibusfatisfacit , quod veilitas eius incogitantibes obfiura eit.
Atque bic eit ille liber Nature,tantopcre facrés celebratus fermonibus;quem Pau-
lus gentsbus propontt,in quo Deum , cen Solem inagquavel fpeculo,contemplentur,
Nam cur Chrifttani minus hac contemplatione nos oblecParcmus;quorum proprium
eit, Deum verocultucelebrare,venerart  admirariz id quod tanto denotiort animo
Sit,quanto rectius,que & quantacondiderit nofler Desws,intelliginus. Sanequam
(plarimos hymnos in Condttorem , verum Deum,cecinit verus Dei cultor Danidess
quibus argumentaex admiratione celorum deducit.Cocli cnarrant,iaquit,glo-
tiam D Er. Videbo carlos tuos, opera digitorum tuorum, Lunam &
ftellas,quartu fundafti : Magnus Dominus nofter, & magna virtus ciuss
quinumerat multicudinem ftellarum,& omnibus nominavocat. Alica-
bi plesnus fpiritn, plenuts facra Letitia exclimat, ipfiamque mundi acclamat  Lau-
dare ca:li Dominum , laudate cum Sol & Luna, &e. Quevox calo? que
Stellis? qua Deum laudent inflar hominis? Nifi quod , dum argumentafippeditant
bominibuslaudandi Dei, Deum ipfeliudare dicnatur? Quam vocem, calis & Na-
turererum dum aperimus his pagellis,larioremane eficimus : nemo nosvanitatts,
aut inutilicr fumsptilaboris arguat.
Taceo, quod hac materia Crestionis , quamnegarunt Philofophi, magnum
argumentum eit: dum cernimus uti Dess more alicuins ex noflrassbus Archites

Gi, ore






(1) Seven months ago I promised you a work which would be acknowledged by
the learned as handsome, and pleasing, and far preferable to annual predictions.
Now at last I add it to your crown, most generous of men —a work, I say, of tiny
bulk, of modest effort, of contents in every way remarkable. For if we look to an-
cient times, it had been attempted (2) two thousand years before by Pythagoras; if
we look to modern times, it is now published among men by me for the first time.
Do you want something bulky? Nothing in the whole universe is greater or more
ample than this. Do you require something important? Nothing is more precious,
nothing more splendid than this in the brilliant temple of God. Do you wish to
know something secret? Nothing in the nature of things is or has been more close-
ly concealed. The only thing in which it does not satisfy everybody is that its
usefulness is not clear to the unreflecting. Yet here we are concerned with the
book of Nature, so greatly celebrated in sacred writings. It is in this that Paul pro-
poses to the Gentiles that they should contemplate God like the Sun in water or in
a mirror. Why then as Christians should we take any less delight in its contempla-
tion, since it is for us with true worship to honor God, to venerate him, to wonder
at him? The more rightly we understand the nature and scope of what our God
has founded, the more devoted the spirit in which that is done. How many indeed
are the hymns which were sung to the Creator, the true God, by the true wor-
shiper of God, David, in which he draws arguments from the marvels of the
heavens.? “The Heavens are telling,” says he, “the glory of God. I shall see thy
heavens, the work of thy fingers, the Moon and stars, which thou hast created.
Great is our Lord, and great is his excellence, who numbers the multitude of the
stars, and calls them all by name.” Elsewhere, full of the spirit, full of holy joy, he
exclaims, and acclaims the very universe, “Praise the Lord, ye heavens, praise him
ye Sun and Moon, etc.” What voice has the heaven, what voice have the stars, to
praise God as a man does? Unless, when they supply men with cause to praise
God, they themselves are said to praise God. And if we reveal this voice for the
heavens and for the Nature of things in these pages, and make it clearer, no one
should charge us with a vain deed or with undertaking useless toil.

I pass over in silence the fact that this very matter, of Creation, which the
philosophers denied, is a strong argument, when we perceive how God, like one
of our own architects, approached the task of constructing the universe with

Mystertvm CosmMoGRAPHICWS. 5

Bisiordinecynormaad mundi mulitionem acceffirit fingulag, fit ita dimenfusiqua-
finon ars naturamimitaretur, [ed Deusipfe adbomins futuri morem adificandi,
refpexiffet.

Quanquam quid neceffe est, divinarum rerum vfs inflar obfoni nummo e-

Simares Nam quid quefoprodeit ventrifamelico cognitiorerum naturalum, quid

totarchqua Aftronsmis Neque tamen anciunt cordati homines illam barbartem,
que defcrenda propteresiflaftudiaclamitat. Pittores ferimus,qui oculos,Sympho-
niacos, gui aurcs oblelant : quamuis nullum rebus noftris emolumentuns afferant.
Et non tantum humana, fedetiam honefta cenfetur voluptas ,queex vtrorumque
operibuscapitur. Quaigitur inbumanitas, qua flultitias menti funm inuidere bone-
Stum gandium,ocults G& auribu: non inuidere? Rerum natura repugnatsqui cum his
pugnat recreationibus. Nara qui nthilin naturam introduxit,Creator Optimus,cui
noncivm ad nece(Statem,tum ad pulchritudinen & voluptatens abuade profpexe-
vit: is mentem homiaw,totius nature dominam foam ipfius imaginem folam nulls
veluptate beaucrit? Imo vti non qusrimus , gua (pe commodicantillet anicula cum
fitarnus ineffe- voluptatem in cantn  propterea, quia ad cantum iftum facta es: ita
nec hoc quarendum , cw’ menshumanatantum furmat laborisin perquirendishifte
carlorum arcanis. Eit enim ideo mens adiunctafenfibus ab Opificenoftre ;non tantit.
ut feipfumn homo fiftentaret, quod longé folertits poffine vel brute mentis minifie-
rio mists aitimantinm genera: fed etiam, vt ab is,quey quod fint oculis cernimus,
ad eatfas quarefint & fiant , contenderemus : quarauts nibtl alind vtilitatis inde
caperemus, Atg, adco vt animalia caters , corpusg, bumanum cibo potug, fuflentan~
tur: fic animnsss ipfe hominis, (3) dinerfum quiddam ab homine,vegetatur, ange-
tur, adolefeit quodammodo,cognitionss ifthac pabulo:mortuog, , quam vino fimsi-
lior eit. feharumyerum defiderie nullo tangitur. Quarevti Natura prousdentiapa-
bulum animantibus nunquam deficit: ita nonimmerits dicere polfumus , propteres
tantam in rebus ineffe varietatern , tamd, reconditos ix calorum fabrica thefauross
ut nunquam deoffet humana miti recens pabulurn,ne faftidiret obfoletum, neu quie~
feeret, (4) fedhaberet in hoc mundo perpetuam exercendi fui officinam.

Neg, werd harum epularum, quas ex ditifsiimo Conditoris penn in hoc libello,
velutin menfadepromo, propterea minor eit nobilitas :quod & maxima vulgi parte
vel ncn guflabunturyvelrefputtur. Anferem laudant plures,qudm phafianum quia
ille consmlnis eHt,ifterarior. Nequetamenvllius Apitit palatus huncilh poflponer.
Stc huius materiedignitus tanto maior erit ; quo pauciores laudatores ,intelligentes
modo fintsreperict. Non eadem vulgoconueniunt és principibusineque hacceleftia
promifeue omninm, fed gencrofifaltem animi pabulum fant:non meo voto, vel ope-
ra,non {ua natura,non Dei inutdia: fed plurimorum hominum, velftupiditate, vel
ignauia. Solent principes aliquamagnt precy ctl pore Aabere menfas , qui.

us vtartur non nifi faturi , leusndi faflidycaufa. Sic hac & huinfmodi frudia
gener ofiffimo & fapientiffiono cuique tum demum fapient,vbit cafa per pagos , oppi-
da, prouincias,regna ad orbis imperium afcenderit , omnis probe perpexerit s neg,
ut fisnt humana.quicquam vllabirepericrit beatum, diuturnum, & tale quo fini
Ch fiturari queat cius appetitus. Tunc enim incipict meloraquarere, tunc aterre
hucin celum afeendee, tunc animum feffum curis inanibus adbanc quietem tranf-
Sferetsuncdicet:
Felicesanimas,quibus hxc cognofcere primum.
Ings domos fuperas fcandere cura fuit,
quareconteranereincipiet ,quaolim praftantifviina es » fol hac Dei operama-

gnifaciet,


Original Dedication

order and pattern, and laid out the individual parts accordingly, as if it were not
art which imitated Nature, but God himself had looked to the mode of building
of Man who was to be.

Though why is it necessary to reckon the value of divine things in cash like vic-
tuals? Or what use, I ask, is knowledge of the things of Nature to a hungry belly,
what use is the whole of the rest of astronomy? Yet men of sense do not listen to
the barbarism which clamors for these studies to be abandoned on that account.
We accept painters. who delight our eyes, musicians, who delight our ears,
though they bring no profit to our business. And the pleasure which is drawn
from the work of each of these is considered not only civilized, but even
honorable. Then how uncivilized, how foolish, to grudge the mind its own
honorable pleasure, and not the eyes and ears, It is a denial of the nature of things
to deny these recreations. For would that excellent Creator, who has introduced
nothing into Nature without thoroughly foreseeing not only its necessity but its
beauty and power to delight, have left only the mind of Man, the lord of all
Nature, made in his own image, without any delight? Rather, as we do not ask
what hope of gain makes a little bird warble, since we know that it takes delight in
singing because it is for that very singing that the bird was made, so there is no
need to ask why the human mind undertakes such toil in seeking out these secrets
of the heavens. For the reason why the mind was joined to the senses by our
Maker is not only so that Man should maintain himself, which many species of
living things can do far more cleverly with the aid of even an irrational mind, but
also so that from those things which we perceive with our eyes to exist we should
strive towards the causes of their being and becoming, although we should get
nothing else useful from them. And just as other animals, and the human body,
are sustained by food and drink, so the very spirit of Man, (3) which is something
distinct from Man, is nourished, is increased, and in a sense grows up on this diet
of knowledge, and is more like the dead than the living if it is touched by no desire
for these things. Therefore as by the providence of Nature nourishment is never
lacking for living things, so we can say with justice that the reason why there is
such great variety in things, and treasuries so well concealed in the fabric of the
heavens, is so that fresh nourishment should never be lacking for the human
mind, and it should never disdain it as stale, nor be inactive, but (4) should have
in this universe an inexhaustible workshop in which to busy itself.

Yet the nobility of this banquet which from the Creator’s sumptuous store I set
forth in this book, as on a table, is no less because by the majority of the people is
will not be savored, or will be spat out. More men praise the goose than the pheas-
ant, because the former is common, the latter rarer; and yet no Apicius’s palate
will rank pheasant lower than duck. Similarly the fewer there are found to praise
this subject, provided they are intelligent, the greater will be its merit. The same
things do not suit the people and the princes, and these heavenly matters are not
nourishment for everyone indiscriminately, but just for a noble mind — not by my
wish, or efforts, not by its own nature, not from God’s jealousy, but by the
stupidity or ignorance of the majority of men. Princes usually have something
very expensive kept for the dessert course, which they use only if they are satiated,
to relieve the monotony. So these subjects, and those like them, will appeal to the
wisest and most eminent of men only when he has ascended from the cottage
through country, towns, provinces, kingdoms to dominion over the world, and
has fully explored all possibilities, yet, as these things are human, he has found
nothing anywhere which is blessed with happiness, everlasting, and able to satisfy
and satiate his appetites. For then he will begin to seek for better things, then he
will ascend from the Earth below to heaven, then he will lift up his spirit, tired
with empty cares, to that tranquility, then he will say:*

Happy the souls whose first concern it was
To gain this knowledge and soar to heavenly homes;
and therefore he will begin to despise what once he thought most important, he
will value only these works of God, and he will derive pure and sincere delight at

4 Toan, Kerrier

gnifacict, atque meram & finceram tandem voluptatem ex his contemplationibut
capict.Contemnant igitur hac & huinfmodi meletemata,quicunque quantumeung,
volent, quarantque fibiundiquaque commoda, dinitias ,thefauros: Aftronoms
Iphac gloria fuffictat,quod Philofophis fua feribunt,non rabulis, Regibus non pafto-
ribus. Pradico intrepide , futuros tamen aliquos, qui {ua fibi feneétutis hinc compa-
rent folatinmstales nempe,quiquoad. ‘aagifiatns gefferuntssta fe gefferunt, ve ibe-
ri morfibus confiientia,babiles eff poffint fruendis hifce delicis.

(5) Exiffes iterum Carolus aliquis,qui,cum Europa quoad imperauerit,non
caperetur; felfus imperiis exiguaS.1uffi cellulacapiatur : cuigque inter tot (pectacu-
la,titulos,triumphos,tot dinittas,vrbes,regna;unica Turrianica yveliam (6) Co-
perntcopythagorea Sphara Planetaria tantopere placeat, vt orbem terrart.-% cum ce
commmtet digitogue circulos quam populos imperiss regere malit.

Nox hac eo dicoyviri ampliffimi,vt nonum paradoxon,fenes difcipulos,in fee
nam, feuin bolas producam;fed-vt appareat quodnam genuinum tempus fit mef-
Semde his ftudits colligendi. Cur enim de fementefacienda aliter egofentiam , atque
viri prudentiffimni de vefira Coronas qui hac fludiainter pracipna cenfuerunt qua
inuenilibus Nobilitatis animisin veftra fchola proponerentur. Sic enim exiflimant,
nequeaptius effe genus hominumadcolenda Mathemata , Nobilitate: vt quibus
artes alia ad vithums comparandum non ita neceffarissnec aptiora NobilitatifIudia,
Mathematicis: propterea,qued occulta cy mirifica quadam facultate, polleant prace~
teris; feroces animos ad humanitatem , adque fobrium rerum terrenarum contem-
ptum inflituends. Quifruttus etfidiffculeate G infolentia materici innenibusob-
feuratur: fenibus tamen,vti modo dittum fio tempore fofe patefacit. Atquehac “G0
haclenus , cum deprafensibus pagellis, tum deomni Aftrononomia,ad vos Aftrone=
mie Cr Literatura totius amatores , Vivi. amplifimi: vt eius-vos admoneam , quod
pridemtenctis:nequemulli vfui fore hoc , quod humilis offero & dedico, opufinlum,
vobis qui vere generofi,vere nobiles eftis: G fi quamalaudem meretur inuentio, il-
lam magna flee ad vospertineresquivefira liberalitate, veflroque flipendio mi-
hi occajiones Cf otissra hacita commentandi feciftis : Accipite: igitur , Virt Amplifgi-
mi,hoc grati animi fymbolum , meque humilem clientemin veftram gratiam [ufci~
pites rdenique (7) affucfcite inter Atlantes, Perfeas, Oriones, Cafares, Alphon~

S0s,Rodolphos,ceterofque Afionemia promoters icenferi Valet. idibus Maii:qué
dies ante annum initinm fuit bins la

vis.
Ampl.-V.

Humilisin Scholaveftra Greciana
Mathematicus

M. Toannes Keprervs
VVirtemberg.

Notz Auttoris.

_ (1) Antefeptemmenfes.] Annors95.die 2, lub poftridie natals decimioftaui Se-
renifimi Ferdinandi Archiducis, Roman, nunc Inperatoris Auguflt, Hungarieque Co Bohemia Re-
Sisscainsin ditione hareditaria Styria tunc merebam flipendia, inueni hoe fecretum: flatimque ad
illus


Original Dedication

last from these studies. Accordingly let these and like occupations be despised by
whoever wishes, and as much as they wish, and let them seek for themselves
everywhere profit, wealth, and treasures: for astronomers let it be glory enough
that they write for philosophers, not for pettifoggers, for kings, not shepherds. I
predict without dismay that there will nevertheless be men who will draw from
here solace for their old age, such men indeed who have conducted not only great
offices but also themselves in such a way that, free from the remorse of con-
science, they can be fit to enjoy these delights.

(5) There will arise another Charles, who, as he was not captivated by Europe,
as long as he held dominion over it, will, tired of dominion, be captivated by the
narrow cell of the monastery of Yuste; who, among so many spectacles, titles,
triumphs, so many riches, cities, kingdoms, will be so pleased by a Torrianan, or
(as it would be now) (6) a Copernico-Pythagorean planetarium alone that he will
exchange the whole round world for it, and prefer to rule circles with his finger
rather than nations with his dominion.*

T do not say this, most generous sirs, so that by a new paradox I may bring old
men as students to schools or college; but so that it may be seen what is the
natural time for reaping the harvest from these studies. For on sowing the seed
why should my opinion differ from that of the sagacious members of your
assembly, who have decreed that these studies should be among those most prom-
inently offered to the young spirits of the nobility in your school? For it is their
view, both that no kind of men is more fit for the pursuit of mathematics than the
nobility, as for them other skills are not so necessary for earning a living; and that
no studies are more fit for the nobility than mathematics, because more strongly
than any others it possesses some hidden and wonderful power of civilizing fierce
spirits and instilling into them a sober contempt for earthly things. Although this
harvest is concealed from the young by the difficulty and unfamiliarity of the sub-
ject, yet to the old, as has just been said, it reveals itself in its own time.

This, then, is what I have to say, both about the pages before you and about
the whole of astronomy, to you who are lovers of astronomy and the whole of
learning, most generous sirs: to inform you of what you have long understood,
that this little work which I humbly offer and dedicate to you will not be valueless
to you, who are truly eminent, truly noble; and that if the discovery deserves any
praise, it belongs to you to a great extent, as by your generosity and by your
salary you have granted me the time and opportunity for this account. Accept,
then, most generous sirs, this symbol of a grateful spirit, take me as a humble
dependent into your favor, and finally (7) accustom yourselves to being included
among the Atlases, Perseuses, Orions, Caesars, Alfonsos and Rudolphs, and the
other benefactors of astronomy.
15th May (the anniversary of the beginning of this task).

lam, most generous Sirs,

the Humble Mathematician in your School of Graz,

Master Johannes Kepler
of Wiirttemberg

(Q) Seven months ago.) In the year 1595, on the 9/19th of July, the day after the eighteenth birthday of

his Serene Highness the Archduke Ferdinand, now Roman Emperor and King of Hungary and Bohemia,
in whose hereditary dominion of Styria I was then earning my living, I discovered this secret; and turning

6 Ioan. Keprert

meoruvsvatio abbocynolibellconfurrexit. Et cur non maguifice me iadkem, dum ecole memoria,
quod demonftratis iam planetarum omniumn motibus tandem ad abfoluend.mstelarn boc libelo ce
ptam, ad opus fe. Harmonicur,ilipfoanno, quo Ferdinandus Archidux inregem Bobemte fufie=
prusestaninurm adiecerim , quod anno fequenti 1618. quo anno Ferdinandus Diadema regni Vin-
garia fifcepitsego lbrum V.Harmonicorum abfoluerim : quod denique anno 1619..quo Ferdinando
firma dignitas Imperialis accefit, Harmonicen ipfe meam codem & loco Gy menfé coronationis
eins publicanerim. Faxit Deus,vt extinctis diftdiorum ciuilium diffonantiis,in toro Monarche buins
innperio,ingue Aufiria fuperiore, moderna rico domicile, fuauiffima pacis Harmonia, quueinaqui-
tateimperiorum & promptitudine obfequiorum confit ,abboc pfo tempore reflauretur , quo go
rina bunc menmn libellum Notis emendatum integratumque denud inpublicum edo. Sic enim
feripeterit,vt vaflicati prouinciarum cicarricibusobdudtis, vtficcatis aquishorrendi dilunij; foli-
bufque rewerfis reflore{censcopiccornu etiam mibideftinatos& Rudolpho Imp. fumpras ( impedites
per fiperioruns temporum turbulentiam ) denique ad tabsularum Aftronomicaruin opusedendwin
affundat.

(2) Antebismilleannos.] Quiadogmade quingue figuris Geometricis, inter Mun-
dana corpora diftributis,refertur ad Pythagoram,a quo Plato hanc Philofophiamelt mutuatus. Vi-
de Harmonics ib.Lfol.3.4. Item lib.I folis8.59.Narn eadem quidem,¢> ills & mibi figure quin-
que crant propofite, idem Gills > ibs Mundus , at nonesdem veringue Mundi partes  filiteram
folain fpeétessnec eadem applicandi ratio,

(5) Diuerfam quiddam ab homine.] Condona ledlor tyroni locutionem wiinus emen-
datare, Corpus equidem agnofeit philofephia quodaimmodd dinerfum quiddam ab homine, quia it~
tnd contizue mutationieft obnoxsumciem homo femper idem fit: Anima vero perbiber id, qué bom
mofithomo:adco woe aniznis diuerfiun qiid ab bomine.Verin illatio manct eadem, eff fam et
ium animo pabulumn,fevrfim dipabulo corporisfuas etiam fiorfim delicis.

(4) Sedhaberetin hocmundo. } Now geram Senecam,quipenéesndem fextentiam
Eloquentie Rovnans fofenlisfcexornauit, Paiillares mundus eft nillin co,quod qucrat,
omnis mundus inueniat.

(5) Exifteriterum Carolusaliquis.] Noequidem cogitaueram tuncfore, vein Imp.
Rudolphi aulam yocarer.Namque une Monarcham vere alterum Carolum bic deprebendi,non ab
Aicationc quidemsat projet fsftidio actionum iniquifimarum, domiforifqueoceurventium ,redet-
ice mentis ab iis , Cr beato , (quantum ad naturales contemplationes, ,) recreationuim exercitio,yt
sins fucritfubdits us potinsiomportunitatibusquasn Regfi fefideo raf :

(6) Copernico-pythagoréa.] adjpheram allufisyflematis Planetariconflrudbam
£2 Orbibusplanetariis, & Corporibus quingue regularibus. Pythagorteis , fais quoque coloribusa (4=
teris dftinitoyorbibus caruleislimbia vero,tn quibusplanetasdecurrerefignificabatur albis:perluci
Ais omnibus fc vt Solin contropendulus videripfft.Saturniorbis, fox crculis,reprefentabatur qui
mutu concifisterai quidem, angulo Cubilocim fignabant. Bini verb, centroplani cubic fuperfla-
bant;Ioussorbinm extinius tribus,intimus fex circults, Martisextimusiterum, 1 fex;intimus vero,non
minus quarn Telluris vterque, Vencrifque extinus; fingulidenis circulis adumbrabantur, quorum
4quini duodecies termi vicies,biniricicsconcurrebant. Veneris Orbisintimus,equalts erat louisexti=
so, Mercury orbis louis intime eFRaculums non inamenurn,cuins rudimentuan quide,at nonpla~
ne genuinun,pidere tin figura tertia fequentiex ere,

(7) AlfuctciteinterAfttonomizpromotores.] Locum inucnit adbortatio mea,
commodo meonon exiguo ; qusin commemerationtm bonori Procerum ex gratitudinisl:getribuo.
Mluffris D.Capitaness de proprio , fRatinn, ceteri, vt crant loco corporis. Prouincialium,expectato eo-
71M conuentu anni 160 0 .magnificam mibituncin Bohemia abfenti renunciationem, quanquam

exhanftocontinuis bells linitancis erarioimpetrarunt.tea Calarumconditor nibioperum
Sioruin preconi ,sunc de viaticoprofpexit,, familiam in Bobemiaim

sranflaturo,


Original Dedication

his death I was appointed in his place by the Emperor Rudolph to supervise the work of the Tables, which
Brahe wished to be named after Rudolph. In completing it I have sweated for these twenty years. Thus
the whole scheme of my life, studies, and works arose from this one little book. And why should I not
make a splendid boast, when | recall to memory that having already derived the motions of all the
planets, I eventually turned my mind to completing the fabric which had been begun in this little book,
that is, my work on the Harmony, in the very year in which the Archduke Ferdinand was accepted as king.
‘of Bohemia; that in the following year, 1618, the year in which Ferdinand accepted the crown of the
kingdom of Hungary, I completed Book V of the Harmonice; and lastly, in the year 1619, in which Ferdi-
nand achieved the highest of fice in the empire, 1 made public my Harmonice in the same place and month
as his coronation. May God bring it about, that the discords of civil conflicts may be extinguished
throughout the empire of this monarch, and in upper Austria, my present abode, and that the delightful
harmony of peace, which consists in justice of rule and readiness of obedience, may be restored from the
very moment at which I finally issue to the public this my first little book corrected and completed with
notes. For thus it may come to pass that scars grow over the devastation of the provinces, that the waters
of this horrible flood dry up, and that with the return of sunshine the horn of plenty flourishes again and
even pours forth the funds intended for me by the Emperor Rudolph (held up by the turbulence of times
past) for the eventual publication of the work of the astronomical tables.

(2) Two thousand years before.] Because the doctrine of the five geometrical figures’ being distributed
among the bodies of the universe is traced back to Pythagoras, from whom Plato borrowed this part of
his philosophy. See Harmonice, Book 1, pages 3-4, also Book 11, pages 58-59. For they and [ had the
same five figures in mind, and the same universe, but not the same parts of the universe in each case, if
you look only at the letter; nor the same method of applying them

(3) Which is something distinct from Man.| Pardon a novice, reader, for this ill-considered expres-
sion. Philosophy indeed recognizes that the body is in a way something distinct from the man, because
the former is subject to continual change, while the man is always the same; but it asserts that the spirit is
that in virtue of which a man is a man, so that the soul is not something distinct from the man. However
the conclusion remains the same, that the soul has its own food too, separate from the food of the body,
and also its own separate delights.

(4) But should have in this universe.| | had not read Seneca, who elegantly expressed almost the same
sentiment as follows in his anthology of Roman eloquence: “The universe is a tiny thing, if the whole
universe does not find in it whatever it seeks.””

(8) There will arise another Charles.) 1 certainly did not suppose then that 1 should be called to the
court of the Emperor Rudolph. For I discovered this monarch to be truly a second Charles, not indeed in
abdicating, but certainly in his disgust for the evil activities which he found at home and abroad, in his
withdrawal of his mind from them, and in his happy enjoyment (as far as contemplation of Nature went)
of his recreations; so that his subjects would have done better to be angry at their own insolence than at
the disgust of their king.

(6) Copernico-pythagorean.} | alluded (o the sphere of the planetary system, constructed of the
planetary spheres, and the five regular Pythagorean solids, each distinguished from the others by their
own colors, the orbits sky-blue, and the bands, in which it was implied that the planets ran round, white;
all transparent, so that the Sun could be seen suspended in the center. The sphere of Saturn was
represented by six circles, which by their common intersection, three at a time, signified the position for
the vertex of the cube, but intersected two at a time over the position of the center of a face of the cube.
The outermost of the spheres of Jupiter was shown by three circles, its innermost by six circles, and the
outermost of Mars again by six; but the innermost of Mars, just as were both those of the Earth, and the
outermost of Venus, were each sketched out by ten circles, of which every five met twelve times, every
three twenty times, and each pair thirty times. The innermost sphere of Venus coincided with the outer-
most of Jupiter, that of Mercury with the innermost of Jupiter. It was a not unpleasing spectacle, of
which the elements, though not an exact likeness, may be seen in the third engraved figure which follows.

(1) Accustom yourselves to being included.| My exhortation found its mark to my not inconsiderable
advantage; and under the obligation of gratitude I assign the credit for this attention to the nobles. The
illustrious captain immediately from his own funds, and the rest, as they represented the corporate body
of the inhabitants of the province, at their expected assembly of the year 1600, although their treasury
was drained by the continuous frontier wars, obtained for me a magnificence remittance, at a time when I
was absent in Bohemia, Thus the Founder of the Heavens provided for me as herald of his works the ex-
penses of my journey, when I was about to move my household to Bohemia.*

Uf
PrazeFatio Antigva Ap LecroremM.
i Rorostrvm ctmihi,Leétor,hoclibellodemonftrare,quod Crea=
\& tor OptimusMaximus,in creatione mundi huius mobilis,& difpofitio=
7% necalorum, (1) adilla quinqueregulatia corpora, indeaPythagora
¥EC3) & Plarone,ad nos vique,celebratiifimarefpexcrit,atquead illorum na-
turam coclorum numcrum,proportiones,& motuum rationemaccom=
modauerit.Sedantequam tead remipfam venice patiat : cum deoccafionc huius li-
belli, tum deratione meiinfticuti,aliqua tecum again: quae & ad raumintelle@um,
&ad meam famam pertinere arbitratus fuero.

Quo tempore Tubingz,ab hinc fexennio clariffimo viro M.Michacli Meftli-
nooperamdabam: motusmultiplici incommoditate vfitarx de mando opinionis,
adeo delectatusfam Copernico,cuiusillein praledtionibus fais plurimam mentio=
nem facicbar: vt nontantum crebro gius placitain phyficis difpurationibus candi-
darorum defenderem: fed etiam accuratam (2) difpurationcm de motu primo,
quod Terre volutioncaccidat , conferiberem. lamque in co cram,vtcidem etiam
(3) Telluri motum Solarem , ve Copernicus Mathematicis , fic ego Phyficis,
feumauis,Metaphyficisrationibus afcriberem. Atqueinhunc vfam pattimex ore
Maflini,partimmeo Marte, quas Copernicusin Mathefi pr Prolemaro habet cé-
moditates,paulatim collegi:quo labore me facile liberare potuiffet loachimusR he»
ticus,quifingula breuiter,& perfpicue prima fua Narratione perfecutus eft. Interea
damillad frum voluosted nagleyus,fecus Theologiam:commode accidit, ve Grg-
tium vcnitem,atqucibi Georgio Stadio,p.n.faccederem: vbi officiiratiome arétius
his ftudiis obfttinxit.Ibiin explicatione principiorumA ftronomiz,magno mihi vfui
fucrunt omniailla,quee antea veld Mavftlino audiueram, cl ipfe affeétaucram.Atqs
vein Virgilio, fama Mobilicate viget,virefque acquirit cundo: fic mihi harum rerum
diligens cogiratio, cogitationis vitcrioris caufa fuit, Donectandemanno1s95.cum
ocium dleétionibus cuperem bene, & ex officiizarione tranfigete,toto animi impe+
tain hancmateriam incubui.

Extvia poriffimum crant, quorum ego caufas, curita ,nonalitereffent, perti-
naciter guatcbam, Numerus,Quantitas,& Motus Orbium. Vt hoc auderem,cffe-
citil!a pal. hra quiefeentium harmonia, Solis,fixarum & inter medii, cum Deo Pas
tre, & Filio, & f2n@oSpiritu: (4) quam fimilitudinem ego in Cofmographia per
fequararzplius.Cumigitacita haberent quie(centia,non dubicabam demobilivus,
quin fe prabitura fine Jnitiorem numeris aggreffus fum,& confideraui,verum vnus
orbisallus duplum,triplum,quadruplum , aug quidtandem haberet; quantumque
quilibet’ quolibet in Copeinicodiffideret. Plurimum temporisifto labore,quafi
Tifiperdidis cdi nulla, neque ipfarum proportionum, nequeincrementorum ap=
parerct equalitas:nihilg; vuilitatisinde percepi,quam quod diftantiasipfas,vt3Co-
pernico prodira:fant, altiffime memoriz infeulpi: quodque hac variqrum cona-
tuum commemoratiotuum affenfam,le€tor, quafi marinis fudtibus, anxic hincin=
deiaétare poteft, quibus fatigatus,denique tanto libentius ad caufas hoclibello ex-
pofitas,ranquam ad tutum portum terccipias. Confolabanturme tamen fubinde,
Sin fpemmcliorem crigebant , cumalic rationes,qua infra fequentur, tum quod
femper motus diftantiam pone fequi videbatur, arque vbi magnus hiatus cratinter
orbes,erat &intermotus. Quod fi( cogitabam) Deus motusad diftantiarum pra
{criptum aprauit orbibus: veique & ipfas diftantias ad alicuius rei pra(criptumac-
commodauir.

Cum igiturhacnon fuccederct, alia via, mirum quam audaci,tentauiaditum,
(5) Inter louem & Martem interpofuinonum Planctam, itemque alium inter Ve=
Aerem & Mercurium,quos duos forte ob exilitarem non videamus,iifque fia tem:
poramescdvxdafcripfiSicenimexiftimabam me aliquam aqualitatem proportionit
¢ffeéturum, qux proportionesinterbinos verfus Solem ordine minuerenturverfus
fixasaugefccrent: vtpropioreft Terra Vencriin quantitate orbis terreftris , he

ars



It is my intention, reader, to show in this little book that the most great and good
Creator, in the creation of this moving universe, and the arrangement of the
heavens, looked (1) to those five regular solids, which have been so celebrated
from the time of Pythagoras and Plato down to our own, and that he fitted to the
nature of those solids, the number of the heavens, their proportions, and the law
of their motions. But before permitting you to come to the actual subject, I shall
discuss briefly both what occasioned this book and my reason for undertaking it,
which I think will affect not only your understanding but my reputation.

At the time, six years ago, when I was studying under the distinguished Master
Michael Maestlin at Tiibingen, I was disturbed by the many difficulties of the
usual conception of the universe, and I was so delighted by Copernicus, whom
Mr. Maestlin often mentioned in his lectures, that I not only frequently defended
his opinions at the disputations of candidates in physics but even wrote out a
thorough (2) disputation on the first motion, arguing that it comes about by the
Earth’s revolution. I had then reached the point of ascribing to this same Earth (3)
the motion of the Sun, but where Copernicus did so through mathematical
arguments, mine were physical, or rather metaphysical. And for this purpose I
collected together little by little, partly from the words of Maestlin, partly by my
own efforts, the advantages which Copernicus has mathematically over Ptolemy.?
I could easily have been relieved of this toil by Joachim Rheticus, who has briefly
and penetratingly treated the particular points in his Narratio Prima.> In the
meantime, while I was rolling that rock, but as a sideline apart from theology, by
a lucky chance I came to Graz, where I succeeded the late George Stadius; and
there the duties of my post obliged me to attend more closely to those studies. For
expounding the principles of astronomy there, everything I had previously either
heard from Maestlin or worked out for myself was of great value. And as in
Virgil “the report Grows by travelling and gains strength as it goes,” so for me
the careful contemplation of these topics was the cause of further contemplation.
Finally in the year, etc., 95, when I had a strong desire to rest from my lectures,
and to have done with the duties of my post, I threw myself with the whole force
of my mind into this subject.

There were three things in particular about which I persistently sought the
reasons why they were such and not otherwise: the number, the size, and the mo-
tion of the circles. That I dared so much was due to the splendid harmony of
those things which are at rest, the Sun, the fixed stars and the intermediate space,
with God the Father, and the Son, and the Holy Spirit.5 (4) This resemblance I
shall pursue at greater length in my Cosmographia. Accordingly, since this was
the case with those things which are at rest, I had no doubt that for things which
move, similar resemblances would reveal themselves. In the beginning I attacked
the business by numbers,* and considered whether one circle was twice another,
or three times, or four times, or whatever, and how far any one was separated
from another according to Copernicus. I wasted a great deal of time on that toil,
as if at a game, since no agreement appeared either in the proportions themselves
or in the differences; and I derived nothing of value from that except that I
engraved deeply on my memory the distances which were published by Coper-
nicus. But as this recital of my various attempts may toss your approval, reader,
anxiously to and fro as if on the sea’s waves, which will tire it, you will at last
come all the more gladly to the causes explained in this little book, as though toa
safe harbor. Yet I was comforted repeatedly, and my hopes were raised, not only
by the other arguments which will follow below, but also by the fact that the mo-
tion always seemed to be in step with the distance, and where there was a great
gap between the spheres, there was also one between the motions. But if (thought
1) God allotted motions to the spheres to correspond with their distances, similar-
ly he made the distances themselves correspond with something.

Since, then, this method was not a success, I tried an approach by another way,
of remarkable boldness. (5) Between Jupiter and Mars I placed a new planet,’ and
also another between Venus and Mercury, which were to be invisible perhaps on
account of their tiny size, and I assigned periodic times to them. For I thought
that in this way I should produce some agreement between the ratios, as the ratios

Nae vider
fequenti
Scbumate.


Mars Terre, in quantitate orbis Martii. Verumneque vnius planet interpofitia
ffficiebat ingenti hiatui £ &¢ J. Manebat enim maior Iouisad illum nouum pro
portio,quameft Saturniad Jouem:Ethocpagto quamuis obtinerem qualemcun-
que proportionem , nullus amen cum ratione finis , nullus certus numerusmobi-
Num facuruserat,neque verfus fixas,v(que dumille ipfe occurrcrent: neque verfus
Solem vnquam, quia diuifio {pati poft Mercurium refidui per hanc proportionem
ininfinitum procederet. (6) Neque enim ab vilius numeri nobilitate conieétari
poteram, cur pro infinitist&
pauca mobilia extitifsér: Ne=
D gue verifimilia dicit Rheti-
K cus in faa Narratione, cum
fandticate fenarii argumenta-
H turad numeraSexCclorum
(7 mobilium, Nam qui deipfi-
us mundi condita difputar,
non debetrationes ab iis nu~
EF meris ducere, (7) qui cx
rebus mundo pofterioribus
(> dignitatem aliquam adepti
fane.

Rurfamaliomodo explo-
raui, verum nonin codé qua-
drante diftantia cuiuslibet
Planet fit refiduum exfinu,
Semotuscius fitrefiduum ex
Fixe ciuscomplementifiau. Fin-
el JB gatur quadracum A B, 4Se-

midiametro totius Vniuerft

ACdeferiprum, Ex angulo
igitur B Soli fiue Centro Mundi A oppofite, (exibacar cam radio BC Quadrans
CED. Deinde in vero mundiradio AC notentur Sol, Fixx & Mobilia proratione
diftantiarum:4 quibus pun@isexcitenturreatz, viquead obuerfium Soli Quadran-
rem porrectz. Que igitureft proportio parallclorurn,eandem virtutis mouentis fin-
xipenes fingulos planetas. In Solis linea infinicas petmanet, quia AD cangitarnon
fecaturd Quadrate.Infinita igitur vismotus in Sole, nempe nil nifi motusipfilfimo
adtu.In Mercurio infinitalinca in Kab{ciffa eft.Quare eius motusiam eftad cateros
<ompa:abilis.In fixis amiffa eft omnind linea, & compreffa in merum punétum C.
Nulla igituribivirrusadmotum. Hoetheorema fait quod ealeulorscexaminane
dum. Quod fi quis probé ponderat,duo mihi defuiffe, primi, quod ignoraui finum
totum, fiue magnicudinemillius propofiti quadrantiscalterum , quod motuum vi
gores non fuerunt aliter expreifi quim in proporione vnius ad alium : qui, in
quam, hac probé ponderat, non immeritd dubitabie, verum aliquatenus hac diff-
cili via peruenire potuerim necne.Ettamen continuo labore,atque infinita finuum
&carcuum reciprocatione tantum effeci, veintelligerem, locum habere nonpoffe
hancfententiam.

Aifias pen? totahaccruce perdita. Deniqueleui quadam occafione propiusin
rem ipfam incidi. Diuinitus id mihi obtigiffe arbitrabar, vt fortuitd nancilcerer,
guodnullo vnquamnlaboreaffequipoteram:idg 3 magis credebarn: quodDeum

femper oraueram,fiquidem Copernicus vera dixiffet,vei ifta faccederent.Igitur die
9-vel19. luliianni 1595. monftraturus Auditoribus meis coniun@ionum magna-
rum faltus per o€tona figna, & quomodo ill pederentim ex vnotrigono tranfeant
in alium, inferipfi multa triangula , vel quafitriangula, eidem circulo, ficvt finis
vniuseffetinitium alterius. Igitur quibus punctis latcra triangulorum fe muruod fe-
cabant, iis minor circellusadumbrabatur. Nam circuli triangulo in(criptiradius,

cht cire


Original Preface

between the pairs would be respectively reduced in the direction of the Sun and
increased in the direction of the fixed stars, as the Earth is nearer to Venus, relative
to the size of the Earth’s circle, than Mars is to the Earth, relative to the size of the
circle of Mars. Yet the interposition of a single planet was not sufficient for the
huge gap between Jupiter and Mars; for the ratio of Jupiter to the new planet re-
mained greater than that of Saturn to Jupiter; and on this basis whatever ratio I
obtained, in whatever way, yet there would be no end to the calculation, no
definite tally of the moving circles, either in the direction of the fixed stars, until
they themselves were encountered, or at all in the direction of the Sun, because
the division of the space remaining after Mercury in this ratio would continue to
infinity. (6) And I could not conjecture from the nobility of any number why so
few moving stars existed in proportion to the infinite number (of the fixed stars).
What Rheticus says in his Narratio, when he argues, from the sanctity of the
number six, for six as the number of the moving heavens, is unlikely. For in
discussing the foundation of the universe itself, one ought not to draw explana-
tions from those numbers which have acquired some special (7) significance from
things which follow after the creation of the universe.®

Again I investigated by another method whether the distance of any planet in
the same quadrant may not be the remainder of the sine, and its motion the re-
mainder of the complement of the sine.? Construct a square AB, described on the
semidiameter of the whole universe AC. Then from the vertex B opposite to the
Sun or the center of the universe A, a quadrant CED with radius BC. Next, on the
true radius of the universe AC mark the Sun, the fixed stars and the moving stars
in accordance with the ratio of their distances, and from these points erect
perpendiculars reaching to the quadrant turned towards the Sun. Then the pro-
portion of the parallel lines has been constructed as the proportion of the power
of the motion belonging to the individual planets."° In the case of the Sun the line
remains infinite, as AD is touched but not cut by the quadrant. Therefore the
motive power in the Sun is infinite, though of course only in the actual bringing
about the motion, In the case of Mercury the infinite line is cut off at K, so that its
motion may now be compared with the others. In the case of the fixed stars the
line completely disappears, and is compressed into the mere point C. Thus there is
no power of causing motion here. This was the theorem which was to be tested by
calculation, But on due consideration, I lacked two things. First, I did not know
the total sine, or the size of the quadrant in question; secondly, the strengths of
the motions were expressed only in terms of a ratio of one to another. On due
consideration, as I say, it will properly be doubted whether I could reach any con-
clusion by this difficult route or not. However, by continuous toil and endless
reiteration of the sines and arcs, I managed to determine that this proportion
could not be maintained.

Almost a whole summer was wasted on this ordeal. Eventually by a certain
mere accident I chanced to come closer to the actual state of affairs. I thought it
was by divine intervention that I gained fortuitously what I was never able to ob-
tain by any amount of toil; and I believed that all the more because I had always
prayed to God that if Copernicus had told the truth things should proceed in this
way. Therefore on the 9/19th of July in the year 1595 when I was going to show
my audience the leaps of the great conjunctions through eight signs at a time, and
how they cross step by step from one triangle to another, I inscribed many
triangles, or quasi-triangles, in the same circle, so that the end of one was the
beginning of another.’? Hence the points at which the sides of the triangles in-
tersected each other sketched out a smaller circle. For the radius of a circle in-
scribed in a triangle is half the radius of the circumscribed circle. The ratio of the

Diagram opposite:
FIXA: Fixed Stars


You see
them in the
Following
diagram
Av Lecrorsam

eft circumfcriptiradij dimidium. Proportio inter virumque circulum videbatuc
ad oculum pene fimilis illi, quz extinter Saturnum & Iouem : & triangulum pri-
ma crat figurarum, ficut Saturnus & Iupiter primi Planeta. Tentaui ftatim qua-
drangulo diftanciam {ecundam Martis & louis, quinquangulotertiam , fexangulo
quaitam.Cumque etiam oculi reclamarentin {ecunda diftantia,qug cftinter louem,
& Martem pres triangulo & quinquangulo adiunxi. Infinitum eft fingula
perfequi. :

Etfinis huius ieviti conatus faitidem , quipoftremi & felicisinitium. Nempe
cogitani, hac via, fiquidem ordinem inter figuras velimn feruare , nunquam me per-
ncniuram vique ad Solem,negs caufam habicurum,cur potius fint fex, quim vigin-
tivel centum orbes mobiles. Ettamen placebant figure, vrpote quantitates, & res
cclis prior. (8) Quantitas enim initié cum corpore cteatajcerlialtero die. Quod
fi(cogitabam.) pro Quantitate & Picea cee >quos ftaruit Coperni-

inque tantiim figura interinkinitas reliquas reperiti poffent, que pra cx-
teris peeuliarés quafdam proprictateshaberent:exvotoreselfer. Arquirurfumin«
fezbam, Quid igure pianzinter folidos orbes? Solidapotiis corpora adeantur.
Ecce, Ledtor, inucntum hoc & materiam totius huius opufculi. Nam fi quis leui-
ter Geomettiz peritus totidem verbis moneatur,illiftarim in promprufunt Quin«
que regularia corpora cum proportione orbium citcumferiptorum ad infcriprosilli

ftaum ob oculos verltur,{cholionillud Euclideumad propofitioné 18.lib.15-Quo

Schema magnarum Coniundtio=
num Saturn é louis, earumque fal-
tusper oftena figna , arquetranfitus

per omnes quatuor Zodiaci triplicita-

B demon-


Original Preface

circles to each other appeared to the eye almost the same as that between Saturn
and Jupiter; and the triangle was the first of figures, just as Saturn and Jupiter
are the first planets. At once I tried the second interval, between Mars and
Jupiter, in a four-sided figure, the third in a five-sided figure, the fourth in a six-
sided figure. Since that was obviously wrong at sight, in the second interval,
which is between Jupiter and Mars, I added a four-sided figure to the three-sided
and the five-sided figure. It is an infinite task to follow up individual cases.
The end of this useless attempt was also the beginning of the last, and suc-
cessful, one. I naturally concluded that by this method, if I wished to keep an
order among the figures, I should never reach the Sun, nor have an explanation
why there should be six moving circles rather than twenty or a hundred. However,
the figures were satisfactory, as they represent quantities, and so something prior
to the heavens. For (8) quantity was created in the beginning along with matter,
but the heavens on the second day. But if (thought I) corresponding with the size
and proportion of the six heavens, as Copernicus established them, there could be
found only five figures, among the infinite number of others, which had certain
special properties distinct from the rest, it would be the answer to my prayer.
Again I set to. Why should there be plane figures between solid spheres? It would
be more appropriate to try solid bodies. Behold, reader, this is my discovery and
the subject matter of the whole of this little work. For if anyone having a slight
acquaintance with geometry were informed of this in so many words, there would
immediately come to his mind the five regular solids with the proportion of their
circumscribed spheres to those inscribed; there would immediately appear before
his eyes the scholium to Euclid’s Proposition 18 of Book XIII, in which it is

Diagram opposite:
Diagram of the great conjunctions of Saturn and Jupiter, and their leaps through eight signs, and cross-
ings through all four quartiles of the Zodiac.

Jo Toan. Kerrert

demonttratur impoffibile effe , ve plura fint aut excogitentut regularia corpora,
quam quingue. Res admirationedigna, cum nondumconftarer nihide fingulo~
tum cotporum prarrogatiuisin ordine,vfum me minime arguta comieétura ex notis
Planctarum diftantiis deducta, adco feliciter fcopum tetigifein ordine corporun My
venihil inillis poftea,cum exquifitisagerem rationibus,immutare potucrim. Ad rei
memoriam afcribo tibi fententiam, ita vtiincidit , & eo momento verbis conce-
pram. (9) Terraeft Circittus menfor omninm : Il circumferibe Dodccaetron: Circulus hoc com-
prebendenserit Mars. Marticircumm(cribe Tetrsedron: Circulas boc conprchendenscrit Iapiter to-
aii cincoanfcribe Cubwn: Circitlus bun comprebendens erit Saturnus. Lam terrainfcribe leofaedron:

lls inferipeus Circulus rit Venus.V exert inferibe Odtaedron: Mliinfcriptus Circulys crit Mercurins,
Habes rarionem numeri plancrarum,

Hac oceafio & facceflus huius laboris: Vide nunc ctiam meum in hoclibro pro-
ofitum. Er quidem quantam xinuentione volupratem percepetim ,niiquam ver
is expreffero.. Non me perditi temporis peenitebatamplius, non verttum eftla-

boris ,moleftias calculi nullas fubtertugi dies nogtefque compurando contami:
donee cernerem, virum concepta verbis fententia cum Copetnici orbibus confen-
titet,an vero ferent mea gaudia venti. Quod firem,vtieffe putabam,deprehende-
rem,votum Deo Opt. Max. foci, me prima occafione hocadmirabile fax fapientiz
{pecimen publicis typis inter homines enunciaturum : ve quamuis neque hee vndi-
quaqueabfoluta fint,& forte reften: nonnulla,queex hisilaant principiis,quorum
inucntionem mihi referuare poffemn: tamen alii , qui valentingenio,quam plurima,
adilluftrationem Nominis diuini, primo quoque tempore iuxtame proferrent, &
Jaudem {apicntiffimo Creatori vno oreaceinerent. Cumigitur paucis poft dicbus
resfaccederetatque ego deprchenderem,quam apte vnum corpus , poltaliud inter
fuos Planctasfederer, totumque negocium in formam prsvfintis opufculiredige-
rem:atqueid Mecfllino celebri Mathematico probarctur : intelligis, amice Leétor,
me votiicum , neque poife morem Satyrico gerere , qui nonum in anaum iubetli-
bros detinere.

Hac vnacaufactt mex maturationis: cui vtomnem tibi fcrupulum, (10) fi-
niftre fufpicandi eximam,addo lubens & altcramn,& recitu tibi,illud Archita ex Ci-
cerone : Si eatin ipfum afeendiffem , Naturanquemunds, &> pulcbritadinenn fideriion pests
peafpesiffenenfixanis lla nibs foretadmiratio, nsf te Letorcm aqua, attentumn & cupidiom , cud
narraren,baberem. Hee vbicognouctis , ficquuses, abftinebis a reprchenfionibus,
guasnon fine caufa prefagio: Sin autem fuo quidem loco rclinquisifia: metals ta-
men, vecerta int, atque vt cgo triumphum cecinerim ante victotiam :ergo vel tan-
dei pagellasipfasaccede, & tem ,quade pridem agimus, cognofee. Nonseperies
nowos & incognitos Planetas, vepanloantea, interpofiras non ea mihi probatur
audacia ; (ed illos vereres parum admodum luxatos , intevicétu vero rectilincorum,

corporum, quantumuis abfurdo,ita munitos, vt porro, quibus vncis ca

Jum quo minus ruar,fifpendatur, quacenuirattico
xefpondere poflis. Vale.

a



Original Preface

shown that it is impossible for there to be or to be conceived more than five
regular solids. It is a wonderful thing that when I had not yet settled the proper-
ties of the individual bodies within their arrangement, yet using such a clumsy
conjecture drawn from the known distances of the planets, I should so successful-
ly have hit the target over the arrangement of the bodies that there was nothing
which I could later change in them when I was working with the ratios calculated
in detail. As an aid to memory I give you the proposition, conceived in words just
as it came to me and at that very moment: (9) “The Earth is the circle which is the
measure of all. Construct a dodecahedron round it. The circle surrounding that
will be Mars. Round Mars construct a tetrahedron. The circle surrounding that
will be Jupiter. Round Jupiter construct a cube. The circle surrounding that will
be Saturn. Now construct an icosahedron inside the Earth. The circle inscribed
within that will be Venus. Inside Venus inscribe an octahedron. The circle in-
scribed within that will be Mercury.” There you have the explanation of the
number of the planets.

This accident was also the happy ending of my toil. You can now also see my
scheme for this book. What delight I have found in this discovery I shall never be
able to express in words. No longer did I regret the wasted time; I was no longer
sick of the toil; I did not avoid any of the tedium of the calculation; I devoted my
days and nights to computation, until such time as I could see whether the prop-
osition which I had conceived in words would agree with the circles of Coper-
nicus, or whether my joy would be scattered to the winds.?? But if I found out
that I was right, | made a vow to Almighty God that at the first opportunity I
would proclaim among men in public print this wonderful example of his
wisdom, so that although the work is not in every way complete, and there may
perhaps remain some points to emerge from these beginnings, the discovery of
which I could reserve for myself, yet others, with powerful talents, would bring
out as many of them as possible, to the glory of God’s name, and at the earliest
possible moment after me, and would sing the praise of our most wise Creator
with a single voice. Therefore, since the success came after a few days, and I
found out how neatly one body fitted after another among their planets, and 1
reduced the whole business to the form of the present work, and had it approved
by Maestlin the celebrated mathematician, you will understand, dear reader, that
Tam bound by my vow, and cannot oblige the satirist who tells us to delay books
for nine years."

That is the sole cause of my haste; but to relieve you of any lingering (10)
adverse suspicion, I gladly add another as well, the saying of Architas from
Cicero: “If I had ascended the very heaven, and beheld completely the nature of
the universe, and the beauty of the stars, the wonder of it would give me no
pleasure, if ] did not have you as a friendly, attentive, and eager reader to whom
to tell it.”"* When you know that, if you are fair, you will refrain from the
criticisms which I have good reason to expect. However, if you leave all that for
its proper place, yet you are afraid that these things are not certain, and that [
have sung my song of triumph before I have won the victory, then go on at last to
the actual pages, and find out about what I have long been discussing. You will
not find any new and undiscovered planets interpolated, as I did a little while ago;
Ido not favor that piece of audacity. You will find the old ones, very little dis-
turbed, though so secured by the insertion between them of rectilinear bodies,
however absurdly, that you will have an answer for the peasant who asks what
hooks the sky is hung on to prevent it from falling.

J. Kepler

om Toan. Kerrert

nequagiar inf fenariumn DEO Creatoriplacuife propter hancindolem. Dico fecundo,hanc affe-
Gionem non competere Numeris,ve numerantibus, Idfacsleprobatur ex Enclidis ib. VU. VIL.
IX. Veenimauétorille demonfiret,ineffequibufdam hanc perfeltionemn , cogitur yti numerisfigu-
rratis,id eff, vt febole loquuntur , Numeris numeratis feu parallelogrammis ,equali menfura diui
fisinlongum &latum. Quarefiqna maximenobilitatisnoracffet, hec ficdilta perfedti ,illa prie
‘mo competeret Geometricisfiguris, Etfivero fenarine veram fuam Cr realem nobilitatem bsbet ex
fexanguto , que figura ipfum prouehit in diftiplina harmonica : non ideo tamen etiam ad con
lituenduim numeram primariorum Mundi corporum fit aptus. Figura enim ila circulum , vt
continuam quantitatem in fex partes diuidit : corpora Mandana non funt partes vnins conti-
nua quantitaris, Ula figura inter planas eft : corporibus vero mundi folida , feu trium dimen~
fionums fpacia data funt peragranda, Rete igitur repudiaui fenary ipfius per fe confiderati do-
tes, ne adfcifcerem illas inter cauffas fenariy Calorum : rette cenfui , oportusffe pracedere caufas
aliquas euidentes, ex quibus deinde fenarins ifle Calorum vltro refultarets, ficut in Harmonica
Asfciplina , catifis pregrefia idoneis , refultat G ternarins confonantiuin in idem fonorum fol.31.
pofter.c> feptenarins duuifionum Harmonicarumfol.z7 pofter.

(8) Quantitaseniminitio cum corpore. ] _Imoldee quantitatum funt erantque
Deocosterna , Devs ipfes funtqueadbuc exemplariter in animisad imaginem Dei ( etiam effentia
ua) fadts,quainreconfentinnt gentiles Philofopbi,é> Dottores Ecclefa

(9) Terraet Circulus.] Scripféram ifta mibi fois inteligebams igitur pro Terra, Or-
Lem, quo illa vebitur , Magnum & Copernico dictum : fic proquolibet Planeta, orbens ipfius. Et
pertinet vleimum comma, Habes rationem, &c. etiam ad hanc ex febedie exferiptam fenten=
tiam,

(10) Siniftre fafpicandi,&<c.] Laboraui puerilitcr , ne quis mibiimputaret, menowa-
torem effe,oftentandi folumm ingenij cana librumn feripfife: his oppofui & vorum & penitifimam
perfuafioncin de veritate corum, qua liber contineret, & denique ardorem conferends
cum als debisinuentia, Et erant, opinor,idonea canfe,profi~

‘gandi pudoris inepti.




Original Preface

confers no more power and flexibility on numbers than the finding of treasure does on a farmer. Thus it
is quite unbelievable that the number six should have pleased God the Creator on account of this quality
Secondly | say that this characteristic does not belong to the numbers insofar ay they are “counting
numbers.” That is easily proved trom Euclid, Book VII, VIII, IX. For in order to demonstrate that cer-
tain numbers possess this perfection, that author is obliged to use numbers represented by diagrams, that
is, as the schools say, “counted numbers,” or parallelograms, divided in equal proportions in length and
breadth. Consequently if this so-called perfection were any mark of the highest nobility, it would belong
primarily to the geometrical figures. Although indeed the number six has its own true and real nobility
from the hexagon, the figure which makes it important in the discipline of harmony, yet itis not on that
account also fitted to constitute the number of primary solids of the universe. For that figure divides the
circle as a continuous quantity into six parts, but the bodies which make up the universe are not parts of
one continuous quantity. That figure is one of the plane figures: but the bodies of the universe are allot-
ted spaces which are solid, or of three dimensions, to traverse. I was therefore right to repudiate the en-
dowments of the number six considered on its own, and not to count them among the causes of the
heavens’ being six in number. I rightly judged that there ought to be some preceding causes in evidence,
of which it would afterwards be the spontaneous result that the heavens would be six in number; just as in
the discipline of harmony appropriate causes come first, and the result is that a set of three notes form a
concord together (the second page 31), and there are seven harmonic divisions (the second page 26)

(8) For quantity. ..12 the beginning along with matter.] Rather the Ideas of quantities are and were
coeternal with God, and God himself; and they are still like a pattern in souls made in the image of God
{also from his essence). On this matter the pagan philosophers and the Doctors of the Church agree.

(9) The Earth is the circle.) had written this for myself alone, and so by the Earth 1 understood the
orbit on which it travels, called the Great by Copernicus, and similarly by each planet, its own orbit. The
last sentence, “There you have the explanation, ctc.,” also belongs to this statement copied from my
notes.

(10) Adverse suspicion, etc.] I struggled in my youthful way to avoid the imputation of being a
radical, of having written a book just for the sake of showing off my own cleverness. | resisted that both
by this avowal and by a most thorough defense of the truth of what the book contained, and lastly by zeal
in discussing these discoveries with other people. Indeed there were, I believe, sufficient reasons for
casting aside any absurd modesty.

Mysrertvm CosmoGRapuicvat. ty


Qurbusrationibus Copernici hyporhefes fiant confentanea. Et
explicatio hypothefium Coperarcs.

Ts1 pium eft, ftatim ab initio huius de Natura difpu-
tationis videre, an nihil Sacris Literis contrarium dica-
tur intempeftiuum tamen exiftimo,eam controuerfiamt
hic mouere, priusatque folliciter. Mludin genere pro-
iitto , nihil me diéturum, quod in Sacras Literas miurid
fit,& fi cuius Copernicus mecum conuincatur, pro nul-
lohabiturum. Arque camens mihi femper fuit, inde a quo Copernici
Reuolut:onum hbros cognofcere ccepi.

Cum igicur hac in parte nulla religione impedirer,quo minus Co»
pernicum,ficonfentanea diceret,audirem : primam fidem mihi fecitilla
pulcherrima omnium,quein coelo apparent, cii placitis Copernici con-
fenfio:vt qui non folum motus pr¢teritos ex vitimaantiquitate repetitos
demonftraret, fed etiam futuros antea, non quidem certillime,fed tamé
longe certius,quam Prolemaus, Alphonfas, & ceter,diceret. Hludau-
tem longemaius,quod qua ex alijs mirari difcimus, eorum folus Coper«
nicus pulcherrime rationem reddit, caufamque admirationis, quz efti-
gnoratio caufarum,tollit. Nunquam id facilius docueroLe&orem,quam
fiad Narrationem Rhetici legendam illi autor & perfuafor exiftam.
Nam ipfos Copernici libros Reuolutionum legere non omnibus va-
cat.

(2) Atqshocloco nunquam affentiri potui illis, qui freti exéploac-
cidentariedemonttrationis , que cxfalfis pramiifis neceffitate Syllogi-
ficaverum aliquid infert. Qui,inquashoc exemplo freti contendebant,
fieri poffe,vt falfe fint , que Copernico placent ny ounce &tamenex
illis vera assp8ve ranquam ex genuinis principijs {equantur.

Exemplumenim non quadrat. Nam iftafequelaex falfis pramiffis
fortuitaeft, & quz falfi natura eft ; primum atque alii rei cognatzacco-
modatur, feipfam prodit: nifi ponte concedasargumentatori illi, veins
finitasalias falfas propofitiones affumat, nec vnquain progreffu, regref-
faque fibiipfi conftet. Aliter fe reshabet cum €0 , qui Solem in cétro col-
locat.Nam iube quidlibet corum,quz reuerainCcelo apparent,ex femel
pofita hypothefidemonttrare, regredi,progredi, vnum exaliocolligere,
& quiduis agere, que veritas rerum patitur: neq illehzfitabicin vilo,
genuinum fir, & vel ex intricatiffimis demonftrationum anfragtibus in fe
ynum conftantiflimercuertetur. Quod fiobijcias idem partim adhue
poffe,partim olim potuiffe dici de tabulis & ByPeteleeeaadyar ne
nempedawopfiows farisfaciant: Atqueillascamen a Copernico, vefalfas
reijci: Pofleigitur cadem ratione & Copetnico refponderi:nempe qua
uisegregiecorum, que apparentrationemreddat, tamen in hypothefi
errare.Refpondeo,primum,antiquas hypothefes precipuorum aliquot
capitum,nullam plane rationcm reddere. Cuiufmodi eft,quod ignorant,

3 numeri,



Although it is proper to consider right from the start of this dissertation on Nature
whether anything contrary to Holy Scripture is being said, nevertheless I judge
that it is (1) premature to enter into a dispute on that point now, before I am
criticized. 1 promise generally that I shall say nothing which would be an affront
to Holy Scripture, and that if Copernicus is convicted of anything along with me,
I shall dismiss him as worthless.? That has always been my intention, since I first
made the acquaintance of Copernicus’s On the Revolutions.

In this area, then, I should not be prevented by any religious scruple from
listening to Copernicus, if what he said was consistent. My confidence was first
established by the magnificent agreement of everything that is observed in the
heavens with Copernicus’s theories; since he not only derived the past motions
which have been recapitulated from the remotest antiquity, but also predicted
future motions, not indeed with great certainty, but far more certainly than
Ptolemy, Alfonso, and the rest.? However what is far more important is that, for
the things at which from others we learn to wonder, only Copernicus magnificent-
ly gives the explanation, and removes the cause of wonder, which is not knowing
causes.‘ The easiest way for me to show the reader that would be for me to incite
and persuade him to read Rheticus’s Narratio, for not everyone has the leisure to
read Copernicus’s On the Revolutions itself.

(2) On this point I have never been able to agree with those who rely on the
model of accidental proof, which infers a true conclusion from false premises by
the logic of the syllogism. Relying, as I say, on this model they argued that it was
possible for the hypotheses of Copernicus to be false and yet for the true
phenomena to follow from them as if from authentic postulates.

For the model does not fit. The conclusion from false premises is accidental,
and the nature of the fallacy betrays itself as soon as it is applied to another
related topic—unless you gratuitously allow the exponent of that argument to
adopt an infinite number of other false propositions, and never in arguing for-
wards and backwards to reach consistency. That is not the case with someone
who places the Sun at the center. For if you tell him to derive from the
hypothesis, once it has been stated, any of the phenomena which are actually
observed in the heavens, to argue backwards, to argue forwards, to infer from
one motion to another, and to perform anything whatever that the true state of
affairs permits, he will have no difficulty with any point, if it is authentic, and
even from the most intricate twistings of the argument he will return with com-
plete consistency to the same assumptions. But you may object that it can to some
extent still be said, and to some extent could once have been said about the old
tables and hypotheses, that they satisfy the appearances, yet they are rejected by
Copernicus as false; and that by the same logic the reply could be made to Coper-
nicus that although he gives an excellent explanation for what is observed, yet he
is wrong in his hypothesis. I reply first that the old hypotheses simply do not ac-
count at all for a number of outstanding features. For instance, they do not give
the reasons for the number, extent, and time of the retrogressions, and why they

14 loan, Kerrert

numeti,quantitatis,tem porifqueretrogradationum caufas: & quareil-
wad amuflim ita (3) cumloco & motu Solis medio conueniant. (4)
Quibus omnibus in rebus, cum apud Copernicum ordo pulcherrimus
appatcat,caufam etiam ineffe neceffe eft. Deinde earum etiam hypothe-
fium,quz conftantem apparentiarum cau fam reddunt, & cum viluccn-
fentiunt,nihil negat Copernicus,potius omnia fumic & explicat. Nam
quod multa in hypothefibus vfiratis immutaffe videtur , id reuera ndita
{ehabet. Ficrinamque poteft,veidem contingat duobus {pecic differé-
tibus prafuppolitis, propterea quod illaduo fab codem geneiefunt,cu-
ius gratia generis primo id contingit,de quoagitur.Sic Pcolemaus Stel-
larum ortus & obitus demonftrauit, non hoc medio termino proximo,
& coxquato;Quia terra fit in medioummobilis. Neqs Copernicus idem
hoc medio demonttrat,quia terraa med 0 diftans voluacur. Veriq; enim
fuffecit dicere(quod &veerque dixit)ideo heec ita fieri,quia inter celum
& terram intercedat aliqua motuum feparatio , & quia nulla inter fixas
ferstiacur telluris 4 medio diftantia.Igicur Prolemzus non demonftrauit
falfo & accidentario medio, fi que demonftrauit daie.Hoc tantum
inlegem *at’ «i peccauit , quod exiftimauit, hac ita propter fpeciem
cucnite,qua propter genus eueniunt. Vndeapparetex co,quod Prole-
inaus ex talfa mundi difpofitione,veratamen, & Ceelo, noftrifq; oculis
confonademonftrauit,cx eo inquam,nul'am effe caufam, fimile quid et-
iam de Copernicanis hypothefibus fufpicadi. Quin potius manet,quod
initio diétum eft:non poffe falfa effe Copernici principia, cx quibus tam.
cOftans plurimord Gevquhiav ratio,ignota vetetibus,reddatur, (5) qua-
tenus exillisredditur. Vidithocfeliciffimus ille Tycho Brahe, Aftrono -
mus omni celebratione maior, qui quamuis omninodelocoterrr2a Co-
petnico diflentiret, tamen ex co retinuit id, cuius gratia rerum haétenus
incognitacum caufas habemus: Solem nempeeffe centrum quing; pla-
netarum.Nam & hoc anguftius eft medidi ad demonftrandas repedatio-
nes:7 Solin centro immobilis. Sufficie enim generalcillud, Solin cétro
Planctarum quinque. Curautem {pecié pro genere fumeret Coperni-
cus,& Solem infuper in centro mundi, terram circa cum mobilem face-
retialia caufzfucrunt. Na vtex Aftrgnomiaad Phyficam, fiue Cofmo-
gtaphiam deueniam,hz Copernici hypothefes non folum in Naturam.
rerum rion peccant,fedillam mu'to magisiuuant. Amat illa fimplicita-
tem,amatynitatem. Nunquam in ipfa quicquam otiofum autfuperflud
extitit:at fepius vnares multis ab illa deftinatur effectibus. Atqu: pencs
viitatas hypothefes orbium fingendorum finis nullus cft: penes Coper-
nicum plurimi motus cx pauciflimis fequuntur orbibus. Ve interim ta-
eam penetrationem orbiti Veneris & Mercurij,& alia, quibus antiqua
Aftronomia in tanta orbium fingendorum libertate ctiamnumlaborat.
Acq; fic Virifte né tantum naturam onerofailla & inutili fupelledtili tor

immenforum orbium liberauit : {ed infuper etiam inexhauftum nobis

thefaurum aperuicdiuiniffimorum ratiociniorum, detotius Mundi,'o-
mniumg;corpori pulcherrima aptitudine.Neq; dubito affirmare,quic-

quida poeriori Copernicus collegir, & vifu deméftrauit,mediantibus

Geometricis axiomatis,id omne vel ipfo Ariftot. tefte, fi viueret ( quod

frequenter optat Rheticus ) 3 priori nullis ambagibus demonftrart pof-

fe.Verum

Chapter I

agree precisely, as they do, (3) with the positions and mean motion of the Sun. (4)
On all these points, as a magnificent order is shown by Copernicus, the cause
must necessarily be found in it. Second, of those hypotheses which give a reliable
reason for the appearances, and agree with observation, Copernicus denies
nothing, but rather adopts and expounds them. For although he seems to have
altered a great deal in the customary hypotheses, in fact that is not the case. For it
can happen that the same conclusion follows from two suppositions which are
different in species, because they are both included in the same genus, and the
point in question is a consequence of the genus. Thus Ptolemy did not derive the
risings and settings of the stars from the proximate intermediate premise of the
same logical status, “Because the Earth is motionless at the midpoint.” Nor did
Copernicus derive the same conclusion from the intermediate premise, “Because
the Earth revolves at a distance from the midpoint.” For it was sufficient for each
of them to say (as both did) that those phenomena follow from the propositions
that there is a difference between the motions of the heavens and the Earth, and
that there is no sensible distance between the Earth and the midpoint in com-
parison with the fixed stars. Therefore what appearances Ptolemy did derive, he
did not derive from a false and accidental intermediate premise. His only breach
of the rules as such was that he believed the consequences which follow from the
genus to follow from the species.* Hence it is evident that Ptolemy derived from a
false arrangement of the universe what was nevertheless true, and in agreement
both with the heaven, and our own eyes, and that there is in that no ground for
suspecting anything of the same sort of the Copernican hypotheses. Rather the
point stands which was made at the outset, that Copernicus’s postulate cannot be
false, when so reliable an explanation of the appearances—an explanation
unknown to the ancients—is given by them, (5) insofar as it is given by them.*
This was seen by the highly successful Tycho Brahe, an astronomer beyond all
praise, who although he entirely disagreed with Copernicus on the position of the
Earth, yet retained from him the point which gives us the reasons for matters
hitherto not understood, that is, that the Sun is the center of the five planets. For
the proposition that the Sun is motionless at the center is a more restricted in-
termediate premise for the derivation of retrogressions, and the general proposi-
tion that the Sun is in the center of the five planets is sufficient.

Yet for Copernicus’s taking the species as the genus, and in addition setting the
Sun at the center of the universe, and the Earth in motion round it, there were other
reasons. For, to turn from astronomy to physics or cosmography, these hypotheses
of Copernicus not only do not offend against the Nature of things, but do much
more to assist her. She loves simplicity, she loves unity. Nothing ever exists in her
which is useless or superfluous, but more often she uses one cause for many effects.
Now under the customary hypotheses there is no end to the invention of circles, but
under Copernicus’s a great many motions follow from a few circles. For the mo-
ment I will not mention the interpenetration of the spheres of Venus and Mercury
and other points on which the ancient astronomy with its extreme freedom to invent
circles is still in difficulty. And so this great man has not only freed Nature from the
burdensome and useless paraphernalia of all those immense circles; but in addition
he has opened to us an inexhaustible treasury of calculations on the fitting together
of the whole universe and of all the bodies in it. Nor do I hesitate to affirm that
everything which Copernicus inferred a posteriori and derived from observations,
onthe basis of geometrical axioms, could be derived to the satisfaction of Aristotle,


fe. Vernmdehis omnibus fufius & pro dignitate pridem egit Rheti-
cinarratio , & Copernicus ipfe : & fi quid copiofius explicari potcft,
(6) aliusidloci & remporiserit , nuncactigiffe {afficit: vt ca mentio-
ne conftarct Iectori altera caufla , qua me in Copernici partes per-
traxerit.

Neque tamentemere, & fine grauiffima praceptoris mei Maftli-
niclariffimi Mathematici au€toritate,hancfeétam amplexus fum.Nam
is,ctfi primus mihidux & premottrator fuit,cum ad ala, rum przcipue
ad he philofophemata, atqueideoiure primo loco recenferi debuitfer:
tamen aliaquadam peculiariratione (7) tertiam mihicaufam prebuit
itafentiendi: dum Cometam anni 77. deprehendit, conftantilfime ad
motum Veicris a Copernico proditum moucri , & capta exaltitudine
fuperlunati conicétura in ipfoorbe Venerio Copernicano curriculum
{uum abfoluere.Quod fi quis fecum perpendat, quam facile falfum a fe-
ipf diflentiat,& econtra , quam conftanter verum vero confonct: non
iniuria maximum argumentum difpofitionis orbium Copernicanz vel
exhoc folo cerperit.

Frautemeaonmi, que ds hyporh.fibus verieque dixé, verifimeita fe habere deprebenda
anc bnstuencexplicationcan byporbsfiun Copernicisduacque tabulas ad boc facientes.

Pro cognofcendo ordine Sphararum Mundi fecundum Coper-
nici fententiam, intuere Tabulam primamin finehuius capitis, & que
ei adfripta func. (8) Terra prodiuerforefpe@utribuuntur 4 Coper-
nico motus quatuor (Copernicus breuitati intentus tres dicit, qui reue~
raquacuorfunt) qui omnes rcliquorum Planctarum motibus aliquam
apparentem varietatem conciliant. 2

Primus cftipfius Sphere feu Orbis, quitellurem ceuftellam circa
Solem annuatim circumagit.Atqsis orbis,cum fiteccentricus, (9) ec-
centricitateinfuper mutabili, (10) tripliciter nobis confideranduseft.
(m) Initio remotacccentricitate  Orbisigitur hic, motufque Terre
has commoditates praftat : quodnon indigemus tribus cccentricis in
vfitatis hypothefibus  {cilicet Solis, Vencris & Mercurij. Nam pro eo,
quod terra circa hos tres planetas circumuchitur, Terricolz cxiftimant
tresilloscitca {¢ immobilescircumuchi. Sic ex vno motutres faciunt.
Quod fi plures eflent ftellx intra orbem terre, pluribus etiam huncmo-
tumafcuberent. Cadunt etiamhocorbe potitotres magni epicycli,
Saturni, louis, & Martis,cumeorummotibus. Idquomodoaccidat,
in adiunétis parallelis, {chematibus videri poteft : Rurfum enim, quia
Terrain confpeétu Saturni (quafi quiefcentis, quia tardior eft) in orbe
fuo circumit, 4 Saturno recedens & accedens: exiftimantincolz, Sa-
_turnum in ¢picyclo fuo circumire,accedere,recedero, feveroin centro
orbis ‘ui quiefcere. Circulumigitur a B puranteffe cpicyclosg,i,l. Item
proptertelluris hunc cundem acceflum ad Planetas& receffum in orbe
fuo, v:dentur nobis ipfir quinque planctarum latitudines aliquam va-
rietatemacciperesquam librationem ve faluaret Prolemzus, neceff¢i-
pai fait quinge alios motus ftatuere: qui omnes, pofito vnico telluris
motucadunt.

Et quamuishiomnes motus,vndecim numero, ¢ mundo extermi-

nati


Chapter I

if he were alive (which Rheticus repeatedly wishes for), @ priori without any
evasions.” However, Rheticus’s Narratio and Copernicus himself have long since
dealt with all this on a broader scale and as its importance deserves. If amore exten-
sive exposition of any point is possible, (6) another place and time will do; here it is
sufficient to have touched on it, so that this mention will make clear to the reader
another reason for my having been completely converted to Copernicus’s side.

Yet I did not embrace this cause rashly, and without the very weighty support of
that famous mathematician, my teacher Maestlin. For, although as my first direc-
tor and guide, both generally and in these philosophical questions especially, he
should rightly have been placed at the head of this list, nevertheless by another par-
ticular argument (7) he furnished me with a third reason for accepting the theory
when he realized that the comet of the year ’77 moved in complete conformity with
the motion of Venus stated by Copernicus, and, by a conjecture drawn from its
altitude’s being greater than the Moon’s, that its whole path was in the actual sphere
of Venus.* Now on careful consideration of how easily the false disagrees with
itself, and on the other hand how reliably truth is consistent with truth, a very
strong argument for the Copernican arrangement of the spheres will quite rightly be
drawn from that fact alone.

To realize that everything which I have said about both hypotheses is absolutely
true, here are a brief exposition of the hypotheses of Copernicus, and two plates
to assist you.

To find the order of the spheres of the universe according to Copernicus’s theory,
look at the first plate at the end of this chapter, and what is written on it. (8) To the
Earth according to its various aspects four motions are attributed by Copernicus
(he himself, intent on brevity, says three, though in actual fact there are four) which
in combination reconcile an apparent variation with the motions of the remaining
planets.

The first is that of the sphere or circle itself, which carries the Earth like a star
round the Sun annually. Now this circle, being eccentric, and (9) furthermore with
a variable eccentricity, (10) has to be considered by us in three ways. (11) To start
with disregard the eccentricity. Then this circle, and the motion of the Earth, pro-
duce the following advantages, that we do not require three eccentric circles as in
the customary hypotheses, namely those of the Sun, Venus, and Mercury. For in-
stead of the Earth being carried round those three planets, the Earthdwellers
believe that those three are carried round themselves when they are motionless.
Thus out of one motion they make three. If there were more stars within the Earth’s
circle, they would also ascribe this motion to more. There also disappear, if this cir-
cle is assumed, three large epicycles, those of Saturn, Jupiter and Mars, together
with their motions. How that comes about can be seen in the attached parallel
diagrams; for again, because the Earth as seen from Saturn (taken as at rest,
because it is the slower) goes round in its circle, moving further from and nearer to
Saturn, its inhabitants believe that Saturn in its epicycle goes round, comes nearer,
goes further away, but they themselves are at rest at the center of its circle.
Therefore they think that the circle AB is the epicycles g, i, |. Further, on account of
the Earth’s coming nearer to the planets and going further away from them, as has
been mentioned, in its circle, the actual latitudes of the five planets seem to admit a
certain variation; and for Ptolemy to save this oscillation, it was necessary for him
to establish five other motions, all of which disappear if we assume a single motion

of the Earth.
And although all these motions, eleven in number, are banished from the universe

16 loan. Kerrert

nati fint, fabfticuto hoc vnice terrz motu: nihilominus adhuc aliarum
plurimarum rerum caufz redduntur,quas Prolemeus ex tam multis mo-
tibus reddere non potuit.

Nam primoa Prolomzo quzrri potuit, qui fiat,quodEccentricitres
Solis,V cneris & Mercurij habeant zquales reuolutiones? Refpondetur
enim,quodnon vere reuoluancur ipfi, fed proipfis vnicaterra. 2.Quare
quinque Planete fiuntretrogradi. Luminarianonitem? Refpondetur
primodeSole,quiais quiefcit: ynde fit, vt motusterrz ,qui femper dire-
dtus eft , ipfi Soli mere & imperturbate ineffe videatur, tantum per par-
tem oppofitam cerli. De Lunavero,quiamotus Terrz annuus, ipfius ca-
lo vere communis eft cum terra. (12) Duo autem quzhabenteun-
dem motum peromnia,videntur inter fe quiefcere. Vndemotus Terra
in Lunanon fentitur,vtin ceteris planetis.De fuperioribus Saturno,lo-
ues Martere{pondetur : Quiaipfi fine tardiores terra: & quiacitculus
& motus ifte Terre putaturipfisineffe. Quareficutillis,quiex x Saturni
globo profpicerent , Terrainterdum progredi videretur ; dum iret pet
medictatem paw fupra Solem:interdum regredi,dumirct per nap, {ta-
Fe vero inn & P'fic neceflceft,venobis ex tetra profpicientibus Sacurnus
volui videaturin partes oppofitas. Vedum eft terrain 8 N a, Saturnus vi-
dcturin b na alteriustabulz. Inferiores Venus & Mercurius ideo re-
gredividentur, quia fant velociores terra; vnde perindeacfiterraftaret
immota, Venus,currensin parte circuli remotioni, contrariam plane de-
{cribit viam illi,quam conficitin parte circuli fui vicinaterrz.

3. Ita queri potuit(fed nihil refpondente Ptolemao ) quarein ma-
gnis orbibus {int tam exigui cpicycli, & quarein paruis orbibus tam im-
manes: hoceft,quare wesded aégecrs Martis fic maior louia, & huius maior
quam Saturni? Etcurnon Mercurius etiam maiorem, quam Venus,ha-
beat,cum ficinferior Vencresfiquidem quatuor reliquorum femperinfe-
tiormaiorem habet? Hic facilis eft refponfio. Mercutij enim & Veneris
veros orbes,veteres cpicycloseffe putarunt. Mercuri autem,vt velocif-
fimi,minimus etiam orbis cft. Superiorum vero vecuique Telluris orbis
propioreft,ficmaioremad eum proportionem haber, & maior apparet.
Marsigitur proximus habet maximam zquationem, Saturnusaltiffinius
minimam. Nam fi oculus in ¢ confticueretur, ei orbis Px videre-
tur fubangulo ray. Acfiin x effet, idem orbis videreturfub angulo

4. Pariternon iniuria miratifune vetetes , cur tres fupetiores fem-
perin oppofitione cum Sole finthumilimi in fuo epicyclo , in coniun-
ionealtiffimi : vefi Terra, Sol & g fintin cadem linea,quare Mars tum
non pollitinalio loco epicycli effe,quam in y. In Copernicocaufa facile
redditur;sNon enim Mars in cpicyclo, fed terrain orbe fuo hancvarieta-
tem caufatur;Hine ficerra ex ins difcefferit,Sol eritinter ¢ Martem
& 8 Terram.Ettum Mats videbiturin Epicyclocx3in y afcendifle. At
Terrain a exiftente, quod eft punéum ipfic proximum: ¢ Mars & Sol
videbuntur ex a inuicem oppofiti. Atq; hac fine quzcxtabulaadocu-
lum demonftrari poffant.

Tam defnceps confideremus etiam eccentricitatem huius orbis.
(13) Copernicus tacit Apogzum Solis(vel Terrz)yt & czterorummo-

ueri,

Chapter I

by the substitution of this single motion of the Earth, nevertheless reasons are sup-
plied for a great many other matters for which Ptolemy for all his many motions
could give no reason.

For in the first place one might ask of Ptolemy how it comes about that the three ec-
centrics of the Sun, Venus, and Mercury have equal times of revolution? For the
answer is, that they themselves do not really revolve, but instead of them the Earth
does on its own. 2. Why do the five planets make retrogressions, whereas the
luminous stars do not? The answer is first, in the Sun’s case, that it is because it is at
rest; and the result is that the motion of the Earth, which is always in the same direc-
tion, seems straighforwardly and uninterruptedly to belong to the Sun itself, though
in the opposite direction with respect to the heaven. In the case of the Moon, how-
ever, as the Earth’s motion is annual, its own motion with respect to the heaven is in-
deed shared with the Earth: (12) two bodies which have the same motion in every way
seem to be at rest with respect to each other. Hence the motion of the Earth is not
observed in the Moon, as it is in the other planets. In the case of the superior planets,
Saturn, Jupiter, and Mars, the answer is: because they are slower than the Earth, and
the circle and motion of the Earth are imputed to them. Consequently, just as to
anyone looking from L (the globe of Saturn), the Earth would sometimes seem to be
moving forwards, so long as it was going by way of PBN above the Sun, and some-
times backwards, while it was going along NAP, but to stand still at N and P, in the
same way to us, looking from the Earth, Saturn must necessarily seem to be turning in
the opposite directions. Thus while the Earth ison BNA, Saturn seemsto be on bnain
the other plate. The inferior planets Venus and Mercury seem to move backwards
because they are faster than the Earth. Hence exactly as if the Earth were stationary,
Venus, passing along the further part of its circle, clearly describes a path in the op-
posite direction to that which it traces in the part of its circle which is next to the Earth.

3. Similarly one could ask (but with no answer from Ptolemy) why in the large
circles the epicycles are so tiny, and why in the small circles they are so huge; that is,
why the correction’ for Marsis larger than that for Jupiter, and for Jupiter larger than
for Saturn? And why Mercury does not have an even larger correction than Venus,
since it is lower, seeing that among the other four planets the lower one always has the
larger correction? Here the answer is easy. For in the case of Mercury and Venus the
ancients thought that the true circles were epicycles. But Mercury’s circle, although it
is the fastest planet, is also the smallest. However, in the case of the superior planets,
the nearer the Earth’s circle is to each of them, the greater the ratio of it to the Earth’s
circle, and the larger it appears. Consequently Mars, the nearest, has the largest cor-
recting factor, and Saturn, the highest, has the smallest. For if the eye were situated at
G, the circle PN would appear to it to be subtended by the angle TGV. But if it were at
L, the same circle would then appear to be subtended by the angle RLS.

4. Similarly the ancients rightly wondered why the three superior planets are always
in opposition to the Sun when they are at the bottom of their epicycles, but in con-
junction when they are at the top; for example if the Earth, the Sun, and g are in the
same line, why Mars cannot be at any other point in its epicycle but at. In Coper-
nicus’s theory the reason is easily supplied. For it is not Mars on an epicycle but the
Earth on its own circle which causes this variation. Thus if the Earth moves from A to
B, the Sun will be between Mars at G and the Earth at B. And at that point Mars will
seem to have climbed up on its epicycle from 5 to y. But when the Earth is at A,
which is the point nearest to G, Mars at G and the Sun will seem from A to be in op-
position to each other. These are the points which can be demonstrated from the
diagram at sight.


ucri,né per deferentes, fed perepicyclium paulo tardiusorbe {uo ad ini-
tium rediens. Hicmotus Apogaictiam aliquid infere in motibus cate-
rorum Planctarum. Nam Prolemzuscaterorum eccentricitates com-
putataccntro terra ; quod ficentrum Eccentrici Telluris & Apogeum.

er confequentiam fignorum difcefferint in aliam partem Zodiaci ,ré-
itis poft fealiorum A pogais tardioribus ; accidctaliqua mutatio eccé-
tricitatum in planctis cgteris.Hoc valde rurfium mirabitur Prolemai A-
ftronomia, arque ad confingendos nouvs orbes confugiets quibus de-
monftretsh2c ita ficri poffe , cum tamé cx motu Telluris ynico fecutura
fint. Atq; hoc quidem multa poft fecula vix demii fiet, (ed tertio (15) mu-
tatio cccentricitatis terrene,qua centrum eccentriciad Solem accedit,
&abco recedit, indea Prolemzo ad nos v{que magnum quid in Marte
& Vencreintulit: quorum eccentricitates cum mutate videancur , quid
Prolemaum diéturum putas?Nunquid rurfum nouoscirculosin cgtero-
rum infinitam turbam afcifceret, fi viueret? quibus omnibus in Coper-
nico opus minimecft. Hc tot & tanta Copernicus per vnius circuli aB
pofitionem & motum preftitit-vnde merito,quamuis exiguus effet, Ma-
cNocognomendedit. Hic primus motus cwlo Lunecum Tellurecom-
munis fuit.

Tam porro videamus, quid reliqui motus telluris efficiant;quiacci-
duntintraillum Lunz orbiculumad a.

Secundus igitur morus non integriorbis, fed (16) grbiculi coele-
ftis,terrz globum proxime ceu nucleum includentis, tendit in oppofitia
ab ortuin occafum, perinde vt cpicycliafuperiorum, quibus eorum ec-
centricitasfaluatur a Copernico. Huius annua coftiutionefit,ve equi
nottialis {emperin candem mundi partemdeclinet. Poli enim AEqui-
noftialis fine corporis ab huius polis per 23. gradu cum dimidio, diftant.
Qui motus cum pauxillo velocior fitmotu annuo orbis magni, facit fe-
tiones circulorum, fiue (17) xquinoctiorum loca paulatimin prece-
dentia moucri. Quareper huncexiguum globulum cadic illa monftro-
fa,ingens, 2es Nona Sphara Alphonfinorum,vt cuius officiumin il-
lum orbiculum anteaneceflarium translarum eft.Cadit etiam motus de-
ferentium Apogzcum Veneris , vequod non aliter mouetur, nififi fixe
moueriftatuancur.

(18) Tertius motus eft Polorum globi terreni,conftans
duabuslibrationibus, quarum vna eftaltera duplo celerior,&
adreétos angulos.Is adminiftratur per quatuor circulos, fic vt
bini circuli fingulas librationes faciant , & librationesipfe
permixtz corolla intortz {pecicm prabeant, in huncmo-
dum: Vnalibratio inColuro folftitiorum fir, &faluat variatio-
nem declinationis Zodiaci, {ero poft Prolemei téporaanim-
aduerfam: tale quid & Prolemzo opus fuiflet confingere , & nonnulli
moderni, vndecimo Mundi orbeiam confiéte, praftareconati funt.

Akera libratio , quz fit in coluro Zquino€iorum , faluat inz-
qualem preceffionem /Equinogtiorum , & eliminat oftauz fixarum

Sphara, qua vicima eft apud Copernicum , motum trepidationis,
illique quictem fuam reftituit, Atque ne non & hic motus aliquid
in catcris motibus focneretur : collit irregularicatem motus , quem

c omaium


Chapter I

Next let us take into account also the eccentricity of this circle. (13) Copernicus
makes the apogee of the Sun (or of the Earth) move, like those of the other plan-
ets, not along deferents, but along a small epicycle which returns to its starting
point a little more slowly than on its own deferent circle. This motion of the
apogee also has some effect on the motions of other planets. For Ptolemy (14)
computes the eccentricities of the others from the center of the Earth; but if the
center of the Earth’s eccentric and its apogee shifts eastwards to another part of
the zodiac, leaving behind them the slower apogees of the others, some change in
the eccentricities of the other planets will result. Again the astronomy of Ptolemy
will greatly wonder at this and will take refuge in inventing new circles by which to
demonstrate that it is possible, whereas it will follow from a single motion of the
Earth. That indeed will only just come about after many centuries; but thirdly,
(15) a change in the Earth’s eccentricity, by which the center of its eccentric moves
closer to the Sun and moves further away from it, between Ptolemy's time and
our own has had a great effect on Mars and Venus; and when their eccentricities
seem to be changed, what do you think Ptolemy would say? Would he again ad-
mit new circles to the infinite crowd of others, if he was alive? All of which are
scarcely needed in Copernicus. All these great and numerous phenomena Coper-
nicus accounted for by the location and motion of the single circle AB, so that it
was proper that he gave it, although it was tiny, the title of Great.!° This first mo-
tion with respect to the heaven was common to the Moon and the Earth.

Now let us go on to see the effects of the remaining motions of the Earth,
which take place within the little circle of the Moon at A.

The second motion, then, which is not a motion of the entire circle but only of
(16) the little heavenly circle which closely enfolds the Earth’s globe like a kernel,
is in the opposite direction, from east to west, like the epicycles of the superior
planets, by which their eccentricity is saved by Copernicus. The result of its an-
nual occurrence is that the equator always slopes towards the same point in the
universe. For the poles of the equator or of the actual globe are 23'4° from the
poles of this circle. This motion, being very slightly faster than the annual motion
of the Earth’s orbital circle, makes the intersections of the circles, that is, (17) the
positions of the equinoxes, move gradually westwards. Hence this tiny little globe
does away with that vast, monstrous, starless ninth sphere of the Alfonsine com-
pilers, as what used to be its essential function has been transferred to this little
orbit. The motion of the circles which carry round the apogee of Venus also
disappears, unless the fixed stars are deemed to move.

(18) The third motion is that of the poles of the terrestrial globe, consisting of
two oscillations, one of which is twice as rapid as the other, and at right angles to
it.2 It is accomplished by means of four circles, in such a way that each oscilla-
tion is produced by two circles, and the oscillations themselves combine together
to form the shape of an interwoven garland, in the following manner: one oscilla-
tion is on the colure of the solstices, and saves the variation in the declination of
the zodiac, which was noticed late after the time of Ptolemy, and is something
which Ptolemy would have needed to invent, and several moderns have tried to
represent by inventing an eleventh circle of the universe.

The other oscillation, which is on the colure of the equinoxes, saves the ir-
regular precession of the equinoxes, and eliminates the motion of trepidation of
the eighth sphere, that of the fixed stars, which is the last according to Coper-
nicus, and restores it to rest. And to make sure this motion also contributes


omtrun ‘eptex *lancfarum, a Apogzoram motus habere debuif-
fenc(né fine minilterio aliquotnovorum. citculorum) quia compertum
eft omnes motus aqualiter per fixas incedcre. .
Quartus denique motus eft ipfius globi terreni & circumfufiaeris
ptoprius,cuius periodus eft 24-horarumin candem mundiplagam cum
cxterissnempe ab occafuin ortum : propter qué totus mundus reliquus
ab ortuin occafum, imperturbatis magno miraculo motibus fecundis
ferri putatur. Cadicigicur illaincredibiliter ala & pernix decima Sphz-
ra dvaspos,cuius & totius midi tanta effetin Prolemao pernicitas, vtvno
niu oculialiquot millia milliarium tranfirent. Acquafote,ad tabel-
Jam refpicias, & cogites,quod tellus hac noftra ,decuius motu difputa-
tur,exigui circellilunaris ad a , feptuageimam vix demum partem dia-
metrixquet: Ab hoccircello deinad$ :urni amphtudinem, &abhac
ad fixarum inzftimabilemaltitudineme ulosintende, & deniquecon-
clude, verum fatucredituque facilius, punétulum illud :ntra a circel-
lum,& fictellurem in vnam plagamrotari,an verototum mundum de-
cem diftinétis motibus(quia decem ab inuicem foluti orbes) infandara-
pidirate ire in plagam altcram, nec quoquam, nifi adiilud pun@ulum,
telluris imagunculam,eamque folam immobilem, refpicere , quia extra
nihileft.

Huc pertinet Tabella prima & fecunda.

Nora Auéoris.

(a) ty cempeftinum.] Occtrrit hyic ferupule Copernicusipfé in prafatione ad Pauluin Ter

tintin Pontif. Maxim. fed paulo rigidin{cule: cutns orarionis panasluit denique, plusquans
70.annis ab editione ibri,, aque morte fua elapfis: fafpentuscnim eft inguitcenfura ,donec
Cottigatur, opinor autem, etiam hoc fubinteligi, donc explicetur. Quomedo enin non fit
feriprurecontrarins , quippein ropofito longifime dinerfo, conatusfumoftendsrerationibus Cr exe
emplis in Introductione in Commentaria demotibus Martis. Ipfius etiam Copernici verbaex=
plicaui dilucidius in finelibri 1. Epitomes Aftronomie: quibus locts fBero religiofis fatisfxdtum iri:
dummodo Cr ingeniam &> cognitionem Aflronomietalem ad boc indicinm afferant, ve gloria diai~
norunn operitm vifibilium , ipforuin patrociniotuto credipofit. Eft{ane aliqua lingua Dei, fed eff,
etiam aliquis digitus Dei. Ee quisneget linguain Deieffe, attemperatam C propojite {uo , c ob id,
lingua populart, hominum? In rebus igatur enidetifiimis torquere Deilinguam,vtilladigitum Dein
natura refuter, id religiofifimus quifque maxime cauebit. Legat,cui cure fut Laudes Creatoris
Domini noftri,legat,inquam,librum mesin qaintum Harmonicorum: c perceptamotuum politia
exquifitifime Harmonica , deliberce fecum, fatiw infle, fatin’ pregnantes cauffe fuerint quefite
conciliationis inter linguam Cr digitwm Dei: anne expediat, ea conciliatione repudiata , famam
banc Operum disivorum pulcbritudinisimmenfe,cenfuris opprimere que fama vt ad rudupopn-
i: quinimo, vt ad vulgi literatorm notitiam vel kuemperueniat , nullis vnquam imperii efick
pofit. Renuit in{citia refpicerein audToritatem,adpugnam vitro prouolat , freta mutitndine , O°
feato confiuetudinis,telis veritatis impenetrabili.

Acies verodolabrein ferrum illifa, poftea necinlignum valet amplins, Capiathoc cuinsin-
rere,

(2) Atquehocloco.} Exndeminflsntiansinparticulari etiam hypotheiuccentricita~

tis Aifeafit in Commentariis Martis cap.21. Oftendsque , quade caufs & quatenus fale hyporhefis
interdum verum prodate

(3) Cumloco & motu Solis medio.] Nondui fiiebans, quod poffeain Comment.
Nani


Chapter I

something to the other motions, it removes an unevenness of movement which
the motions of all the seven planets, and of their apogees as well, would have had
to have (not without the services of some new circles) because it has been found
that all their motions are regular relative to the fixed stars.

Lastly, the fourth motion is that of the terrestrial globe itself and of the at-
mosphere which immediately surrounds it. Its period is twenty-four hours and its
direction with respect to the universe is the same as the rest, that is from west to
east, and because of it the whole of the remainder of the universe is thought to be
carried along from east to west, its secondary motions being by a great miracle
undisturbed. Consequently there disappears that incredibly lofty and swift tenth
and starless sphere, the swiftness of which and of the whole universe would be se
great according to Ptolemy that it would traverse several thousand miles in the
blinking of an eye. I ask you to look at the plate and consider that this Earth of
ours, the motion of which is in dispute, scarcely equals a seventieth part of the
tiny little lunar circle at A. Next turn your eyes from that little circle to the
spaciousness of Saturn’s, and from that to the incalculable loftiness of the fixed
stars; and finally decide whether it is easier for it to happen and to be believed
that that small point within the little circle A, and hence the Earth, rotate in one
direction, or that the complete universe goes with ten distinct motions (as there
are ten mutually independent circles) with inconceivable rapidity, and is subject
to nothing but that small point, which alone is motionless, because there is
nothing outside.

Here belong Plates I and I.


(1) Premature.| Copernicus himself faces this doubt in his preface to Pope Paul III, but with a little

too much inflexibility. His discourse finally paid the penalty, after more than seventy years had elapsed
since the publication of his book, and his own death; for “it is suspended,” said the censorship, “until it is
corrected”; though I think “until itis explained” should also be read between the lines. For I have tried to
show with arguments and examples the way in which it is not contrary to Scripture, admittedly with a
greatly different intention, in the introduction to the Commentaries on the Motions of Mars. Also I have
explained Copernicus’s own words more clearly at the end of Book | of the Epitome of Astronomy. In
those passages I hope to have satisfied those with religious scruples, provided that they approach their
decision on this point with sufficient intelligence and knowledge of astronomy for the glory of God’s
works, which are themselves visible, to be safely entrusted to their protection. Certainly God has a
tongue, but he also has a finger. And who would deny that the tongue of God is adjusted both to his in-
tention, and on that account to the common tongue of men? Therefore in matters which are quite plain
everyone with strong religious scruples will take the greatest care not to twist the tongue of God so that it
refutes the finger of God in Nature. Let him read, if any man is concerned for the praises of our Creator
and Lord, let him read, I say, the fifth book of my Harmonice; and when he has perceived the most
skillfully Harmonized Republic” of the motions, let him debate with himself whether sufficiently sound,
sufficiently prolific reasons have been discovered for reconciling the tongue and the finger of God; or
whether he will repudiate that reconciliation and hasten to suppress with censorship the renown of the
immeasurable splendor of the works of God. That this renown should come to be known to the common
People, nay rather, to the generality of the even superficially educated, could never be brought about by
order. Ignorance refuses to respect authority, it resorts spontaneously to combat, relying on numbers and
on the shield of habit, which is impenetrable to the weapons of truth.

Truly once the edge of an axe has been blunted on iron, it can no longer cut wood

Let those who care understand.

2) On this point.) 1 have also discussed this instance, with reference to the particular hypothesis of ec-
centricity, in the Commentaries on Mars, Chapter 21, and have shown for what reason and to what ex-
tent a false hypothesis sometimes reveals the truth."

(3) With the position and mean motion of the Sun.) 1 did not yet know what I afterwards

1zz ‘d 295,

umuBeyc uso prinsryrinl finan rind y Z +L 59 maze refer mys:
PAE YZ Durning vargy axrqrandyamur XXPURY XAT angf ngayny quf mumcasunfena Wb 1 Dy STV YeNgO 2 IG Ad ut
cunays suvargdf pr moar § XL98"S TA mf “oop many Se 9 4G uae g Pr oLomeRsL radesd ms pos
HAD mr arsayds pe nxrprardsudypengio wapfoyy AT ds sruanpeasurx yf worgey mrpura pr sagas un? unisrd vue sad nsom mrpes *y 9

‘SIV NAT HAF HAS ‘mpbrdrmpampogee yruampl rut

-ahyim unfe anded ‘SANDY W STELO and “anrpnb
So tumunp omens mins ys SLA TTS Sg V anninbsf ny
“afvquane 6 mesap
ruses umpun nap momen mI TD SICA NTA maby sot
mop anpusb rixaovanyymosunusnnmann sa
“migeneus TOS pasted rateasear wT

*prurodog urenusiugy umpans
-2y‘sTINT]D | OUSeyyaqzo ur boa alae a sopnSue wos :sepueyrp

seny setipour eaXnr wma sturpnaruSeur ussuorzodosd wes anbynuny : umypqour wn AyoP25

iy

saved
wdes you

wnat
Mout gananuunnpesS nippy
‘pdsinges mguuabenr ema
sophasdeD nynp v12130 199
24 Ssraar gurssefornen fap sip
outs sy fob d oynsais
oataanuos uy poGida © aay

id) samen 9 108TAS

“sunmnyfrsranporiad'rgand
sunsoufuogdyVvixns ‘syne
“aonb ayris x1 unr o8 rua
suanves 9 nunbra ‘surxead
ern sonny “<1 soy annua
seponuurte nase ney nenp
i gpayed Yb oudca oina
aqyesfop soxSayus wanasuas 199
suacvsava vous son § srsivuava

pete reauy raags bb.

repaadul waniay eansonbyey

“sagen awesfamutes

ssapignyoa aworfurnaen
saute srauv'vargds $1708
snfid OS 1MANT Amid
oof wash : rrpauny s1949 31h
Trou znnrausoad sings

pos
noua snsora ns © awausfrad
waa uavaryds IV NAT mp

“psgorues

weld
-uraauagy wumoz94

ppunoay senueytp serp
ou eaxny  wiopmso9 103)0
~maoyyond siaze 1 01
enduvonbse “uinioj9ho1do
2 wmigzo wauomodoxd
‘bunoaazy wunryo}209 wns
-xg3ydy waurpso susqyxg


x *
XX oe xe FR


Martis d:mouftrs1i; Anomaliam orbis Magni}. Commutationis , qua Kerrogradatigiles Caujarny,
roftitui ad ipfum verwam Solis morum C locum. Id veroinveters aftrononns forma unmet albuc
magis murat coguntur quieamretinent. Adeoqueex boc ipfo,demonflratina nafcuntur arguanen-
t4,retrogradationes non orvri ah aliguo motu rcal,vel planctarumn,rel rorius Syftematis calsflis fed
ex motu Telluris vnice per imagmationem m Planetasonmes transfer,

(4) Porro fententia fequens, Quebus onnib. Grc.oftitanter eitconceptasbocenim dicerey ole~
bamscsina on Copernico apparcat ordopulcherrimnus qitalis ft inter caunfem Cr fiuos effects; neceffe
effet hac ipfafit vera cafe retrogradationsan,quain Copernicus dscit; vt fe. Hypathefis ifta non fit
fictitia tantum.

(5) Quatenuses illisredditur.] Quia,vt iam feyuitur non vrex fpecialiearum con
fornatione, fed vt ex gencrali,queeit Copernico cum Brabzo communts, quorundain casifareddt=
tursat nonillorun tasnen cauff.cex is tex fpeciali Copernicivedditur:item, Quiafiparticularifi~
nas Hypothefiums Copernics couditiones dumenfionefqucrefpictamiss caffe minutiarun nonnulla~
rumin Obfernationthusreddi non potuit : eaquede cauff Copernican Bypotlefes circa particulas
ria tamquoad forma, quam quoad dimenfiones , ims corrigi ad prs{uriptum Obfervatio ium de
buerunt. Fafiqaein forma dixicmendanda finffeilla tals fune pve ad perf Gionem porous Mypothe-
fim Copernica hoce dyad longiorem earum deductioncm ab vfitata via fpedentquariad nonam ax
Legian conforinattoncms;t-a vt inCommentariis Martis alicubi dist, Copernicus fitaraan ipfe dini~
tiarum fut ignaris.

(6) Alius id loci & temporiserit.] Potifina huius opere moles, quod Affronsmica
attinet,in Comimentarta Martis incubuit; in Phyficis vero feu Metaphy/fcis arguinentis conregandis
Sufior furn in Epicommes Aftr teh IV qua liber ipfim 7 «¢2v coutince, gited hoc loco fin pellicats
detorum,

(7) Tertiam mihi caufam ita fentiendi.} Idem tamen mevlero admounit paflea,
noneffe neciffariam banc collect oxem, Nan cum Con:eta mnoturn nonin multos dies conzinuet ,
‘uont habeaanus ibertatem intendendiremittendive cius motumnin Hypothefifufcepta, vbi obferua-
riones(qute plerumgue craffe funt) idrequirere videntur , binc adso fit, date inparticularibus
Hyporheles, eaftem Comete ob ervationes reprefentent.Et Braheus librode Cometisfol.282, Maftli=
ling Hyporbefia examinat,cum fits comparatredarguitque, Ipfevero ,fol..06. Hypothefintalem
proponut,in que notus Comete proprius circularis initio tardus, nox intenfus, in fine varfumtare
dusexbibstur. Uague edo hoe genere argument, fic quidem informatiy vt ex co quod poruerunt ar
tifices preflare, nudacredulitate, &> generaliilla conicélura, quod verum veroconfovet, de veritate
Sippofitronui quid prafimatur. at vicifim alia viaeandesn arccmrurfiam occupo. Si enim motus
telluris ad hoc vtiliseSt,vt Comeraruns motus rectilineieorungue perpetua, vel equabilitas, vel au
ginentatioaut contraria diminutio perpetua, fatisf ciant objaruatis, tuncfane, quantum eft verifie
amiletudinisin motu redilineo squabils, corporim vanefcentium; tantum fidei accedet motui Tella
ris prsfertion fiflexusitiaerum apparentium rrregularcsaccafione motus Telluris proueniffe conflet,
alarumque,qua in Comctis apparent,ratio reddatur.V erbi cauffaille ipfe Cometa anni1s 77. ortus
ex vltisnispartibus fagittarg, maximum ibi motum diurnum,caput 7.aninutorums ,caudamn 22.87.
longamexhibuit; becomntsfucrunt diminuta verfusfinem , adeo vet in figna Pifeiums ,quod qua-
drantediflar 3 fgittario,ftationem peradturusvideretur, nif eusnuiffet. Queritur que cauffa, cur
Comete circa qucadratuan eins lociin quo maximi apparucruut co velocifuma , appropinquent fla-
toni,cur ftationi vicini,al occultentur fib Solem,v¢ iffe,alijetiawn in oppofito folu paulatim enane=
Seane, ve illeanni 1618 faciunt enim iffaplerique. Quod fipotiaris libertate circularis motus Come=
tatribuendi,cauffizm per omnes Cometaseandem dicere nonpoteris, At fi teipfiam redigas ad angu=
fis traiectionis rectilinee fatim apparce necefitss phsnomeni tage planum traicétcrium Come-
teanni1s77.c9a0rdinaffem ines linea, per qu.tin paucispott difparitionem dicbus , videndus fuilfer
Caiffelongitudinis fi fuperfuilfe : traicétionemipfam primuon velocem, in fubeuntes diestardio~
rem fecifem , idque pro ratione propinguitatis partiuun, traieéloria linca ad Solem , quia Com
meta via obliqua fusiebat a Sole , Tcllusquefimul.iCometa, Qua ratione cfficiebatur,vt Cometa
initio quidem dimndiam folis altitudinem baberet ex co fpharas Veneris,Telluris, Martistraiiceret,
Cin finepls quam triplo altior Sole euadcrce.Non mirum igitur, quod parallaxisinco nulla depre-
hendepotuit, Sed de hac re plus fatis: hoc loco plurs fipetit ledtor , adeat meuun de Comctislibellam,
quem nundinis Autamnalibus anni 1619 .emifi,

Cc. (8) Ter-

Chapter I

demonstrated in the Commentaries on Mars, that the anomaly of the Great Orbit or of the parallax,
which causes the retrogressions, is restored to the actual true motion and position of the Sun. Those who
keep to the old form of astronomy must find this fact much more surprising in that form, Consequently
this fact in itself gives birth to compelling arguments that the retrogressions do not arise from some real
motion either of the planets or of the whole system of the heavens, but that from the motion of the Earth
alone they are transferred by imagination to all the planets.

(4) Furthermore I was nodding when I composed the following sentence, On all these points, etc., for I
meant to say, that as a magnificent order is shown by Copernicus, as there is in the relationship between a
‘cause and its effects, the true cause of the retrogressions must necessarily be this very fact, as Copernicus
says, that is, this hypothesis is not merely a fiction.

(5) Insofar as it is given by them.] Because, as now follows, the explanation of certain points is given,
not from this special conformation of the planets, but from the general one, which is common to Coper-
nicus and Brahe; but nevertheless the explanation of some points is given by them from Copernicus’s
special one, Further, because if we consider the particular detailed conditions and dimensions of Coper-
nicus's hypotheses, no explanation could be given of some minor points in the observations; and for the
sake of that explanation the Copernican hypotheses had to be corrected by me on particular points as dic-
tated by the observations, both with respect to the arrangement and with respect to the dimensions.
However, the points which I have said had to be emended in the arrangement were such that they con-
tribute rather to the perfection of the hypotheses of Copernicus, that is, to drawing them further from
the traditional path, than to some new conformation, since as I have said elsewhere in the Commentaries
‘on Mars, Copernicus was himself unaware of his own riches."*

(6) Another place and time will do.| The chief burden of this task, as far as it concerns astronomy, fell
fon the Commentaries on Mars: but I have assembled a more copious collection of the physical or
‘metaphysical arguments in Book IV of the Epitome of Astronomy. That book contains the actual oeuvre
which I promised in the present passage. See the whole,

() He furnished me with a third reason for accepting the theory.| However the same person later in-
formed me of his own accord that this argument is not necessary. For since a comet does not persist in its
motion for many days, and since we are at liberty to intensify or relax its motion in the hypothesis
adopted, when the observations (which are frequently rough) seem to require it, it therefore follows that
hypotheses which differ in detail represent the same observations of a comet. And Brahe in his book
about comets,'* page 282, examines Maestlin’s hypothesis, compares it with his own, and refutes it. On
the other hand on page 206 he proposes a hypothesis such that according to it the proper circular motion
of the comet is shown as slow to start with, then intensified, and finally slow again. I therefore abandon
this kind of argument, which indeed is so constructed that it leads to the presumption, by sheer credulity
and the general conjecture that truth is consistent with truth, that what the practitioners have been able
toadduce has some bearing on the truth of the suppositions. However I again capture the same citadel by
another route instead. For if the motion of the Earth has the useful result that taking the motions of the
comets as rectilinear, and either constantly regular or increasing, or on the contrary constantly
diminishing, will satisfy the observations, then plainly just as much credence will be given to the motion
of the Earth as there is probability in attributing regular rectilinear motion to bodies which disappear,
especially if it is accepted that the irregular shifts in their apparent paths are the effect of the Earth’s mo-
tion, and an explanation is given of the other appearances observed in comets. For example, the comet of
the year 1577 itself when it rose in the furthest region of Sagittarius showed its greatest daily motion
there, a head of 7' and a tail 22° long. All these were reduced towards the end, so much so that in the sign
of Pisces, which is a quadrant away from Sagittarius, it would have seemed almost to have reached a sta-
tionary point, if it had not disappeared. The reason is therefore required why comets, at about a
quadrant from the place where they have appeared to be largest and fastest, approach a stationary point,
why when they are close to a stationary point some are hidden by nearness to the Sun, as that one was;
while others disappear little by little in opposition to the Sun, as did that of the year 1618. For this is what
they mostly do. But if you grant yourself the liberty of ascribing circular motions toa comet, you cannot
speak of the same explanation for all comets. But if you confine yourself to the restrictions of a rec-
tilinear path, the necessity of the phenomenon is at once apparent. Thus I would have placed the rec-
tilinear path of the comet of the year 1577 on the line on which it would have been visible a few days after
its disappearance, if it had survived. Its actual passage I would have made first rapid, for the succeeding
days slower, and that in proportion to the nearness of the regions where the line of its path lay to the Sun,
because the comet was moving away from the Sun at an angle, and the Earth at the same time from the
comet, For that reason the result was that the comet, to start with, had half the altitude of the Sun; after
that it traversed the spheres of Venus, the Earth, and Mars; and in the end it finished by being more than
three times higher than the Sun. It is therefore not surprising that no parallax could be detected in its
case. But that is more than enough on this point: if the reader wants more on the subject, he should go to

le book on the comets," which I published at Michaelmas 1619,

20 loan. Kerrert

(8) Terretribuuntur motusquatuor.] Scribendocgoidscmporis athe dulicis ne
perturberisigitur muiltitudive ifs motusm:proprie duo tantum fin, vnues ab rterropendenspria~
Gipio,conuoktionis djurne,circa proprium centrum, alter extovnfeeus + Sole Telluriillatus, anit
circa Solem, etfi moderaturilluon formatque vis magneticasfibris Telurisinfita, qui vero tectins hie
cenfebatur ile quies et potins axis Tellurisin fitup.rallelo,dum exntrusn carca Solem firtur , Cr qui
quarts bic ventitatur, i eft lenicuta perturbstio buins quietis , orta ex aberratione ducrusu primo-
ruin C> prepriorum.Sed debis infra plura,

(9) Eccentricitateinfuper mutabili.) oc coadi funt flatuere auctores ,cxteri de Sole,
Copernicus de Tell:sre , quia niminms trabuunt Obferuationtbus Hipparcht G Prolemai:fed que non
funttanta fubsilitatis, vt dogna tantimomenti poft iis fuperadificari. Itaque ns Commentariis
‘Martialtuin fpeculatiozum, cy lub. V 1 Epitontes parte 1. opinionem iftam, vt £ hyfice caleft inion
Cam adinodum.fidenterreiect, nc din cedo featentia: enidenters imbecillatern opinationts hints
alibidemonftrabo.

(10) Tripliciter nobis confiderandus.] Non quod triplex ipfefits ed quis vow
idem exiftens tria diftndha habes,qus fingula fuos mulsiplices rfus & munia babent in Aftrononia
reformata,

(a1) Initio remotacccentricitate.] Id eff, fepofita confideratione Eccentricitatis. Que
dans emmpraffat orbisifte Ecccntricus,non ipfa fua Eccentrucitate, fed ila fola re , quod circa Solum
vertitur.

(12) Duoautem.] CalimputaLune(non Luna per fe) ce Tellus ,babent eundem mo-
tion tyanjlationis de loco in Locum per Orbe mnagniin,ergo ewan fewper Terralaco codem fit,
loco amneniter & calum Lung; Calum igitar boc Lune, C per td, Luna ipfa, canfa qirdera cas fit,
nitllamealen ex motu Telluris apparentiam firjeipit motus fru , quavcin ex Terra tranflatione , Sol
Lefupit, ipfeverequiefeens, 1d fecus effet, fi Terra promora, Cerlusn Luncquicfveret, aut moncre-
sur deloco m loci, moti alto dyftincto: tunc enim motus centri Telluris per imaginationent etiam
calm Lune tranfiribsrotur, C>ficetiars totum caluan Line, prorationefitusfuipofitrstrogra-
dum videri,non minns,quam Planets quinque,

(13) Copernicus facit Apogzum Solis. Duo bicinatuntur, alrerum Solemi-
Pfunattiner, alt.rum cx Soleredundatin Planctas, Prolvimeus Solenicollocat in Eccentrico , Ee~
centricum ixcludit ducbus dcfercrtibus: Copernictes Epicyclo affigit planctam, Epicycluin concentrie
Co. Prolemeusigitur, ve Aponea promoucat, D-fercntibusfiis actribait mote petultarcon tarde
finuns, Copernicus ids preftat, per aberrationcm veftitutionis Epicyels areflituti
cuinfitveraqueannuafire. Verifimilius autem eff, motusillostardos,ex aberraticse eff,
motu pofitine, Prafertim can prcyclo motusannuns, tantum refpeilu Eccentric fi
Gircumaito Epicyclusfé cuoluat on plagem contrarian s at reffettu fivarusn , quivtpotivs fpecicm
pre fe fer 5 ques im hac cuolutione jit vt eaidcm Epicyuli partes ifdews fixarum placie tamper ob=
aertantur nifi quantiluin turbat aberratio.Ezo veroin Commentaviis Martis,crin Epit, Apr.libre
IV. cauffain trado phyficam, tam Eccentricitatis,quam traiifpofitionis Apogecrum, que canfsine
(fits eft fibres corporis planeta,nccidiget,veldt-furentibus,vel Epicyclis, Sed hac membrum, Solem
‘pfiums (feu Terran: )attinens , mtcllige obiterfalteminculeatum 5 vtex coiamoficudatur, quidex
Apigsci Solus tranjpofitionc redundet in Planctas cateros,

(14) Eccentricitates computar 4 centro Terra] Hecdilzcidiora firnt per intuie
tum Tabule V. Est quidem hoc raticinart,diccre quid post multa fecula fit fururim, cum ifleforu-
pulus de prafeitivondum vrgeat Affronomiam veterem, Sed fic ccmpasatum eftcxau tranffumptic
ne placitorum Prolemai partictlarinsn in Hypothsfin Copernic; vt non potuerit 2 meemitti mentia
iff, Nam etiam Copernicus Escentricitates quinque planct arin ccmputauit velut 4 Centro Orbis
nagns :quafiillad/ non vero ipfim Solis centrum vicinifimum) fic genuina bafi Syflematis plane>
tary. Per hos vero xs.anios ex quo libellura bunc edidi, fic eft a me conflituta Affronomia, vt Ec~
entricitates onnes{ primariorum Planetarum) ad ipfifimum Solis centrum,cen reramMundi bafin
referautier, Itaque mancre poffunt Eccentricitates Planctarum oumiuin,quorfumeungue fercciprat
Apagetm Solis.Vide in Martialsbus Commentariispartem primam de aquipoll:ntia Hyporb:fium,
prssertim Caput VL.

(15 Mutatio Eccentricitatisterrena.] Hee ¢x admonitione ipfius Copernici tranferi=
pta funt.Erverum eft, qué Centrum Orbis Solis a Tellure( vel Telluris a Sole) nimium dimouct, vt
Seosffecontcndo Prolemsum G Mipparchum ; is fi Planetarum Eccentricitates ad boc pu Qura refrty

alias

Chapter I

(8) To the Earth... four motions are attributed.] From writing this at that time I have continued to
learn. Do not, therefore, be perplexed by this multiplicity of motions. Properly speaking there are only
two, one depending on the internal origin of the daily revolution about its own center, the other the annual
motion round the Sun conveyed from outside by the Sun to the Earth, though the latter is controlled and
shaped by the magnetic power residing in the bowels of the Earth. The motion which was here reckoned as
the third is rather the immobility of the Earth’s axis in a parallel position, while its center is carried round,
the Sun; and the one which is here held out as the fourth is the slight disturbance of this immobility which
arises from the aberration of the two first and primary motions. But more on these points below.

(9) Furthermore with a variable eccentricity. The authorities were forced to this conclusion, the rest
with reference to the Sun, Copernicus with reference to the Earth, because they rely too much on the ob-
servations of Hipparchus and Ptolemy; but they are not of such precision that a doctrine of such impor-
tance can be based on them. Consequently in my Commentaries on the Observations of Mars, and Book
Vi of my Epitome, Part 1, I rejected that opinion very confidently as repugnant to celestial physics, and I
do not abandon that conclusion. The obvious stupidity of this supposition I shall demonstrate elsewhere.

(10) Has to be considered by us in three ways] Not because itis itself triple, but because though it is
one and the same it has three distinct aspects, which each have their own multiple applications and func-
tions in the reorganized astronomy.

(1) To start with disregard the eccentricity.] That is, consider the eccentricity separately. For this ec-
centric orbit produces certain effects not by its own eccentricity but solely because it revolves round the
Sun.

(12) Two bodies.) The heaven of the Moon that is to say (not the Moon as such), and the Earth have
the same motion of translation from place to place on the Great Orbit. Consequently since the Earth is
always in the same place as that in which the heaven of the Moon is also found, then the heaven of the
Moon, and by its means the Moon itself, does not on account of its own heaven receive any such appear-
ances of motion as the Sun does from the displacement of the Earth, though itis itself at rest. It would be
different if the Earth moved and the heaven of the Moon were at rest, or moved from place to place with
another distinct motion; for in that case the motion of the center of the Earth would also be transferred by
the imagination to the Moon's heaven, and thus the whole of the Moon's heaven could, according to its
location, seem to retrogress just as much as the five planets.

(13) Copernicus makes the apogee of the Sun.] Two points are hinted at here. One applies to the Sun
itself, the other extends from the Sun to the planets. Ptolemy places the Sun on an eccentric, and encloses,
the eccentric between two deferents: Copernicus locates the planet on an epicycle, and the epicycle on a
concentric circle. Consequently Ptolemy, in order to make the apogees move forward, attributes a special
very slow motion to his deferents. Copernicus produces the same effect by a divergence of the period of

revolution of the epicycle from that of the concentric, each being almost annual. Now it is more probable
that those motions are slow on account of a divergence than on account of an additional motion, par-
ticularly since the epicycle possesses an annual motion only with respect to the eccentric, as the epicycle
rolls in the opposite direction from the rotation of the eccentric; but with respect to the fixed stars it
presents rather a kind of immobility, because in this rolling motion it comes about that the same parts of
the epicycle are always turned towards the same regions of the fixed stars, unless an aberration disturbs
them slightly. Now in my Commentaries on Mars, and in Book IV of my Epitome of Astronomy, | relate
the physical cause both of the eccentricity, and of the shift of the apogees, a cause which resides in the
bowels of the planet, and does not need either deferents or epicycles. But take this clause, referring to the
‘Sun itself (or to the Earth) as an interpolation inserted merely in passing; so that the effect which extends
from the shift of the Sun’s apogee to the other planets is now made clear from it.

(14) Computes the eccentricities. ..from the center of the Earth.} These points are much more readily
apparent from a glance at Plate V. To say what is going to happen many centuries later is indeed prophe-
sying, though such a compunction about the present does not yet affect ancient astronomy. But the com-
parison between this extrapolation of Ptolemy's particular conclusions and the hypothesis of Copernicus
was such that I could not omit mention of the point. For Copernicus also computed the eccentricities of
the five planets as if from the center of the Great Orbit, as if that (and not the actual center of the Sun
which is very near) were the true basis of the planetary system. During these twenty-five years since I first
published this little book, I have so established astronomy that all the eccentricities (of the primary
planets) are referred to the actual center of the Sun, as the true basis of the universe. Hence the eccen-
tricities of all the planets can remain unchanged, wherever the Sun’s apogee shifts to. See in my Commen-
taries on Mars the first part on the equivalence of hypotheses, especially Chapter 6.

(15) A change in the Earth's eccentricity.| These words were included in accordance with the advice of.
Copernicus himself. And it is true that anyone who moves the center of the Sun's orbit too far from the
Earth (or the Earth’s too far from the Sun), as I contend that Ptolemy and Hipparchus did, if he refers
the eccentricities of the planets to that point must necessarily apportion different quantities to them than

Mystertvm CosmoGraPiicy

alissis quantitateslargiatur nec. (fees. quam qui hodie Sols Eucentricitatea eancsadaram bahet,
Atfidcezoitvicitatescomputentsr abipfo centro Salis,vt eof cso, tua aibr! rts artinet hee minta=

tay Eccenz1icatatis Solis feu Terre, feu verailla fit,vt credid:t Copernic
fione nuts nixa. Infpive fuper bac retabulam VC nairationens Rhetie
rum 1 vlna.

§ Osbiculi cerleftis, Terre globum ceunucleum. } Tanazinationi buic an’
Pee operaicus: feu forvive volueri capri fiaereura &infeheferse in peaplexitaterst, 4
fibers 2s plants fubleari nequt, foidts pofft quidem ,fedalla dificillime appaarantur. Vi
resbabet, motss Mlerewera morusivoneltquacspotias dicend.t:
rigtan
ila, Terre
fice biz

clinat.

feayveeco,falfae porfuse

ver Maztealiam meo-

fe
ec meluns vilare potest reprsfinta-
ipfisuiia fiea canffaphyfica qucex Martialibus, o Epuones Ajlr tb IML oF VLet

slobsis dum an ino mots cirsunfertur circa Solem , tenet interim axemtconuolutionis
i fer paralluan in deuenfsiubus,prapter fibrarion naturalem Gp magneticam in=

ad guicfecnetacm vel etnama propter consimuatatem dit ne conaolutionss circa bunc aa
sxean,gira una tenct eredtuon,yt fin turtane incitato o> difisfitante. Quare fice mots iflerc~
era nar: tib, fod quics potiues, fc ettamorbicula commessticio n bil eSopus = a inre bic sme annique
Servers per fuafionisdefolidirate Orbium rewm eget Tyche Brahcus, qui lecto libello literashacde
Caufead mededis,

(27, Aquinodtiorum loca paulatin in precedentia.]_ Omnis dodtvina prace
nis aqu-0.Forcam , contemplatione axis Gy Polorum Tellurisabfoluitur: vt nec Nona Sphara , nec
erbicula so circa terram fit opus. Vide Comment. Martis partem V. Et Epit. Afr. bb. U1. UL.
ov.

48) Tertiusmotus cf Polorum.] Secindum motum in meram axis quietenrede-
gianasrcrtius tin ad facundam cS redacendus, C came invniinconflandus, Srenim canffarum
Ploficar Zobussrione axist ella spol nam renalationcin annucd inuenicurinfenfbili agua retror-
famine. fitu priflino; c> fituetur nibilominus coiyftantem: inclination? ad laters mun

gi; fitertioetiam Ecliptica,quippe Orbita Telluris,vt reliqzvoruin Planctariim orbite,la~
sutudines fuasbabet a viaregisaseafqueper fimilemprenentionem tranjlocabilesde loco inlocum fub
iss: ex bis obtentis equitur vlero fine vila Polorum libratione, > declmationcin Ecliptice mutari,
Ge aquis:aitia nonntbiluunc mestarinanc retardari, quin imo fequitur bocetia amplins ,quod Co-
pernico zraniniaducrfim, Tycho Brabeus o> Landgravins Hagie derexerunt , fixarum mutari la
Hitudins, Eifivevo libratioequinodtiornm non tantanectamceler tuncelicitur, quantsex libra
tionibws Copcrnici:.at de lla quantitate non tantum nondum liquet, fed conftaias equalitas ante &
polt Bre“cmania deprebenfa,, totamusgotin , vna cum obferuationibus Ptolemai propemodum in
dubinave vacat. Sola enimn atas Prolemaieit, qus exorbitar: reliqutrum atatum obferwationescon-

grist ad aquabileinregutamsCopernicum enins qui fusetatis affociationelibrationem banc
enixis cB, proximi atate obferuatoresfidedignifsimi refutant. Vide bacde re
mea Commentaria de Marte Cupitibus pltinis, & Epi-
tom. Aftron.libr. VI.


q

Chapter I

if he knew the present-day corrected eccentricity of the Sun. But if the eccentricities are computed from
the center of the Sun itself, as I compute them, then this change in the eccentricity of the Sun or the Earth
does not affect them at all, whether it is genuine, as Copernicus believed, or as I believe, false and based
‘on mere opinion. On this matter see Plate V and the Narratio of Rheticus, also the last chapter of my
Commentaries on Mars.

(16) The little heavenly circle which. ..enfolds the Earth's globe like a kernel.| Copernicus provided a
handle for this simile; whether he wished to accommodate our capacity for understanding, or whether he
himself was truly caught in perplexity over the point, which could not be relieved by flat diagrams,
though it could be by solid models, which are extremely difficult to supply. However that may be, the
motion in question is not truly a motion, but should rather be spoken of as rest; and it cannot be better
represented by anything than by its actual physical cause, which from the Commentaries on Mars, and
the Epitome of Astronomy, Books 1, Il, Ill, and VI, is the following. While the Earth’s globe travels
round in its annual motion about the Sun, all the time it keeps its axis of revolution always parallel to
itself in its various positions, on account of the natural and magnetic tendency in its inner parts towards
staying at rest, or even on account of the continuity of the diurnal rotation about this axis, which holds it
upright, as happens with a top which has been set in motion and is spinning. Consequently just as thi
not truly a motion, but is rather rest, similarly there is no need of an imaginary little circle; and Tycho
Brahe rightly accused me of this ancient and erroneous belief about the solidity of the spheres, and when
he had read my little book wrote to me on this topic.'*

(17) The positions of the equinoxes. ..gradually westwards.) The whole theory of the precession of
the equinoxes is disposed of by consideration of the axis and poles of the Earth, and there is no need
cither of a ninth sphere or of the little circle round the Earth, See the Commentaries on Mars, Part V, and
the Epitome of Astronomy, Books II, II, VIL.

(18) The third motion is that of the poles.) We have reduced the second motion merely to the axis’s
staying at rest. The third must now be assimilated to the second, and combined with it. For if by the in-
tervention of physical causes it is found that the Earth’s axis after a single annual revolution is inclined by
an insensible amount backwards from its original position; and if nevertheless it maintains a constant in-
clination towards the edges of the universe, or the poles of the Royal Way;:* and thirdly, if the ecliptic
(that is to say the orbit of the Earth), like the orbits of the other planets, takes its latitudes from the Royal
Way, and they are by a similar forward motion capable of moving from one position to another as
against the fixed stars; then from these premises it follows that of their own accord without any libration
of the poles not only does the declination of the ecliptic alter, but also the equinoxes sometimes move
faster, and sometimes more slowly to a certain extent. Furthermore, it follows even more strongly, as
Copernicus failed to notice and Tycho Brahe and the Landgrave of Hesse revealed, that the latitudes of
the fixed stars alter. Even though the libration of the equinoxes which then emerges is not as large or as
rapid as from the librations of Copernicus, yet not only is there still no agreement on that quantity, but
the fact that it has been discovered to be constant and regular before and after Ptolemy calls almost the
whole affair along with Ptolemy's observations into doubt. For it is only Ptolemy's age which is out of
keeping. The observations of other ages are consistent with a regular law. For Copernicus, who
originated this libration by comparison with his own age, is refuted by thoroughly reliable observers who
are very close to his age. See on this point my Commentaries on Mars, in the final chapters, and the
Epitome of Astronomy, Book VII

22 Toan. Kepreri

“Primaria demonflrationss delineatio.

Visvs ita pramihs, vtad propoficum veniam ; arque
modo recentitas Copernici hypothefesde mundo nouo,
nouo argumento probem-rem a primo,quod aiunt,ouo,
§ nouo qua breuitate ficri porerit,repetam.

Corpus erat id, quod initio Deus creauitscuiusde-
finitionem fi habcamus,cxiftimo mediocriter clarum fo-
re,cutinitio corpusnonaliamrem Deus creauent. Dicoquantitacem
Deo fuiffe propoficam : ad quamobtinendam omnibus opus fui, que
ad corporis cllentiam pertinent: veita quatitas corporis, quatenus cor-
pus, quicdam forma , Definitionifqueorigofit. Quanturacemautem
Deusidco ante ommiaexiftere voluics veeflet curui ad Reétum compa-
ratio. Hacenim vnare diunusmihi Cufanas, alijque videntur: quod
Reéti, Curuiquead inuicem habitudinem tanti fecerunt, & Curuum
Deo,Reum creaturisaufifiat comparare : vehaud mulzo vtilioré ope=
ram praftiterint,qui Creatorem creaturis, Deum homini,iudicia diui-
na hamanissquam qui curuum recto,circulum quadiato xquiparare co-
nati fun. .

Cumque vel in hoc folo fatis conftitiffet penes D £ v M quantira~
tum aptitudo,& curui nobilitas: acceflicramen &alterum longe maius:
Deicrnuniimago in Spharicafuperficie, Patris{cilicetm centro, Filij
in fuperficic , Spiritus in equalitate ¢xéts inter punctum & ambitum,
Nam que Cufinas circulo, alij forte globo tsibucrentiea ego {oli Sphee-
rice fuperficiciarrogo. Necperfuaderi poflam, Curuoram quicquam
nobilius effe,aut perfe@iusipfa Spharrica fuperficie. Globus cnim plus

eft Spharica fuperficie , 8 mixtus reétitudini, qua {ola impieturintus.
Circulus vero miiin plano rego exiftar, hoc eft, nifi Sphatica fuperfici-
es,aut globus plano recto fecetur; circulus nulluserit. Vande videreeft,
multas illic a Cubo in globum, hic quadrato incirculum fecundario

deflucre proprictates,propter diametrirectitudinem.

Sed cur denique Curui & Reéti difcrimina , curuique nobilitas

Deco fucrunt propofitain cxornando mundo? Cur enim? nifi quiaaCé-
ditore perfectiffimoncceil¢ omnino fuit,vt palcherrimum opus confti-
tuctctur, Fas enim neceit,nec vaquam fuit (vtloquitut ex Timxo Platonis
Cicero inhbro de yniuerfitate ) quicquam nifi pulcberriaium facere eum aque
effet optimus. Cum igitur dzam mandi Conditor animo preconcepe-
rit (loquimur humano morc , vthomines intelligamus ) atque Idaa fr
rei prioris, fit vero, ve modo dictum eft, rei optim, yeforma futuri ope-
ris & ipfa fiat optima; Parct quod his legibus quas Deus ipfe fua bonita-
tate fibi prefcribit,nullius rei dam pro conttituendo mundo fufcipe-
repotuerit,quam fax ipfus effentix: que bifariam, quam praftansatqs
diuina fit.confiderari poteft,primo in fe,quatenus eft yna in effentia,tri«
nain perfonis,dcinde collatione facta cum creaturis.

Hane


After these preliminaries, to come to the point, and to demonstrate by new
evidence Copernicus’s hypotheses about a new universe, which have just been
reviewed, I shall repeat the argument, as they say, from scratch, with as much
brevity as possible.

It was matter which God created in the beginning; and if we know the defi-
nition of matter, I think it will be fairly clear why God created matter and not any
other thing in the beginning. I say that what God intended was quantity. To
achieve it he needed everything which pertains to the essence of matter; and quan-
tity is a form of matter, in virtue of its being matter, and the source of its defini-
tion. Now God decided that quantity should exist before all other things so that
there should be a means of comparing a curved with a straight line. For in this one
respect Nicholas of Cusa and others seem to me divine, that they attached so
much importance to the relationship between a straight and a curved line and
dared to liken a curve to God, a straight line to his creatures; and those who tried
to compare the Creator to his creatures, God to Man, and divine judgments to
human judgments did not perform much more valuable a service than those who
tried to compare a curve with a straight line, a circle with a square.

And although under the power of God this alone would have been enough to
constitute the appropriateness of quantities, and the nobility of a curve, yet to
this was also added something else which is far greater: the image of God the
Three in One in a spherical surface, that is of the Father in the center, the Son in
the surface, and the Spirit in the regularity of the relationship between the point
and the circumference. For what Nicholas of Cusa attributed to the circle, others
as it happens have attributed to the globe; but I reserve it solely for a spherical
surface. Nor can I be persuaded that any kind of curve is more noble than a
spherical surface, or more perfect. For a globe is more than a spherical surface,
and mingled with straightness, by which alone its interior is filled. Furthermore a
circle exists only on a flat plane; that is, only if a spherical surface or a globe is cut
by a flat plane, can a circle exist. Hence it may be seen that many properties are
imparted both to the globe by the cube, and to the circle by the square, that is
from an inferior source, on account of the straightness of the diameter.

But after all why were the distinctions between curved and straight, and the
nobility of a curve, among God’s intentions when he displayed the universe? Why
indeed? Unless because by a most perfect Creator it was absolutely necessary that
a most beautiful work should be produced. “For it neither is nor was right” (as
Cicero in his book on the universe quotes from Plato’s Timaeus) “that he who is
the best should make anything except the most beautiful.”? Since, then, the
Creator conceived the Idea of the universe in his mind (we speak in human
fashion, so that being men we may understand), and it is the Idea of that which is
prior, indeed, as has just been said, of that which is best, so that the Form of the
future creation may itself be the best: it is evident that by those laws which God
himself in his goodness prescribes for himself, the only thing of which he could

adopt the Idea for establishing the universe is his own essence, which can be
considered as twofold, inasmuch as it is excellent and divine: first in itself, being
‘one in essence but three in person, and secondly by comparison with created
things.

14 loan. Keprert

ordinisminimecapaces,¢ mundo finito,ordinatiffimo,pulcherrimo eij-
ciamus. Rurfum ex corporibus, quorum infinities infinita funt genera,
feligamus aliqua céfu habito per certas notas : puta, quautlatcra aut
angulos,aut plana, fingula velalterna, vel quouis conftantimodo mixta
habeant inuicem aqualia: veita bona cum ratione ad finitum aliquid
veniatur. Quod fi quod genus corporum perccrtas conditiones detcri-
ptum,intra {pecics quidem numero finitas confiftic; fed tamen in ingen-
tem numerorum copiam multiplicatur: corum corporii angulos & cen-
traplanorum (2) profixarum multitudine, magnitudine, fituquede-
monttrando, fi poflumus,adhibeamus: finautem is labor non efthomi-
nis,ergo tantifper differamus numeri, ac fitus earum rationcm quarere;
dum quis nobisad vnum omnes, quot quanczue fint,defcripferit. Miflis
igitur fixis,atq; ei permiffis , qui folus numeratmultitudinem ftellarum,
&fingulas nomine vocat, (Pf£147.)fapientifimo Arufici;nosoculosad
propinquas, paucas & mobiles conucrtamus.

Denique igicur delectum corporum fihabucrimus ,atqueomnem
mixtorum curbam cieccrimus,retineamus vero {ola illa, quorum omnia
plana & aquilatera,& equiangula fuerintsreftabunt nuvishzc quinque
Corpora Regularia, quibus Greci hc afcripfére nomina, Cubus feu

F Hexacdrum,Pyramis feu Tetraedrum,Dodecaedrum, Icofaedrum, O-
Aacdrum. Quodque his quinque plura efle non poifine, vide Euclid lib.
13.poft prop 18.{cholion.

Quareficuthorum definitus & exiguus admodum eft numerus,cx-
terorumaut innumerabiles,aurinfinitz {pecics , ita decuitin miido duo
effe ftellarum genera, cuidenti difcrimincab fe inuicem diftindta (cuiuf-
modi motus & quies eft ) quorum vnum genus infinito fimile , ve fixz,al-
terum anguftum vt Planeta. Non cft huiusloci difputare decautis,cur
hacmoucantur , illanon. Sed pofito , quod Planete motu indigue-
rint, fequitur, (3) vehune obtincrent , rotundos orbes accipere de-
buiffe. .

Habemus orbem proptet motum, (4) & corpora propter nu-
merum & magnitudines; quid reftatamplius, quin dicamus cum Plato-
ne, dy de yewuerav,atqsin hac mobilium fabrica corpora orbibus, & or-
bes corporibus infcripfiffe tantifper, dum nullumamplius corpus refta-
ret, quod non intra & extra mobilibus orbibus veftitum cflet. Naex 13.
14.15.16.17.lib.13. Euclidis videre eft: qua hac corpora natura fua fint apta
ad hancinfcriptionem & circumfcriptionem. Quare fiquings corpora
mediantibus& claudentibus orbibus,infcrantur fibi mutuo:habebimus
numcrumfex orbium.

Quod fialiqua mundi xtas hoc pacto de mundi difpofitionedifpu-
tauit,yt {ex orbes poneret mobiles circa Solem immobilem ; illa vtique
veram Aftronomiam tradidit. Argui eiufmodi fex orbes habet Copernicus, eof-

Propof. quebinos in eiufmodt ad inuicem, proportione: vt hac quingue corpora omnia aptif-
Lioneiner loan poffnt qusfummacrit corumque fequuntur. Quare tantifper
audiendus eft, dum quis aut aptiores ad hac Philofophemata protulerit

hypothefes;aut docuenst,forcuito in numcrosatque in mentem hominis

inrepcre poffe,quod optima rationcex ipfis nature principijs dedudtum
eft. Nam quid admurabilius,quid ad perfiradendum accommodatius di-

ciaur


Chapter IT

surfaces, as they are infinite, and consequently scarcely admit of order, from this
complete, thoroughly ordered, and most splendid universe. On the other hand let
us select from among solid bodies, the varieties of which are infinitely infinite, by
picking out a few in accordance with particular distinguishing features—such as
those which have edges, or vertices, or faces, singly or in pairs, or combined in
some regular way, respectively equal, so that in this way we may arrive at some
finite number on a logical basis. But if a class of bodies defined by definite
characteristics, though it falls among those species which are finite in number,
nevertheless proliferates to a vast number, let us use the angles and centers of the
plane faces of those bodies, if we can, (2) to derive the myriad number, the size,
and the position of the fixed stars. Yet if that task is not humanly possible, then
let us put off seeking the logic of their number and position until such time as
someone has given us an account of the number and size of every last one of
them. So let us pass over the fixed stars, leave them to that most wise Craftsman
who alone numbers the multitude of stars and calls them each by name (Psalm
147),” and turn our eyes to those which are near, few, and moving.

So if in the end we make a selection of bodies, and reject the whole crowd of
hybrids, but retain only those which have all their faces equilateral and
equiangular, we shall be left with those five regular solids to which the Greeks
allotted the following names: cube or hexahedron, pyramid or tetrahedron,
dodecahedron, icosahedron, octahedron. There cannot be more than these
five—see Euclid, Book XIII, scholium after Proposition 18.

Therefore, just as the number of the latter is limited and very small, while the
species of the rest are either innumerable or infinite, so it was proper that there
should be in the universe two kinds of stars, distinguished from each other by an
obvious criterion (such as motion and rest), of which one kind is apparently in-
finite,® like the fixed stars, and the other is restricted, like the planets. This is not
the place to discuss the reasons why the latter move and the former do not. But if
it is assumed that the planets required motion, it follows that (3) they had to
receive round orbits in order to acquire it.

We know the orbit by the motion, (4) and the solid bodies by their number and
sizes. What else remains except to say with Plato, “God is always a geometer,”
and in this structure of moving stars he has inscribed solids within spheres, and
spheres within solids, until no further solid was left which was not robed outside
and inside with moving spheres. For by Propositions 14, 15, 16, and 17 of
Euclid’s thirteenth Book, it is clear how these solids are suited by their nature for
this inscription and circumscription. So if the five solids are fitted one inside
another, with spheres between them and inclosing them, we shall have a total of
six spheres.

Now if any age of the universe has discussed the arrangement of the universe on
the basis of the assumption that there are six spheres moving round a motionless
Sun, it has undoubtedly given a true account of astronomy. But Copernicus has
six spheres of that sort, and each pair of spheres in such proportion to each other
that all these five solids can very readily be fitted in between them, which is the
essence of what follows. So we must concur with him, until someone has either
put forward hypotheses which give a better solution to these problems, or
asserted that a system which has been deduced by excellent reasoning from the
very principles of Nature can creep accidentally into the numbers and into the
mind of Man. For what could be said or imagined which would be more
remarkable, or more convincing, than that what Copernicus established by obser-
vation, from the effects, a posteriori, by a lucky rather than a confident guess,


Propo-
sition

dici aut fingi poteftiquam,quod ea qua Copernicus ex daurauicis,ex ef
feétibus,cx pofteriorib. quafi cazcus baculo greffum firmas)vc ipfe Rhe-
tico dicerc folitus cft(felici magis quam confidenti conieéturaconttitu-
it,atque ita fefe haberecredidit}cainquam omniarationibus a prior, 2
caufis, a Creationis ida deductis rectiflime confticuta effe deprehen-
dantur.

Nam fi quis philofophicas iftas rationes,finc rationibus, & folori-
fu excipereatque cludere voluerit: propterea,quod nouus homo fub fi-
nem feculorum,tacentibus illis philofophizluminibus antiquis , philo-
fophica ifta proferamiilliego ducem ,auétorcm & przmonftratorem ex
antiquiffimo feculo proferam Pythagoram:cuius multain {cholis mentio,
quodcum praftantiam videret quinque Corporum , fimili planeratio-
neante bis millcannos,quanunc ego, Creatoris cura non indignacen-
facritad illa refpicere;atque rebus mathematicis phyfice, & ex fua qua-
libec proprictate accidentaria cenfitis , res non mathematicas accom-
modauerit. (5) Terracnim Cubo zquiparauit, quia ftabilis veerque,
quod tamen de cubo non proprie dicitur. Cee!o Icofaedrum dedit,quia
verumque volubile: Igni Pyramida,quia hxc volantis igniculi formasre-
liqua duo corpora interacrem & aquamdiftribuit , proper fimitemy-
trinque cum vicinis cognationem.Sed enim Copernicus ili virodefu-
it,qui prius,quid cfferin mundo, diceret: abfque co non fuiffet, dubium
nonelt,quin quare cflet,inueniffet, atque hac ceclorum proportio tam
nota nuncefler, quam ipfa quinque corpora;tam item recepta, quam
hoc temporum decurfu mualuit alla de Solis motu, deque quicte T ellu-
ris opinio.

Verum age vel tandem experiamur , verum inter orbes Copernici
finciftz corporum proportiones. Acinitioremcraffiufcule cenfeamus.
Maxima diftantiarum differentia in Copernico eft inter louem & Mar-
tem: Vt vides in explicatione hypothefium Tab. 1. & infra cap.14. 8 15-
Martis enim diftantiaa Solenon zquattertiam partem Iouiz. Quara-
turigicur corpus,quod maximam facit differentiam inter orbem circa-
friptum & infcriptum (6) (concedatur nobis hc x«@z¢ena* cauum
profolidocenfendi) quod eft Tetracdrum fiue Pyramis. Eftigiturinter
Jouem & Martem Pyramis.Pofthos maximam faciunt differentiam di-
ftantia Tupiter &Saturnus.Huius enimille paulo plus dimidium equat.
Similis- pparetin cubiintimo & extimo orbe differentia. Cubum igitur
Saturnusambit,cubus Iouem.

ZEqualis fere proportio eftinter Venerem & Mercurium, nec ab-
fimilis inter orbes O@aedri. Venusigitur hoc corpus ambit, Mercurius
induit.

Reliquxduz proportiones inter Venerem & Terram, inter hane

& Martem minima fant, & ferexquales.nempe interior exterioris do-
drans autbes. In Icofacdro & Dodecaedro fant etiam zqualesdiftantiz
binorum orbium: Etproportione vtuntur minima inter reliqua regula-
tiacorpora. Quareverifimile eft, Martem ambire terra mediantealter-
utro horum corporum:Terramautema Venere fummotain, mediante
reliquo.Quare fi quis ex me quarat,curfinttatum fex orbes mobiles, re-
{pondebo,quia non oporteat plures quinque proportiones cffe, toridem
D nem-


Chapter IT

like a blind man, leaning on a stick as he walks (as Rheticus himself used to say)
and believed to be the case, all that, I say, is discovered to have been quite cor-
rectly established by reasoning derived a priori, from the causes, from the idea of
the Creation?

For if anyone who listens to this philosophical reasoning wants to evade it
without reasoning and merely with a laugh, because I am putting forward this
piece of philosophy almost at the end of the ages as a newcomer, though the an-
cient luminaries of philosophy say nothing of it, then I will offer him as a guide,
authority, and demonstrator from the earliest age Pythagoras, who is much
spoken of in the lecture rooms because seeing the pre-eminence of the five
solids," by plainly similar reasoning, two thousand years before I now do so, he
judged it not unworthy of the Creator’s concern to take account of them, and he
made things which were not mathematical fit mathematical things physically, and
by classifying them according to some accidental property of their own. (5) For he
compared the Earth with the cube, because each is stable, although that is not an
essential property of the cube. To the heaven he gave the icosahedron, "! because
they both rotate; to fire the pyramid, because that is the shape of a rising flame.
The other two bodies he divided between air and water, as each has a similar af-
finity with its neighbors on either side. Yet Pythagoras did not have Copernicus
to state beforehand what there was in the universe. If he had not been without
him, there is no doubt that he would have discovered the reason why it was; and
this proportion in the heavens would now be as well known as the five solids
themselves, and also accepted to the same extent as during this lapse of time the
belief in the motion of the Sun and the immobility of the Earth has weakened.

But come, let us at length test whether the proportions of the five solids are
found between the spheres of Copernicus; and to start with let us assess them
rather roughly. The greatest difference of the distances in Copernicus is between
Jupiter and Mars, as you may see in the explanation of the hypotheses, Plate I,
and below in Chapters XIV and XV. For the distance of Mars from the Sun is not
as much as a third of that of Jupiter. A solid is therefore required which makes
the difference between the circumscribed and inscribed sphere a maximum (6) (if
we may be allowed the solecism of counting it as hollow instead of solid), and that
is the tetrahedron or pyramid. Therefore there is a pyramid between Jupiter and
Mars. After them Jupiter and Saturn yield the greatest difference in distance. For
the former amounts to a little more than half the latter. A similar difference is
found in the interior and exterior sphere of a cube. Therefore Saturn goes round a
cube, and a cube round Jupiter.’

The proportion between Venus and Mercury is almost equal, and that between
the spheres of an octahedron is not greatly different. Therefore Venus goes round
outside that body, Mercury inside.

The remaining two proportions, between Venus and the Earth, and between
the latter and Mars, are very small, and almost equal, that is, the interior sphere is
three quarters or two thirds of the exterior. In the icosahedron and the
dodecahedron also the radii of the two spheres are equal, and stand in the
smallest proportion to each other, compared with the rest of the regular solids.
Therefore it is probable that Mars goes round the Earth with one or other of these
solids in between; whereas the Earth is separated from Venus by the interposition
of the other. Therefore if | am asked why there are only six moving spheres, I
shall answer that it is because there ought not to be more than five proportions,
Lom
fol. preced.

Supr ibid.

26 Toan, Keprerti

nempe,quotregularia fancin matheficorpora.Sexautem terminicon-
fummanthunc proportionum numerum,
Hacpertinet Tabula tertia.

Annotatio in Caput fecundum,antiqut.

Quodgue his quizzque) Corporum nobilitas eft ex fimplicitate, & ex
aqualitace diftantix planorum a centro figure. Sicutenim norma & re-
gulacreaturarum Deuseft;fic Sphzra corporum. Atquieahabet di€tas
proprictates.1.Eft fimpliciffima,quia vno clauditurtermino,{eipfa {cilic.
2, Omnia cius puncta xqualiffimea centro diftant. Ex corporibusigitur

roximeaccedunt regulariaad Sphara perfeétionem. Eorum definitio

ac eft, vt habeant, 1-omnia latcra, 2. plana , & 3.angulos,fingularqua-
Jes & {pecie & magnitudine, quod eft fimplicitatis ; quam pofitam defi-
nitionem {cquiturillud vitro,quod 4.omnium planorum centra xquali-
ter medio diftent, 5. quodin{cripta globo omnibusangulis tangantfu-
perficiem, 6. quodin ca hzreant, 7. quod infcriptum globum omnibus
planorum centris tangant,8.quod proinde in{criptus globus hzreatim-
motus,9.& quod idem centrum habeatcum figura, Quibus rebus cffi-
cituralrera fimilicudo cum Sphzra, qua eft cx xqualitate diftantiz pla-
norum.

(7) Scholion autem illud ita fonat: Aio vero.prater dias quings
figuras non poffe aliam conftitui iguram folidam , qu planis & xquila-
teris & xquiangulis contineatur, inter fe equalibus. Non cnim cx duo-
bustriangulis , fed neque exaliis duabus figuris folidus confticueturan-
gulus.

Sed ex tribus triangulis,conftat Pyramidis angulus.

Ex quatuor autem, Oftacdri.

Ex quinquevero,Icofaedri.

Nam extriangulis {ex & aquilateris , & xquiangulis ad idem pun-
um cocuntibus,non fict angulus folidus. Cum enim crianguli yuila-
teriangulus, recti vnius beffem contineat, erunt ciufmodi {ex angulire-
tis quatuor aquales. Quod fier: non poteft. Nam folidus omnisangu-
lus, minoribus quam reétis quatuor angulis continctur,per 21.11.

Ob eafiiem fane caufas,neque ex pluribus quam planis fex ciufme-
di angulis folidus conftat.

Sed cx tribus quadratis Cubi angulus continetur,

Ex quatuor nullus poteft.Rurfas enim reéti quatuor erunt.

Extribus autem pentagonis equilateris, & xquiangulis Dodeca-
edri angulus continetur.Sed ex quatuor nullus poteft. Cum enim Pen-
tagoni xquilateri angulus rectus fic, & quinta reéti pars, erunt quatuor
anguli reétis quatuor maiores. Quod fierinequit. Nec fane ex alijs po-
lygonis figuris folidus angulus continebitur, quod hincquoque abfurda
fequatur. Quamobrem perfpicuum eft,pracer didtas quinque figuras a-
liam figuram folidam non poffe conftitui, quz fub planis aquilaceris &
zquiangulis contineatur.

Planum

Chapter IT

that is the same number as there are regular solids in mathematics. But six bound-
aries make up this number of proportions.
Here belongs Plate 11.1%

Original note to Chapter II."

There cannot be] The nobility of solids depends on their simplicity, and on the
equality of the distance of théifaces from the center of the figure. For just as God
is the model and rule for living creatures, so the sphere is for solids. Now the
sphere has the following properties: 1. It is extremely simple, because it is en-
closed by a single boundary, namely itself. 2. All its points are at a precisely equal
distance from the center. Therefore among bodies the regular solids approach
most closely to the perfection of the sphere. Their definition is that they have: 1.
all their edges, 2. their faces, and 3. their vertices respectively equal both in kind
and in size, which is a sign of simplicity. From the adoption of this definition it.
follows automatically that 4. the centers of all the faces are equally distant from
the midpoint, 5. that if they are inscribed in a globe all their vertices touch its sur-
face, 6. that they are fixed within it, 7. that they touch the inscribed globe at all
the centers of their faces, 8. that consequently the inscribed globe is fixed and im-
mobile, 9. and that it has the same center as the solid. These properties yield
another resemblance to the sphere, which results from the equality of the
distances between the faces.

(7) Now the scholium expresses it in this way: I say that apart from the five
stated figures no other solid figure can be formed which is bounded by equilateral
and equiangular faces which are equal to each other. For a solid angle cannot be
formed from two triangles, nor from any other two figures.

But from three triangles the vertex of a pyramid is formed.

From four, that of an octahedron.

From five, that of an icosahedron.

Now from six equilateral and equiangular triangles meeting at the same point, a
solid angle cannot be produced. For since the vertex of an equilateral triangle
contains two thirds of a right angle, six angles of that type will be equal to four
right angles. Which is impossible. For every solid angle is bounded by four angles
less than right angles, by Euclid Book XI, Theorem 21.

For the same reason, neither can a solid angle be formed from more than six
plane angles of that type.

But the vertex of a cube is bounded by three squares.

From four such, no solid angle can be formed, for again there will be four right
angles.

However, the vertex of a dodecahedron is bounded by three equilateral and
equiangular pentagons. But from four such, no solid angle can be formed. For
since the angle of an equiangular pentagon is one and one-fifth right angles, four
such angles will be greater than four right angles. Which cannot be the case. Nor,
plainly, can a solid angle be bounded by other polygons, because from that also
an absurdity would follow. Consequently it is evident that apart from the five
stated figures no other solid figure can be formed which is bounded by equilateral
and equiangular faces.


page 97

Above, at
the same
place.
‘877 ‘d 295

gx who]
rope aout
fesmayy wnaruth ome woes ge
oie <

} en hi

‘fag vy mae uO ~OIVIITAL 40mm,
ve rum 7 facial nie ra
a : rage 1092S
“ayo SADINWIIC) Pubep ponb rrurmuer mig wont
“quanop anbusnb vaoboe mpeg 77010
aaui gs vquonh navodlp unurmuny Fur A
Gan van” vin] weber aab vp yy

<S ido nosvariag ‘edo sursy rea73

11 Planum | Plana | Latera | Angul. | Orbem infcripré.
Cubus | _ | avadrangulam 7 6 | R 8 | mediocrem.
Oacdron | 3 | ciangulum | 8 Pa | 6 cuboxqualem.
Dodccaed. | 2 | quinquangulum J 12 30 | 20 | maximum.
Tcofaedron |” | triangulum 20 | 30 | 12 | dodee. equalem.
Tetracd. | Joriangulum = |g | 61) 4) | minimum.

Notz Auétoris.

@) Lhe vero & fuperficics.] O male f2itum.E mundone ejciamus? Lo poStliminiores

uocaniin Harmonicis.Cur autem eijciamns? An quia infinite, cy proin ordinis mininseca-
paces? Atqui non ip(e,fed mea illins temporisinfcitia communis mibi cum plerifyue , ordinisiltarum
‘minime capax erat lt.agite ib.1.H.armonicorum, Cy deledtum aliquem inter infinitas docti,cz ordi
sens in igspuleherramumn in lucem protuli, Nam cur lineas nosex archerypo mardi eliminemusy cum
Lness Deus opere ipfo exprefferit,motus fc.Planctarum? Lingua igitur corrigenda,menstenenda. In
Corporusnsnumnero,fpberarum amplitudine conflituenda primitus,climinenturfane lince: at inmo=
ribias,qat linea perficnmntur ,exornandis,necontenmantis lineas e fuperficies , que fole proportio-

num Harmonicarum funt origo,

(2) Profixarum mulciudine.] Ingens diferimen argunento nominum , einter fica
& mobiles; car non fit aliquod etiam in virinsque generisexornatiom? Quis Ordinispulchritudi~
nent iatelligeret finon inxta cerneret fixarumexercituim ordinis expertem? Quis Aftronomiam dif-
ceret fipsrpetna efor finuitudo fcbematifinorum, feu conftellationum?Es fins formis ornatus.ell Cr
Materie.Sit igitur propria, materie cs puldbra exornatio, que fab. eit por infinitam cr molem
multitudinen, c> vavictatem,tam fitue, quam magnitudinis Claritarifque.

(3) Vehunc,motum,obrincrent, rotundos orbesaccipere debuiffe.] Nonillos
folidos,anale bicuin inellecius a Tychone Braheo, fed fpacia,prorfum quidem circularia, vt reuolue
tones yderui in feifesredire cr perpetnecffepoffentsverfuspolos vero itidé circularia,id et fuperfi-
Ciesplbericaspropter motnslatitudinumsyon quod polis opus habuerint,d quibve,ve fphara materia
Uassaffigerentur.

(4) Etcorpora propter numerum.] Corpora intellige Geometrica regislaria folida
quingucshec ve archetypum, Orbes vero,vt opus exflruendum.

(5) Terramenim Cuboxquiparauit.] Videlib.1.Hannon.inproam. fol.4. & lib.
UL prop.XXV.C lib. V cap...Et Epit. Aftr ub.V fol. 456.

(6) Concedaturnobishaec.] Vere quidem aut phericum etiaminterfolida cenfen=
dum,quod globurn dicimus;ant bec corpora folida dici non merentur?necerant a foliditate, hocelt &
perfectionetrium dimenfionum argumenta texenda pro Orbiam exornatione per ea.Nam Cy ipfior-
bes’ feu fpacia) cai fuse . C= figura he obid nobiles funt , quia fpharictperfe€tionem omnimodacon=
clufionefpaciy, quod amplexe fant , quam proximeamulantur. Soliditas verotam inglobo,quamin

bus figuris est genuina materiaidea vt fuperficicsforme.
(7) Scholionautemillud.] Hoceft dimidiam libri mei 11. Harmon.de Congruentia
planorumin{olide,


Chapter II

No. of No. of No. of
Type of face faces edges _vertices_Inscribed sphere

Cube Quadrilateral | 6 12 8 | Medium
Octahedron Triangle 8 12 6 _| Equal to cube
Dodecahedron] | Pentagon 12 30 20 | Largest
Icosahedron Triangle 20 30 12 | Equal to dodeca-
hedron
Tetrahedron Triangle 4 6 4 | Smallest


(1) Straight lines and surfaces.| What a mistake! Are we to reject them from the universe? Instead, I
reinstated them, as was their right,-in the Harmonice. But why are we to reject them? Is it because they
are infinite, and so scarcely admit of order? But it was not they themselves which scarcely admitted the
possibility of their being ordered, but my own ignorance at that time, held in common with most people,
So in Book I of the Harmonice | not only explained a certain principle of selection among these infinities,
but also brought to light a most splendid order in them. For why should we eliminate lines from the ar-
chetype of the universe, secing that God represented lines in his own work, that is, the motions of the
planets? Therefore my language should be corrected, my intention retained. In establishing the number
of the bodies, and the size of the spheres, lines should indeed be eliminated in the first place; but in
displaying the motions, which are accomplished by lines, let us not despise lines and surfaces, which
alone are the origin of the harmonic proportions.

2) To derive the myriad number. . .of the fixed stars.| There is a vast difference as far as names are
concerned between the fixed and moving stars: why should there not also be some difference in the
displaying of the two kinds? Who would understand the splendor of order, if he could not perceive
alongside it the host of the fixed stars without order? Who would learn astronomy, if there was an endless
similarity between the patterns or constellations? Their display is provided by their forms, and by their
matter. Then let them be displayed in a way which is appropriate for their matter, and splendid, as it is
the result of both their infinite bulk and their infinite number, and of the variety in both their position
and their size and brightness.

(3) They had to receive round orbits, in order to acquire (this motion). Not solid spheres (Iwas
misunderstood here by Tycho Brahe), but spaces, which are indeed completely circular, so that the
revolutions of the stars could return to the same positions and be perpetual. In the direction of the poles
they are likewise circular, that is, their surfaces are spherical, on account of the motions in latitude, not
because they needed poles to be fixed to, like a material sphere.

(4) And the solid bodies by their number.) Understand this to mean the five regular geometrical
solids: they are, so to speak, the archetype, and the spheres the work to be constructed.

(5) For he compared the Earth with the cube.} See Book I of the Harmonice, page 4 of the preface,
Book II, Proposition 25, and Book V, Chapter 1; and the Epitome of Astronomy, Book IV, page 456.

(6) If we may be allowed.) In fact, either the spherical shape, which we call the globe, should be
counted among the solids, or else these bodies do not deserve to be called solid, and arguments that the
spheres were displayed in accordance with them should not have been drawn from their solidity, that is
from the perfection of their three dimensions. For the spheres (or spaces) themselves are hollow, and also
the reason for the nobility of these figures is that they emulate as closely as possible the perfection of the
spherical by their complete enclosure of the space which they have surrounded. Solidity indeed both in
the globe and in these figures is the true Idea of their matter, as surface is of their form,

(1) Now the commentary.| That is the half of Book Il of my Harmonice, which is about the concur-
rence of planes in a solid figure.

a8 Ioan. Keprert


Quod hac quingue corpora in duosordines diftinguantur, oe
quodterrarefle locata fit.

Orr oautem fortuitum hoe videri poffet,atque a nvl-
lafluens caufa , quod fex orbes Copernici recipiantin-
tra fuas vnius abalio diftantias hac quing corpora,nifi
is ip{e ordo effec inter illa,quoordine ego fingulainter-
Jocaui.Nam fi Saturnus Ioui,tam propinquus cfler,qua
eft Venus Telluri, viciffim fi hz duz abinuicem tanto
lo diftinguerenturin Copernico, quanto diftinguuntur lupiter
& Mars: alio ordinevrendum fuilletin inferendis corporibus. Forcte-
nim inter duos primos orbes primo loco Dodecaedron vel Icofacdron,
quarto veroloco Tetraedrum. Qui ordocum non poflitadmitti ratio-
nibus Mathematicis, facile foret futilitatem concept: Theorematis pa-
tefacere.Nunc autem videamus nos, ecquibus rationibus probetur, de-
buiffe hocipfo ordine difponi corpora inter orbes. Initiod:ftinguuntur
hzc corpora intria primaria, Cubum, Tetracdrum, Dodecaedrum, 8
duo fecundaria, Otaedrum & Icofaedrum. Quodque veriffimum hoc
fit difcrimen,nota vtriufque generis proprictates.1. Primaria plano inter
fe differunt: fecundaria vtuntur codem tridgulari.2.primariorum quod-
libet proprium habet planum : cubus quadratum, Pyramistriangulum,
Dodecaedron quinquangulum: fecundaria planum trianguluma Pyra-
mide mutuantur.3.primaria omnia fimplici veunturangulo, nempc tri-
bus planis comprehen(o : fecundaria quatuorautquinque planisin yn
folidum adfcitcunt. 4. Primarianemini fuam debent orig nem & pro-
prictates: ecundatia pleraque ex primariis,faéta commutatione,adepta
fant, & quafigenita exillis.s. Primaria non moueri cécinne poflunt, mifi
aéta diametro per centra vnius aut oppofitorum planorum: fecundaria
vero acta per oppofitos angulos diametro. 6.Primariorum eft proprium
ftare:fécundariorum pendere.Siue enim hacin bafin pronoluas, fiucilla
inangulum erigas:vifus verinque deformitatem afpeetus refuict.7.Ad-
de denique quod primaria perfeéto numero tria (unt: fecundariaimper-
feé&o duosquodqucilla omnes anguli fpecics habent, Cubus reétum,Py-
ramisacutum , Dodecaedrum obtufim; hxc veroambo in obtufifolius
genere verfancur. Ec Oftacdri quidé angulus per omnes tres {pecies va~
gatur, iniunauralaterum obrulis ; inter cocuntia duo latera ex oppofi-
to,rectussipfevero folidus,acutus. Cum igitur manifeté eflecdiferimen,
inter corpora, conucnicntius fieri nihil potuic, quam vt Tellus noftra,
totius mundi fumma & compendium, atqueadeo digniffima ftellarum
mobilium,orbe fuo inter diétos ordines diftingucret,locum-
que cum fortirctur, quem ipfi faperius
attribuimus.



Now, it might seem fortuitous, and not the result of any cause, that the six
spheres of Copernicus accept these five solids into the spaces between them, if
their actual pattern were not the pattern in which I have placed them. For if
Saturn were as close to Jupiter as Venus is to Earth, or on the other hand if the
last two were separated from each other in Copernicus by a gap of the same size
as are Jupiter and Mars, it would have been necessary to use a different pattern
for interpolating the solids. For between the first two spheres in the first position
would be a dodecahedron or icosahedron, but in the fourth position a
tetrahedron. Since this pattern could not be acceptable to mathematical reason-
ing, the futility of the theorem which I have adopted would easily be exposed. As
things are, however, let us see the reasoning by which it is confirmed that the
solids had to be arranged in this precise pattern among the spheres. To start with,
these solids are classified into three primaries, the cube, tetrahedron and
dodecahedron, and two secondaries, the octahedron and icosahedron. For the
correctness of this distinction, note the properties of each class. 1. The primaries
differ from each other in shape of face; the secondaries both have triangular faces.
2. Every one of the primaries has its particular type of face: the cube has the square,
the pyramid the triangle, the dodecahedron the pentagon; the secondaries borrow
the triangular face from the pyramid. 3. All the primaries have a simple vertex,*
that is, one which is included between three faces; the secondaries combine four
or five faces in one solid angle. 4. The primaries owe their origin and properties to
no one; the secondaries have got several things from the primaries by borrowing,
and are so to speak generated by them. 5. The primaries cannot move ap-
propriately except on a diameter drawn through the centers of a single or of op-
posite faces; but the secondaries on a diameter drawn through opposite vertices.
6. It is characteristic of the primaries to stand upright, of the secondaries to
balance on a vertex. For if you roll the latter onto their base, or stand the former
on a vertex, in either case the onlooker will avert his eyes at the awkwardness of
the spectacle. 7, Add finally that the primaries are three, the perfect number, the
secondaries two, an imperfect number; and that the former have all types of
vertex, the cube a right angle, the pyramid acute, and the dodecahedron obtuse,
but the latter both employ a single type of angle, the obtuse. In fact, in the case of
the octahedron all three types of angle occur: the obtuse at the junction of the
faces; a right angle between two edges running from opposite vertices; whereas
the actual solid angle is acute. Therefore, since there was an obvious distinction
between the solids, nothing could be more appropriate than that our Earth, the
pinnacle and pattern of the whole universe, and therefore the most important of
the moving stars, should by its orbit differentiate between the two classes stated,
and should be allotted the position which we have attributed to it above.?

Mystertvm CosmoGRAPHICVM. 29


Quare tria corporaterram ambiant, duo reliqua induant?

Ar eRe nunc, Leétorxquanime,vtludam aliquantif-
} per inre feria,& nonnihil Allegoriisindulgeam. Etenim
‘@ exiftimo ex amore Dei in hominem caufas rerum in ma-
{do plurimas deduci poffe. Certeequidem nemo negabit,
in domicilio mundi exornando Deumad incolam futu-
tum identidem refpexiffe. Finis enim & mundi & omnis
creationis homo cft.T crram igitur;quz genuinam Creatorisimaginem
datura & alitura effet,exiftimodignam a Deo cenfitam, qua circumiret
inter medios planetasfic, vt totidemilla haberct intra orbis fui comple-
xum,quot extra habituraeffet. Vthoc Deus obtineret , Solem reliquis
quinque Stellis accen{uit, quamuisilletoto genere difcreparet. Idgue
co magis confonum videtur, quod cum fupra Sol Dei patris imago
rit.credibile eft, hac affociatione cum reliquis Stellis argumenta ventu-
rocolono prabere debuille ¢rerspurias , & ai Spuremdas quam Deus v-
furpesucus erat erga homines, ad domefticam familiaritatem v{que fefe
demittens. Namin Vereri Teftamento, frequenterin numerum homi-
numvenit, & Abrahami amicus audire voluit; ficuti Solem videmusin
numerum mobilium yenire. Cum autem Sol aterraambiretur : pofitis,
que diéa fant, neceffario ille ordo corporum intra terram includendus
fuit,qu: duo faltem complettitur : nempe ve mobilia duo cum immobili
Sole cundem efficerentnumcrum ternarium, quicftin exclufisab orbe
terre.Sicigitur Luna prafertim terramambeunte,domicilium noftrum.
optimus Creator in medio feptem Planctarum collocauit.Nam fi rium
reliquorum ordoad Solemaccefliffets fuiffent igitur incraterramcum
Sole quatuor Stellz,duz vero tantum extra.Q uz numeri dmetiocum ra-
tione careat,omiffa eft 4 Creatore: Cum item continere fit perfectioris,
veactio,contineri ve paflio im perfectioris ; primaria vero perfeétiora fint
caterissconuenit,vt trium ordo contineretterram, reliqua contineren-
turintraorbisterreniambitum. Atquefichabemus obiter caufam,cur
extraterram tres mouedtur Planetz,intraduo;quz fiminus LeStori pro~
batur,cogiter,honorarium hoc effe,non precipuum. Nametfinefcire-
mus caufam ob quam fupraterram (vel Solem Prolemzi) tres irent Stel-
Iz,tamen fequentia ftarentcum prxcedentibus ; quia nobisde REcon-
ftac.Nec quifquam vnquam dubitauit,quin b¥ SF fuperiores fine. Ta-
tum illudcencamus ; cum tresin Copernico Planeta fint fupraterram,
oporcere nos ordinem trium primariortm corporum Cubum, Pyrami+
da, Dodecacdron extra orbem tclluris collocare , O&aedrum ve-
ro & Icofacdron intra; fi palmam in hoc negotio

velimus obtinere.

c-



Bear with me now, patient reader, if I trifle for a moment with a serious subject,
and indulge in allegories a little. For [ think that from the love of God for Man a
great many of the causes of the features in the universe can be deduced. Certainly
at least nobody will deny that in fitting out the dwelling place of the universe God
considered its future inhabitant again and again. For the end both of the universe
and of the whole creation is Man. Therefore in my opinion it was deemed by God
fitting for the Earth, which was to provide and nourish a true image of the
Creator, to go round in the midst of the planets in such a way that it would have
the same number of them within the embrace of its orbit as outside it. To achieve
that, God added the Sun to the other five stars, although it was totally different in
kind. And that seems all the more appropriate because, the Sun above being the
image of God the Father, we may believe that by this association with the other
stars it was bound to provide evidence for the future tenant of the loving kindness
and sympathy which God was to practice towards men, even as far as bringing
himself down into their intimate friendship. For in the Old Testament he fre-
quently came among their number, and was willing to be known as a friend of
Abraham; just as we see the Sun is numbered among the moving stars. However,
since the Sun was encircled by the Earth, granted what has been said, that class of
bodies which in fact includes two had necessarily to be contained within the
Earth’s orbit, that is, in order that those two moving stars along with the unmov-
ing Sun should make up the number of three, which is the number of those out-
side the Earth. Thus with the Moon as a special case encircling the Earth, the best
of Creators placed our domicile in the middle of the seven planets. For if the class
of the other three had been added to the Sun, there would have been four stars in-
cluding the Sun inside the Earth, but only two outside. Since this irregularity of
number lacks order, it was dismissed by the Creator. Also, since containing is
proper for the more perfect, as it is active, but being contained for the more im-
perfect, as it is passive, but the primaries are more perfect than the rest, it is fit-
ting that the class of three should contain the Earth, but the rest should be con-
tained within the circuit of the terrestrial orbit. And thus we have in passing the
reason why three planets move outside the Earth, two inside; and if this meets
with less approval from the reader, let him reflect, that this is a by-product, and
not the main point. For even if we did not know the reason why three stars moved
above the Earth (or the Sun in Ptolemy), nevertheless what follows would be con-
sistent with what precedes, because we are certain of the facts. Nor has anyone
ever doubted that Saturn, Jupiter, and Mars are superior. Let us just hold on to
this point: since in Copernicus the three planets are above the Earth, we should
locate the class of the three primary solids, the cube, pyramid, and
dodecahedron, outside the Earth’s orbit, but the octahedron and the icosahedron
inside it, if we wish to win the palm in this affair.

. Kertert

30 Toa


Quod cubus primum corporum, o inter altifvimos planetas.

SES F533 Ent amv s modoad primariatria, fuaque fingulis (pa-
\5 IE cia ribuamus. Et Cubus quidem ad fixas appropinqua-
S77 tedebuit,primamque proportionem, quaincer Satur-

£3) Bum &louem cft,conftitucre;quiadigniflimamundi pars
So) (egey) extra terram fant fixe: vecirculi (poft centrum) circum-
= ee fi : Cubus vero primum corpusin fuo ordine.1.So-
lus enim {ua bafi generatur, cum reliqua quatuor non generétur facie-
bus fuis,fed autfecta fine é Cubo, ve Pyramis, reiedtis 4.pyramidibus re-
Ctangulis:auraudta,ve Dedocaedron,appofitis fex pentaedris. 2. Solus in
homogencos cubos fine prifmate refolui potelt. 3. Solus eft quaqua
verfum,& in tres direétas dimenfiones porrigitur. Nam eliquotum fa-
cigsinclines fune,&alicubi, cum fe duabus diredtis fetionibus prebeit,
inireliqua {eétorem fruftrantur. 4. Hinceft, quod folus habet tot facies,
quothabet cernaria diméfio terminos,nempe fex, &duplum numerum
Jaterum,{cilicet duodecim. 5.Solus vndiquaqs habetxqualemangulum,
feilicetretum. Atin Pyramideregula,qua: fedccadhibita medijs planis,
difcrepat, fi cd verfus angulum intorqueas ; nec folidianguliad eam nor-
main quadrant, quzintcriectum longum lateralem angulum metitur.
6.Hincetiam folicépetit, quod ex #2¥8/Baw Prolemai citat Simplicius
faper Arift.lib.1.de culo cap.1. pro caufa perfeQionisin ternario ; quod.
{cilicet non plures tribus reétis perpendiculatibus ad locum folidum.
in folidos rctos diuidendum concurrere poffine. 7. Eftfolidorum recti«
lincorum omnium fimpliciffimum corpus. Quodetfin Pyramideam-
bigitur, tamen ex co facile euincitur , quod pyramidis men fara Cubus
eft,menfaram autem priorem effe conuenit. Menfuraveroeftnon tan-
tum exinftituto hominé, qui quicquid folidorum metiuntur, cius quan-
titatem in patuis cubifcis cocipiuntanimo-fed multo magis natura. Re-
us enimangulusxqualis eft alteri, quocumin planum extenditur.Eft
igitur perpettio fibi aqualis ipfi, atque adeo vnus, cateracum ytrinque
infiniti fant. Menfurain autem decet vnam & candem, atque ctiam fini-
tameffe.8, Hine (1) tam feccunda cftreéti in circulum infcriptio, fine
quo mediante,nec triangulum , nec quinquangulum,nec ab cis deriuata
infcribi poffunt.9.Sedneque illud pratereundum quod peifectiffimo a-
numali folers natura {ex cafdem dasteus perfedhffime attribuit: non ob-

{curo argumento , quam hoc corpus penesillam fitin pretio.
Nam homo ipfequidam quafi cubuseft,in quo
fex quafi plagas funt,fupera, infera,
antica, poftica, dextra,
finiftra.



Let us now come to the three primaries, and allot to each its own space. Now the
cube should be close to the fixed stars, and establish the first proportion, that be-
tween Saturn and Jupiter, because the fixed stars are the most important part of
the universe outside the Earth, just as that of a circle (after its center) is the cir-
cumference; and the cube is the first solid in its class. 1. For it alone is generated
by its base, whereas the other four are not generated by their faces, but are either
parts cut from the cube, as is the pyramid, which is derived by cutting off four
rectangular pyramids, or compound, as is the dodecahedron, which is derived by
the addition of six pentahedra. 2. It alone can be resolved into homogeneous
cubes with no prism. 3. It alone faces in all directions, and extends in three direc-
tions at right angles. For the faces of the others are oblique, and at some point,
although they allow division in two directions at right angles, frustrate it in the re-
maining direction. 4. It follows from this that it alone has the same number of
faces as the three dimensions have directions, namely six, and twice that number
of edges, that is twelve. 5. It alone has an equal angle, that is, a right angle, in
every respect. But in the case of the pyramid, the formula which holds good when
applied to planes of symmetry fails if you turn it on a vertex; and the solid angles
do not square with the rule which governs the intervening angles between the
lengths of the edges. 6. Hence also it alone agrees with what Simplicius quotes
from the Monobiblos' of Ptolemy on Aristotle, De caelo, Book 1, Chapter 1,
about the reason for the perfection of the number three—that is, that not more
than three straight lines perpendicular to each other can meet at a point to define
a solid angle consisting of right angles. 7. It is of all rectilinear solids the most
simple. Even if that is disputed with respect to the pyramid, nevertheless it is easi-
ly substantiated from the fact that the cube is the measure of the pyramid, and it
is accepted that the measure is prior. Indeed it is the measure not only by the con-
vention of men, who whenever they measure a solid conceive its quantity in their
minds in terms of tiny cubes; but it is the measure much more by Nature. For one
Tight angle is equal to any other which is spread out in the same plane. It is
therefore perpetually equal to itself, and therefore one: of the rest there is an in-
finity on both sides. But a measure should be one and the same, and also finite.
8. (1) It is for this reason that the inscribing of a right angle in a circle is so fruit-
ful, for without its intervention neither a triangle, nor a pentagon, nor any of the
figures derived from them can be inscribed. 9. Yet further we should not pass
over the fact that sagacious Nature has most perfectly allocated the same six direc-
tions to the most perfect animal —a clear sign of how she prizes this body. For a
man is himself like a cube, in which there are so to speak six regions: upper,
lower, fore, hind, right, left.
Mystertvm CosmMoGRraPHIcyM. 3r


Quod inter louem & Martem Pyramis.

RESIN Am cur Cubumexcipiat Pyramis, nemo admodum mi+
Pay } rabitur,cum rillafere de principatu aufitcum cubocon-
aN ¥ tendere. 2.Infuper vel ipfa, vel¢#2a2eirregularia faciunt
iG 3B ad cxtercrumcompofitionem. Nam Icofaedroncom-
es apecasy ponunt 20.Pytamides, paulo breuiores Tetracdricis:O-

ctacdrum otto adhuc breuiores. Dodecacdron erfi qua-
drato occulto conftat,tamenin pyramidasrefoluinccefleeft. 3. Neque
conremnendem hoc, quod Tetracdrum in quatuor perfedtas pyrami-
das & vnum O@acdron lacerfidimidiominorum refolui potett. 4.Sicut
in planis omnia multangulain triangularcfoluuntur , ita reliqua folida
menfarandi caufa in pyramidas,quas deinde cubis, ve triangula quadra~
us,metimar.Eftigitur reliquorum menfura, & omnium facilimea cubo
menfilis. 5. Hine pleraqueeiuslinex,vt & cubica tam facilequantitaré
exratione diagoniyaccipiunt, non tamenaliter quam quadratis nume-
ris.6.Pyramidhs etiam regularitas cx folis laceribus pendet:cubi etiam ex
angulis.Atqs fic pyramidum inter zquilatcra non plus vnacft, atin tfde_
equamuis ¢qualibus lateribus,tamen infinita varictas cft Angulorum,
Quonomine,finullzaliz eflentrationes fitne praferéda cubo,an poft-
ponenda,in dubio relinquo. 7. Hanc natura folertiam imitati homines
primum materiam ad perpendiculum erigunt, rectif{queangulis conti-
gnane,deinde trianguhis frmant & ftabiliune. 8. Infuperacutum angulit
cum habeat pyramis,prior eft obtufangulis. Nam id femper primumeft
in ordine,quod iuftam haber quatitatem shoc fequi videtur minusiufto,
quia & longiusabeffe videtur ab infinitate , qua plusiufto, & fimplicius
etiam eft. Nam obtufangulum videtur quodammodo multiplex exreéto
& acuto.Quo minus mirandum,cur paucitas angulorum in bafi,& ipfa-
tum ctiam bafium tctracdri non detoget cubo. Namangulorum & ba-
fium numerusadfufcepram anguli fpeciem neceffario fequitur. Vndefi
redtus prior eftacuto, prius etiam t=i3}2, quam Tetraedron, Tetrago-
nocdrum quam Trigonoedrum. 9. Atqueidetiaminde colligi poteft,
quod perfectum vbique primum, poft,id,quod deficit,demum,quod ex-
cedit. Cumigitur Senarius facierum numerus perfectus fit, fequicur py-
ramidem, que deficit, non quidem pracedere deberecubum ,atimme-
diate fequi.

Habemus curinter loucm & Martem fecundo loco fit pyramis.Su-
piain {nfpenfo fuit, quod corpus tertioloco fitinter Martem & terram.
Uiud vero hic facile deciditur. Cumenimé primarijsrefiduum fit Do+
decaedrum, eritillud ordinetertium, inter Martem & tertam; decuius

prop.ictatibus quid fentiendum fit,collatione cum prioribus facta, faci-
le patebit.




Nobody will now greatly wonder why the pyramid follows the cube, since 1. the
former has almost dared to contend with the cube for the chief place. 2. In addi-
tion either they themselves or the irregular solids which are similar to them con-
tribute to the composition of the rest. For an icosahedron is composed of twenty
pyramids, slightly shorter than in the tetrahedron; and there are eight, which are
shorter still, in the octahedron. Though a dodecahedron is based on a concealed
square, yet it must necessarily be analyzed into pyramids. 3. Nor must we
disregard the fact that a tetrahedron can be analyzed into four perfect pyramids
and one octahedron with edges which are half as long. 4. Just as in plane surfaces
all polygons can be analyzed into triangles, so the other solids are analyzed for
mensuration purposes into pyramids, which we then measure by cubes, just as we
measure triangles by squares. It is therefore the measure of the others, and of
them all the easiest to measure by cubes. 5. Hence most of its lines, and also the
cubes in it, take their magnitude as easily from the dimensions of the diagonal,
though only in terms of the squares of numbers. 6. The regularity of a pyramid
depends only on its edges: that of a cube also on its vertices. Thus there is only
one type of pyramid among those which are equilateral, but in the case of a hex-
ahedron, even though the edges may be equal, there is an infinite variety of
angles. On this showing, if there were no other arguments, I should leave it in
doubt whether the pyramid should be placed before or after the cube. 7. In imita-
tion of this sagacity of nature, men first set up building material perpendicularly,
and join it at right angles, and then fix it and strengthen it by triangles. 8. Fur-
thermore, since a pyramid has an acute angle, it takes precedence over obtuse-
angled solids. For that which has the exact measure is always first in order; and
that which is less than the exact magnitude seems to come next, because it both
seems to be further from infinity than that which is more than the exact, and is
also simpler. For the obtuse angle seems in a sense compounded of a right angle
and an acute angle. So we need not wonder why the fewness of the angles at the
base of a tetrahedron, and also of the bases themselves, does not detract from the
cube. For the number of the angles and bases is necessarily less important than
the type of angle which is formed. Hence if the right angle takes precedence over
the acute, so does the hexahedron over the tetrahedron, and solids formed from
squares over solids formed from triangles. 9. And from this it can also be in-
ferred, that the perfect everywhere has first place, the next, that which is defi-
cient, and the last, that which is in excess. Therefore since the sixfold is the
perfect number of faces, it follows that the pyramid, which is deficient, should
not indeed come before the cube, but immediately after it.

We have shown why the pyramid is in the second position between Jupiter and
Mars. Earlier it was undecided which solid is in the third position between Mars
and the Earth. But that is now easily determined. For since the dodecahedron is
the remaining one of the primaries, that will be the third in order, between Mars
and the Earth. What we should conclude about its properties will easily appear
from a comparison with those which come before it.
32 loan. Kerrert


Defécundariorum ordine @ proprietatibus.

RAS EcvNnDARIA quod attinet, cum Oétaedron fit prius
x Gs 2% Icofaedro,miri alicu videri poffit, cur quod ordine Na-

3S cure pofterius cft, inmundo pracedat? Nam quia Mars
3 Dodccacdron fortitus eft cum Tellure,equiturex ijs que
SS} diximus, inter Telluré & Vencrem interefle Icofacdron.
Ec prius efle Octacdron Icofaedro multa probant. Prima
cnim Ogtacdron natum eft/non vere quidem, fed ita quafinatum fit) ex
Cubo & pyramide primis in {uo ordines quorum illius numerum lacera,
hutusbafin trrangulam mutuatur. Icofaedron vero a pyramide, & Do-
decacdro poftremis in fuo ordine nafcitur, Ruxfam cnim cx illa bafin,ex
hoc numerum laterum mutuatur. 2. Odaedron & Icofaedron fi ex an-
gulisa(picias,illud cubi bafin quadratam oftécat,hoc Dodecaedti quin-
quangulam. 3. Oacdrum cubo xquealcum eft, vevidebimus, & Icofac-
dion Dodecaedro. 4. Odacd:on cum cubo,Icofaedron cum Dodecae-
dro permutantnumcrum bafium & angulorum. Nam Cubibafes & O-
laedrianguli func fex,illius anguli & huiusbafes ofo. Sic Dodecaedri
bates & Icofaedri anguh funtverings duodecim: viciifim illius anguli &
huius bafes funtyiginti. 5. Odacdron Cubi reétum angulum imitatur,
IcoQedron Dodecacdri obtufum.Ex quibus patet Octaedron caput ef-
{ef4i ordinis, ficut cubus primorum eft princeps.

G ‘ni
Kane


Quod Oétaedron fit intra Venerem 8 Mercurium.

4 Vo p autem propterea ftatimad Dodecacdron in mun-
\ 7 do fequi debeat,non fequitur. 1. Nam quiarcuera duo di-
\. Jq\] Nertifuncordines, poflunt ctiam in diuerfas mundi pla-
Grey, Saas {pedtarc fuis capitibus.2.Atque adco, quia Cubus di-
9) gnillima mundi regioni extra Tcrram appropinquat,cir-

cumferentia {cilicet fiuc fixis: par crat,vt & altcrius ordi-
nis caput dignior loco mundi intra Telluris orbem aceederet.Nihilau-
tem dignius centro & Sole. 3. Quod fi ctiam veriufque ordinis fitum pro
vno cenfeamus, quid elegantius heri poterat , quam veille vtrinque fimi-
libus & primis corporibus clauderetur. 4. Pulchrius etiam eft, multifaria
corpora adinuicem fequiin medio, & a pluralitate bafium verinque fen-
fimvad paucitatem difcedi,finihilaliud prohibeat: quam fiad multarum
batium , corpus fequercrur, vaum paucarum bafium, & denique fucce-
deret rurfim aliud Jonge plurium , quam crat verumque. 5. Atque cum
Dedecacdron effec in {uo ordine yltimum, conueniebar, veilli fuccede-
retex

oe



As far as the secondaries are concerned, although the octahedron takes
precedence over the icosahedron, could anyone think it puzzling that the one
which comes after in the order of Nature, comes before in the universe? For
because Mars together with the Earth has been allotted the dodecahedron, it
follows from what we have said, that between Earth and Venus is the
icosahedron. And there are many proofs that the octahedron takes precedence
over the icosahedron. For first the octahedron was born (not literally born, but in
a manner of speaking) from the cube and the pyramid which are first in their
class: it borrows from them the number of edges of the former, and the triangular
base of the latter. On the other hand the icosahedron is born from the pyramid
and the dodecahedron which are the last in their class. For similarly it borrows
from the former its base, and from the latter its number of edges. 2. If you look
at the octahedron and the icosahedron from their vertices, the former shows the
square base of the cube, the latter the five-sided base of the dodecahedron. 3. The
octahedron is the same height as the cube, as we shall see, and the icosahedron as
the dodecahedron. 4. The octahedron interchanges with the cube, the
icosahedron with the dodecahedron, its number of bases and angles. For the cube
has six bases and the icosahedron six vertices; the former eight vertices, the latter
eight bases. Similarly the bases of the dodecahedron and the vertices of the
icosahedron are twelve in each case: correspondingly the vertices of the former
and the bases of the latter are twenty. 5. The octahedron copies the right angle of
the cube, the icosahedron the obtuse angle of the dodecahedron. For these
reasons it is clear that the octahedron is the chief member of its class, just as the
cube is the leader of the first class.


Nevertheless, what should come immediately after the dodecahedron in the
universe does not follow from this argument. 1. For because there are in fact two
different classes, it is even possible that their principal members may face towards
different directions in the universe. 2. And because the cube is close to the most
important region of the universe outside the Earth, that is, the circumference, or
the fixed stars, it was proper that the chief of the other class should come to the
more important position in the universe within the orbit of the Earth. However,
nothing is more important than the center and the Sun. 3. Moreover, if we take
the arrangement of both classes as the same, what could be more elegant than for
it to be bounded on both sides by similar and principal solids? 4. For it is more
beautiful for many-faced solids to follow one after another in the middle, and to
move out bit by bit on both sides from many bases to few bases, if nothing else
prevents it, than if a solid of many bases were followed by one of few bases, and
then there succeeded another of far more bases than either. 5. Also since the
dodecahedron was the last in its class, it was suitable for it to be succeeded by the

Mysrtertvm Cosmooraputcve, 33

retexalcero ordine,quod effect {ui fimile. 6. Etiam hocad Telluris digni-
tatem pertinee, vevtrinquefimilicer, quantum ficri poffer, ftiparetur,
Cum igituritacecidiffet,ve exterius proximeambiretur multifacio,par
erat, veinterius ctiam proxime complecteretur multifacium. Duoigi-
tur hi ordines quinque horum corporum ita funt a fpientiffimo Condi-
tore in vnum redacti, ve calcibus inuicem ad Tellurem, que macetiesi-
pforum eft, obuerterentur, capitibus in diucrfas mundi plagas difce-
derent.

Nota Auttoris.

P surescorporsm diptinctiones, cy hac ipfafufiusinuenies ib.1V Epitomes, aliqua etiam,ortum o#
combinationem [peantia,lib.V Harinon.cap.1.Et infra in hocipfo tibello cap.XI.
In Caput V.Notz Au@oris.

(1) Hinctam fecundaett,Regtiin circulum infeviptio.] Ex anguli {lil rectiaptn
tudine,ce quod omnia in femicincuto redtus eft angulne,


Diftributa corpora tater Planetas , proprietates aptate , demons
firata ex conporibiss cognatio planetarum mutua.

Ow poffum pracetire, quin hic aliqua ex ea Phyfices
j parte, qua eftde Planctarum qualitatibus,delibem ; ve
FS appareat, etiam vires ipforum naturales huncordinem
x feruare,camq; ad inuicem proportionem retinere. Nam
SG cos planetas, quiterramambeunr, illis etiam corpori-
bus , que fibi in{cripta continent accenfeas , inclufis
autem Planetis 4 Telluris orbeillacorporatribuas, quibus vterque cir+
cum{cribitur, quod optima ratione fieri pofle exiftimo: Saturnusha-
bebic Cubum, Iupiter Pyramida, Mars Dodecacdron, Venerem Ico-
faedron , Mercurium Oétaedron. Terra vero cum nihil fie nifi li-
mes , neutri accenfetur. Solem etiam & Lunam Aftrologi maximo
interuallo a cxtcris quinque diftinguunt, ve ita non opus fic illorum
hic meminiffe , & numerus corporum pulchre cum quinque Planetis
conucniat.
Tupiterigitur (1) in medio malcficarum beneficus ipfe multosin
admirationem rapuit, & Ptolemaum etiam ad caufarum inueftigatio~
nemextimulauit. Nos fimile quid videmus in pyramide,quz inter duo
corpora partim cognata, partim abhorrentiainuicem adcoab vtroque
difezepat, ve fere de loco pericliteturin ratiocinijs {uperioribus. Tum
fuperiorum quilibet cum reliquis (2) hoftiliacxercct odia. Tribus ct-
iam corum corporibus nihil penitus conuenit corum, qua appatent.
Mars amen cum Saturnoin folamalitiacon{pirat. Huic ego comparo
E incon-

Chapter IX

one similarly placed in the other class. 6. Also it is fitting for the importance of
the Earth that as far as possible it should have similar attendants on both sides.
Therefore since it had so fallen out that on the outer side it was most closely en-
circled by a many-faced solid, it was proper that on the inner side it should also
embrace a many-faced solid most closely. Thus the two classes of these five solids
have been assembled together by the wisest of Creators in such a way that they
respectively turn their heels towards the Earth, which is the barrier between them,
and with their heads face outwards towards different directions in the universe.


You will find further points of distinction between the solids, and a more extensive treatment of those
above, in Book IV of the Epitome, and some which concern theit origin and combination in Book V of
the Harmonice, Chapter 1. Also below in the present little book, Chapter 13.


(1) Iris for this reason that the inscribing of a right angle in a circle is so fruitful. That is to say, on
account of the adaptability of the right angle, and because every angle in a semicircle is a right angle,


I cannot avoid here abstracting a little from that part of physics which concerns
the properties of the planets, to make it apparent that their natural powers also
observe this order and keep this proportion to each other. For if you allocate
those planets which encircle the Earth to the solids which they contain and which
are inscribed in them, but allot to those planets which are included within the
Earth’s orbit those solids by which they are each circumscribed, which I think
could follow from the best line of reasoning, Saturn will have the cube, Jupiter
the pyramid, Mars the dodecahedron, Venus the icosahedron, Mercury the oc-
tahedron. The Earth, however, since it is only the boundary, is allocated to
neither. Also between the Sun and Moon and the other five the astrologers make
a very great distinction, so that there is no need to mention them in this connec-
tion, and the number of the solids agrees excellently with the five planets.

Jupiter, then, benign (1) in the midst of the malevolent, has driven many to ad-
miration, and also stimulated Ptolemy to enquiry into causes. We see something
similar in the pyramid, which, between two solids which are partly akin and partly
abhorrent to it, is so different from both of them that from our earlier reasoning
its position is almost in peril. Everyone of the three superior planets (2) has hatred
and hostility for the others. Also among their three solids absolutely none of their
observable properties agrees, though Mars conspires with Saturn in malice alone.
To this I relate the variability of their angles, which is peculiar to them, and

54 Toan. Kerrert

inconftantiam angulorum, quz illorum propria , & communis cft veri-
que. Igiturbonitatis argumentum erit contrarium,{e.ftabilitas angulo -
rum in (folislateribus. Argumentum cur lupiter, Venus & Mercurius be-
neficifint. Cubus, Saturni corpus,mctitur omnia reliqua fua rectitudi-
nesEt planetaipfe menfores cfficit, eftque quoad ingenium rigidus, re&ti
cuftos,ne latum yngucm cedens,inexorabilis,iaAexibilis.Sic fertangu-
liredtitudo.

Cognatio evidentiffima eft in bafibus, qua cum Iupiter, Venus,
Mercurius (planctam dico pro corpore) eadem veantur , cauffam ha-
bemuscorum amici, vefupra. Nam ftabilitas ineft triangulo pri-
mum,

Alter gradus eft,planum apparenscum anguloceu vmbilico. Ne
miremur igitur amplius ecquid delitiarum penes durum & igneum
Martem lateat , cuius cauffa delicacula Venus mariti fruftrata thala-
mum cum Marte confpirauerit. Nam Martis quinquangulum eft in
Venere, Sic Saturm quadrangulum in Mercurio conciliat eofdem veri-
quemores. Tertius gradus eft, cum idem ciufdem in duobus eft velap-
paret: Ectumillisin caufis communis amiciconuenit, Igiturin rebus
Jousjs conuenit Veneri cum Mercurio,quia commun Iouis veunturba-
In Saturnijs confentit Mercurius cum Marte parumper, quiain il-
Jo Saturn: quadratum ,inhoc tectus cubuselt. Apparct etiam hinc cur
Veneri cum Saturno nulla cognatio, & qux potitlima, 8& cur Merce-
rij verfatile ingenivin omu:bus quatuor fee apphicet, minimum ramen
Marti.

Etiam Saturnus folitarius cft , amanf{que folitudinis, plane,ve eius
anguli reétitudo non poteft ferre vam inzqualicatem yelminimam,cu-
jus gratia multiplex fiat. Contra Jupiter ¢ genere infinitorum acuto-
rum vnum angulum nactus popularis ideo taétus eft, nioderate tamen
& temperanter, Auétor cnim cft am-citiaram honeftiorum. Ira
Mars & Venus populares & ipfi funt, fed nimium. Nam obtulus &
prodigusipforum angulus intemperantiamnorat. Mercuriusde natu-
ra Sacurni & Iouis eftrationeanguli. Etamantliterati quidem folicu-
dinem , fed inhumani ramen non fant. Amanteos , qui ijfdem ftu-
dijs oble@ancur : modumque ftaruuntin conuerfatiombus, plus quam
Tupiter, cniusomnisattio cftin ccetubus hominum, interque purpura.
tos.

Jupiter & Venus feecundifunt. Sane quia Iupiter facitad plerorum-
que compofitionem 5 Venus autem Iouis quafi foboles eft, cum vna Ve-
nus viginti Ioues breviufculosin fe contincat. Iupiter autem in mares 2-
quior, Venus in fminas;ynde ille mas dicitur,hecfomina. Pyramise-
nimefficax eft, Icofacdroneffedtum, & foboles Ex his ijfdem ptincipijs
aliquanto explicatior caufaredditur, quare Mercurius promifcui fexus
fit,& quare in foccunditate mediocris.

Touis primum,dein Saturni,g¢ demi Mercurij tranquillitas & con-
ftantia morum eft & paucirate planorum: Veneris& Marcis turbulentia
& leuitas a multicudine. Vanum & mutabile fempet farmina. Ecfigura
Vencris omnium maxime varia & volubilis, Atque higradusfunt: ynde
medius Mercurius,media fide.

Mercurij

Chapter IX

common to both. Therefore the contrary, that is the constancy of the angles be-
tween their edges alone, is evidence of benignity, which is evidence that Jupiter,
Venus, and Mercury are benevolent. The cube, the solid of Saturn, is the measure
of all the rest by its uprightness. And the planet itself produces measurers, and is
rigid in temperament, a guardian of the right, not yielding a finger’s breadth, in-
exorable, inflexible. This is the effect of the rightness of its angle.

Kinship is most evident in the bases; and as Jupiter, Venus, and Mercury have
the same base (I use the name of the planet for that of the solid), we know the
reason for their friendship, as above. For stability is a property of the triangle
first and foremost.

The second type of kinship is in showing a plane section which is associated
with its vertex as if with an umbilicus.? Consequently we should not wonder any
longer what attraction lurks in the harsh and fiery Mars, on account of which
dainty Venus betrayed her husband’s bed and intrigued with Mars. For the pen-
tagon of Mars is in Venus. Similarly the square of Saturn in Mercury assimilates
the same behavior in both. The third type is when one and the same feature of a
planet is found or appears in two others; and then they share with each other the
characteristics of their mutual friend. Consequently Venus shares Jovial
characteristics with Mercury, because they both have Jupiter’s base. Mercury
resembles Mars a little in Saturnine characteristics, because the square of Saturn
is in the former, a concealed cube in the latter. It is also evident from this why
Venus has no kinship with Saturn, and which is the strongest kinship, and why
Mercury’s versatile temperament is related to all four, but least to Mars.

Also Saturn is solitary, and a lover of solitude, plainly because the rightness of
its angle cannot bear any irregularity, even the slightest, which might make it in-
constant. On the other hand Jupiter, having taken one out of the infinite class of
acute angles, has therefore become sociable, though moderately and temperately.
For it is responsible for the more honorable friendships. Also Mars and Venus are
themselves sociable, but too much so. For their obtuse and lavish angle betokens
intemperance. Mercury partakes of the nature of Saturn and Jupiter by reason of
its angle. Men of letters do indeed love solitude, but nevertheless they are not
churlish. They love those who love the same studies, and set a limit in their inter-
course, more than does Jupiter, all of whose activity is among assemblies of men,
and among the blue-blooded.

Jupiter and Venus are prolific, plainly because Jupiter contributes to the con-
struction of so many, and Venus is like an offspring of Jupiter, since Venus alone
contains twenty tiny little Jupiters within herself. Jupiter however is more
favorable to males, Venus to women; hence the former is spoken of as male, the
latter as a woman. For the pyramid is the producer, the icosahedron the product
and offspring. On these same principles a clearer explanation is given why Mer-
cury is of both sexes indiscriminately, and why it is not very prolific.

In the case of Jupiter first, of Saturn next, and lastly of Mercury, their calm
and the steadiness of their character are the result of the fewness of their faces; in
the case of Venus and Mars their turbulence and changeability are due to their
large number of faces. Woman is always fickle and capricious;? and the shape of
Venus is the most capricious and variable of all. These, then, are the types of kin-
ship; and hence Mercury is intermediate and of intermediate reliability.


Mercuri verfatile & ccler ingenium refert Oaaedri mobilitas.Na
fi fuper duosangulos voluas, quatuor continua larera per medium figu-
radire&tumitertranfeunt. Cxteras figuras,quomodocunque voluas,vi~
debis permedium tranfucrfa & impedita incedere latera.

Marsmultis latcribus pauciora plana efficic, Venus totidem hate~
ribus plura plana; Martisctiam multi conacusirrici fants Venus conati-
busilli par,profperiore ramen veitur fortuna. Necid mirum effe deber.
Facilius enim chores inftitutitur quam bella, & par crat, citiusad finem
perucnireamores , quam iras ; quia hx perimunthomines, illigignunt.
Eodem paéto Mercurius Saturno felicior cft.

Nota Auctoris.

Er teri hc caput. eis rola, necparsopeiscnfr diet fed exci: ofa
‘tamen illud lector cum Ptolemeirationibus, tam in Tetrabiblo , qucin in Harmonicis : videbit
noftras Ptolensaics non inferiores,ac forte meliorreseffe.

(1) Inmedio maleficarum.] Loquor cum aftrologis. N.m fi meam fententiam dicam,
snullusin coclo maleficus nnbi cenfetur sidque curm ob altasrationes, tum maine propter batt, quia
hominisipfius Natura ef, bicintervis verfans,que radiarionibus Planctaram concliat effeam in
Siishicut auditus,inftrudlusf.ecultate dignofeendi concord.antias vocum , conciliat Muficshanc vim,
villa incitet andientem ad faltandum. De hacre cgi multts in Refponfo ad Obiecta DoGtoris Roftini,
contra librum de Stellanowa , Cr alibipafim , etiamd, in lib.LV Har monicoruim pafin, prefertime
cap. VIL,

(2) Hofiliacxercer odia.] Hoc allegorice intellectum phyyfcisrationibus def.ndi potest:
vt fi fib od vocabulodiferimen gitatccungue intelligatur fitus,mnotus, luminis, coloris, VideleGtor
capi vleimum Ptolemai Harmoricorum, vbiprodicrine queque ind snnotaneri prafertim vlti~
mam meam fpeculationens, de Saturni Gr Martis mutuisexcefsibus vel: defeltibus , louis vero medio
critate,


(1) De origine numerorum nobilium.

eappees: Nr init vM eft fingula perfequi: neq; fine fructude his
(0) A ftrologus amplius cogitet: Videamus modo Aftrono-
USES morum Arithmeticam, facrofquccorum numertos , 6.12.
. os Go. Igitur excepto quadrante& fextante, {cilicet, 15. 10.
EEG omnes fexagenarij partes multiplices reperiuntur in his
“© quinquecorporibus. (2) Viciflim exceptisanguhs planis
Odaedri & cubi , quorum veerquehabet 24. Cateraomnia, quznu-
merantur,funt pars multiplex fexagenarij: vt cxiftimem vix vilinumero
poffenea Pythagora quidems vilam rem naturalem aflignari,qu¢ illi ma-
gis fit propria,quam hicnumerus ft di&tis quinque corporibus.
Vnus eft Cubus, Vina pyramis, Vnum Dodccacdron, Vnum Icofae-
dron, Vnum Odtaedron, Vnum folitarium fine fimili.

Eo: Duo


Chapter X

The quick and variable temperament of Mercury is represented by the mobility
of the octahedron. For if you roll it on two vertices, four edges in continuous suc-
cession trace a path straight through the middle of the figure.* However you roll
the other figures, you will see that the edges pass the middle obliquely and jerkily.

Mars produces fewer faces with many edges, Venus more faces with the same
number of edges; also Mars makes many useless attempts, Venus makes the same
number of attempts, but enjoys better fortune. Nor should that be surprising. For
dances are more easily started than wars, and it was proper that lovemaking
should achieve its goal more quickly than anger, because the latter destroys men,
the former begets them. By the same token Mercury is more successful than
Saturn.


Although this chapter is merely an astrological game, and should be considered not a part of the work
but a digression, yet the reader should compare it with Ptolemy's arguments, both in the Tetrabiblos and
in the Harmony.* He will see that our arguments are not inferior to the Ptolemaic ones, and perhaps
better.

(1) In the midst of the matevolent.| 1 am addressing astrologers. For if | express my own opinion, 1
consider nothing in the heaven malevolent, for the following reason particularly among others, that it is
the nature of Man himself, exercised here on Earth, which by the emanations of the planets gains their in-
fluence for itself; just as the hearing, which is endowed with the ability to discern the concordance of
notes, gains this power of music to stir the hearer to dance. I have dealt fully with this topic* in my reply
to the objections of Doctor Réslin against my book on the new star, and elsewhere generally, as well as in
Book IV of the Harmonice generally, especially in Chapter 7.

(2) Has hatred and hostility.) Mf this is understood allegorically it can be defended by physical
arguments, that is, if the word “hatred” is understood to refer to some difference of position, motion,
brightness, or color. See, reader, the last chapter of Ptolemy’s Harmony, and the notes I have made on it,
especially my last investigation on the amounts by which Saturn and Mars mutually exceed or fall short
of each other, and the way in which Jupiter falls in between.


It is endless to pursue details; yet it is not fruitless for an astrologer to ponder fur-
ther on these topics. Let us now look at the arithmetic of the astronomers, and
their sacred numbers, 6, 12, and 60. Now except for the quarter and sixth, that is
15 and 10, all the aliquot parts of sixty are found in these five solids. (2) Converse-
ly, except for the plane angles of the octahedron and cube, of which each has 24,
everything else, which is countable, is a factor of sixty. I believe therefore that
there is scarcely any number to which any natural entity could be assigned even by
Pythagoras which would be more appropriate to it than this number is to the
aforesaid five solids.

The cube is one, the pyramid one, the dodecahedron one, the icosahedron one,
the octahedron one — one solitary and unique.

36 Ioan. Kerrert

Duo corpora fecundatia;Duo ordines corporum:; Bina femper fi-
bifimilia;Duz ciufmodi fimilitudines.

Tres angulibafium in pyramide, Icofaedro, Oacdro, quiabafes
trilatera, Tria primaria corpora. Tres angulorum differentiz.

Quatuor anguli & latera bafisin Cubo. Quatuor folidi pyramidis
anguli.Q uatuor eiufdem bafes.
re Quingue corpora. Quinque anguli & latera in bafi Dodecae-

rica.

Scxanguli Octaedri. Sex latera pyramidis. Sex bafes cubi. Pulcher
numerus.

Odtobafes Ofaedri. Odo anguli cubi.

Duodecim bafes Dodecaedri.. Duodecim atera Oétaedri.Item &
cubi. Duodecim anguli Icofaedri. Duodecim plani anguli pyramidis.

Eccehicnumerus in omnibus quinque eft.

Viginti bafesIcofaedri. Viginti anguli Dodecaedri.

Viginti quatuor anguli, plani Ogtaedri & cubi. Hic alienuseftnu-
metus, fednec pracipuz rei, necitaalienus; ct enimbis 12. ter 8. quater
6.quiomnes funtin 60.

Trigintalacera Icofaedri & Dodecaedri.

Sexaginta planianguli Dodecaedri & Icofaedri.

Pr¢tereaque nihil numeratur,nififammas omnium laterum &an-
gulorum inire velimus,quod alicnius eft. Tum prouenient anguli deno-
minancium bafium 18. Facies 50. Anguli totidem,latera 90. Anguli pla-
ni180.Numeri cognatiomnes.

Notz Auétoris.

(1) Eetigine Numerorum nobilium.) 1 fiprsicm dium ef, omnis Numeroram

nobilitas(quam pracipue admiratur Thealogia Pythagorica,rebufque dininiscomparat eft
primitusex Geometria, Cum vero multefine cuspartes: ba quidem quingue figure falidenon fine
prinsa nce ynica caufs nobilicaie buns fed accidit,vt malta incundens numserun convent, Prima
vim origoapttidinis muerorummefex fgurisplaniregularibus circuloinfcrptlabusearumque
congruentia,ynde poftea folide oriuntur.Vide lib..co L.Harmonicorum. Nevero confundaris,vbi
legeris, Demonffrationeslaterum,, quibusvtuntur figures arcoft.d numerisangularum: quaftideo
Numerus, vt numerans prior fit & dignior, Minime,non enim ideo murmerabilesfiunt anguli figure,
quia pracefse conceptus ilins numeri,fed ideo fequitur conceptusnumeri,quia res Geometricebabent
illam multiplicitatem in fesexiflentesipfe Numerusnumeratus,

(2) Viciffim exceptis,8c.6 infis, O80 bafes.] Ecce manifefam ballucinationem,
Ofto,noneltpars fexagenary, fed beneparsell numeri t20.quich bisGo.



Chapter X

The secondary solids are two; the classes of solids are two; twofold and in all
cases like each other; two likenesses of the same kind.

The angles of the bases in the pyramid, icosahedron, and octahedron are three,
because the bases are three-sided. The primary solids are three. The different
classes of angle are three.

The angles and the sides of the bases in the cube are four. The solid angles of
the pyramid are four. The bases of the same are four.

The solids are five. The vertices and edges in the base of the dodecahedron are
five.

The vertices of the octahedron are six. The edges of the pyramid are six. The
bases of the cube are six. An excellent number.

The bases of the octahedron are eight. The vertices of the cube are eight.

The bases of the dodecahedron are twelve. The edges of the octahedron are
twelve. So are those of the cube. The vertices of the icosahedron are twelve. The
plane angles of the pyramid are twelve.

Behold—this number is in all five.

The bases of the icosahedron are twenty. The vertices of the dodecahedron are
twenty.

The plane angles of the octahedron and cube are 24. This number is foreign,
but neither in an important respect, nor altogether foreign; for it is twice 12, three
times 8, and four times 6, which are all contained in 60.

The edges of the icosahedron and dodecahedron are 30.

The plane angles of the dodecahedron and icosahedron are sixty.

Apart from these there is nothing countable, unless we wish to proceed to the
sums of all the edges and angles, which is more foreign. In that case the angles of
the defining bases will come to 18, the faces to 50, the vertices to the same, the
edges to 90, the plane angles to 180. All these numbers are akin.


(1) On the origin of the noble numbers. As has been said above, the whole nobility of the numbers
(which is especially a source of wonder in the Pythagorean doctrine, and is there ranked with the divine)
is originally from geometry. Yet there are many parts of it: indeed these five solids are not the first and
unique cause of this nobility, but it happens that many features coincide in the same number. For the first
origin of the aptness of the numbers is in the regular plane figures which may be inscribed in a circle, and
in the way in which they fit them. It is from this that the solids subsequently derive. See Books I and Il of
the Harmonice. Yet do not be confused, when you read that the arguments for the edges which occur in
the figures are derived from the numbers of the angles, as if on that account number, as a means of
counting, were prior and more important. Far from it, for the angles of a figure do not become capable
of being counted, because the concept of number preceded them; but the concept of the number follows,
because geometrical objects have this multiplicity in themselves, and themselves constitute a number
which is counted.

(2) Conversely, except for, etc., and below The bases....are eight.| This is an obvious aberration:
ight is not an aliquot part of sixty, but is in fact of 120, which is twice 60.

538 loan. Kerrert

apparent,verfanturin codem quadrati O&acdrici continuate plano.1d-
gue prxcipucin mulnifacis vecognatisapparct. (11) Namcaterorum
latera diéta non fimul congtue poni poffunc. Dodecaedron. igitur,de-
cem lateribus, talem deferibit viam, per medium tranfeunte quadrato
Oétaedri , in planum ex-

| tenfo:

Icofacdron vero manifeftam Zonam hoc patto, tranfeunte rur-
fum O€aedri quadrato in

LA 0 Z\ rectum extenio:

Quod fi hxc duocognata corporaita applicentur per circumfe-
rentiam(nam anguli duo wntus, & centra planorumduorum alteriusad-
hue, vefupra ,tanquam policohercreintelliguncur)veapparentia bina
quinquangula Icofaedri, & bina vera Dodecacdri, angulis congruant,
progignetur circularis fe-

A A A A Gio ,quzin planum exten-

V Y faycum Octacdri quadrato,
= fichabct.

(14) Sin angulus ynius medio lateri alterius in fupradictis quin-

7 7 quangulis applicetur , talis
Behe ese4 ctit feo.

(3) Quid reftacigicur, qu n dicamus Planetasillam viam totma-
nifclts pundtisnoratam a Creatorciuffosire, pracipue cum inter fupra
aflium ptacolligataque centra & angulos,tanquam polos media fit.

Notz Au@oris.

Op £ fitu corporum, & origine Zodiaci.] Totum hoc caput quantum ad fcopum omit
tipotuit,nullinsenimm momenties, Neque enim bic et gentcinus fitus fen coaptatio inter fe,
corporum, quingue Geonsetriccrum,vt infra patebit:neque fieffer,Zodiacus indeeffer.

(2) Creator,cum mens fit.} Ecce vt fenerauerit mibi per hos 2s.asnosprincipiumiam
tuune firmsipsime perfnafium: ideo foil. Mathematica caufas fieri naturaliuon; (quod dogma Arsfloteles

tot lois vellicauit) quia CreatorDeus Mathematica vt archetyposfecum ab stern habit in abftra=
lions fimplicifuma cy dinina , abipfis etiam quantitatibus, materialiter confideratit. Aviforeles
Creatovein negauit,mundumn atermum flatuit: non miruan, fiarchtyposreiecit:fateer enim vllame

alles view futurams fuiffe , finon Devs ipfein illosrefpexiffer in creando, Ergo etiain Eccentricitatiios
éafeex hee principio tandem inuentafunt; quarum inequalit tem vebementer neccffeStadimira-
rasquicunnque des feria cogitat:qutcuongue ccm Ariffotelederebus caleftibas ic querit:.Quare no

quo quilibct Plancta humilior,co pluribus orbibus vehitur? Naw quiin hoc inguiren-

dum fibiputanitin Aftronomia fui temporis ingaeperfutfioneillafalfa folidorumorbinm:idem ho~
die fiviverct, cy puram arque genuinam nofiramn de cao do€lrinzm cognofeerct , multe maxime fbi
quer.nduneexiftimaret, Quare non, quo quilibet Planeta interior, hoc minorem eriam
Eccentricirarem habee? agus omnibus rationibus,quas ii fa principia fagcercrent, confam-
pista perpetua voce, Quare non; fitandentcdoverctur riflotiles, caufas butus reipulcherrimas
© plaise necefferias ex Harmoniis tex Archetypo redui peffe;puco lnm plenifione affenfu Cr Arche-
1yp05, qitta horam per fervull.efficacia off, Dewan miunds archite Pum receptia van fuiffe.Hec igitut
dehefitpfa> qua rainen ad bypotlefin in boc quideon cap. ve capidivere , non faliciterfuit applicata.
(3) Opor-


Chapter XI

apparent intersections, as seen by someone looking from the center, in the same
extended plane of the square in the octahedron. That is particularly evident in the
case of the many-faced solids, as they are akin. (11) For the edges referred to in
the others cannot be placed so that they all correspond. Therefore the
dodecahedron, with ten of its edges, describes a path like this as the square in the
octahedron opened out into a straight line passes through it."
See top diagram, opposite.

However the icosahedron describes an obvious belt like this, again as the square
in the octahedron opened out in a straight line passes through it:

See second diagram, opposite.
But if these two solids, which are akin to each other, are aligned with respect to
the surrounding sphere (for up to now the two vertices of one, and the centers of
two faces of the other, have been understood to be related as if they were poles) in
such a way that the two apparent pentagons in the icosahedron, and the two real
ones in the dodecahedron, correspond at their angles, a circular cross section will
be generated which, if laid out flat, along with the square in the octahedron, is
like this:

See third diagram, opposite.

(12) But if the angle of one is aligned with the middle of the edge of the other,
in the aforementioned pentagons, then the cross section will be of this kind:

See bottom diagram, opposite.

(13) What remains, then, but to state that the planets have been commanded to
follow that path marked out by such obvious signs by the Creator, especially since
it is midway between the centers and vertices which have been assumed and linked
together above, as if they were its poles.


(1) On the arrangement of the solids, and the origin of the zodiac.| The whole of this chapter, as far
as its aim is concerned, could be omitted, for it carries no weight. For this is not the true arrangement,
but a fitting together of the five geometrical solids among themselves, as will be apparent below; and if it
were, the zodiac would not result from it.

(2) The Creator, since he is a mind.) Notice how the prineple of which I was then already so firmly
persuaded has repaid me with interest over these 25 years—that is, that the reason why the
Mathematicals? are the cause of natural things (a theory which Aristotle carped at in so many places) is
that God the Creator had the Mathematicals with him as archetypes from eternity in their simplest divine
state of abstraction, even from quantities themselves, considered in their material aspect. Aristotle deni
the existence of a Creator, and decided that the universe was eternal? —not surprisingly, if he rejected the
archetypes, for I confess that they would have possessed no force, if God himself had not had regard to
them in the act of Creation. Consequently the causes of the eccentricities were eventually discovered from
this principle; and their irregularity must decidedly be a source of wonder for whoever seriously reflects
on them, whoever with Aristotle asks concerning things in the heavens, “Why is not each planet moved
by more spheres the lower it is?” For he who thought that this was the question he ought to ask in the
astronomy of his own time, and in the false belief in solid spheres, would today, if he were alive and
learnt of our pure and true theory of the heaven, consider that much the most important question to ask
was, “Why does not each planet have a smaller eccentricity the further in it is?” Thus when all the
arguments which his principles suggested were used up against that persistent phrase, “Why not?”, if in
the end Aristotle was persuaded that splendid and plainly necessary causes for this matter could be de-
rived from the harmonies as if from an archetype, I think he would accept with the fullest agreement both
the archetypes and, since they are ineffectual by themselves, God as the architect of the universe. These
remarks, then, refer to the thesis itself; but, as I began to say, their application to the hypothesis in this
particular chapter was not happy.


(3) Oportetaurem principia fineratione conftitucre.] Hoc deijsdidum eft qua
in gencrequaratitatuan,rationem havent materia. Verbs caufa .fphsricuon ipfiempor fe vnurn totum
forge vadiqu:fimile eff formaliter:at materialiter vt fuperfiutss,babet partemextrapartem, Hig
ci ratione partium dominetur n fphavico vafinitas dinsfionis fphsricum igstur ratione ea quainpar~
tes cI dtuiduum,non confideratur formaliter fed ssaterialiter:fiue quod idem, Partesfpbcriat for~
maleswallefunts, que veroin illo confiderantur partesymatcrials funt , inqnantum figura fphericé
veitur materia quantitatina,diaidiquepotell, Lam vero atti mfcribstur Cubus phsrica ; fifpheri-
cx formaliter confiderattr vt figuaa , locus queftioni non fF , quabus nam in punttislatuendi fine
angultcubi,fin autem matercaliter confideres, vt fiperficeean inpisitoram pundtorum :tune quidem
qies{tioni locus est , quabusin punéta? at refponderi non potest , carn ratio nulla fit, cur potiusin bie
petnntis, quam in als: quippe potcét on infinitisalvs atque alts,

Huts generis funt > sflequeftiones;, Cum fingitur patinm vltramundanum infinitum,@
ceo queritsr cur potins inbac parte (pacij, quam in alia collocatus it mundus.cem cum tempus a=
sera oppofitumn in aduclo fingitur queriturque, Cur demum ance fex millis anmorum conditus
fit mundus, Deo ab omni arernirate abftinante x creandoz Nam Cy fpatili G tempus, ingencre quan-
f1t.stumrationem habent materia, refp.Eu quidem figuratarum ynantizatum, Materia vero de fe
rationesnul.s furpedstat,ipfa in fe vnam Cr folam proprietacem babet,tmjinstatem partinm, aBtua-
Lem quadem,vel numeri,vel quantitatis ,ffipfiam totum atu infinitum:potentialem vero numeri, fi
tun alts fitum quod folusn est ppibsle , cum quantitas extra matertacerporal: phyfica vel cale=
‘ffi-Videlib.Fpitom.x, Aftr fel. 40.764 defiguracaliagitur.

(4) Neinfinisus fiatregrellus} Ratio Arifloteli familiaris bic impertinenter adbi-
betur simone principrum quidein datur alicuins regreffisin af.gnandisratiombus, vbi ratio plane

nulla ci.

(5) Etvealiquando tranficum habeanvus.] Si, inqiam, nomedtinitium operis faci
endam fine rations yullum ynquam initium erit facienduin  rationescnin ad boc vel slud initium,
vbidantur infintea,plane nulle fiont. Quod igitur in infisitispundtis fiers equepoffe, id cum fit
corm v0 aliquo, prater omnentrationemeit, quod ineo potifimum jit prateritis alia,

(6) Debecbafin bafi cubi parallelam. ] Atqui Geon.tria cet locarionemn Pyrani~
disin Cubo longe coucinniorem ¢ porf:Biorem:concinniorem , quia querasia elf inferiptionis Geo-
nactriceillinsin iffo,eadem etiam in mundo concinna erit: At Geometrice Pyramis Cuboficinferibie
sur,ot,quodlibet Lats Pyramnidis fiat diagonios vninsplani cubici:perfechiovemn vero, qui fimaxime
Dafs wna Pyransidis fist parallels bafivnt Cubs: tamen adbucincertacftlocatio laterum bafistrian=
‘glectriunirefpsitu laterum bafisquadrangule quatuor.Porestenion quodliberillorums,cuilibet bo
‘0m parallelumn featuispotest & angulorum vniobtendi:vt perpendicular potins plant triangularie
cus Latere Cubi in id:mn planuum competat. Deniqueperfetba lacatio non eft , vbinon omnibus planis
ions fieuscomtingune at cum ynum Pyramidisplanum firparallellum piano cubi reliqua ils,
‘nll buins crunt para'telas1dean Cy de Latoribus G de anguils diftum efto.

(7) Dodecaedron bafi Pyramidis.] Hiciam fieusab veraque figura abborret , Ga
Pyramide , G4 Cubo, Nav inferiptio Geometrica docet , angulospotius quatuor Pyramidis debere
izangi (vel fx >erpont)totidem angulis de dodecaedri viginti. Sic cadem inforiprio Geometrica Cubi in
Dadecatdron cocet draganios Dodecaedri afto de duodecimn  fieri ofto latera Cubi:itaque fi Dodecate
dro vicigim fic intra Cubuan; oportet de trigintalateribus Dodecaedri fona fubordimarifenis planis
Cubificuparallelo.

(8) Sufpendendumerit O€acdron.] Hoc patto refpondebit quidem fitus OFatdri
intimin Cubo cxtimo,in{cription: Geometricaciu{demin Cubo : at Pyramids, Dodecesdro,lcofae~
dronon legitime accommodabitur mfifitusillorum in Cubo ad legesisn prefcriptasemendetur, Tie
enim concurrentin yna redta linesex centro (Smuni figurarumomnusm edudla, 1. angulus O8at-
dri, Latcrum Icofedri,3 Dodecatdri,g. Pyramidis,mediapuntha, 5.centrum planicubici:eruntg,
ealinon loncaruin fox, fitus undig, fibnpfifimilie. .

(9) Qua quis inter didtos angulos & centra,mediacenfere potelt.] Quia in Py-
rranide per bun vitiofim fitumimpedimsur,yt medialatera nequeamnus cenfere,

(10) Siregulariter ponantur.] Tancfaneetiam in Pyramideinuenientur quatuor me-
dia latera;tunc etiam fitusfigurarum in fe mutuo,refpicit leges unferiptionum Geometricarum.

(11) Nam cxterorum didtalateranon fimul congrue poni poffune.] Nonpof-

fun


Chapter XI

(3) Now it is necessary (0 establish some principles without reason.| This was said of things which are
in the class of quantities and have a material aspect. For example, as form the spherical in itself is a single
whole and is alike in all directions; but as matter, being a surface, it has sepatate parts. In this case since,
in respect of its parts, an infinity of division dominates in the spherical, therefore from that aspect in
which itis divisible into parts, the spherical is not considered as form, but as matter; or, which is the same
thing, formally the spherical has no parts, but what are considered as parts in it are material, inasmuch as
the shape of the spherical takes on quantifiable matter, and can be divided. But in actual fact the cube is
inscribed in the spherical. If the spherical is considered formally, as a shape, there is no room to ask at
what points the vertices of the cube are to be placed; but if you consider it materially, as a surface of in-
finite points, then indeed there is room to ask, “At what points?” Yet there can be no answer, since there
is no reason why it should be at certain points rather than at others, for it can be at an infinity of different
points.

The following questions are also of that kind. Since space outside the universe is supposed to be in-
finite, the question arises in its case also, why the universe has been located in this part of space, rather
than in another. Further, since time is supposed to be eternal (the opposite to its use as an adjective), the
question arises why the universe was established six thousand years ago, and God abstained through all
eternity from creation? For both space and time in the class of quantities have a material aspect, in
respect that is of quantities which have shape. Matter, however, supplies no reason for itself, but in its
‘own self has a single unique property, the infinity of its parts, actual indeed, either in number, or in
quantity, if in totality it is actually infinite, but potential in number, if in totality it is actually finite,
which alone is possible, since quantity resides in corporeal matter, physical or heavenly. See the Epitome
of Astronomy, Book 1, page 40, where the shape of the heaven is discussed.

(4) So as to avoid an infinite regress.| It is inappropriate to apply the familiar argument of Aristotle
here, Rather not even a starting point is offered for a regress, in assigning reasons, where there is plainly
no reason,

(8) And fo have at some point a transition.) If, Lsay, the task should not be begun without a reason, it
should never be begun; for plainly there are no reasons for this or that beginning, when an infinite
number are offered. Therefore as it could equally well be at an infinity of points when it is at a particular
one of them; there is no possible reason why it is at that one particularly to the exclusion of the others.

(©) Must have its base parallel to the base of the cube.| But geometry teaches us a far more ap-
propriate and more perfect way of locating the pyramid in the cube: more appropriate, because the
reason in geometry for inscribing the former in the latter will also be appropriate in the universe (and in
geometry the pyramid is inscribed in the cube in such a way that each edge of the pyramid becomes a
diagonal of one face of the cube); more perfect, because even if one face of the pyramid is made parallel
to one face of the cube, yet the location of the three sides of the triangular face with respect to the four
sides of the quadrangular base is uncertain. For any of the former can be established as parallel to any of
the latter, and can also subtend one of the vertices, in such a way that it is rather a perpendicular of a
triangular face which falls in with the edge of the cube in the same plane. Lastly, the location is not
perfect when similar positions do not occur for all the faces; but if one face of the pyramid is made
parallel to a face of the cube, the remaining faces of the former will be parallel to none of the latter’s. The
same applies to both the edges and the vertices.

(7) The dodecahedron to the base of the pyramid. This position is now antagonistic to each of the
two figures, both the pyramid and the cube. For the geometrical method of inscription teaches us that the
four vertices should rather be linked with (or superimposed on) the same number of vertices among the
twenty of the dodecahedron. Similarly the same geometrical method of inscription of the cube in the
dodecahedron teaches us that eight of the twelve diagonals of the dodecahedron become eight edges of
the cube. Therefore if the dodecahedron is in its turn inside the cube, six of the thirty edges of the
dodecahedron should be arranged opposite the six faces of the cube in a parallel position.

(8) The octahedron. . .must be suspended.) On the same showing the position of the octahedron in-
side within the cube outside will correspond with the geometrical method of inscription of the same in the
cube; but it will not correctly fit the pyramid, dodecahedron, and icosahedron unless their position
within the cube is emended in accordance with the rules now prescribed. For in that case there will fall on
one straight line drawn from the common center of the figures 1. a vertex of the octahedron, 2. the mid-
point of edges of the icosahedron, and 3. of the dodecahedron, and 4, of the pyramid, $. the center of a
face of the cube; and there will be six of such lines, and their position will be alike in all respect.

(9) Which may be reckoned as intermediate between the vertices and centers referred to.) Because in
the case of the pyramid, on account of its imperfect orientation we are prevented from being able to
reckon the edges as intermediate.

(10) If they are regularly placed.) In that case indeed even in the case of the pyramid four intermediate
faces will be found; and in that case the orientation of the figures towards each other will mutually
respect the rules of geometrical inscriptions.

40 Ioan. Keprerr

fant inquam congruere Latera visius omnia,Lateribus alterius, minimeomninm Fyramidis. Scilicet
ideo congrueponi nonpoffunt,quis initinin poftionisnonfaltumcftregulare,

(12) Sinangulus vnius mediolatcrialterius.] Hicequidem legitimus duorum he-
rum corporums ftus eft ad femutuo:at Octaedri fitus qui bic adfea{citur ilegitirans ef, '

(43) Quid reftatigicur,quin dicamus. } Omnino multareflant,quo minus boc dicere
popinsns.N.rm fitns,quipolos hic fignat,ilegitimus est. Quatenus vero induobus, Dodecedro Cr Ico-
feedro,firus eff legitimus;totidein poffint eff poli, quot anguli huins plana illius,duodecim fe. quare
Zonsintermedic fex:Eritt igitur incerti Planete,quorfum eant, Ingencre objfat hoc, quod figura ifle
reali ftupartium ad fe mutuo,non unt expreffein mundo, fed folum proportio orbinm figuralium ex
disdefiompta imorbescacleftesfuie tranflata,numerusg,orbinen 2 figuris conflitutus.Rectiusigitur be
quaftionem; cur hanc potius, quam aliam viam currant planetae , vt abferdam repellinims,
Namum flit inintentione Deistrcilus,motibusplanet aru neceffaris; ili Deus per imentionemn
conftituto materiale &flellarum [pharicum circumiecit. Nec dubitatio aliqua Deum ab operertti-
uit quo niinus initinon einsfacerepofet, quafifineratione: nanstunc corpus nulluon preexiftebat,
cus ile artinme refpedhu dubitaret. Spatium vero fine corpore puracst negatios{ating,rationtselt ad
{faciendumsinitium in infinito Nibilo, vel cogitare liter de aliqua:taleenim cogitatum ian flatin
infinitis modis eft pre(Lantins,reliquo infinito non actu,necexiflenti,nec cogitato, G> fic prins illo,
initio aptum,Neg, vero prints ego fumt,qui mieipfim hac inuttli queftione fatigans,Cur {cilic.hac

traduétus fit Zodiacus,cum potuctitalialocis infinitis:Inueniasfimilems buinsin Ariflor
Cur hanc potiusin plagam eant Planete,quamin cius contrariam? Namne hic quidem
ratio eft vlla vnins pre aleero,cumt onnnis linea, longitudinis cBdivione, dss cbrineat plagas,qiee fut
invettaverfius duos eins terminos, Fatetur quidem sbi Arifloteles in genere,von oimnitan rationes,eo~
dem modo queripoffe:adoritur tamenquefhoneimbanc; Naturam ut interpopubilia fimper quod
optimumcligercimeline vero ofe vt ferantur fidera mplagam dignior®; at qui digniorcm effeplagam
prorficem, quam retror fiom, Ridicule. Nam prius quam motus effet neutra plaga, ned,prorfum neque
etvorfuin dicebatursprincipium petitur. Argutatur quidem a fimilicudine nundt wn animalibas,
Animalia cunsplagisfiisfex,jidcam mundi flatuens. Atqui ric fans principivon petitur, Densus enim
aaundim effefactim ad finulitudiné animalis; dicat igitur prius de ipfo animal, car hoc ill fit prov
fier, jllud retror fins non vicfsi, hoc e8,car oculi,aurefg, Cr naves,cr lingta. > os verfus tot
iginem in fpeculo dirigantur, brachiorum mannum digitoruang, articlsillorfum flectantur , peduas
alone illor fim extend.antur,& non potius,vt imagins in feculo membraeadem, retro verfits hoini—
‘nem potuit enimn etiam fic feri:boc eff poruit cor quod nunc eit in finiftra,collocarsin ede qui nunc
putamusdextram. Etvtconfletratioin bac Idea mundi,quid? annon aque facile contrria potuit
ius ad Latera mundi fieriapplicatio? quid impedinit quo minus iniftram ad Meridwem tenderct, de=
xtram ad Septentrionems quando plagas mundi metari iff ef? fc enim faciem vertiftin plagam,
(que nobis nuncoccafus dicitursficcontrariam fidera plagam profam habuifint, in quam morib.fuis
tenderent.Retinsitaque fuperfedifet Arifloteesfolurione butusinepta queffionis: five ipfis admtoni~
sioni obtemperans. Nam inter ea,queomniaex equo contingere poffent, natura nullan inuenie Me
Tioris cr Deterioris clectionem;hec enim inuoluit contradictionem, Quinimo fic argumentemur: Cit
Ensnon Ente,praftet: nondum igitur exiflente Mundo,quacunque eins plag.a prof concepta fuit ini
tio, poriores nunc ex [ua parterationes habe, cu prof t,quam eis contraria,boc ipfo,quia con
travia ius concipitur cffesm non Ente: qua fi tri profafatta effet, Mundus tamen, ifirnilis hute prefenti
fal fet.Comp ratio locum non habet Mundorum, vbi nus foluse. Valvaneitaque queftiones
uinfinodi materiales, Gr cam ifdem etiam metatio Zodiaci, feupotins , (quia ic loc fuisexcedit
ai fuce fia vie Regie,a Solaris corporis cinculo inter eins polos medio monftrate, Nam [poli ce a~
xis corporis folarisin plagasrmunds alia verfi fuiffentetiamn vie Regiaalia fuift radudla. Quod
idem & defiguris Dodecacdro GIcofacdro dicendit. Demuscnim,munus pferum effeymetsari Zodia~
‘au edtionibus mutuislateruan, > cert quidem ordinisex fox quos divin eff pope
Uilesscerte tranflato figurarum fituin Mundofenfii,fedes etiam alia
Zodiacoobtingeret,



Chapter XT

(11) For the edges referred 10 in the others cannot be placed so that they all correspond.) 1 mean that
the faces of one cannot all correspond with the faces of another, least of all in the case of the pyramid,
Obviously, the reason why they cannot be placed so that they correspond is that the starting point taken
for their location is not governed by rule.

(12) But if the vertex of one is aligned with the middle of the edge of the other] In this case indeed the
mutual orientation of these two towards each other is legitimate; but the orientation of the octahedron,
which is the case in point here, is not legitimate.

(13) What remains, then, but to state.| Decidedly, many obstacles remain to our saying this. For the

position which in this case represents the poles is not legitimate. Certainly, insofar as in two cases, those of
the dodecahedron and the icosahedron, the orientation is legitimate, there can be as many poles as the lat-
ter has vertices, or the former faces, that is twelve, so that there can be six pathways in between them; and
consequently the planets will be uncertain where to go. In general the difficulty is that these figures are not
‘embodied in the universe with a real mutual orientation of their parts to each other; but the proportion of
the spheres based on the figures has been taken from them and applied to the celestial spheres, and the
number of the spheres has been established from the figures. It is therefore more proper for us to reject as,
absurd the question, “Why should the planets traverse this path rather than another?” For since in God's,
design a circle was necessary for the motions of the planets, when he had established it in accordance with
his design he surrounded it with a material and starry sphere. Nor did any doubt restrain God from the
task and prevent his taking a starting point, so to speak, without a reason; for at that time there was no
body previously in existence to cause him doubt over the relationship to its parts. Indeed mere space
without body is a contradiction; and in an infinite Nothing itis sufficient reason for taking a starting point
even to consider one fleetingly; for one which is thus fleetingly considered is at once distinguished in an in-
finity of ways from the infinite number remaining which have no reality either in existence or in considera-
tion, and is consequently prior to them, and suitable for a starting point. Nor indeed am I the first to have
tired myself with this useless question, “Why was the zodiac drawn round in this position, when it could
have been elsewhere, in an infinite number of positions?” You will find a similar one in Aristotle:: “Why
do the planets move in this direction, rather than the opposite?” For even here there is no argument for one
rather than the other, since every line, in virtue of having length, possesses two directions, which point
directly towards its two ends. In fact Aristotle there admits in general that arguments cannot be expected
to account for everything in the same way. However, the following question arises: “Nature,” he says,
“always chooses among the possibilities that which is best. It is better that the stars should travel in the
more fitting direction; and furthermore the forward direction is more fitting than the backward.” This is
ridiculous. For before there was motion, neither direction was spoken of as forwards or backwards. The
argument is circular. Indeed he makes great play of the analogy between the universe and animals, setting
up animals with their six directions as the Idea of the universe.* But again the argument is circular. For let
us grant that the universe was made on the analogy of an animal. Then let him say first of the animal itself
why this is its forwards direction, and that its backwards; that is, why the eyes, ears, and nostrils, and
tongue and mouth, point towards their image in a mirror, why the joints of arms, hands, and fingers bend
in that particular direction, why the soles of the feet extend in that particular direction, and not rather, as
do the same members in their image in a mirror, in the opposite direction to a man. For that could have
been the case: that is, the heart, which is now on the left, could have been located in the position which we
now think of as the right. And to refer the argument to this Idea of the universe —after all, could it not
equally easily have applied to the sides of the universe the other way round? What was there to prevent its
turning the left hand side to the south, the right hand side to the north, when it was instructed to mark out
the directions of the universe? For in that case it would have turned its front in the direction which we now
speak of as the west; and in that case the stars would have had the opposite direction as their forward direc-
tion, towards which they would turn in their motions. It would therefore have been more sensible for Aris-
totle to have left off trying to answer this pointless question, conforming with his own advice. For among.
things which could all equally happen, Nature found no way of choosing better or worse; for that involves
contradiction. Let us rather argue as follows. Since Being has precedence over not Being, therefore when
the universe did not yet exist, whatever direction in it was conceived as the forward direction at the start,
now has arguments on its side, why it should be forwards, rather than the opposite direction, for this very
reason, that its opposite is conceived to be in not Being; and if the latter had been made forwards, yet the
universe would have become similar to what it now is. There is no room for comparison of universes when
there is only one. Then let us say farewell to material questions of this sort, and with them to the marking,
‘out of the zodiac, or rather (since the zodiac departs from its position with the passing of the ages) of the
Royal Path,? which is shown by the circle of the solar body between its poles. For if the poles and the axis
of the solar body had been turned towards different directions in the universe, the Royal Path would also
have been drawn in a different position. The same also applies to the figures of the dodecahedron and
icosahedron. For if we grant that their function is to mark out the zodiac by the mutual intersections of
their edges, and in particular order, out of the six which as we have said are possible, certainly if the posi-
tion of the figures in the sensible universe were shifted, the zodiac would also be differently situated

Mysrrrivm Cosmocrarnicye, 4!

Diwifio Zodiaci,ey afpettus.

Vitr diuifionem Zodiaciin duodecim precifa f=
gna pro figmento humano habuere, talincmpe, cut
nibilreinaturalis fubfic. Neque enim hae #2 viri-
4) bus,aut affectionibus differre naturalibus arbitran-
\\ cur 5 fed aflumpta propter numeri ad rariones apti=
tudinem. (1) Quibus etfinonomnino Tepugno,ta-
men ne quidtemerereijciatur, ex ij{dem principijs
diuifionis huius caufam proponam , ad quam Creatorem proprieta=
tes (fi quas ill diftingtas habent ) accommedaffe vero non erit ab-
fimile.

Numerorum fubic&um quodnam fic,fupra vidimus. Ec (2) cer-
te preter quantum,aur quanto fimile, pocentia qualicunque praditumy
nihil eftin toto vniverfo numerabile, prater Deum, qui ipfiftima vene-
randa Trinitas cft. Iamigitur (3) corpora omnia diflecuimus per Zo-
diacum. Videamus, (4) ecquid feétione hac Zodiacus ipfe adeptus vel
paflus fit. Sectorumigitur dicto modo, Cubi facies ex fectione refulrans
eit quadrata,vt & Octaedri,Pyramidis triangula,Reliquorum duorum
decangula.Quater tria decies faciuntfummam centum & viginti. Igi-
tut infcripta ie aaes Curae une idem pti-
um,arcusvariosin circumferentia diftinguunt,quos
omnes metitur portionon maior centefima vicclima
totius circuli. Naturalis igitur diuifio Zodiaci in 120.
exregulari fitu corporum interorbes. Cuiustriplum
cum fit 360. videmus hanc diuifionem non omnino
nullarationcniti.Iam fi quadratum & triangulum rur-
fum ex eodem punéto feparatim defcribamus , por-
tio circuli minima crit pars duodecimaambitus,nempeSignum. Vt mi«
rum fit, (5) &motum Solis & Lunzmenftruum, & (6) coniunétio-
nes magnas Superiorum tam apte quadrarcad portiones,qug ab corun-
dem corporibus pertriangulum & quadratum diftinguuncur. '

(7) Atqueadeo quam hac duodenaria diuifio penes naturam in
pretio fit,exemplo cape extranco ; vt quamuis caufanon omninocogni-
tafit,tamen occafio pateat ,fubinde praclarius de his quing; figuris fen~
tiendi.

Eo propofita fides aliqua ,ciufque fonusT ve. Igitur quot=g=
occrrruntyeces aW vigsad o€tauam confonantes cumY (8) to. =
ties, [epius,potes fidem rationalitcr diuidere, fic vt divife fidis par~
tes inter fe & cum integraconfonent. Porro quotnamilliufmodi yo-
ces occurrantaurces indicant. Ego {chemare & numeris dicam.



a

Vide nunc & ipfas harmonias , & fidium proportiones in nut
F meri



Many have held the division of the zodiac into twelve exactly delimited signs to be
an invention of Man, that is, of a kind which has no basis in Nature. For they con-
sider that these portions do not differ in their natural powers or influences, but
are assumed on account of the ready divisibility of the number. (1) Although I do
not altogether object to that, nevertheless to avoid hasty rejection of the idea I
shall from the same principles propose a reason for this division, so that it will not
seem unlikely that the Creator should have accommodated their properties to it
(if they have any distinguishing properties.)

What is the concern of numbers we have seen above. And certainly (2) apart
from quantity, or what is similar to a quantity, and endowed with a power of
some kind, there is nothing in the whole universe capable of being numbered ex-
cept God, who in himself is the venerable Trinity. Now (3) we have already cut
through all the solids with the zodiac. Let us see (4) what the zodiac itself has ac-
quired or experienced in this cutting. Then of the solids cut in the way stated, the
figure resulting from the cutting of the cube will be square, and similarly those
from the octahedron and pyramid will be triangles, those from the remaining two
decagons. Four times three tens make a total of a hundred and twenty. Then if a
square, a triangle, and a decagon are inscribed in a circle, from the same point,
they mark off different arcs at the circumference, the common measure of which
is a portion no greater than one hundred and twentieth of the whole circle. The
division of the zodiac into 120 is therefore natural because of the regular align-
ment of the solids among the orbits. As three times that is 360, we see that this
division does not wholly lack rational foundation. Now if we similarly describe a
square and a triangle separately from the same point, the smallest portion of the
circle will be a twelfth of the circumference, that is a sign. So it is a wonderful
thing that (5) the motion of the Sun, and the monthly path of the Moon, and (6)
the great conjunctions of the superior planets are so neatly fitted to the portions
which are marked off from their solids by the triangle and square.

(7) Furthermore how greatly this twelvefold division is valued by Nature may
be gathered from an extraneous example, so that although the reason is not com-
pletely known, yet there will be an obvious opportunity for forming at once a
much clearer opinion about these five figures. Suppose there is a string, and its
note is G (ut).? Then the number of notes from G to the octave which are concor-
dant with G is (8) the number of times, and no fewer, you can divide the string in-
to rational fractions so that the divided parts of the string are concordant both
with each other and with the whole. Furthermore our ears tell us how many of
such notes occur. I shall make the point with a diagram and numbers.

See bottom diagram, opposite.

Now look at the actual harmonies, and the proportions of the strings in

numbers, where the lowest symbol represents the note of the whole string, the

42 Ioan. Keprert

meris:vbi Nota ima fignificat vocem integre fidis; fuprema, vocem par-
tis breutoris; media, vocem partis longioris; Numerus imus indicarin
quot partes fides diuidenda fitsreliqui,longitudines partium.

= saa
es (seh
(9) Atque hx folz voces mihi naturales videntur,propterea quod
habentindubitatum numerum. C2terx voces non poffunt certa pro-
portioncad iam pofitasexprimi. (10) Nam vocem F fave, alam ex C
{ol fa vt, defuper, aliam ex B mi molli inferius clicies, vecunque hz duz
perfecta quintz effe videantur.Sed ad rem. Prima & fecundaconcordia
quodammodo fociz funt;fic etiam quinta&fexta. (11) Cumenimim-
perfcétz omnes fine: bina: femper,vna dura,altera mollis,confpirant, vt
fingulis perfetis quodammodo xquiparentur. Nccadmodum diuerfas
diuifioncs habent.Nam ! & 3 fefehabentad inuicem,ve 38 & 38, quieta-
tum yna trigefimadifferunt. Sic } & } fc habentadinuicem, ve33 &38-
Diftcruncigitur cantum yna quadragefia particula. Atque ita proprie
loquendo, tantum quinquein Mpfica habemus concordias, ad nume-
rum quinquecorporum. (12) Quod fifeptem diuifionum in 6.5. 4.3.8.
j-2-communem minimum diuiduum quasras, rurfum inuenics 120. ve
{upra,cum de diuifione Zodiaci ageremussperfe€tarum vero concordia-
rum minimum dividuum rurfum 12. (13) plane quafi perfeétz concor-
diw a quadrato & triangulo Cubi, Tetraedti & Octacdri,imperfedte ve-
roa decangulo reliquorum duorum corporum proucnirent.Atquchze
fecunda eft corporum cognatiocumconcordiis Muficis. (14) Sed quia
caufas huis cognationis ignoramus, difficile eftaccommodare fingulas
hatmonias fingulis corporibus.

(15) Videmus quidem duos harmoniarum ordines, tres fimplices
perfectas,& duas duplices imperfeétas; ficut tria primaria corpora, duo
{ecundariasverum cum reliquanon conueniant,deferenda efthec con-
ciliatio, &aliatentanda, Nempe ficut Dodecacdron & Icofaedron fuo
decangulo fupra auxcrunt duodenarium viqi ad 120. ita hic imperfecta
harmoniz idem faciunt.

Eruntigiturad Cubum,Pyramida & Odtaedron accommodandz
perfeétx harmoniz ad Dodecacdron & Icofacdron imperfeétz. Quo
accedit &illud, atquehercle (16) indicem digitum ad caufam harum
rerum occultiffimam intendit, quod proximo capite habebimus: (17)
duos nempe effe Geometriz thefauros, vnum, fubrenfein redtangulo
rationem ad larera ; altcrum , lineam extrema & media ratione {e-
&am , quorum exillo Cubi, Pyramidis & O@acdri conftruétio Auit,
ex hoc vero conftructio Dodecaedri & Icofaedri. Vnde tam facilis &
regularis eft infcriptio Pyramidis in cubum, O€tacdriinvtrumque, fic-
ut Dodecaedri in Icofaedron. (18) Ve autem fingulz Harmonix
fingulis corporibus accommodentur , non ita in promptu eft. (19)
Mlud folum patet, Pyramidi deberi harmoniam , quam quintam di-
cunt,quartam in ordine,quiain ca minor portio cht ; pars integra, ficut
lacus trianguli (quo Pyramis vticur)fubtendit } circu. Hoc plurainfra

confir-


Chapter XIT

highest the note of the shorter part, and the middle symbol the note of the longer
part, the lowest number indicates into how many parts the string is to be divided,
and the others the lengths of their parts.
See diagram opposite.
(9) Now these seem to me to be the only natural notes, on account of their hav-
ing an unmistakable number. The other notes cannot be expressed by a definite
proportion to those already set out. (10) For the note F (fa ut) is different if you
derive it by coming down from C (sol fa ut) and if you derive it by going up from
B flat (mi) even though both of them seem to be perfect fifths. But back to the
point. The first and second concords have some affinity, and so have the fifth and
sixth. (11) For though they are all imperfect, if one of the two is major, the other
minor, the two always combine together so that they are in a sense equivalent to
single perfect concords.? Nor are the differences between them very wide. For 1/6
and 1/5 are to each other as 5/30 to 6/30, which only differ by one-thirtieth.
Similarly 3/8 and 2/5 are to each other as 15/40 to 16/40. Hence they differ by
only one-fortieth part. Thus properly speaking we have only five concords in
Music, corresponding to the number of the five solids. (12) But if you look for the
lowest common multiple of the seven divisions into 6, 5, 4, 3, 8, 5, and 2, you will
again find that it is 120, as above, when we were dealing with the division of the
zodiac; but of the perfect concords the lowest common multiple is again 12, (13)
plainly because the perfect concords come from the square and triangle of the
cube, tetrahedron, and octahedron, but the imperfect ones from the decagon of
the other two solids. This is the second point of kinship of the solids with the
musical concords. (14) But because we do not know the causes of this kinship, it
is difficult to fit the individual harmonies to the individual solids.

(15) Indeed we see two classes of harmonies, three simple and perfect, and two
double and imperfect, just as there are three primary solids and two secondaries;
but as the other points do not agree, this matching must be abandoned, and
another tried. Now just as the dodecahedron and the icosahedron above by their
decagon increased the twelvefold to 120, so in this case the imperfect harmonies
do the same.

Therefore the perfect harmonies must be fitted to the cube, pyramid, and octa-
hedron, the imperfect to the dodecahedron and the icosahedron. From that
another point follows (and heavens! (16) it points the finger at the most secret
cause of these matters), which we shall include in the next chapter: (17) that is,
that there are two treasure houses of geometry: one, the ratio of the hypotenuse
in a right-angled triangle to the sides, and the other, the line divided in the mean
and extreme ratio. From the former of these is derived the construction of the
cube, pyramid, and octahedron, and from the latter the construction of the
dodecahedron and icosahedron. That is why the inscription of the pyramid in the
cube, and of the octahedron in either of them, is so easy and straightforward, like
that of the dodecahedron in the icosahedron. (18) However, the fitting of the in-
dividual harmonies to the individual solids is less readily achieved. (19) Only one
thing is obvious, that to the pyramid should be attributed the harmony which
they call a fifth, the fourth in order, because in it the smaller portion is a third of
the whole, just as the side of the triangle which the pyramid employs subtends

Mysterivm CosmoGrarHicye. 4g

confirmabunt,vbideafpectibus agemus, qua vt hic ctidm intelligamuss

omnino ita cogitemus,quafi fides fit nonrectalinea,fed cicculus. Dabit

igitur diuifioharmoniz di&tztriangulum : in quo angulus laceri oppo-

nitur,plane ve in pyramideangulus plano. Remanétigitur Cu-

bo & Octaedro octaua & quarra diéte, teria & f{eptima ino: &

dine. Sed verumcorum verara harmoniam tenebit? yerum d:

cemus (20) fecundaria reciperecas,qua lineas {cribant,& pri-

maria,quz figuras?cum Cubo debebitur quarta dicta. Nam fi ex fide cir-

culum facias, & ex vna quarta rectam vique aliam ducas tamdiu, donec
in primum punétum redeas,fict quadrangulum, quate
planum etiam Cubus obtinet. Contra Octaedro de-
bebituroétaua,qua eft dimidiz fidis. Namin circulo
dudtus ad dimidiam, éad idem punétum facit nil nit
lineam, SicDodecaedro debebitur priorimperfed&a
duplex. Nam duétus per quintas & per fexcas circuli
faciunt quinquangulum & fexangulum. Reftabirigi-
turIcofacdro pofterior imperfeéta duplex , quia du-
tus perduas quintasrepetiti vfqucinidem pundtum;
(21) faciunt tancum lineas. t- Sic & duétus per tres
ofauas.* (22) Anmalumus Odaedroquartamda-
re,quiais duodecies quarta circuli{ubtendit. Id quod

nullum lacus cubi facit? Sicrelinquetur Cubo o¢taua.
+ harmonia perfectiffima,ve ipfe perfectiffimum corpus

oft. Forfan &illudconucnientius eft, (23) relinque-
reIcofaedro priorcmimperfectam propter fexangu-
lum, quod bafitriangulz cognatum magiscft, quam
quinquangulz: Dodecacdro vero dare diuifionem o~
€onariam proptet numerum cubicum 8. quia cubus
Dodecaedro in{criptilis. Hacfanein medio fita fintt
doneccaufas quis repericrit.

(24) Veniamus modoad afpeétus. Erquandoqui-
dem modo ex fidecirculum fecimus ; facile eft videre, (25) quomodo,
tres perfcétx harmonix pulcherrime cum tribus perfectis afpedtibus.
comparati poffint,fcil.cum 92,11. Imperfedta vero prior B. mollisad
ynguem fimilis eft {extili,cuiushacnota, (26) *quemquedebilifimum
efleferunt.

Habemus caufam (27) (qualem quidem Prolemzus non dedit)
cur planet diftantes vno autquinquefignisnon cenfeantur in afpeau.
Nam yt vidimus, (28) nullamtalemin vocibus agnofcit Natura con-
cordiam. Cum enim incateris eadem fit ratio influcntia & harmonia+
rumscredibile eft & hiceffe. (29) Caufaverinque procul dubio cadem
eft,& cx quinque corporibus,quam alijs quarendam relinquo. (30) Cé
igitur omnes quatuor harmonig confonent luis afpedtibus,& veroadhue
tres reftentin Mofica harmoniz s fufpicatus aliquando fum , non negli-
gendum efle in indicijsnatiuitatum, fi Plancta 7 2.aut144.aut 35.gradi~
bus diftent, prafertim cum videam, vnam eximperfedis habere fuum
afpectum. Quamuis cuilibet oculato Metcororum fpeculatori facile pa-
tebit,vtrum aliqua in his tribus radijs vis ee te afpeétus acris

a moras


Chapter XIT

one-third of a circle. This will be confirmed below in many ways, when we deal
with the aspects; but to gain acceptance of it here as well let us think of the string
in all respects as if it is not a straight line but a circle. Therefore the division re-
quired for the harmony named will yield a triangle, in which the angle is opposite
to the side clearly in the same way as in a pyramid the vertex is opposite to the
face. Therefore for the cube and octahedron there remain the harmonies named
the octave and the fourth, which are seventh and third in order. But which of
them will take which harmony? Shall we say that the secondaries (20) accept those
which describe lines, and the primaries those which describe figures? Then the
harmony named a fourth will be attributed to the cube. For if you make a circle
of the string, and draw the chord of each quadrant, one after the other, until you
come back to the starting point, the result will be a square, which is also the kind
of surface the cube contains. On the other hand to the octahedron will be at-
tributed the octave, which is half the string. For in a circle, drawing the chord ina
semicircle and another back to the starting point produces nothing but a straight
line. Thus to the dodecahedron will be attributed the former of the imperfect
pairs. For lines drawn in fifths and sixths of a circle make a pentagon and a hex-
agon. Therefore the icosahedron will remain the second imperfect pair, because
drawing a line repeatedly in two-fifths of a circle, back to the same point, (21)
makes only lines. So does drawing lines in three-eighths. (22) Or do we prefer to
allot the fourth to the octahedron, because it subtends a quadrant of a circle
twelve times,‘ which the edge of a cube never does? In that case for the cube will
be left the octave, the most perfect harmony, as it is the most perfect solid.
Perhaps it will also be more appropriate (23) to leave for the icosahedron the
former of the imperfect pairs, on account of the hexagon, which is more akin to
the triangular base than to the pentagonal base; but to give to the dodecahedron
the eightfold division on account of the number eight’s being a cube, because a
cube may be inscribed in a dodecahedron. These are in fact open questions, until
someone finds the causes.

(24) Let us now come to the aspects. Since we have just made a circle of the
string, it is easy to see (25) how the three perfect harmonies can be most beautiful-
ly related to the three perfect aspects, that is to opposition, trine, and quartile. Of
the imperfect harmonies, the first, B flat® resembles to the last detail the sextile,
of which this is the sign. >, (26) and which they say is the weakest.

We have the reason (27) (which Ptolemy at least did not give) why planets
which are one sign or five signs apart are not reckoned under an aspect. For as we
have seen, (28) Nature recognizes no such concord among musical notes. For
since in the other cases the ratio is the same for the influence and the harmonies, it
is easy to believe that it is in this case as well. (29) The cause in both cases is un-
doubtedly the same, and depends on the five solids; but I leave it for others to
seek. (30) Since, then, all four harmonies agree with their aspects, and yet there
still remain three musical harmonies, I suspected at one time that we should not
overlook it in casting horoscopes if planets are 72° or 144° or 135° apart,
especially since I see that one of the imperfect harmonies has their aspect.
However, to any observer of things on high who has eyes, it will be easily ap-
parent whether there is any power in these three aspects, since in the case of the
other aspects it is verified by invariable experience of the changing atmosphere.

44 Toan. Keprert

mutationes conftantiffima ratificent experientias (31) Caufe quidem
quas probabiliter quis reddat,quod 3 $3 in fide fonent, in Zodiaconon
operentur,hz effe poffint.

1. Oppofitus folus,duo quadrata, trinus cum fextili, abfoluunt fin-
guli femicirculum:at tres hi radij nullum habentfocium adhoc munus,
quem Mufica non penitus repudiet.

2. Reliqui radij rationem habent facilem ex diametro,latus quin-
quanguli,& fubrendens duo lateraquinquanguli, tria octanguli,funcin
graduremotiore & irrationales.

3. Caufa, quia trinus cum fextili, quadracum cum quadrato cffi-
ciuncreétum angulum,Radij reliqui nullo paéto cum vllarecepta linea.
4Amperfeéta B mollis eft quodammodo perfecta, quia vtitureadem di-
uifione cum perfectis,& eft dimidiaquinta. Vndenon mirum, folam ex
imperfedisrefpondereafpectui alicui,(c. fexuli,qui itidem eft dimidius
trinus.Cgtera cnimnec apt funtin duodenarium,nec perfecti alicuius
pars fant.

3. Denique fex trigonianguli,quatuor quadrati,tres fexanguli, &
duobus femicirculis comprehenfa duo fpacia implent omnem ia plani-
ticlocum. Attresanguh in quinquangulominores funt quatuor redctis,
quatuorfunt maiores. Wade &illud paret, quare nec odtangularis, (33)
necduodecangularis radius, nec vilus reliquorum operctur. (34) Atqs
hicfere feparo caufasa{pectua acaufisconcordiari. (35) Certeenim,
quaexangulis fit,genuina radijs eftratocinatioscum propterangulum
in punéo fuperftcici terrenz factum,in quo mifcentur,exiftatoperatio,
(36) nonvero propter figuramin Zodiaco circulo defcriptam,qug ima~
ginatione potius quam rei veritate conftat. Diuifio vero fidis nec in cir-
culo fit, nec angulis veicur, fed in plano per reétam lineam perficitur.

(37) Poffunttamen nihilominus & concordantia & afpeétus habere
commune quid,quod eadem vetinque caufatur,ve fupra dictum. Id vero
aliorum induftrix relinquo ferutandum, (38) Pcolemxi Mufica , que
Regiomontanus cum expofitione Porphyrij,cditurus erat, fed nondum
excufa Cardanus afferitin hac matcria proculdubio verfantur. Vide et-
iam (39) quidcx Euclidis Muficis huc referri poffit.

Notz Auétoris.

(1) Cy Vibusetfinon omnino repugno.] Hocthemaex prof fotradauiin libro defiel-

Q: noua ingue refponfo ad obiecta Roflina:nempe, quatuor quidem circili Zodtaci qua
drantes monftrari iconditiunibusduorum motuuan,diurni, Co Solisannni, quas fequuntur etiam
Luminis & Calcfaitionis meta: at quadrantum fingulorun fibdinifioncm internaprecifefigna ni-
hideale necex motanec ex virabus habere,cuius effectus cenferi pofisnifi tantum gencralijimamil-
Lam diftindtionemn, quanti vniufeuin{quein principinm, Med:um, c> Frnem-quas tainen partes nulla
necefiesiubet aqualeseffe,ac ne partes quidern:fuffict enimn,yt pro medio cenfeatur., totaquadran-
tis lnvea,pro principio c> fine,duo linea termins feupunts,qua nen funtparsde lines.

(2) Prater quancum,aur quanto fimile, potentia qualicunquc predicum, ni-
hilef in toro vniuerfo numerabile. }Ridicula mubifententis excidit vere non fntentia. Quid
enim eit, Nihil practer Omnia? Numeratio,adtio Ment, fupernenit rebus onniabas,distinis &
brurmanis: mulls leuipimsa quidem diftindtio ei, feu realit, fou ntcutionalia (fitalla prima , vel fe«

sundae


Chapter XI

(31) The reason which one may with probability suggest why 3/8, 1/5, and 2/5
produce notes on a string, but do not operate in the zodiac, may be the following:

1. A single opposition, two quartiles, or a trine combined with a sextile each
make up a semicircle; but these three aspects cannot combine together for that
purpose with anything which Music will not completely repudiate.

2. The other aspects are simply related to the diameter; but the side of a pen-
tagon, and the diagonal stretching under two sides of a pentagon, or three sides of
an octagon, have a more distant relationship and are irrational.

3rd reason: that a trine with a sextile, a quartile with a quartile, form a right
angle; the other aspects do not by any device with the addition of any line.®

4, The imperfect harmony B flat’ is in a way perfect, because it uses the same
division as the perfect harmonies, and is half a fifth. So it is not surprising that
alone of the imperfect harmonies it corresponds with an aspect, that is, the sex-
tile, which in the same way is half a trine. For the rest neither fit into the
twelvefold division, nor are part of a perfect chord.

5. Lastly six of the angles of a triangle, four of the angles of a square, three of
those of a hexagon, and the two angular distances included in two semicircles
complete the whole circuit in a plane. But three of the angles in a pentagon are
less than four right angles, and four of them are greater. From that the reason is
clear (32) why neither an octile (33) nor a duodecile aspect, nor any of the others,
operates.* (34) It is precisely here that I make a distinction between the reasons
for the aspects and the reasons for the concords. (35) For the argument based on
the angles is sound in the case of the aspects, since their operation is due to an
angle formed at a point on the Earth’s surface, at which they meet, (36) and not to
a figure drawn on the circle of the zodiac, which exists in imagination rather than
in reality. However, the division of the string is not done on a circle, and does not
use angles, but is carried out on the flat along a straight line. (37) Yet concords
and aspects may nevertheless have something in common, which suggests the
same reasons in each case, as has been said above. However I leave the examina-
tion of that problem to the industry of others. (38) Ptolemy’s Music, which
Regiomontanus? was going to publish with Porphyrius’s exegesis, but which Car-
danus'® asserts has not yet been printed, undoubtedly discusses this topic. See
also (39) what can be applied to it from Euclid’s Music.


(1) Although I do not altogether object to that.| This theme I have dealt with openly in my book on
the New Star, and in my Reply to the objections of Roslin. That is, the four quadrants of the zodiac
cle are indeed shown by the specifications of two motions, the diurnal motion and the annual motion of
the Sun, which are also followed by the turning points of its light and heat-giving; but for the subdivision
of each quadrant into precisely three signs, no such warrant is given either by the motion or by their
powers, of which the implication could be taken into account, with the single exception of the very
general distinction of any quantity whatever into beginning, middle, and end. However there is no
necessity which dictates that these parts should be equal, or even parts; for itis sufficient that the whole
line representing the quadrant should be taken as the middle, and for the beginning and end the two ex-
tremes of the lines, or points, which are not a part of the line.

(2) Apart from quantity, or what is similar to quantity, and endowed with a power of some kind, there
is nothing in the whole universe which is capable of being numbered.) | tet slip a ridiculous opinion, in
truth not an opinion. For what is “Nothing except for everything”? Counting, an action of thé mind, ap-
plies to everything, divine and human. There is no distinction, not even the slightest, either real or in in-
tention (whether it be of first intention, or second, or third, or whichever intention you like) which does


level tertig vel quora libet nntentionis,)qus non quandam fimilitudinem babest cum dinifione
rectginp.nees, Videque de num.risdifputati lib IV. Harmoutcorum Cap. fol.iv7. Hoc autem
abseratine Aitnta,ciuns banc fententram conciperem; quicquid nsmsratur 4 nobis (prater diuinas
pesfor.ssin SS. Cranstate rd refpectum aliquem habere quantitatinwn , falters in intentione numes
rants,

(3) Corpora diffecuimus per Zodiacum.] Per imaginationem plant per feltiones
illssheteru & per couteum figuraramomainm tradudti , & vfgue fab fixasextenfi, cuins feltia
cua jpheraficarain nobis pepurit in conceptione illa Eclipticam,

Zodiacas ip{eadeptus.] Si nimirum ex centro communi.
perf ctenes icH plani cum Latersbus figurarumeincsantur vfque fb fas: adden
scan atcinsSctoatin onmnesquing, figura tals irregalartfitu iaicem co.ptentur, vt fingularuam fine
gla latera fectionibus fits let m yn tali recta linea : tune enim Zodiacus diflinguetur in partes
toilets main iskctat ity & vicefirna totins. Cunt autern fitus ifle fit irregularis; regularie
vero per angilos Dodecactr C Icafaedrioctonos vtringue inplanum dictum incidentes, diflinguat
Zodtacum se wrration.sliaspst-t bane disifionem non fepropriam quingue figurarum. Em igisur
it Epttont. Yr fal. 8 1.derizonftraui propriam off: figuranunplanarim, Regularinin demons
fieabiluan fille crete infirisancur ab yno etus puncto.

{5) Motum Solis & Lana menftraum.] Solisintellige annuum, Nam dam Solan-
nctmpermeat:Livaa duodccria vaentfes conficit fers, Adeoque bancdiftributionean anni , Cr accommo
dstiv:nem inotsiin Solis & Lure. faltemin primo proportionis illarui concepta , Ego archetypicam
festuasexquebacordinacione,c3 ex concur fi naturaliumn caufarua motricuan,catfascrue quarun=
dain inssqualitatuin in Lunasyt vaoiti a Prolegoments Ephemeridum, é doceo plene tn Epit. aftr.
fs AV Sumi qued abidem mnstentas etiam de proportione anni ad reuolutiones diurnas 560. (inpri=
ma mzzntione qusbus accad std indeob concur fim caufarum,, reuolutiones 5. & quadrans: vnde
elitr noua sq:eatio temporis.E5j3dalibsro adbuc,objertationefqueexpendo,

(6) Coniunétioncs magnas{uperiorum.] Hocqusdem accidentarium eft, non are
chetyptcuin. Noam ve doceo li). iL.svinonicorum: , Periodica Planetaruin tempora fint ex Harmo
mucus coutemperationabis itotttitie cxtremorumin Apheliis enim debuit offi mornin proportio que
auad sfore, 1m Persiclsisvero , ques.ad 12. vt filter miter Saturni Aphelium & louis peribelinm
pojfet effe Diaperte Eps Diapafon, rater vero Saturni peribeliuim & lonis Aphelium , perfedtum Diag
4,00 quia biedise Harmonie Cito cognatefint. Hac enim prima G Archetypica in motibuse®
can fa. Oita. fiigitir ve Apbclisison morn, fic totarin peviodorum proportio fet qua rad 5 .t6ne,
in.wanisGo.cu itngerentt pracife dive reuolutiones Saturni,quingue vero Louis; tn annis 12. pna lo=
seis: Co Satcrnts c Lepiter conssancti verbs caufa , in principio Artetis, pracife pot 20. annos in ipfo
principio Sagittary coment stiri. Lupiter enim fuperato Saturno,dum Zodracum emenfus Satur
ruin figicr-tean per fequatur : r'le vuterimex Artete abjt tatu, ve Iupiter in quinquerenolutiont~
stor fol::ntmodo a ipjiem,quiaeffugit Saturnusper duas ex quingue; ita replant tres con=
juisdTiones in guingue Lowialibes periodisperfecto triangule diftribute, Ecce vt hic triangularis con-
jneelronuin tus fit necefariamn confequens caufe archetypicaex Harmoniis defumpta; accidat vere
ini Za. fe per pyrarntdas fou per tring, fquisilam, vt in boc capite ponebam, Ar
cbetypicam (ffecontenderit. Vicifim ftotarsan periodorum t & IL proportio effet lla, que propter
Harmonicas contemperationesdebutt effe motsium Periheliorw, fc. 5..ad 12. tune inannis1s0 t=
piterreuerterttr duodecses colin anni 1afomis,Ablatisigiturs.de ra.roflarent 7 totes felupi=
ter affequerctur Saturnion Iraque Zodtacus per has coniunctones dinideretur in partes 7.quarum
qitnia,id SB 257 gradibus bine coniundtiones dé iauicen remouerentur verbicafa, post ynamin
© V, coutigever alzerains 7.7 -tertia in 4.n Sed quis periodica tempora componuntur ex mo~
tibus tain Apbeliis,qu.am peribeliisexque interiedlis onnibus;bine nafcitur etiam mtermediaperion
dorsin proportio,coniunitionumng,per Zodiscum diftributio, vt prima in principio Arietis collocata,
fecsinda nequein pfurnprincipizein fagittari veniat , necetiam vfqueint7. 2 excurrat, fedmedia
G equabiliratione ad tresgradus yltratriangularemlocumprogrediatur. Quod tipfa Zodiaci di-
‘lanitioin eres ricntes,por figuras Geometricas,genina Cy archerypica caufa fuifer hui difpofitio~
nis coniunctionums; vtaque cxprepiffer ila perfectum triangulum; non aberrat eniin dininum opis ab
archerypo fia.Non igitur amplius mirum effe debet cur Saturni louisg, congreffus ad triangulum als
ludantsqiuia necperfetta cy plane accidentaria cH allufe


F 3 (7) Atqie

Chapter XI

not have some resemblance to the division of a straight line into parts. See my discussion of numbers? in

Book IV of the Harmonice, Chapter 1, page 117. However, what I had in mind when I conceived this
opinion was that whatever is counted by us (except the divine persons of the Holy Trinity) has some
quantitative aspect, at least in the intention of the one counting.

(3) We have already cut through all the solids with the zodiac.) By imagining a plane drawn through
these intersections of the edges and through the center of all the figures, and extended right up to the
fixed stars, the intersection of which as thus conceived with the sphere of the fixed stars has produced for
us the ecliptic.

(4) Whar the zodiac itself has acquired. ..in this cutting.) Certainly if straight lines are taken from the
common center of the figures through the intersections of the plane mentioned with the edges of the
figures, right up to the fixed stars. However, we must add the following words: also if all the five figures
fitted together in such an irregular arrangement among themselves that the individual edges of individual
figures at their intersections fall on a single straight line. For in that case the zodiac will be divided into
parts which are measured only in units of one hundred and twentieths of the whole. As, however, that ar-
rangement is irregular, and the regular arrangement, in which eight each of the vertices of the dodeca-
hedron and icosahedron fall on the said plane, divides the zodiac into irrational parts, it is evident that
this is not the division which is proper to the five figures. I have therefore shown in the Epitome of
Astronomy, Book Il, page 181, that it is proper to the plane, regular figures which are capable of being
constructed, if they are inscribed in a circle from a single point on it

(5) The motion of the Sun and the monthly path of the Moon.] Understand “annual” motion of the
Sun. For while the Sun passes through a year, the Moon completes twelve months about. And so I
establish this apportionment of the year and accommodation of the motions of the Sun and Moon as the
archetype; and from this orderliness and from the concurrence of the natural causes of motion, I extract,
the causes of certain irregularities in the Moon, as I have commented in the Prolegomena 10 the
Ephemerides, and report fully in the Epitome of Astronomy, Book IV." You will also find in the same
place a similar point on the ratio of the year to the 360 diurnal revolutions (in the first intention), to
which are added on account of a concurrence of causes 5 1/4 revolutions. Hence a new equation of time
is elicited. However 1 am still pondering, and weighing up the observations.

(6) The great conjunctions of the superior planets.] This indeed is accidental, not archetypal. For as |
report in Book V of the Harmonice, the periodic times of the planets are derived from the harmonic con-
sonances of the extreme motions. For at the aphelia the ratio of the motions should have been as 2 is to 5,
about; and at the perihelia as $ is to 12. Thus, for instance, between the aphelion of Saturn and the
perihelion of Jupiter could be the chord of the fifth above the octave and the tonic, * and between the
perihelion of Saturn and the aphelion of Jupiter the perfect octave, as these two harmonies are akin to
the cube. For this is the first and archetypal cause among the motions, Then if the ratio of the motions of
the aphelia were as that of the whole periods, which is as 2 is to 5, then in 60 years there would occur
precisely two revolutions of Saturn, and five of Jupiter; and in 12 years one revolution of Jupiter. If there
were a conjunction of Saturn and Jupiter, for example at the beginning of Aries, after precisely 20 years
they would come together again at the beginning of Sagittarius. For while Jupiter, after overtaking
Saturn, has traversed the zodiac and again pursues the fleeing Saturn, it has meanwhile departed from
Aries by such an amount that Jupiter in five revolutions catches up to Saturn only three times, because
Saturn escapes for two revolutions out of five. So there remain three conjunctions in five of Jupiter's
periods, distributed in a perfect triangle, Notice that this triangular arrangement of the conjunction is a
necessary consequence of the archetypal cause, derived from the harmonies; whereas it is an accident of
the trisection of the zodiac, whether in accordance with the pyramid or with the triangle, if it is main-
tained, as I supposed in this chapter, that it is archetypal. On the other hand, if the ratio of the total
periods of Saturn and Jupiter were what the ratio of the motions of the perihelia should have been on ac-
count of the harmonic consonances, that is as 5 to 12, then in 150 years Jupiter would complete twelve
revolutions, or one in 12¥% years. Then on substracting 5 from 12 the remainder would be 7, that is,
Jupiter would catch up to Saturn that many times. Consequently the zodiac would be divided by these
conjunctions into seven parts, and five of them, that is 257°, would be the separation of two conjune-
tions from each other. For example, after one in 0° of Aries, another would occur in 17° of Sagittarius, a
third in 4° of Virgo. But because the periodic times are compounded of both the motions at the aphelia
and perihelia, and of all the motions in between, it also arises that the ratio of the periods is intermediate,
and that the conjunctions are distributed round the zodiac in such a way that if the first is located at the
beginning of Aries, the second neither comes to the beginning of Sagittarius nor presses on as far as 17°
oof Sagittarius, but in an intermediate and uniform proportion goes forward three degrees beyond the

triangular position. But if the division of the zodiac itself into three thirds, in accordance with
geometrical figures, had been the true and archetypal cause, it would in any case have expressed the
perfect triangle; for the divine work does not deviate from its archetype. There should not therefore be
any further reason to Wonder why the meetings of Saturn and Jupiter make sport with a triangle; for
sporting is not perfect and is plainly accidental

46 loan. Keprert

(7) Acqueadeo quam hae.) Hic fun ipfiftina principia mei operis Harmonict, eaque
on tantumn opinationsns,qite poflerioribus emporibuscorrigende fuerint, fd etiam verifima rest
pfi:Onmnisenins philofophica fPeculatio deber initia capere a fenfinmn experiment: bit vero, que
ferfusauditus teflecur denumero vocum , cumyna aligua covfonantinms quaitem fonfus oculo-
rum , de longitudine chordsrum confonantium ; enendatipone cr plene expriffian ha~
bes.

(8) Toties;nec fepius.} Mirumei equident, cum tor ex antiquo extiterint feriptores
Harmionicorum nufpiam penes ipfos occurrere obferuationem hanc,de numero feéhionsm Hatrmont-
caruinplanefundanentaler, cy gute recba ad catfas ducitseum tam fir obwivms cuilabet, tdin chor
dis quacunqueexten{,cuinsfpatium fubie€umn carcino dinidi poe fimplici applicatione reidura,yt
cultrt aut cauis ad chordam,manuyna , & percipione partivanetusinterflinctarit, cua pledtroin
manu altera,experimentariLeag, funni fu sta felicitas in principio fpeculationis t@denti ad opus
Harmonicum feribenduon: quaminis tunc quidem nonduanid anime deflinaneram, Canfaautcm,
cur fiptemordine voces, vfque.ad Diapaforscum ima fufcept.e confonent,eit ft, quia chorda fepries
Harmonice dinidi potelE; finguliseninn it actibus finguli conflitauntur fom, confonantescuan fouo
totins. Videlib.1t, Harm.cap.1,

(9) Atqueha fol] Verumest, Naturale id dices,quod prima flatim coaptatione fe-
ionum,inipfo quafiveftigio caufarumpragrefferum cliciturs vt diflinguatur ab eo,qued fecidaria
ratione,velut artificialiter Cr imitatione Natura conflituitur. At finon ordinem ortus , fed propor
tionem ipfam refpicias, naturaliacrunt &illainterualta dicenda , quueproportiones fic ante conjli
sutas, imitation Nature fiefeipiunt. Vein fequela vocum Re, Mi, Fa , Sol, La Naturale eSt inter-
uallum, Fat, Sol,Tonus maior dict, quippe primitus conflituitur, quardointcrallum Re, Fa ,ad=
bucnondums edt diuificm : iam etiam inter Re, Fa, defignetur vox Mi, tali proportionechorde Ms,
ad chordam Re quali cM Sol,ad Fa, tune & ipfavox Mi Nusturalis baberi debet. Quod vero canfam
hic reddididiftindionis,quafiFa,Sol,babeant indubitatos nxmeros, Mivero,non item:id condonan-
dum e8 tyrocinio tunc pofito. Xam lib1LL.Harinon.cap.V «ce VMcatifusoptionas tiadidt, quibus et
iam foro Mi, fimilibus funsindubitatus numerus afiignatur.

(10) Nam vocem F fayt,aliam ex.] Hoc verum cit, fivtrinque vellesperfedum Dia~
pente conflituere. Atqui,quod tunc ignorabam,parsnon minimaeftdifcipline,de Confouantiis adul-
terinis,quamtradidy lib I-Harmion.cap.XIL.

(11) Cum enim imperfeéte omnes fint.] Ita vfitate appellantur 5 veteres ne pro
Confonaritis quidem habuerunt. Inaneo Opere Harmonices,fol.83.pofteriorinec minus cy cap. 1.
G1. libri 11. & pafion etiam imperfectss appelland, fedvox ifla non aque valet adultering.
Dest enim adultering mininum aliquid , quo minus fit plena confonantia , mibil dee tortie
GS fexte ligitime , quo minus inter confonantias referamur. Itaque diftinitionis caufa praflat
tertias C fextas, minoresdicere confonantias , idue non quantitatis tantumrfpestu, fed evar
foucici.

(12) Quod fi feptem diuifionum.] Hunc ego neruit argumentitunc conflitui, Diniditur
Zodiacus in partes12.G-120 .diuiditur c chorda in torsd® harmentcc:ergo numer bi fast apitd naa
turamnin pretio, At cum Zodiac dinifi fit 4 quing, corporibus( vtatunc exiffimabam)verifimilesin~
didem G Chords dinifioncm effe, & fic quingue sllas figurasetiam Harmoniarum Ideaceffe;, tune
quidem fequi videbatur.Sed nusicex opcre Harmonico lector catfas Harmonicornm genuinas petat:
fo enim non illa quingue corpora Geometrica: fed potius figura plane in civculum inferipte,

be

(13) Plane quafi perfeéta concordia 4 Quadrato & Triangulo.] Incundum ef,
primasinuentionun conatuseriam errantesineueri Ecce caufas genuinas Cr archerypicas concordan.
tiarum, quas manibus verfabam cecutiens, velut abfentes,anxiequefiut. Figure plane funt caufa
concordantiarun feipfissnon quatenus fiunt folidarum figurarwn fupcrficier, Fruftra ad folida re~
Sexi in conflituendis Harmonicis motuum preportionibus,

(14) Sed quia caufas huius cognationisignoramus.} Atgui caufas iais nomina-
1.16 Vides, figuy asplatas: Atquinon cognatio non confanguinitas, fed nuda affinitases?. Figura exim
plane ex rnapartedinidunt circulum harmonice,cx altera parte congruuntin figuras quingue foli-

dae. Ergo & Harmonica circuli dinifio, C quinque figura, in vno tertio, in figuriefeilplanis cone
unt,

(15) Vide-

Chapter XIT

(7) Furthermore how greatly.] Here are exactly the principles of my work on harmony, and principles

not of mere opinions, which will be open to correction in later times, but the genuine principles of the ac-
tual fact. For every philosophical speculation ought to take its starting point from what is experienced by
the senses: in this case, what the sense of hearing testifies about the number of notes which are in har-
mony with a given note, and also what the sense of the eyes testifies about the length of strings which are
in harmony, you know with great accuracy and in complete detail

(8) The number of times, and no fewer.) It is indeed surprising, as there have been so many writers on
harmony since antiquity, that nowhere in their works does there occur this observation on the number of
the harmonic divisions, though it is plainly fundamental, and leads straight to the causes; and as it is so
easy for anyone to make trial of it, on any stretched string, of which the length can be covered by a pair
of compasses, and divided by the simple application of a hard object, such as a knife or a key, to the
string with one hand, and the striking of the parts of it which are divided off with a plectrum in the other
hand. Consequently this was a great piece of good fortune at the beginning of my investigation to one
who was leaning towards writing a work on harmony, though I had not yet decided on it in my mind.
Now the reason why seven notes in order up to the octave are in harmony with the lowest which is
sounded is this, that the string can be divided seven times harmonically; for by each act of division is
established a particular sound which is in harmony with the sound of the whole. See Book III of the Har-
monice, Chapter 2.

(9) Now these... the only.) This is true, if you call natural that which is produced immediately at the
first fitting together of the divisions, in the very footsteps, so to speak, of the preceding causes; so that it
is distinguished from that which is set up by a secondary ratio as if artificially and in imitation of Nature,
However, if you take account not of the order but of the actual proportion, those intervals should also be
called natural which take the ratios thus previously set up in imitation of Nature. Thus in the sequence of,
notes re, mi, fa, sol, la, the natural interval is fa, sol, called a whole tone; for it is set up, first of all, when
the interval re, fa has still not yet been divided off. If the note mi were now also to be designated between
re and fa, with the string for mi in the same proportion to the string for re as sol to fa, then the note mi
itself ought also to be taken as natural. Indeed my here giving this as the reason for the distinction, as if
fa and sol have undoubted numbers, but mi has not, must be pardoned because I was then serving my ap-
prenticeship. For in Book II] of the Harmonice, Chapters $ and 7, I have expounded excellent reasons
for assigning their own undoubted number to mi and similar notes.

(10) For the note F (fa ut) is different if...) Thisis true, if you intend to set up a perfect fifth in both
cases. But, although I did not know it then, a considerable part of the discipline concerns adjusted con-
sonances,* and I have expounded it in Book III of the Harmonice, Chapter 12.

(11) For though they are all imperfect.) This is what they are usually called. The ancients did not even
accept them as consonances. In my work Harmonice, the second page 83, and equally in Chapters I and 4
of Book III and generally, I have also called them imperfect; but that name is not so appropriate to an
adjusted concord. For an adjusted consonance is a very small amount short of being a full consonance;
and the third and sixth lack nothing which would stop them being included among the consonances. Thus
the distinction is best made by calling thirds and minor sixths consonances, and that not only in respect of
quantity, but also of species.

(12) But if. .of the seven divisions. At that time, | made this the mainspring of the argument, The
zodiac is divided into 12 parts and 120 parts; division of a string into the same number of parts is har-
monic; therefore, these numbers are important in Nature. But since the division of the zodiac is based on
the five solids (as I then thought), by the same argument it is probable that the division of the string is
also; and so it then seemed to follow that the five solids were the Ideas of the harmonies. But the reader
should now look for the true causes of the harmonies in my work on harmony; for they are not the five
geometrical solids, but rather the plane figures inscribed in a circle, ete.

(13) Plainly because the perfect concords (came from) the square and triangle.) It is pleasant to con-
template my first efforts at my discoveries, even though they were wrong. You can see that I anxiously
sought for the true and archetypal causes of concordance, as if they were not there, like a blind man,
when I had them in my hands, The plane figures are the causes of concordance in themselves, not in
tue of being surfaces of solid figures. It was in vain that I turned to the solids in establishing the harmonic
proportions of the motions.

(14) But because we do not know the causes of this kinship.] But you see that the causes have now
been named: the plane figures. But it is not a kinship, not a consanguinity, but a mere affinity. For the
plane figures on the one hand divide the circle harmonically, and on the other hand agree with the five
solid figures. Hence both the harmonic division of the circle and the five figures meet in a single third fac-
tor, that is, the plane figures.

48 Ioan. Keprerti

erat comminifeendum, quo ficlla Oangularis affociarctur Diametro,{ubeodem, quafi genere,recla-
mante Naturd.Reéle igitur factumn, quod non acqitieni huicdiftributioni,

(22) Anmalumus Odtaedro quartam.] Hoc plane fin fecutus in lib. V. Harmonic,
Sed ininflituto dinerfo, Hicenim quarebam ortum Harmoniarum fingularum:at lib.V Harmoni~
Corum; delectus inter iam ortas eSt inflitutus , que Harmonia, quibus Planctis, qua mediasste figura
folida,confaciarctur.Cubo igitur etfinon reéte hic adfersbitua ipfe ortus confonantia Dispafon  redte
tamnen didto ub.VHarmonicorwan,affociatur ipfian Diapafon;non canfa ortus, fed caufa cobabita~
tionisinter Planctas cofdemy recte affociatur Octaedro, quod Cubiconiunx c8 , Difdiapafon , cui in
barmouica fectione adbaret Diateffaron.Videlib.V .cap.1X.Prop.VII.Cr XI.

(23) Relinqucre Icofacdro priorem imperfeétam.] Hiciterum fortuito ( quip=
pesnfpeculatione non propria )in verum incidi quadamtenus.Nam Prop.XV « XXV U.dicl capitis
1X, Dodecaedro quident, Diapente obtigit ,Icofacdro vero, vtrag, Sextarum, Tertits locum nullum ef=
e,probatur Prop.V1.

(24) Veniamusmodoad Afpc@tus.] Dehac materiacft meus liber 1¥., Harmoni=
corum.

(25) Quomodo tres perfeéte Hatmoniz cura tribus.] Parum aliquid in hac come
parationeemendandum, vide lb.1V Harm cap.VI fol.15 4.

(26) Quemque debiliffimum elle ferunt.] Nequaquam pero debilesn experientia
teflatur,fed fortiorem fepe ipfo Trino,caufam ex meis principys do lib.LV Harm.

(27) Qualein quidem Prolemaus non dedit.] PutainTetrabiblo de Aftroligia
{Seripto. tin Htarmionicis qua tunc nondumn videramncaufam hanc angi, fed male, vt ex meis nota
ad Proleinsum patebit:Ormninoeniin,& vnum,& qning, fignd, afpectus conflstunnt (fficaces, quas
cappello, Semifextian,ce Quincuncem,

(28) Nallam calem in vocibus agnofcit Natura concordiam.] Hoc adliteram
falfion est Nam inter chords 1.G 12.8 Tiflsspafon epi diapente  ficinter chordas 5c 12.681 Ter=
ia ininor fupra Diapafon. Alied sgitus habelam in anino cain bec verba ferbercon: feilcet,mullam
ff feAionenn tripliciter Harmonicam, qua rifpondeat bifce diuifionibus circali : quia erfi1.12. items
5-12. confonent: at refidua 11. &> 7.abborrent ab vtrifque terminis, Atnon ¢ffe eandem rationem
Afpectaum , quzei Confonantiarum , doceoper torum libruml V. Marmonicorum spracipue cap,

(29) Caufa virinque,&c.ex quinque corporibus.] Minime ex bis, at bene,ex fia
planis,quarum non ignobilifima Dodecazonus,

(30) Cumigituromnes.] Hocinitio falto,capi angere numerum afpedtuum :erff male
adfiini Sefquadruan, fou gradus.5 5.anale omifi femifextum, feu gr.3.0, Vide ape allegatum cap.¥'1,
Ub.1V. Harmon,

(31) Caufe quidem quas probabiliter.} Fruftra: Nam confirmat experientia Quin-
tulesn,c Biquantilent; De Sefguadro vero,cur ille minusficefficax,quain reliqui omnes,caufe bl,
Harmcap.v araduncur longe diner[e.tfte verobicrecenfitequing,caufe.funt nobisiterum refutan-
dane tencant Quintilem & Biquintilem.,

Nam quad caufams primcam atcinet  ficut cam Trine fextilisimpleecirculun,com quadrato
quadrats alias, ficetiamn cuin quantile Tridecls cuom Biquintiledeciliss cum fifquadro fequadrus
implent femicirculuon,nec repudeat hos Mufica. NoneSl gitur ffcacia abhac adequatione femi=
ara,

Secunda canfaad rem e3:atilla non penitus epudiant Quiztilemn, fed folummodo imperfe-
hora facit Trinoc fextilt,qwantum quidem ipfa pollet cum fola ronyit.Irrationale autem fic nia
catpo cum vulgo,quod in Warmonicis mihi dicitur, tneffabile.

Tertia cafe coincedit camprimajomnisenim in emicircato angulus reCluseit. Et fialiter ine
formerur hac catefa,quod bini femper afpectus officiant fummam duorun:redtorum, nunc femicircn=
lusiterum est eorum menfura,

Quarta caufa fatilis eS,Si enim Tertia mollis ideo eit quod.xmmodo perfecta, quia vtitur cade
disifionc cum perfcidis fel. Duodenarias[ane Cs dinifia vicenasia confiituitur adiumento quaterna-
rie,c fexaginariatornaiie. StTertiadura non quadrat ad duodsnarinm ,maiori termina 5.fane
nequetertia nollisquadrat ad Vicenarium , maioritermino 6. Rurfium fitertia mols ideo habetur
propirf. cla qua eit dronid.nnipfius Daspente,magistertia dura babebitur pro perfcéha, quia cr ipfa
if duonidiuia pfius Diapente fuer ans tantum quantum tertia mollis deficit a dimidto, Utag,cauen-

dum

gis


Chapter XIT

(21) Makes only lines.] Implying that the “stars” are not also figures? Forsooth there should have been
some fictitious association made between the octagonal star and the diameter, as if included in the same
class, though Nature would object. I was therefore right not to agree with this distribution.

(22) Or do we prefer (to allot) the fourth to the octahedron.| I have plainly followed this course in
Book V of the Harmonice, but with a different intention. For here I was seeking for the origin of in-
dividual harmonies; but in Book V of the Harmonice my intention was to select, among those already
originated, which harmony was the partner of which planets, and with which solid figure in between,
Therefore, although the attribution here of the origin of the consonance of the octave to the cube is not
correct, yet as is said in Book V of the Harmonice, the octave is correctly associated with the cube, not as
the cause of its origin, but as the cause of its dwelling with it between the same planets. The association
with the octahedron, which is the spouse of the cube, of the double octave, to which the fourth is linked
in the harmonic division, is correct. See Book V, Chapter 9, Propositions 8 and 12.

(23) To leave for the icosahedron the former of the imperfect pairs.| Here again | accidentally (that is,
in an investigation which was irrelevant) stumbled on the truth to a certain extent. For by Propositions 15
and 27 of the aforesaid Chapter 9 the fifth indeed has fallen to the dodecahedron, and both sixths to the
icosahedron; but it is proved that there is no place for the thirds by Proposition 6,

(24) Let us now come to the aspects.| Book IV of my Harmonice is about this matter.

(23) How the three perfect harmonies (can be most beautifully related) to the three.) There is a small
point to be corrected in this comparison: see Book IV of the Harmonice, Chapter 6, page 154,

(26) And which they say is the weakest.| Experience does not by any means testify that it is weak, but
that itis often stronger than the trine itself. I give the reason from my own principles in Book IV of the
Harmonice.

21) Which Ptolemy at least did not give.| Understand this to refer to what he wrote in the Tecrabiblos
about astrology. But in his Harmony, which I had then not yet seen, he touches on this reason, but
wrongly, as will be evident from my notes on Ptolemy. For in all respects both one sign and five signs
constitute potent aspects, which I call the semisentile and the quinduodecile.

(28) Nature recognizes no such harmony among musical notes.] Taken literally this is untrue. For be-
tween strings in the ratio 1:12 is a triple octave and a fifth; and similarly between strings in the ratio 5:12,
is the minor third above the octave, Thus I had something else in mind when I wrote these words, that is,
that there is no ratio of division which is harmonic in three ways, which would correspond with these divi-
sions of the circle, since although 1:12 is concordant, yet the remainders 11 and 7 are repugnant to both
terms.” However, I explain in the whole of Book IV of the Harmonice, especially Chapter 6, that the
reasoning is not the same for the aspects as it is for the consonances.

(29) The cause in both cases, etc...on the five solids.] Hardly on them but decidedly on the plane
figures, of which the dodecagon is not the most ignoble.

(30) Since, then, all. Having started in this way, I began to increase the number of aspects. Although
I was wrong to include the trioctile, or 135°, 1 was wrong to omit the duodecile, or 30°. See Chapter 6 of
Book IV of the Harmonice, to which I have often referred.

(31) The reasons which (one may) with probability.] In vain. For experience confirms the case of the
quintile and biquintile; whereas quite different reasons are reported in Book IV of the Harmonice,
Chapter S, why the trioctile is less potent than all the rest.1* Indeed we must once again refute the five
reasons listed here, so that they do not include the quintile and biquintile.

For as far as the first reason is concerned, just as a sextile with a trine makes up a circle, along with a
quartile and another quartile, so also a tridecile with a quintile, a decile with a biquintile, and an octile
with a trioctile make up a semicircle; and Music does not repudiate them. Consequently potency does
not come from this property of equalling a semicircle.

‘The second reason is relevant; yet it does not entirely repudiate the quintile, but merely makes it more
imperfect than the trine and the sextile, as far as the effect of this reason itself is concerned, though it is not
alone. However I here use the common word “irrational” for what in the Harmonice I call inexpressible.

The third reason coincides with the first. For every angle in a semicircle is a right angle. And if this
reason is rephrased as “a pair of aspects always make up the sum of two right angles,” in that case a
semicircle is again the measure of them,

The fourth reason is worthless; for if the soft third is to a certain extent perfect on account of using the
same division as the perfect harmonies, that is, into twelfths, certainly division into twentieths is also
established with the help of quarters, and into sixtieths with the help of thirds. If the hard third does not
fit its greater term, 5, in respect of divisibility into twelfths, certainly neither does the soft third fit its
greater term 6 in respect of divisibility into twentieths, Further, if the reason why the soft third is taken as
perfect is that it is half the fifth, the hard third will more readily be taken as perfect, because it is itself
half the fifth, as much in excess as the soft third falls short of the half. Thus we must here beware of the

Mysterivm Cosmocrapiiicvm, 49

dumbic collufione ifha.ccidentaria , quod cris fextilisfitpracafe dimiisaeusTrinus , G> Sextils
Tertia molli refpondeat.Nam docui c4p.V Llib.AV Harmonicas, Sexttli rfpondere non Tertiam
aall:mfed drapente cpidifdiapfon: spfam vero Tertiam mollzmconzmacin ff: fobel.im tani quin~
quarigulisquans fexnguli quia bisnumeris .6.comprebenditur Eftyus canfadisi fifima,que Tric
asan in duosperf:itosfextilis dimidit ab lla caufa,qite Diapentein dassTertiss, maiorem Ge miro-
rematuidit 1d quidem velex boc apparet , quodpartes funt ilicaqialis,bicinsquales Nibiligitur
detrabitur nolilttati Tortie dura, nibil accedit Tertie molli , quod fextilisc&t dinsidinm deTrinos
Quineilisnon itsin;, > poffe non miinoris boc aftimari, quod Quineilis fit dimidium de biguintslis
Ge. Equidden non minima parses folertis , ah huinfinods concurfibus accidentariis cawere, qui,vt
quondam Surenficala nauigantescanta 5 fic ipfiphilofophantes voluptate apparentis pulcbritudinis,
aptiquercfponfus (fiquidem bic adbarefiant ado atione captisvbicanfa nulla eft alterins in altero)
detincnt,yt ad feopuin prafinitum feientisperuenire non pofint.

Quinta cafe etefidusfecund,c oficit , vt QuinsilsimperfecTior afPedtus, Tertia dura
iinperfedtior petine alteriusgenerisyconfonantia fit non cfficit ,veille afpeltus plane niillins efficacies
becconfonantia nultiusft fuaustatis, Nain boc iam dudum de omnibus quingue obietionibus erat
dicendurn, quod fivalirent,jin Mufica.sque valerent,acin nogotioafpeéluum :necratio vila redditurs
cur becanfe vateant illicnon valeant bic.

(32) Quarenec OGangularis.] De Odlangulaflelaresedtalia, Cur enim illa , cum
Sf quadro climinetur feu magispojlponatur ex afpeétibus, nonitem & Mufica eliminetur Sexta mi
nor ex Odangilo notazeins rei caifesego explicuslub.1V Harmon, cap.VI. Scilicet etiam circa bune
atta fit,t.amn in Mufica quar inter afpeitss,quoad proportionesp{4s 3.675.ad 8,fiunt eniin vtrin=
aque viles: at propter concurfurn in vna felHione trim proportionsmn 3.5.C5 5.8. G> 3-8. cris ratio
jater afpectushabetur nulla, nobilior e5t hac Ofogonica fectain Mufica.

(53) NecDuodecangulatis radius. } Imo vero > hicoperatur, tele expericntia, &
contrarian Oangulariexperitur fortunain,in Mafica,nullam enin feionein peculiarem confit
tit Vide [ope alleg.stum cap.V 1. ib. IP. Harmon, V ides igitur canfaan illam quintam ofe de mbilo,
queafisquinon implent planiticm,j non pofint feriafpeltus, Namsetfifingularum [pecicrum nonim=
pleistyat implent unélarum. ;

(34) Atquchicfere feparo.] Scparatio aliqua necefferia uit, fed illaob caufss longe
alias,quain qus hic loco quintocommemoratur.

(35) Certeenim,quec-ex angulis fir,genuinaradijs.] Optime:valer enim hocipfium
eiamin vera caufa.V ide Harmon lib.

(36) Nonvero propterfiguram.] Hoc nimninmest,crcontrariumpramiffo. Si prow
pter angulum, pique etiampropter figuramsNans & fguaaper angulos conflituitur, G> angulorvan
ddelcitusper figuras fit.Sed vide ferupulum de figura centrali & decurcumferentialexcuffion ib.AY.
Harmcap.V.

(37) Poffunttamen.} Hic paragraphus compledtitur totam fere difpofitionem Harmo-
icoruns mcortum.N ama commuuneillud Geomnetricum,tangucam cesar cpa pri
L.CrU quid vero ills canfecur in Mufica explicani ib.111 quid in afpeibus tb.1V.

(38) Prolemai Mufica.) Fruflra hs caufas,ex Prolemai Muficisexpedlatas a meeffe,
IeGor ipfe dicet,fiquando auctores bi cm meisnotisedantur,Deo vitam prorogate. Haret enim Pro-
Acimeusin numeris,ytcaufacfine vefpelu figurarum, ye numeri numeraticitaque ge Harmonias non
nitllas cum veteribus iniufte proferibit , & interualla quadarn inter concinna recipit nullo illorums
smerito.Vide Harm.mee lib 10 fol.27.

(39) Quid cx Euclidis Muficis.) De bis prater propofitiones & Dafypodio exfériptas
nihil vidi Nequee tamen fpesest,in Euclide repertumn iri,que Prolemaus,que Porphyrius, atate pofte=
noresnoz babeut.



Chapter XII

accidental coincidence that the sextile is also precisely half the trine, and the sextile corresponds with the
soft third. For I have explained in Chapter 6, Book IV of the Harmonice, that it is not the soft third
which corresponds with the sextile, but the fifth above the double octave; whereas the soft third is the
common offspring of both the pentagon and the hexagon, because it is compounded of the numbers 5
and 6. Also the reason why the trine is divisible into two perfect sextiles is quite different from the reason
why the fifth is divisible into two thirds, major and minor. That is indeed apparent from the fact that the
parts are equal in the former case, unequal in the latter. Thus there is no detraction from the nobility of
the hard third, and none is added to the soft third, because the sextile is half the trine and the quintile is,
not; and it could not be reckoned less important that the quintile is half the biquintile, etc. Indeed it is not
the least important part of being shrewd to beware of accidental associations of this kind, which, as the
Sicilian siren once detained seafarers with her singing, detain those engaged in philosophy by the pleasure
of their apparent beauty and their neatness of fit (if indeed they are struck with wonder and cling to
them, when there is no cause for the one in the other), so that they cannot attain the predetermined goal
of knowledge.

The fifth reason is an effect of the second, and its effect is that the quintile is a more imperfect aspect,
the hard third a more imperfect consonance (but with the other kind of imperfection). It does not have
the effect that the aspect is absolutely impotent, or that the consonance is entirely disagreeable. For it was
long ago necessary to say of all the five objections that if they had any force, they had equal force in
Music and in the affair of the aspects; and no reason is given why these reasons have force in one case and
not in the other.

(82) Why neither an octile.| The octagonal star is a different matter. Take note why it is banished
along with the trioctile, or rather excluded from among the aspects, and the minor sixth which comes
from the octagon is not banished from Music. I have expounded the reasons for that in Book IV of the
Harmonice, Chapter 6. That is to say, even in this case they are equivalent, both in Music and among the
aspects, inasmuch as they represent the proportions 3:8 and 5:8 in themselves. For both ratios are base,
However, on account of the occurrence together in one division of the three ratios 3:5, 5:8, and 3:8, of
which no account is taken among the aspects, this octagonal division is more noble in Music.

(83) Nor a duodecile aspect.) On the contrary in fact, this aspect also operates, on the evidence of ex-
perience, and meets with the opposite fortune to that of the octagonal in Music; for it sets up no par-
ticular division. See Chapter 6, Book IV of the Harmonice,* which has often been referred to. You see,
then, that the fifth reason given above is void, as it implies that those which do not complete a plane sur-
face cannot constitute aspects. For although they do not complete one if taken as separate kinds, yet they
do if joined together.#

(34) Jt is precisely here that I make a distinction.| Some distinction was necessary, but for reasons far
different from that which is mentioned here as the fifth reason.

(35) For the argument based on the angles is sound in the case of the aspects.| A very good point, for
it also applies to the true reason. See the Harmonice, Book IV.

(36) And not to a figure.) This goes too far, and is contrary to what has already been said. If it is due
ton angle, it is certainly due to the figure; for the figure is established through the angles, and the choice
of angles is on account of the figures. But see my reservation about both the central figure and the sur-
rounding figure, which is hammered out in Book IV of the Harmonice, Chapter 5.”

(87) Yet (concords and aspects) may.| This paragraph embraces almost the whole of my Harmonice.
For [ have stated at the beginning, in Books I and II, their shared geometrical properties, as being the ar-
chetypal cause. Furthermore, I have explained in Book II] what it gives rise to in Music, and in the case of,
the aspects in Book 1V

(38) Prolemy’s Music.| You wait in vain for these reasons from Ptolemy's Music to come from me, the
reader himself will say, if ever these authors are published with my notes, should God prolong my life,
For Ptolemy clings to the numbers as the cause, without reference to the figures, inasmuch as they are
counted numbers. Consequently he unjustly proscribes some harmonies, as do the ancients, and accepts
as melodic®* certain intervals which do not in the least deserve it. See Book III, page 27, of my
Harmonice.

(39) What (can be applied to it) from Euclid’s Music.| On this, apart from the propositions set out by
Dasypodius,** I have seen nothing. Nor is there any hope that in Euclid will be found what neither
Ptolemy, nor Porphyry, have who were later in time.

5° Ioan, Keprerti


De computandis orbibus qui corporibas inferibuntur, © circum-
Seribuntur.

22 is Acrtenvs nihildi@um,nifi confentanea quedam fi-
Vy gna,& dx fufcepti Theorematis. Tranfeamus modo
FA) ad éweiveG orbium Aftronomiz & demonftrationes
ie) Geometricas;quz nifi céfentiant,proculdubio omnein
precedentem operam|luferimus. Primum omnium vi-
deamus , in quanta proportione fint orbes fingulis his
quinque corporibus regularibus in{cripti ad circum{criptos.

Etradij quidem fiue femidiametri circumfcriptorum aquant fe.
mudiagonios corporum. Nam nifi omnes anguli figure retigerintcandé
fuperftcié,corpus regulate nonerit. Biniautem anguli oppofitimutuo,
& centrum figura femper fancin eademlinéa fiue axi orbis. Excipicur
vnum Tetraedron, quodhabct fingulosangulos fingulis facierum cen-
tris oppofitos.

Tam reéta connectens centra figure & bafis elt radius, fue femidia-
meter in{cripti per vicimam lib.15. Campaniin Euchdem. Orbisenim
infcriptus tangere debet omniacentrafigure ; & figurein{criptz cum
circum{criptis omnes poflidentidem centrum.

Quod cum ita fir, facile eft videre , porentiamradij, quo circulus
baficircumfcribitur,auferendam de porentiaradij orbis circumfcripti,
verefidua fit porentia queefitelinew.{euradij orbis infcripti. Inadiuncto
f{chemate HOMett axis circumtcripti orbis, cuius ve & figurcinfcriptae
commune centrumin OHGL plunum vaum figura, quod hic fit bafis,t.
centrum bafis, HI radius circum{cripti bafi. Et reta ex cé- M
tro orbis O in] centrum minoriscitculidemiffa perpen-
dicularis crit circulo &linee HI. Incriangulo igitur HI °
Oangulusad I retus.ErgoH O potentiarquatpotentias gq
HI IO. Etporentia HJablataex H O potentia, relin-
quit I Q potentiam quaficam,per 47.primi.

Hinc apparet, ve habeatur IO inomnibus figutis,
quarendam effe prius H I radium bafis. Habetur autem &HI radius co-
gnitolatere figure, cui circulum circumferbit, Hincrurfam , veradius
bafis habcatur,quzrendum priuslatus cuiuslibet figura.

Affumpto igitur radio circum(cripti cuiuslibec in quantitate fi-
nis totius 1000.partium (fufficit noftro inftiturohac radij magnitudo)
potétia lateris cubici per 15.prop.lib 13.clem. Euclidis , cft pars tertiapo-
tentia axis, ve fiaxis habet 2090. lacus cubihabet 1155. Lateris Octaedri
potentia perr4.ciufdem , eftdimidium potenti axis. Lateris Tetrae-
drici potentia eft per 13. ciufdem,(e(quialtcra pars de potentiaaxis. Ate
que hactenus viui fuit aureum illud theorema Pythagorz de potentijs
Jaterum in trianguloreétangulo, prope47.lib.1. In ceteris duobus cor-
pori-



So far all that has been said is that certain signs agree with the theorem proposed
and make it probable. Let us now pass to the distances between the astronomical
spheres and the geometrical derivations: if they do not agree, the whole of the
preceding work has undoubtedly been a delusion.* First of all, let us see in what
proportion the spheres inscribed in each of these five regular solids are to those
circumscribed.

Now the radii or semidiameters of the circumscribed spheres are equal to the
semidiagonals of the solids. For unless all the vertices of the figure touch the same
surface, the solid will not be regular. However, pairs of vertices mutually opposite
to each other, and the center of the figure, are always on the same line, or an axis
of the sphere. The only exception is the tetrahedron, which has a vertex opposite
to the center of each of its faces.

Now the straight line connecting the centers of the figure and of the base is a
radius or semidiameter of the inscribed sphere, by the last theorem of Book XV
of Euclid in the edition of Campanus.? For the inscribed sphere must touch all the
centers of the faces of the figure; and the inscribed figures all have the same
center as the circumscribed.

That being the case, it is easy to see that the square of the radius of the circle
circumscribing the base must be subtracted from the square of the radius of the
circumscribed sphere to give as the remainder the square of the required line, the
radius of the inscribed sphere. In the adjoining diagram, HOM is an axis of the
circumscribed sphere; the common center of that and also of the inscribed figure
is at O; HGL is one face of the figure, which in this case is to be the base; I is the
center of the base; HI is the radius of the circle circumscribing the base. Now the
straight line from the center of the sphere, O, to I, the center of the smaller circle,
will be perpendicular to the circle and to the line HI. Then in the triangle HIO the
angle at I is a right angle. Therefore the square of HO equals the squares of HI
and IO; and subtracting the square of HI from the square of HO leaves the square
of IO, which it was required to find, by Euclid, Book I, Theorem 47.

Hence it is evident that to find IO in all the figures, we first need to find the line
HI, the radius of the base. But the radius HI is known if the side of the figure
about which the circle circumscribes is known. Hence to find the radius of the
base we first need to find the edge of any figui

Then taking the radius of each circumscribed circle in terms of the whole sine?
as 1000 units (a value of the radius which will give sufficient accuracy for our pur-
pose), the square of the edge of the inscribed cube by Proposition 15 of Book
XIII of Euclid’s Elements is one-third of the square of the axis, so that if the axis
is 2000 units, the edge of the cube is 1155. The square of the edge of the octa-
hedron, by Proposition 14 of the same Book, is half the square of the axis. The
square of the edge of the tetrahedron is by Proposition 13 of the same Book one
and a half times the square of the axis. So far we have been able to use the golden
theorem of Pythagoras on the squares of the sides in a right-angled triangle,
Proposition 47 of Book I. For the other two solids we need that other treasury of

Mystertvm CosMoGRaPHICVAt st

poribusalteroillo Gcomerrix thefauro opus ft, delinea fecundum ex-
tremam & mediam rationem fe&ta,qui eft propofitio 30.fexti. Nam Do-
decaedricum latus eft maior portiolateris cubict fecti, fecundum €xtre-
mam & mediam rationem per corollat.17.decimitertij. Sic pro Icofae-
dricolatereinueniendo primum queritur radius illius circuli,quiquings
Icofaedri cangitangulos,qui eft A C in circulo AB. Eius
potentia eft quinta pars de potentiaaxis » percoroll.16,
wedecimi.Igitur per 5.8 9. ciufdem, radyjiftius A C,fe-
cundum extremam & mediam rationem feéti,maius fe-
gmentum A Deft lacus decanguli,quodcidem A B cir-
culo infcribi poteft.Iunctg igitur potentig A C radijto-
tius,& A D maiorisfegmenti huius,faciunt potentiam E F lateris quin-
quangularis in illo citculo, per to. decimitertii. Quod cum fit inter
duos Icofaedri angulos , crit vtique lacus Icofacdri, per 11. 8 16. ciuf-
dem.

Habemus facera omnium figurarum in proportionead axin orbis
circum{cripti. Sequitur ve radios circulorum qui bafibus circum{cribi-
tur,inueftigemuy ex iam notislateribus: id quod adminiculo finuum fa+
cilime affequetur quilibet, quireputabit, hic exquifitiffimis
numeris non opus effe.Sitamenalicui placet artificiofius la~
borare ; ci fundamentareiex Euclideapponam. Cumigitur
tres falcem forma fintbafium, trangula, quadrangula, quin- # @
quangula: in triangularibus quidem, latus G H poteft tripla
guafiti radij HI, perr2. fepeallegatisInquadratolaus = ¢
GH potelt duplum quatiti radij: in quinquangulo deniqs 1
G Hlaceris & K H fubrendentis(datarum linearum)iun-
&x porentiz poflunt quintuplum redij H I quafiti, per 4: H
dec:mi quaru fecundum Campanum. Habemus radios
circulornm in bafibus in eadem proportione,qua latera. e

Sudrractisigitur potentijs radiorum de potentia fi-
nus totius, qui eft quantitas femidiametri fiuc radij in cir- K
cumferipto: reftabunt, vt fupra probatum cft, porétiz ra-
diorum,quos quatimus,in{criptorum {c.orbium. Commodius tamendz
facilius veéris,ve dixi,finubus.

Sed hicncquealia quedam prxtereunda compendia,nenimium
operofelaboremus. Primam orbesin{cripti Dodecacdro & Icofaedro
fant ciufdem amplicudinis,fi frgura eidem orbiinfcribancur. Habente-
nim bafes veriufque figure cundem radium per 2.decimiquarti.Idem iu-
dicium efto de cubo & oftaedro. Nam axis poteft triplum cubicilaceris,
& hoc duplum radij in bafi,ergoaxis potcft fextuplum radi) in bafi:in o-
acdro viciffim , axis poteft duplum lateris, & hoc triplum radij in af.
Poteftergo etiam hic axis fextuplumradij. Cum ergo ficex hypotheft
idem radius circifcriprorum fiue H M (in primo huius capitis {chema-
te)fitq; idem etiam radius bafium H I,&1 OH femperreétus: Ergo ct-
iam radiusin{criptorum,tertium nempelatus O I,idem erit per26. pri-
miconuerfam.Quarc habitis cubi & Icofacdri infaxiptis,de O&tacdroée
Dodecaedro nihil opus inquirere.

Deindcin cubo cum ipfam Jatus fit altitudo figura: dimidiom Ie-

G3 tusdi-


Chapter XIII

geometry, on the line divided in the extreme and mean proportion, which is Prop-
osition 30 of Book VI. For the edge of the dodecahedron is the larger portion of
the edge of the cube divided in the mean and extreme ratio, by Corollary 17 of
Book XIII. Thus to find the edge of the icosahedron we require to find the radius
of the circle which touches five vertices of the icosahedron, which is AC in circle
AB. Its square is a fifth of the square of the axis, by Corollary 16 of Book XIII.
Then by Propositions 5 and 9 of the same Book, if the radius AC is divided in the
mean and extreme ratio, its larger segment AD is the side of the decagon, which
can be inscribed in the same circle AB. Then the sum of the squares of AC the
radius of the whole, and AD its larger segment, gives the square of EF, the side of
the pentagon in the same circle, by Proposition 10 of Book XIII. In that case the
distance between two vertices of the icosahedron will naturally be the edge of the
icosahedron, by Propositions 11 and 16 of the same Book.

We have found the edges of all the figures in terms of the axis of the cir-
cumscribed sphere. The next thing is to investigate the radii of the circles which
circumscribe the bases, from the edges which are already known, That will easily
be achieved with the aid of sines by anyone who realizes that here there is no need
of very precise numbers. However if anyone wishes to work it out more
laboriously, I will append the principles of the method from Euclid. Then inas-
much as there are three shapes for the bases—triangles, squares, and pentagons —
in the case of the triangles the square of the side GH is three times that of the re-
quired radius HI, by Proposition 12 of the Book frequently cited; in the case of
the square, the square of the side GH is twice that of the required radius; and last-
ly in the case of the pentagon, the sum of the squares of the side GH and the
chord KH (which are known lengths) is five times that of the required radius HI,
by Proposition 4 of Book XIV according to Campanus. Thus we have found the
radii of the circles on the bases in terms of the axis like the edges.

Then subtracting the squares of the radii from the square of the whole sine,
which is the measure of the semidiameter or radius of the circumscribed sphere,
the remainders will be, as has been proved above, the squares of the radii which
we require, that is, the radii of the inscribed spheres. However as I have said it will
be easier and more convenient for you to use sines.

But here we must not miss certain other ways of saving effort, to avoid excessive
labor. First, the spheres inscribed in the dodecahedron and icosahedron are of the
same size, if the figures are inscribed in the same sphere. For the bases of both
figures have the same radius, by Proposition XIV.2. The same conclusion must ap-
ply to the cube and the octahedron. For the square of the axis is three times that of
the edge of the cube, and the latter is twice that of the radius of the circle about the
base; therefore, the square of the axis is six times the square of the radius of the cir-
cle about the base. In the octahedron on the other hand the square of the axis is twice
that of the edge, and the latter is three times that of the radius of the circle about the
base. Therefore in this case also the square of the axis is six times that of the radius.

Then since by hypothesis the radius of the circumscribed circles, or OH (in the
first diagram in this chapter) is the same in each case, and the radius HI of the cir-
cle on the bases is the same, and OH is in all cases a right angle; therefore the
radius of the inscribed sphere, that is the third side OI, will be the same by the
converse of Proposition 26 of Book I. Consequently, since the spheres inscribed
in the cube and icosahedron are known, there is no need to investigate the oc-
tahedron and dodecahedron.

Second, since in the case of the cube the edge is itself the height of the figure,

52 Toan. Keprert

tus dimidia eritaltitudo,ncmpe linea conneétens centra figura & bafis.
Nihil igitur opus inquifitione radij in bafi.

(1) Tertio Oétaedri & pyramidis equalium laterum eft eadem al-
titudo.Quanto maius igitur lacus pyramidis,tanto altior etiam ipfa figu-
ra.Ipfa Odaedron & pyramis duplo maiorum laterum habentcundem
orbem infcriptum. Nam pyramis fi fecetur medb}s lateribus, conciditin
quatuor pyramidas & Odaedron vaum,duplo minorum laterum.Cun-
que pyramis habeat quatuor facies , nulli caruin refeéta pyramis minor
adimitcentrum,vepote quod feétione longe inferius eft; manctigitur in
O€tacdro ex (e&o orbis infcriptus,antiqua quatuor centra,&pex defint-
tionem regularis corporis ctiam noua quatuor ex {eétione accedentia fi-
mul tangés.Siueigitur pyramidis,fiue O&aedri vel cubi infcriptus prius
habeatur, facilime per proportionem laterum habebitur etiam quanti-
tasalterius infcripti.

His adde qua Candalla,& quzalij decorporibus iam demonftra-
runt,ve quod potentia N M dimetientisin {phara , qua Tetracdrocirct-
{crmbitur, fit potentix H 1 radij in bafi tetraedri 44 perco-
roll..prop-tlibs. Quodibidem raleitudostine per- |

pendicularis corporistic bes nm dunecientis,Sillius x 1 & IN
potentia fic bes potentia latcris cH. Quod infcripti py- WEN
ramidi radius 0 1 fit pars quarta ipfins N 1 perpendicala-
ris,tertia ipfius n ocircumfcripti, vel fextaw M dimetié-
tis, Coroll.3. prop. 13. lib.13. iuxta Candall. Breuiter fic mM

funtinter fe Potentiz.or. 1. 1P. 2.HP. 6. HI. 8. NO. 9. NILIO.NP. 18, NH.

aug 3
Se ale {ss femidiame- we] ie
" rer circuli J943C 3 Q795
aa 714° Splanocitr- S607 = Si
ee 10st { cumferipti_ (607 } 32 | 597
E 1414} ee eS ie quadrato
*  Odaedriinferi-
pticirenli,
Quodnora.

Nota Auctoris.

(1) Entio 02. &¢Pyr, xqualium lt.cftendemalitndo.] rye quitemaiede
cenfetur & centco bafisofg, ad cppofiram angulum : Octacdriveroaltitudo bic ila confider, =
tur, que cSt inter duasbalesparallelss.Demanftratiof cil, Pyramidis enim Lateribus bifedts,
‘aiedts quatuor pyramidibus minoribusrflat O@tacdron, laterium fubduplorun Lateribis Pyranni=
dis magnescuina quatuor plana, vr tifa, tria circum, fant partes quatuor bafiwon magus Pyra=
andi: haben igitur tri circum earsdeminclinationcm cura trabus furgentibus 4 bafi Pyramids ad
Safliguuon anguls: quarnuisangulos habeant deorfum verfosrelta :crgo adem efl proportiopcrpendi=
_ in talt plano ad perpendicularem corporis , qua cSt in Tetraedro perpundicularinm ilius ad
banc.


Chapter XUT

half the edge will be half the height: that is, a line connecting the-centers of the
figure and base. Therefore there is no need for investigation of the radius of the
circle on the base.

(1) Third, the height of the octahedron and a pyramid of equal edges is the same.
Therefore the longer the side of the pyramid, the higher is the figure itself. Also,
the octahedron and a pyramid with edges twice as long have the same inscribed
sphere. For if a pyramid is divided at the midpoints of its edges, it falls into four
pyramids and one octahedron, with edges half as long. A pyramid has four faces;
and the smaller pyramid formed by the division does not take away the center from
any of them, as the center is far below the point of division. Therefore the in-
scribed sphere remains within the octahedron formed by the division, and touches
the four original centers, and by the definition of a regular solid the four new
centers resulting from the division as well. Therefore whether the sphere inscribed
within the pyramid or the octahedron or the cube is considered first, by the ratio of
the edges the size of the other inscribed sphere will also easily be found.

To these points is to be added what Candalla has proved, and what others have
now proved, about the solids, for instance, that the square of NM, the diameter of
the sphere circumscribing a tetrahedron, is 41 times the square of HI, the radius of
the circle on the base of the tetrahedron, by Corollary 1 of Proposition 13 of Book
XIII ; that in the same figure NI, the height or perpendicular of the solid is 2/3 of
NM the diameter, and the square of NI is 2/3 of the square of the edge GH; and that
the radius O1 of the sphere inscribed in the pyramid is 1/4 of NO, the radius of the
circumscribed sphere, or 1/6 of NM the diameter, by Corollary 3 of Proposition 13
of Book XIII according to Candalla.* In short, the squares of OI, IP, HP, HI, NO,
NI, NP, NH, NM are to each other respectively as 1, 2, 6, 8, 9, 16, 18, 24, 36.

Therefore: in units in which the semidiameter of the sphere circumscribed
about each figure is 1000 units,

semidiameterof — semidiameter of

in lengthofedge circle circum- inscribed sphere
scribing a face
the Cube is 1155 is 816% is 577
Pyramid 1633 943 333
Dodecahedron 714 607 795
Tcosahedron 1051 607 795
Octahedron 1414 816% 577
707 incase

of circle inscribed
in square of octa-
hedron.$Note this.


(1) Third, the height of the octahedron and a pyramid of equal edges is the same.) The height of the
pyramid indeed is reckoned from the center of the base up to the opposite vertex; whereas the height of the
octahedron is here considered as the height between two parallel faces. The proof is easy. For if the edges
of the pyramid are bisected, and the four smaller pyramids rejected, there remains an octahedron, with
‘edges half the edges of the large pyramid, of which four faces, one below, and three at the sides, are parts
of the four faces of the large pyramid. Then the three faces at the sides have the same inclination as the
three faces rising from the base of the pyramid to the vertex at the top, though their vertices are turned
straight downwards. Therefore the ratio of the perpendicular to a given face and the perpendicular height
of the solid is the same in the pyramid as the ratio of the corresponding perpendiculars in the tetrahedron.



Primarius feopus Lbelli, 8 quod hac quinquecorpora fiat tater
orbes, Aftronomica probatio.

Gry r vead principale propofitun veniamus :notum

1 Ben vias p'anctacum cile c. centricas: & proinderecepta
QON phyficis featentia , qued obtineant orb2s tantam craffi-
S435 tiem, quantaad demonttrandas motuum varictates re-
SER quiritur. Echactenus quidem (1) noftris Philofophisaf-
fentitur Copernicus. Verumiam potroné paruum cers
nituropinionum difcramen. Namcentent Phyficiab ima culilunaris
fuperficie ad decimam (pharam vfque nihil eflececleftibus orbibus va-
cuum; fed tangifemper orbem ab orbe , imamque (uperioris fuperfi-
ciem cum fumina inferioris penitus vniri.Sic enim quarenti.quisexem=
pli caufa coeli Martiilocus fit Phyficus, refpondent: interiorem Touis
fuperficiem. Ecapud Prolemzum, atque viitatam A ftronomiz defcri-
ptioncm obtinere fortafle poflunt hanc caufam : propterea , quod orbi-
um proportiones inueftigandi nulla illic occafio, nullum adminiculum.
Quemadmodum enimijs, qui de nouis Indijs {cripferunt, nemo facile
contradicit, quiillalocanon ipfeluftrauit : fic phyficorum ratiunculas
decontactu orbium Aftronomus reijcere non poreft , quem obferuatio-
num expericntia & hypothefium conditio inceclum ipfum , interq; or-
bes né cuexit. Iam vero ex Copernici hypothefibus,& ex illo terre mo-
tu fequitur, nullam eflcorbium vicinorum differentiam, qucnon mul-
tis patcibus orbis vtriufque eccentricitatem fuperet. Atque huius reica-
peexemplum ex Telluris & Veneris orbibus, ijs nempe, quiminimum
abinuicemabfunt. Qualium Telluris4 centto mundi diftantia medio-

cris eft 60. tahum Vencris ab codem diftantia mediocris eft 433 Diffe- Copers.

rentia 16${crupula. Iam Tellus in perigzo appropinquat Veneri fcru-
pulis 2} Veiusilliobuiam procedit in A pogzo fcrupulis itidem 24 fum-
ma,s-fcrupulorum. Ergoduodccim refiduis {crupulis hxc due corpora
diftant ctiam cum proximeab inuicem abfunt. Quod fi quis hocinter-
medium {pacium compleriafflerat deferentibus nodos, & circulis lati-
tudinum, is cogitct: poffe ea officia etiam a longe tenuioribus orbibus,
quam quitantum hiatumimpleanct, adminiftrari: neque naturam im:
mani moletantorum orbiumonerandam. Quamuis hercle Coperni-
ci hypothefes omnes itacomparatz,itaaptz funt , itainuicem inferui-
unt, vehaud facile vilo orbe, qui vitra planeta viam euagatur, admotus
yeddendos indigere videamur. Sed efto , vt in propinquis {pacia his
impleantur orbibus: quafo illud quale fit,videamus. Cuma perigaalo+
uis diftantia ad Martis Apogzam, duplo longius numeretur {patium,
quam ab ipfo Marte ad centri Mundi (louis nm diftantia triplaeft ad
Martiam ) ergonead pufilli Pianeta vixad fenfum variandas motiuncu+
las,inlongum,in larum,totum hoc {pati duplo craffius omni Marte,re+
pleturtam portentofis orbibus: Quahac Naturz luxuries?’ Quam ine+

3 ptai


lib.5.c. 416
22, Efi
inTabula.


Therefore let us come to the principal purpose. It is known that the paths of the
planets are eccentric, and consequently the received opinion among physicists is
that the spheres have the thickness which is required for representing the varia-
tions in the motions. So far indeed (1) Copernicus agrees with our philosophers;
but from now on a considerable difference of opinions is perceptible. For the
physicists believe that from the inner surface of the heaven of the Moon right up
to the tenth sphere there is no empty space in the celestial spheres, but sphere is
always touched by sphere, and the inner surface of the upper is completely united
with the higher surface of the lower. Thus if anyone asks, for example, what the
position of the heaven of Mars is, physically, they reply: the inner surface of
Jupiter’s. And from Ptolemy and the customary astronomical description they
may be able to draw the following reason: that in them there is no opportunity for
investigating the proportions of the spheres, and no assistance. For just as
nobody easily refutes those who have written about the new Indies if he has not
inspected those regions for himself, so an astronomer cannot reject the physicists’
feeble figuring about the contact of the spheres unless the test of observation and
the agreement of the hypotheses has transported him out to the actual sky and be-
tween the spheres. Now in fact it follows from the hypotheses of Copernicus and
from this motion of the Earth that in no case does the difference in size between
neighboring spheres fail to exceed many times over the eccentricity of either
sphere. Take as an example the spheres of the Earth and Venus, that is, the ones
which are at the smallest distance from each other. In units in which the mean
distance of the Earth from the center of the universe is 60, the mean distance of
Venus from the same point is 43 %. The difference is 16 5%, units. Now the Earth
at perigee comes 21 units closer to Venus, and Venus at apogee similarly goes out
2% units to meet the Earth, a total of 5 units. Therefore these two bodies are the
remaining twelve units apart even when they are at their closest distance from
each other. But if anyone were to assert that this intervening space is filled by the
deferents of the nodes, and the circles of latitude, let him reflect that those duties
could be performed by far thinner spheres than those which would fill such a
large gap, and that Nature should not be burdened with the vast bulk of such
large spheres. Yet I swear the hypotheses of Copernicus are so neatly adjusted, fit
so closely and support each other so well, that we do not readily seem to need any
sphere which wanders outside the planet’s path to account for its motions. But
suppose that the spaces between neighboring planets are filled with these spheres:
I should like us to see how it would be. Since the space between the distance of
Jupiter at perigee and the distance of Mars at apogee amounts to twice as far as
that from Mars itself to the center of the universe (for the distance of Jupiter is
three times that of Mars), then for the scarcely detectable variations of the tiny
motions in longitude and in latitude of a feeble planet, is this space which is twice
as thick as the whole of Mars’s filled with such portentous spheres? What ex-
travagance of Nature is this, so out of place, so pointless, so little like herself?


Copern.
below in
Plate.
Hucperti-
net Tabu-
La quarta,

54 loan. Keprert

pta?Quam inutiliseQuam minime ipfivfitaca? Atqueexhocvidereeft,
in Copetniconullum orbem abalio tangi,fed ingentia relinqui fyftema-
tum interualla vtique plena celefti aura, fed ad ncutrum tamen propin-
uorum fyftematum pertinentia. (Hactabulaab oculos proponotibi
orbium & interftitiorum magnitudinesiuxta veras proportiones 5 vti e&
numerisa Copernico expteflz funt.) Eorum autem {paciorum cil ini-
tio profeffus fim caufas ex 5. corporibus reddere, cur tanta fingula inter
binos planetas relicta finta Creatore Opt.Maximo, nempe quod fingu-
le figura fingula incerualla efficiant: videamus modo, quam id feliciter
tentatum fir, caufamque hance coram Aftronomia ludice, & interprete
Copernico difceptemus. Orbibus ipfis tantam relinquo craffiticm,qua~
tam requiritafcenfus defcen{ufque planctz; quz tamen verum fufficiat,
infra,cap.22.videbis. Quod fi figurz intericéta funt,vedixi: oportetima
faperioris orbis fuperficiem xquari circum{cripto figure, fummam infe-
rioris infcripto; figuras autem cenferi co ordinc, quem fuprarationibus

confirmaui. Quare
Lib. 5. Copern.

333, Cap.14.
757 Cap.ag.
794 Cap.21. 822,

723 Cap.27.

Martis 333 |
Telluris 795
Veneris 795
Mercurij577
vel707 5

c | {louis 5777
eftrooe.

ne
Si ima son debcbatef:
(e" fe fumma

Quod ficraffitici orbis terreni accenfeacur fyftemalunare: ergo fi
ima fuperficies orbis terreni, etiam Luna coclum comprehendens, eft
1000.{umina Veneris cftin Copernico 847. Etterreni orbiscum Luna
fammus margo elt 801.41 o ima habet 1000. Hic velim teidentidem re-
fpiceread tabellam capitis fecundi,nempead huius interpofitionis qua-
Jemcunqueimaginem.

Ennumeros (2) parallelos propinquos inuicem, & Martis quidé
atque Veneris cofdem.Telluris vero & (3) Mercurij nen admodumdi-
uerfos, folius louis immodice difcrepantes, fed quod in tantadiftantia
nemo miretur. Erin Marte quidem atque Venere, vicinis orbi Telluti
vides quancam efficiat diuerficatem orbiculus Lunzaccenfitus craffitiei
orbisterreni: (4) quitamenorbiculus vix3.fcrupula equat, qualium
orbis terrz habet 60.

Vndecolligere potes,quam facile animaducrfum fuiffer, quantaqs
numcrorum extitiffetinzqualitas:fi hc contra ceeli naturam tentaren-
tur,hoc eft, fi Deus ipfe in Creatione nonad has proportiones refpexif-
fet.Certe enim fortuitum hoc effe non potelt, ve tam propinquz fintin-
teruallis hifce proportiones corporumscum propteralia, tum maxime,
quiaidem ordo eftintcruallorum, quem fuprarationibus optimis, cor-

poribus aferipfi, vide cap. 3. Nam etfi 635.4 577.difcrepat:nul-

licamen propinquior eft ,atque

huic ipf.



Chapter XIV

From this it may be seen that in Copernicus no sphere is touched by another, but
huge gaps are left between the various systems, certainly full of the air of heaven,
but concerning neither of the neighboring systems. In this plate I set before your
eyes the sizes of the spheres and the intervening spaces according to their true pro-
Portions, just as they were elicited numerically by Copernicus. However since I
claimed at the start to draw from the five solids the reasons why such large spaces
were left between each pair of planets by the best and greatest of Creators, name-
ly that particular figures produce particular intervals, let us now see how suc-
cessfully this has been undertaken, and argue the case with astronomy as judge
and Copernicus to plead for us. For the actual spheres I leave as great a thickness
as the upward and downward movement of the planets requires. Whether that is
enough you will see below in Chapter 22. But if the figures are interposed, as I
have said, the inner surface of the sphere above ought to be equal to the cir-
cumscribed sphere of the figure, the outer surface of the sphere below to the in-
scribed sphere, the figures being ranged in the order which I have established
above by argument. Therefore

of Copernicus

Saturn of Jupiter $77 635. Ch.9

Hee Jupiter 5 ae Mars 333 cording 338. Ch. 14

“ _ .. saeseasaéaeaéaéaéaéésew
Earth Pod Venus 795 Pg 794 Ch. 21. & 22

Venus Mercury $77 CPEMICYS 293 Ch. 27

be itis
or 707

But if the lunar system is allocated to the thickness of the Earth’s sphere, then if
the inner surface of the Earth’s sphere, also including the Moon, is 1000 units, the
outer surface of the sphere of Venus is 847 units in Copernicus. Also the outer
boundary of the sphere of the Earth? along with the Moon is 801 units if the lowest
point of Mars is 1000. Here I should like you to look repeatedly at the plate in the
second chapter, that is, at the representation such as it is of these interpolations.

(2) Notice that corresponding numbers are close to each other, and indeed in
the cases of Mars and Venus, the same. Indeed in the cases of the Earth and (3)
Mercury they are not very different: only in the case of Jupiter is there an undue
discrepancy, which however at such a great distance should surprise nobody. Also
for Mars and Venus, which are next to the Earth’s sphere, you see how great a dif-
ference is made by the allocation of the little sphere of the Moon to the thickness
of the Earth’s sphere, (4) although that sphere scarcely amounts to three units
where the Earth’s sphere is 60.

Hence you can realize how easily it would have been noticed, and how greatly
unequal the numbers would have been, if this undertaking had been contrary to
Nature, that is, if God himself at the Creation had not looked to these propor-
tions.? For certainly it cannot be accidental that the proportions of the solids are
so close to these intervals, for various reasons but particularly because the order
of the intervals is the same as that which I ascribed to the solids above for ex-
cellent reasons; see Chapter 3. For although 635 differs from 577, nevertheless
there is no number to which it is closer than to that.


Here
belongs
Plate 1V.
“6¢7 “dag
“nsf emus wx 2 unuenrs 120

sumer Z ‘upHfn \uprope
‘upein § ipsa reds

np
semoursnes uss mmbonp

rnb sansa
gunfip maria var
“anpuayap gi

my

spuryfv ombunu H{ reese vant
-aunamnboaayisuenbo a mpl
esdo myer rnb diana mestarues mnaay near angonp sanse4. TY
“wot
suorarnbay ernaqoig wanb ya motina9 sunfurts  wsteopfnruaetiad nd
anbse® us ypplrdo mavrSedo 24-0 * YX 10d mproanyfomb pa yuan roa
sarnpusa rf nama
coo guanrrs via rnb afryves noancaoues mmsan mgonp sage TH,
-vesipuou aang vous tarsus yuna pf pa won map
nour “rouguarnirinend rin yr nastany ended mag na iunetade and

sprurodop ures

saafod rnasuayys Sods Cy srploydouunrSusd snd bio sad yen vor
“prunes mplsdanen

29 moardss mae onp wunsurisy m00)runfurts 1-9 Ossdmrsu

“auras dion uunfdssdoxd Q'umnuounasnzy 2

cumufadms

mrs hanosoperyayfuomnson susueonb amp sniusumrds 5
innsous od

uo womb sane funag navuaf(fuonngrarsmuruepip send 3
7 “aursourp warsifoas PUNT

cofasiunsr1in nguoserymsuyroneiey ong “vptdde woongenp usury
rufa ena xavoanarvesiaua me Sagas visanfonmny. CL

Uafjrunis
“yom myfusfn uns

-inBruu ug ayes xa wnssnin

promiypcmgac mao vin ourunn ogo xo urn pa

ated
rd py



re


Mystertvm CosmoGRapist Cvat 55


‘Nora Audctris.

(ON OtisPhilofophisaffenticur Copernicus.] Intillige de patio Orbinm Geometri=
code materia cnim, hoc et, decorpulentisadamantina ne Ptolemaus quidem adeo craffe
philofophatur.

(2) EnNumcrosparallelos{ Eregione fitos,vt 577.63 5.fi¢ 333-333

(3) Mercurij non admodum diuerfas.] Sin ¥ nom fiumas 577. radium inferipti
Oftsedro , fed 7.07. radium infiripti qiadrato Ofaedri : tunc ifle non multum diferepat a

(4) Quitamen orbiculus.] Hicproportio Orbitum Solis & Lune affumitur ea que 20.
4d qusittam tradit Aftronomia antiga circiter. Atdoccolib, 4. Fpitomes quod illa fitfere triplo
anaiaryetfiin Ephemeridibus modaflsa quaadam vfus,vjierpai iam fefquiplo maiorem, ferlcam qua
304d t.nterim dm plane concladerem.


Correttio diftantiarum ¢x dinerfitas profthapharefeon.

Evero tibi,Le€tor amice,occafionem vllam prebeam to+
tum hoc negotium propter Icuiculam difcordiam re
ciendi,moncndus hic es,quodte probememiniffe velims
{ Copetniciintentum non in Cofmographia verfari , fed

in Aftronomiashoc eft,verum nonnihilin veram orbium
proportionem peccct, parum ipfi cura eft: modo nume-
ros ex obferuationibus cos conftituat,qui fintad demonftrandos motus,
Planctarumqueloca computanda , quantum fieri potuit, maxime apti.
Atfiquisaptiores dareconctur » & hos Copernici numerositacorrigar,
venihil interca autparumin profthapharefeficurbet id illi per Coper~
nicum facilelicebit. ; . .

Vtigiturfummam denique huic negotio manum imponam, atqs
vtapparcat, quid quancumque penes fingulos Planetasin parallaxibus
orbis terreni murctursnouum ftruam mundum;& cum priusinueftiga~
ta fucritab artificibus cuiushber x2 ad orbis femidiametrum
ptoportio : idco fiquidin longiffima vel proxima orbisa centro mundi
diftantia mutabitur per interpofitioné corporumsid inéxnersgéamn anim
aduertendum crit proportionaliter-Initium erita maxima terre diftan~
tia furfum,minimadcorfum,centrum verfum.

Ante omnia autem recexcrdi numeri Copernici, atque peculiari-
ter accommodandi funtad prefensinftitutam. Nam efile fine dubio
centrum totius vniuerfitn corporefolariconftituits tamen vecalculum
inet compendio , & ne nimium a Prolemxo recedendo , diligentem
cius lectorem turber ; (1) diftantias omnium Planctarum maximas
atqiic minimas, ve& loca carnm in Zodiaco (quz Apogtorum & Peri-
gxorum nomen retinuerunt)computauitnon a centro Solis,feda cen-

to orbis


Chapter XV


(1) Copernicus agrees with our Philosophers.| Understand this to refer to the geometrical space of the
spheres: for on their material, that is, on their solid corporeality, not even Ptolemy offers such a crass
piece of philosophizing

Q) Notice that corresponding numbers.| Located according to the region to which they refer. For in-
stance 577 against 635, 333 against 333.

(3) (Of) Mercury they are not very different.} If in the case of Mercury you take not 577, the radius of
the sphere inscribed in the octahedron, but 707, the radius of the circle inscribed in the square of the octa-
hedron, then the discrepancy from 723 is not great.

(4) Although that sphere.| Here the ratio of the Spheres of the Sun and of the Moon is assumed to be
as 20:1, about as reported by ancient astronomy. But I explain in Book IV of the Epizomet that itis about
three times larger; though in the Ephemerides | leaned to the modest side and took it as one and a half
times larger, that is as 30:1, for the time being until I came to a definite conclusion.


In case, friendly reader, I should offer you any occasion for rejecting the whole of
this enterprise because of a trifling dispute, I must here mention to you something
which I should like you to remember carefully: Copernicus’s purpose was not to
deal with cosmography, but with astronomy. That is, he is not much concerned
whether there is a mistake relating to the true proportion of the spheres, but only
with establishing from the observations the values which are best suited for deriv-
ing the motions of the planets and computing their positions, as far as possible.
But if anyone should try to give better suited values, and rectify Copernicus’s in
such a way as to upset in the process nothing, or very little, in the system of equa-
tions, that will readily be permitted as far as Copernicus is concerned.

Therefore, to put the finishing touch to this enterprise and to make clear the
alterations in the parallaxes of the Earth’s orbit produced by the particular
planets, and their amounts, I shall construct a new universe; and as the propor-
tion of each eccentricity to the semidiameter of the orbit has previously been in-
vestigated by the authorities, if by the interpolation of the solids any alteration is
produced in the greatest or smallest distance of the orbit from the center of the
universe, it will be noticed proportionately in the eccentricity. The starting point
will be from the greatest distance of the Earth upwards, from the smallest
distance of the Earth downwards, towards the center.

First of all Copernicus’s values must be reworked, and particularly adapted for
the present undertaking. For although he undoubtedly established the center of
the whole universe in the solar body, yet to help his calculation by shortening it,
and to avoid upsetting the diligent reader by too great a departure from Ptolemy,
he computed the maximum and minimum (1) distances of all the planets, and also
their positions on the zodiac (for which he retained the name of apogees and
perigees) not from the center of the Sun, but from the center of the Earth’s orbit,

Hue perti=
netTabula
quinta,

38 Toan Kerrert

troorbis Magni, quafiillud cffecVniuerfitatis centrum;cum tamen illud
4 Sole tanto femperinteruallo diftet,quanta eft quouis tempore Telluris
(vel Solis)maxima ¢2x<76!7, Quos numeros fi rctinerem in pr¢efenti ne-
gotio;illud incommodum fequeretur, quod aut ctror committerecturin
in{criptione,dum terra orbis pro corpore cenferctur, qui fuperficies fal-
tem elletsve videre cftin praced. TabellalV. aut orbi terreno nullam, ve
catcrisrelinquerem cralliticm. Effencigitur Dodecaedricorum plano-
rumcentra & Icofaedrici anguliin cadem fuperficie {pharica; atque ita
totus mundus ardtius confidercr, fierctquelonge anguftior, quam expe-
rientia motuum & obferuationes patiuntur. Atqshunc fecupulum cum
ego Michacli Mzftlino, praceptori mco Clarifimo apetirem,cxplora-
turus,an probare velletmodo pofitum hoc Thcorema: is infperato mei
iuuandiftudio hunclaborem in fefufcepit, & non tantum ex Prutenicis
Tabulis ipfas Planetarum diftatias denouo computauic, fed ctiam pra-
fentem Tabulam mihi confecit; acqs fic me tumalijs non paucis occu-
pationibus derentum magno & diflicili atq; molefto labore fubleuauit.
Quam tabulam ipfo permitcente Autoretecum , Leétor ,communico:
tibique ficcam commendo, vt qui non tantumin prafenti negotiotbi
profutura, fed etiam incricatiffimum nodum ad oculum folutura,, atque
adco te in ipfa Prutenicarum acque Coperniciadyta,quafi manu,dudtu-
rafic. Ecenimex caiucundum eft difcere, quomodo Auges Planetarum
diuerfz,in diuerfa Zodiaciloca cadant;quod in Venere plusintegri trié-
tis diuerfitatem,parit-Nam eius Apogeum eft in 8 & Iba diac in $ &ue.
Videre etiam eftlongealiaseffelincas diftantiarum a Sole, quam Acen-
troterreniorbis.Qua diuerfiras in b maxima cit: propterea quod inte-
gra Telluris caxers¢?ms cius diftantizaccedit. In Iouc aucem parum mu-
tatur,quia is,non ve Saturnus é regione Solis ficaltiffimus, fedin , vbi
fere aqualiter abcftab vtroque centro Solis & Orbis magni. Atqucin-
de ctiamad oculum patetdemonftratio cius,quodCopernicushb.5.Re-
uol.cap. 4.16. & 22. fub finem , de mutabili Eccentricitate Martis & Ve-
nerisad mutationem tcrrenz,breuiffimisverbis innuit; Rhicticus vero
in fua Narratione copiofius perfequitur. Alind ctia ct, cuius nosifthee
tabulaadmoner, quod quia commodius alioloco dici poteft, nunc dif-
feram.- Nuncad rem. Pandamautem quadruplicem ordinem numero-
rum.In primo crunt Planetarum ab{ceflus 4 centro magni Orbis;ficutij
abfceflus & numeti ex Copernico & Prutenicis fimpliciter & fine mu-
tatione cliciuntur-In fecundo cruntab{ceffus orbium 4 centro Solis, qui
proucniuntex Copernicopoftillam refolutioncm numerorum, de qua
modo vidiftitabulam. Intertio & quarto venient rurfum ablceflus pla-
netarum 4, proutilli per interpofitioncm corporum mutati fant. Et
tertius quidem ordo eritex ftru@ura mundi ea, que pro fundamento
habebit orbis terreni craffitiem fimplicem ,nonaccenfito fyftemate Lu-
tus denique prodct craffitiem orbis terreni tantam,que fupra
{emidiametrum orbis Lunaris contcgerc poffit.

% Alu

Chapter XV.

as if it were the center of the whole universe, even though that is always separated
from the Sun by a distance which depends on the maximum eccentricity of the
Earth (or of the Sun) at any given time.? If I retained those values in the present
enterprise, the inconvenient consequence would be that either an error would be
committed in the interpolation, the Earth’s sphere being counted as a body when
it was really a surface, as may be seen in Plate IV above, or I should leave no
thickness for the Earth’s sphere, such as was left in the other cases. Then the faces
of the dodecahedron and the vertices of the icosahedron would be on the same
spherical surface; and thus the whole universe would be more restricted in size,
and would become far narrower, than our knowledge of the motions and obser-
vations allow. When I revealed this difficulty to Michael Maestlin,? my famous
teacher, to find out whether he wished to verify this theorem which has been just
proposed, he took that labor on himself, with an enthusiasm for helping me
which was greater than I had hoped, and not only computed the actual distances
of the planets afresh from the Prutenic Tables but also executed the present plate
for me; and thus he relieved me, at a time when I was busy with quite a number of
other commitments, of this great and difficult and troublesome labor.? This
plate, reader, I communicate to you with the author’s permission; and I commend
it to you as something which will not only be of great help to you in the present
enterprise, but will also disentangle a most intricate knot at sight, and thus lead
you as if by the hand into the very tabernacles of the Prutenic Tables and of
Copernicus. For it is pleasing to learn from it how the various apsides of the
planets fall in different places on the zodiac, which in the case of Venus brings
about a difference of more than a whole thirty degrees. For its apogee is in Taurus
and Gemini, its aphelion in Capricorn and Aquarius. It may also be seen that the
lines of the distances from the Sun are far removed from those from the center of
the Earth’s orbit. This difference is greatest in the case of Saturn, because the
complete eccentricity of the Earth is added to its distance. In the case of Jupiter,
however, it is little changed, since unlike Saturn it reaches its highest point not op-
posite the Sun but in Libra, where it is almost equally distant from both the center
of the Sun and from the Great Orbit of the Earth. Moreover the derivation is ob-
vious at sight of what Copernicus hints at in very brief words in Book V of the De
Revolutionibus, in Chapters 4, 16, and 22 at the end, on the variable eccentricity
of Mars and Venus compared with the variation in the Earth's eccentricity, but
Rheticus pursues at greater length in his Narratio. There is also another point of
which this plate reminds us, which I shall postpone for the present because it can
more conveniently be made in another place. Now to business. I shall set out a
fourfold table of values.‘ In the first column there will be the distances of the
planets from the center of the Earth’s Great Orbit, just as these distances and
values are extracted from Copernicus and the Prutenic Tables, simply and
without alteration. In the second column will be the distances of the orbits from
the center of the Sun, which emerge from Copernicus after the revision of the
values illustrated by the plate which you have just seen. In the third and fourth
columns again will come the distances of the planets from the Sun, as they have
been altered by the interpolation of the solids. Thus the third column will be
derived from the arrangement of the universe which will have as its basis the
thickness of the Earth’s sphere on its own, without the addition of the lunar
system. Finally the fourth column will give the Earth's sphere a thickness suffi-
cient to cover the semidiameter of the lunar sphere above and below.


Plate V
belongs
here.
Joho a9 may S90 9 SF g9 1yEO
Fort aibér XAV Hy Wok gt TF ot ACE
ashe: Gore govt gary & A Ss yOG
awd B

roin Gide Wave Fito de Osage
ef ar

omos myadiuh y we pirua imps nnghef ran ti g NaHS A
os mizediu Sy I pas “vasay mypadiea mprigouanby wren i LHD
“ost igoium 4y-0¥ L966

4? N08 yn nuasynaxyne munirs “06173 © y srnstanien Dd y OSB La
DE SLO9e ye ping mms evstanusye nse merd foumarea>
“$9666 purse oSt6oS

Og rinffipranxrannsy -oS9ot Ov ry ps 00098 Gx -naMnNsg

tie gavO ZB Lt Gots oO -~

stunmn (vunbifea

Bro0ott yap emt 7f mires resarme poms

oofGtt rum *Ogl+51 rusx¥u © ¥ P ruaryfiprpua “thot zee

samme LY ps tLogte exe) Sem aquege wroumem ag 9Tte9e
9 sarap vm 1B ar nsiamosatpasus sand 199 eT) 7

+S Seton ort viig) ae

oat © y sinensfip hansiay vuerxrm syn 4+oLtO1 Reg ym mnEKINTY CL

vf TSYEL gq rium nque oxmuasy mrursasearse Ufntr Yost memsertr dal rab
wapunara ya tzsuriprnaf smoney “f mornsuanarra psy ys meen

“31969 re

wera DT L vinunsfip © 9 meas 4 ranxru mH TOT BTy NOs 2 TOS

unr wungfna > y ns OT pg iuSrm age oan yaa mardunas wT

am adoud ai nesayna rp airy nro mas
106956 rumimp *0L14O% yomseuas mes ¥ P49] PulxrM WH *OOCOOE

Srédey

49 Bia gion saxruripiefrumenbs*0L1¥ po savage) snd GY

tpauapia sormayyfoasxip anf
aq espa omfoanad pps vensoyy racunas 13h ruurcrvasdeud ps7
smqio srenisurs9 ya sen “runndesd one wt wenaader susduas pol ‘romper ated
_glareaffues wnasisr Bowngig nunaneines enenmasen mf alyfefo
Cag urna ura ngio mernsiamaxn men ys'g prinnerd mp

“gern Rar YS VF

Yes

x
/ al

{| Ww Ayn


Sr
a
a
“Sere ‘ofr an
unuuy rorptpiuadoy tiodua py wonauy forp'rruzjorq uiodulaa py
‘giao sumaeorusing wr sosaumnu 29 orurodo> urenusiuzy tnpuns9;punyy wnTerzud]
i rd pn Durer oj tunpun 35% yy

Mystertvm CosmocRrariicvm 37

I I oo “fo “fo Fb Te. "|

5 [Ali 9 42 of 59 15

35 s6 [ur 18 16
Humil. 8 39 0 20 30 y

gt 849 26 26

ce

Altifl 5 27 29] 5 29 33

Humil. 4 58 49]4 59 58


Pa! oS

Alc. 1 39 s6]1 39 sat 33 alr 39 5
? |Humil 1 22 26,1 23 35] r 18 39] 1 2: 2

ter {Alnifl. 1 0 of1 2 jo]r 2 jofa 6 6
ta |Humil. 1 0 ofo 57 30]0 57 300 53 54

Alufl$ 0 45 40/0 44 29]0 45 41] 0 42 50
Humil. 0 40 40]0 41 47/0 42 js5]o 40 14

Aluf. 0 29 24/0 29 19/0 30 2fo 28 2
¥ |Humil. o 18 2]0 14 Of90 14 oO|o HB 7

Ala. 0 2 30

© {Hum o 1 56{° 9° 972 9 Of9 9 0

Hediftantix.Iam porro fabiungam laterculum arcuum , qui finu-
bus debenturijs,quos cfhiciune Veneris quidem & Mercuriialeaflimi ab-
fceflus,G mediaterre diftantia fic finus totus : Tclluris vero media difta-
tia, fi fupcriorum abfceflus longiffimi fint finus tocus; quorum arcui ili
quidé elongationibus maximis Veneris & Mercurija Sole, hivero pro-
fthapharefibus¢=y«sr Saturnilouis & Martis proximicrunt. In:primo
ordine fantarcus,qui proucniunt ex corporibus exclufa Luna, in fecun-
doarcus,qui proueniune cx diftantijs 4 Sole Copernicanis, in tertio de-
nique,arcus qui ex corporibus, adiuneta Telluri Luna fequunturs Et in-
terponentur vering; differentiz.

[| ié_s=s=sesesemhe
Je |2 3 —o 20] 5 45| = 0 4t 54
Mio a7} —o wlio 2a9f — © 6lio a3f
abs 9) #2 47437 22] Bo 20437 FI

1$ 149 36) er 4sta7_sti— 2 Blas 35
u

L¥ 130 asl er 4teo 9} p28 18

In Caput XV. NotzAuéors.

(1) Pp itansissomniam Planetarum.] Quidpeceur pean vel lxationem Sf

matis Planetarg, Cr quomodo peccativ hocredarguatur obferuationib.Brabeanisin Mar=
tesdiligenterexplicauiin Comment.de motibusillins Plantte,idg, ex profeffesparte prima, que eft de
aquipellentia byporbefiam.Ee quia ad declinandos hoserrores, nece(fefuit fundaimentit veluti munds
inipfuan folu centrumreponere :hine adeo factum,vtloca Zodiaci quibusplanete fiunt altifirni e
bumilisisoniam ampline Apogaorumn Cr Perigeorum nomen rerinerepoffent,vt quidem in Coper~
nico retinyerunt abufine:fedproprie c> fignificanter indigetarentur 2 me pbs c Perla. A


Chapter XV

Saturn Highest _. > | a 11 18 16
oo Lowest 8 39 0 82030 851 8 9 26 26
Jupiter Highest 5 2729 5 29 33 5 639 527 2
Lowest 4 58 49 4 59 58 439 8 4 57 38
Mars Highest 1 39 56 1 39 52 133 2 1 39 13
: Lowest 12226 12335 11839 1:23:82
aan Highest 1 00 1230 1230 1 6 6
Lowest 100 0 57 30 22 = —
Venus Highest 0 45 40 0 44 29 0 45 41 0 42 50
Lowest 0 40 40 0 41 47 oo. 0 40 14
Highest 02924 02919 03021 02827
Mercury Lowest. 018 2 014 0 014 0 013.7

Highest 0 2 30
Sun ©0900 000 00 0
Lowest 0 1 56

Those are the distances. | shall now also append a list of the angles subtended,
which correspond with the sines defined by the greatest distances of Venus and
Mars, taking the mean distance of the Earth as the whole sine or with the mean
distance of the Earth, if the furthest distances of the superior planets are taken as
the whole sines. Of these angles the former will be very close to the greatest
elongations of Venus and Mercury from the Sun, the latter to the equations for
Saturn, Jupiter, and Mars in apogee. In the first column are the angles derived
from the solids with the omission of the Moon; in the second the angles derived
from the Copernican distances from the Sun; in the third, finally, the angles
which follow from the solids, with the Moon added to the Earth; and in between
them are the differences in either direction.>

Saturn 5 25 —0 20 5 45 -041 5.4
Jupiter 10 17 —0 12 10 29 -0 6 10 23
Mars 40 9 +2 47 37 22 +0 20 37 52
Venus 49 36 +1 45 47 51 —2 18 45 33
Mercury 30 23 +1 4 29 19 -11 28 18


(1) Distances of all the planets.| What is at fault in this, so to speak, dislocation of the planetary
system, and the way in which this fault is at odds with the observations of Brahe on Mars, I have carefully
explained in my Commentary on the motions of that planet, and specifically in the first part, which is on
the equivalence of the hypotheses.’ And because in order to avoid those errors it was necessary to restore
the foundation, so to speak, of the universe to the actual center of the Sun, the result was that the posi-
tions in the zodiac at which the planets are highest and lowest could no longer retain the name of apogees
and perigees, though indeed in Copernicus they did retain it, improperly. But they were appropriately
and significantly named by me aphelia and perihelia.

53 foan. Keprert

DeLuna peculiare monitum,¢y de materia corporum 8 orbium.

On ergoexiguum fcrupulum Lung Orbis,vcutexiguus
fit,mouet. Quare porrode Luna tempus clt, vealiquid
dicam. Etincipio quidem fineambage, tibi Letorsfin-
‘\ cere meam mentem exponeresfecuturum nempe mein
haccaufa, quocunque propinquitas numcrorum prait.
Vefiinterpofitio Lung numeros & arcus Copernici ve-
rius reddit: dicam accenfendum illud fyftema craffitiei orbis magni. Sin
autem ciecta Luna melius nobis cum Copernico conuenire poteft: et-
iamegodicam, orbem magnumnontam craffum effe circumcirca, vt
coclumlunare tegatsfed emincre interdum furfum , interdum deorfum,
integrum Lung hemitphzrium fupravel infra margines orbis magni, in-
terdum & plerumque quidem minus hamifphatioextaresomnino pro-
utipfum corpus telluris, quod eft Orbis Luna centrum vel atcenderit,
yeldefcendcrit per orbis fui fpiffitudinem, (1) Nec herele fcio, quor-
fam magis inclinent Cofmographica vel etiam Metaphyfica ratio-
nes. (2) Concinnum quidem negotium effe videtur 5 venon fitin coe-
Jo orbisaliquis, quitalemgeratnodum , velutannulus gemmam ,cuius
eminentia obfit, quo minus abfolutiffima conftet orbi rotunditas. Ac
viciffim in cenfenda figura orbis quid attinet Lunz rationemhabere,
cum illa non proprie ad orbem terre yelutt cxterorum Planctarum
euagationes in altum, in protundum ( quz Phyfice commodiffime per
epycyclia demonftrantur ) velut , inquam , hac epicyclia ad funm
quodque orbem pertineat? Tellus enim cft cui Orbis ille certius a Sole
debetur, ipfa cius remigio inter catcros Planetas Solem circumic, ipfa
per fc,perque fua epicyclia nulload hoc Lune vfa minifterio faas per-
ficit varictates, vtdocent Copernici placita: Luna vero hane circa
tellurem exiguam domunculam quafi precarioaut conduétam obtinet,
Luna fequicur vel trahitur potius, quocunque Tellus quacunque varie.
tate graditur. Finge Tellurem quicfcentem, nunquam Luna viam circa
Soleminuenict,nedum circumuenict. Difcurfitat enim hinc indeangu-
ftis inclufa {pacijs circa terram, lucis humorumque Telluriminiftra,
veluti Atrienfis aliquis circa hcrum, autveluti quiin naui obambulant,
neque tamen fefc tatigando proficiunt in itinere, nifimagnavisaqua-
rum inccrtos quorfum cant, & velquictos promouear. Atque vefpa-
tium Luna ex orbe terreno,motumque fortita eft,fic & * multas condi-
tiones globi terreniadeptam, puta, continentes, maria, montes,aerem,
vel his aliqua quocung; modo correfpddentia,multis cdie€turis Maftli-
nus probat,nec nullas ego habeo; ve vel ob hoc folum verifimilior fit Co-
petnicus,qui candem Ieci motufq; communionem duobus hifce corpo-
rib. largicur.Ac certe @Aatums Creator vitimo veltiuiffe videtur Tellu-
rem hoc orbe Lunaris quia fimilé ci fitd atcribuere voluit,fitui Solis 5 vefi
& ipfa orbis alcuius centrum effet (vt Sol eft a aes
cuiuf



It is therefore by no means a small doubt which the Moon’s orbit raises, however
small it is itself. It is time, then, for me to say something about the Moon. And I
begin indeed without prevarication, reader, by frankly revealing my intention to
you, which is to follow in this debate wheresoever the closeness of the numbers
leads. Thus if the insertion of the Moon makes the numbers and angles of Coper-
nicus more accurate, I shall say that that system should be added to the thickness
of the Earth’s sphere. But if the rejection of the Moon can give us better agree-
ment with Copernicus, I shall also say that the Earth’s sphere is not so thick all
round as to cover the lunar heaven, but that the complete hemisphere of the
Moon sometimes juts out upwards, sometimes downwards, above or below the
boundaries of the Earth’s sphere, and sometimes, usually indeed, projects by less
than its hemisphere, in general in proportion as the body of the Earth, which is
the center of the Moon’s orbit, either ascends or descends through the breadth of
its sphere. (1) Nor, great heavens, do I know in which direction the cos-
mographical or even the metaphysical arguments tend more. (2) Indeed it
seems to be a question of tidiness, about whether there is not in the heaven some
sphere which carries such a lump, like the jewel on a ring, that its protuberance
prevents the sphere being perfectly round. On the other hand, in reckoning the
shape of the sphere, how is it relevant to take account of the Moon, although it
does not properly belong to the sphere of the Earth, as the wanderings of the
other planets in height and in depth (which physically are very conveniently ex-
plained by epicycles), as, I say, these epicycles belong each to its own sphere? For
it is the Earth to which this third sphere from the Sun is allocated, the Earth by its
impulsion goes round the Sun among the other planets, the Earth on its own and
by its own epicycles with no assistance from the Moon for this purpose performs
its variations, as Copernicus’s theories tell us, but the Moon holds its tiny home
round the Earth as if as a favor or on lease, the Moon follows, or rather is
dragged, wherever the Earth goes in any of its variations. Imagine the Earth at
rest; the Moon will never find its way round the Sun, much less make its way
round. For it keeps flitting to and fro, shut into its narrow spaces round the
Earth, and serves the Earth with light and moisture like some steward about his
master, or like people who wander about in a ship, who although they tire
themselves out make no progress in their journey unless the great power of the
waters carries them on, uncertain where they are going to and even if they rest.
And just as space and motion have been assigned to the Moon from the terrestrial
sphere, similarly Maestlin proves by many inferences, of which I have not a few,
that it has also got * many of the features of the terrestrial globe, such as con-
tinents, seas, mountains, and air, or what somehow corresponds to them; so that
on this account alone Copernicus is more convincing, as he endows these two
bodies with a common position and motion. And certainly the Creator, loving
Man, seems finally to have clothed the Earth with this lunar sphere, because he
wished to allot to it a position similar to the Sun’s, so that if it too was the center
of a sphere (as the Sun is the center of all things), it could be considered as like a

60 loan, Kerrert

Notz Auttoris.

@ JN Echeicleio,quorfam magisincunentrationes.] atiam in lucemprolatiscon-

templationtbus Harmonicis,dccifaedt hac concrouerfia,lib.V.Harmon, Primum enim core
poribus fis quingue adempta funt proportiones Orbinom ex pate: vltimsa fe. C> abfolutifimna Orbit
proportio communisesl fatta Cr corporibus & Harmontis Prop.XLVULGy XLIX.cap.1X. Quono~
gnine nibilex foliscorporibusin banc vel illainpartern de Luna difputan potest. Deinde fimaxime
tx Solis quiingue corporibus formarentur proportiones orbum;burustamen formationis modus alins,
‘prin quo inferiptio orbiuan Phyfica gradusperfettionis proportions Geometricaruim emularetur
fiabilituseSt Prop. XLVIXLVIL. Tertio conffat ex ommibusilins libri axiomatibus C* propofitiontb,
‘pleimans linsitationem proportionisdiaftematum fieri necelfariam  propter motus Planetarion; vt
{fiinter extremos motus efepoffent harmonie certe.Sibocnalla igitar potest haberi ratio Lune,ter-
am cincumcuoftantis, vt qua nibil confert ad incitandune vel retardandr vllias Plancte motu,
nec curriculum fusm circa Solem exercer,nct ex Sole regularis apparet cius motus, Nam ex Sole in~
(fbecina Lune motus videretur faltuatimn incedere.Sicigitur de orbeTlluris et difputandum,acfi Lu-
nacelum nullanneicrapitiem adderet

(2) Concinnum quidem, venon fittalis obis cum Nodo.} Hac gemino fenfa
poffent accipisprimus,textticonneniens,ei1 bic: efit quidem Orbis cuan nodo , fed oncludatur Orbt-
44 Planete,cants pifitudini,ve nodus bic, feu Lune calum,lateat totum intus , nibil impediens ex~
me intimaque fuperficiei rotunditatem abfolutam, Alter (enfushorum verborums poffet arripisfte
quod in genere abfierdsm fit Lunam corcumire Terram,dum hec interim circa Solem incedit. Vt igi-
tur banc etiam ebiectionem diluam: dico,quod boc tung conciunum videri porucrit cst nondun de-
techieffint louiales Planeta, cy caters in calo noua, Atex quo ila fciinus, concinawm nequaguam
annplius vider’ debet non effe,quod ominino eft, Nodus e.quadruplex csrcaloxemn , fipro Noda corpo-
eo fpatia curriculorum intelligas fic circa louem ordinatorum,
pt circa Terram Lunecurrictlum ordinatumest. "Nam decor-
jorea Orbiumn foliditate fupra fatiscantum , cr canetur etiam
antexty fequenti.

* Multas conditiones globi terreni adeptam.]
Confenfusin boc multorumper omnes tatesphilofopborum , qui
‘fugra vulgus fapere funt anfi. Diogenes Lacrtins Anaxagora tri-
buitslibro meo,cui Titulus,Ad Vitellionem Paralipomena, capi=
sede Lucefiderum,allegasti Plutarchum de facie Lune. Citatur
G Ariftotelesab Anerroe.Verum boc dogwia poftrenus Galilens
Telefcopio Belgicoconfirmatifvimum reddidit. Videctiam differ~
Zationean meam cum nuncio fiderio Galitei,

(3) Eadem fere proportio globi Tellurisad Orbem Luna.) Certa quidem eff
proportio ifta,f¢.queer.ad 59 circiter: at propurtio corporis Solis ad erbem Mercurij eft paulo alia; (c.
rion medins orbis Mercury , (ed intimus Gr anguflsfrinus eit afumendus; cniin Tabella capitis XV.
tribuuntur gr.14.cum Solis fermidiameter ex cade Tellure infpectus , contineat minuta rs.quare
fereeStproportio que.ad 56.

(4) Moderante cutfas,intelleétu proportionum.] Its quidem tunccenfebam; at
Pofteain Comment. de Marte, ne hoc quidem intellectu in motore opus effe demonflraui. Nam ctfi
Proportiones certe funt praferipte motibus omnibus, idgue ab Ilutelligentia ipfa jae or vnica,

hoc 8,4 Deocreatore:illetamen proportiones motuum inde & creatione bucufque conférvantur in=
suariabils,non per intelleftum aliquem Motori concreatum, fed per duas es altas,prima ct, equabi-
Jifsma & perennis rotatio corporis (olaris,cumn fpecie fuk immatcriata,in totum mundum emanante,
que fpeciesvicems motor praflat, alters caufa, funt libramenta &> magnetice diredtionescorporum
ipforum mobsliuonimmutabilia &perennia, Veficequenon magis fiopnscreaturis ifisintelledin
aad swendas motuuan proportioncs,atq, ibrelancibus Gx ponderibus micnte eft opus ad prodédampra
portionem ponderum.Etfifunt alia argument quibusprobatur,ine{fein corporibus Planctarum,fal-
sem Tellsris G Solis , sntelteCum aliquem,non quidem ratiocinatinnm vein homing; attamen ine
flindture


Chapter XVI


(1) Nor, great heavens, do I know in which direction the. . .arguments tend more.| But now that my
meditations on harmony have been brought out into the light of day, this controversy has been settled, in
Book V of the Harmonice. For, first, the proportions of the spheres have been partly removed from the
five solids. That is to say, in its final and most finished state the proportion of the spheres belongs to both
the solids and the harmonies in common, by Propositions 48 and 49 of Chapter 9. On this showing, no
argument for one side or the other about the Moon can be drawn from the solids alone. Second, even if
the proportions of the spheres were formed chiefly from the five solids alone, yet it has been determined
by Propositions 46 and 47 that the mode in which they were formed was different, and such that the
physical inscribing of the spheres reflected the degrees of perfection of the geometrical proportions.
‘Third, it is established from all the axioms and propositions of that book that there must necessarily be a
final limit to the ratios of the intervals on account of the motions of the planets, so that there could be
definite harmonies between the extreme motions. If that is so, then no proportion can be found for the
Moon's flitting round the Earth, as it contributes nothing to hastening or retarding the motion of any
planet, and does not perform its circuit about the Sun, nor have a motion which appears regular from the
Sun. For the motion of the Moon viewed from the Sun would seem to proceed in jumps. Therefore we
must argue on the sphere of the Earth as if the heaven of the Moon added no thickness to it.

(2) Indeed (it seems to be a question of) tidiness, about whether there is not such a sphere with a
Jump.) These words can be taken in two senses. The first, agreeing with the text, is this: that there is in-
deed a sphere with a lump, but it is contained within the orbit of the planet, the thickness of which is so
great that this lump, or the heaven of the Moon, is completely concealed within it, and does not hinder at
all the absolute roundness of the outer and inner surface. The other sense which could be forced into
these words is the following: that it is categorically absurd for the Moon to go round the Earth, while the
latter at the same time proceeds round the Sun. Then to clear away this objection as well: I say that this,
could have appeared tidy when the satellites of Jupiter, and the other new objects in the sky, had not yet
been detected. But ever since we came to know of them, it should no longer seem at all tidy that what
decidedly exists should not exist, that is to say the quadruple lump round Jupiter, if by corporeal lump
you understand the spaces occupied by the courses which have been appointed round Jupiter in the same
way as the course of the Moon has been appointed round the Earth. For on the corporeal solidity enough
‘caution has been shown above, and caution will be shown in the following text.

* It has (also) got many of the features of the terrestrial globe.| The consensus of many philosophers
on this point throughout the ages, who have dared to be wise above the common herd. Diogenes Laer-
tius® attributes it to Anaxagoras; and in my book entitled Paralipomena Ad Vitellionem, in the chapter
“On the Light of the Stars,” I have referred to Plutarch’s On the Face of the Moon. Aristotle is also cited
by Averroes. However Galileo has at last throughly confirmed this belief with the Belgian telescope. See
also my Conversation with Galileo's sidereal messenger.*

(3) The proportion of the Earth's globe to the sphere of the Moon is almost the same.| Indeed this
proportion is certain, that is about as 1 10 59; but the proportion of the body of the Sun to the sphere of
Mercury is slightly different, that is, not the mean measurement of the sphere of Mercury, but its inside
and smallest measurement must be assumed. To that in the table in Chapter 15 are attributed 14°, where-
as the radius of the Sun, as seen from the same Earth, contains 15'; so that the ratio is about | to 56.

(4) The understanding of the (geometrical) proportions governing their courses.| So indeed 1 then
supposed; but later in my Commentaries on Mars* I showed that not even this understanding is needed in
the mover. For although definite proportions have been prescribed for all the motions, and that by the
supreme and unique Understanding himself, in other words, by God the Creator, yet those proportions
between the motions have been preserved unchanged from the Creation right up to the present not by
some understanding created jointly with the Mover, but by two other things. The first is the completely
uniform perennial rotation of the solar body, along with its immaterial emanation, which is diffused to
the whole universe, an emanation which takes the place of a mover. The other cause is the weights and
magnetic directing forces of the moving bodies themselves, which are immutable and perennial proper-
ties. Thus there is no more need for these created things to have understanding to observe the proportions
of their motions than there is for the scales and weights of a balance to have intellect to declare the pro-
portions of weights. Nevertheless there are other arguments by which it is proved that there exists in the
bodies of planets, at least of the Earth and Sun, some understanding, not indeed rational as in Man, but

MysrertvM CosmoGRAPHICVM. or

firm vtinplanta,quo conferuatur fpeciesfloris,c nurnerusfaliorum. Deboc vide Epilegesibre-
im VG VHarmonices nofire.
(5) Nullum enim punétum graue eft } Ita conceptumesthoc arguinentum, vt au=
Aire velim phyficos,quid contradicere pofint. Nam ab bis 25.annis nen quod fciam extitit, qui ud
cxcuteret.At me candor folus mouet,ve ipfecxcutism. Vides igitur Lector quid voluerim, Centrum
folumeffequod primo circa Solem agaturiingyrum:ld vero vel folonutu fieripaffe, cumgraue nan fit,
Ve euinspars nulla. Hane propofitionein non potest mib eriperephyficus,qui contendit, quod bic fequi-
turomnia centrum equi. Et quia vulgata doctvina phyfica tenet hoc de centro mundi quod omnia
Sratiaid centrum querantideoexiftimantiego, poffe grauia edidem operacentrum fui corporis que~
rere, Verumin Epitomes Aftyonomia ib. 1. demonflraui,falfum efe boc phyficorum axioma, quod
gratia quarane vllumn centrum ve tale falfifimnum quod centruin totins mundi, vesum, fed per acti
dens, quod centrum Tellurisappetant, non quam id punctumelt, fed quia corpus Telluris appetunt,
quod cum fc rotundumex eo ferivt appetentiaftaferatur verfus muedinm , C> fic verfus centrum,
aadco quidem, terra figurans haberetdiftortain fenfiiliter, Grauia non verfus vnum yndig, pun-
um tenfiura fuerint, Hocigitur fundamento corrnente, flructura etiam euertitur buit nina, Sci~
Ticet corpora Plantarum in motu, feu tranflatione fui circa Solem, non funt confideranda vt punta
mathematica, fedplane vt corpora materiata,ccum quodam quafipondere(vt in librode ftellanoua
Seripfi boc e5t,in quantum funt pradita facultate renitendi motui extrinfecus lato , pre molec
1s, denfitate materia.Nam quia omis materia ad quictems inclinatinlocoila in quo eSH(nificor-
pusvicinum vi magneticaillam ad fealliciat) bine adeo fit vt virtus Solie motoria pugnet cum hat
smertia materie,ficurin lancepugnant duo pondera ,exque vtrarumgue virium proportione tandem
enafcatur celeritas veltarditas Planeta, Vide introduitionem in Comment. Martis , & ipfa Com-
mentaria pafim; precipue vero librum 11 Epitomes Aftronomia,

Nequetamenex-cofequitur quod bic per fulfiun ratiocinationen amlitum ibam, dubios
titubantes motoris greffus effc, fi laborat in pondere,vincitque in pugna. Namcerta cy conftans e&
proportio virium iter fe verarumgue , & vidloriapartiblis,pro viriua modulo; ve neque Planeta
in codem hareat locosnequerotationis Solaris eleritatem affequatur.

(6) Non magisarque magnes, dum actu ferrum habuit,ingrauefcie.] Manife-
fisexperimentisboc falfiun deprebenditur. Ponders feorfim ferrum, feorfim cr Magnetemscollige
pondcrain ynam fuminam Sufpendatur deindeferrum 4 Magnete viille inuifiili, Magnes vero ne~
Glatur 4 lance, aut inijciatur quia vispermeat lancern,finon fitferrea: videbis, Magnetem, dum ails
tenet attradbumnferrum,equeponderaturumn veri{que prius ab inuicem feparatis,

(7) Influxuum ccleftium in media corpora vehiculum.] Non equidem, quod
inflaxusceclefesindigeant aliqua matcvia qua ed nosdeuchantarsfalfumenimeillad Arifioelis,
acre opuseffised enfionem corporis Solaristranfportandam vfque ad oculum; ve in Opticis demon=
flraui: quin potius, quo minus occurrit materia, initinere medio, boc minusimpeditur lax in traie~
Gionefua. Hoc igitur fbi volume if verba cut corpora non impediant ,quo minusinfluxuscelefls
in imimapenetrent: ficetiam Motoriasfacilrates non indigere corporibus aliquibus intermedi,
quubus veri catbenis aut vedtibusmouenda Planctarum corpora prebendant.Ludereplacuitin voce
acvispauto audacins. Quid Orbis vel celum? Quid nifi acr?Et quid aer? Quid nifi fpecies imimateriaa
4 corporis quod motum Planctis infert,in gyratione verfanti? Atqui fepofite Lufu,concedamus, at~

rem naftrumm efecorpus ateriatumn permesbilefaculratibusmagnetics,motoris,calefadto~
riis,illuminatoriis, C finilibus: vt fit vapor non toto genere diuerfumm ab acre,
fed faltem gradubuscrafitic ditindtus a circumfupe
cris campie,


Chapter XVI

instinctive as in a plant, by which the type of flower and the number of leaves are perpetuated, On this
point see the Epilogues to Books IV and V of our Harmonice.

(5) For no point. . .has weight.| This argument has been conceived in such a way that I should like to
hear from the physicists what they can say against it. For in the course of the ensuing 25 years no one as
far as I know has come forward to examine it. However I am moved by honesty alone to examine it.
Therefore, reader, you see what I mean, which was that itis the center alone which is propelled in a circle
round the Sun; and that indeed could be brought about by a single nudge, as it has no weight; as
something which has no parts. This proposition cannot be snatched from me by a physicist who argues
what follows here, that everything conforms with its center. Also because the commonly received doc-
trine in physics holds with respect to the center of the universe, that everything which has weight seeks
that center, for that reason I supposed that things which have weight by the same token seek the center of,
their own body. However in the Epitome of Astronomy, Book 1, I showed that the axiom of the
physicists stating that things which have weight seek any center as such is false, and that they seek the
center of the whole universe is more false. It is true, but by accident, that they strive towards the center of
the Earth, yet not insofar as it is a point, but because they strive towards the body of the Earth; since itis
round, as a result of that it comes about that the direction of this striving is towards the middle, and thus
towards the center. In fact, it follows that if the Earth had a shape which was sensibly distorted, things
which have weight would not tend towards the same point from every direction. Consequently with the
collapse of this foundation, the edifice, which was excessive for it, is also demolished. Clearly the bodies
of the planets in motion, or in the process of being carried round the Sun, are not to be considered as
‘mathematical points, but definitely as material bodies, and with something in the nature of weight (as I
have written in my book On the New Star), that is, to the extent to which they possess the ability to resist
a motion applied externally, in proportion to the bulk of the body and the density of its matter. For
because all matter tends to remain at rest in the place where it is (unless a neighboring body attracts it to
itself by magnetic force), it comes about as a result that the motive power of the Sun contends with this
inertia of matter,® as two weights contend on a balance, and from the relative strength of the two forces is
at length produced the quickness or slowness of the planets. See the introduction to the Commentaries on
Mars, and the Commentaries themselves throughout, but especially Book IV of the Epitome of
Astronomy.?

However, it does not follow from that, as I was setting out to refute here by false reasoning, that it
makes the progress of the mover doubtful and faltering, if it struggles in the balance and wins the contest.
of weights, For the proportion of the two forces to each other is definite and constant, and the victory
‘can be shared in accordance with the rating of the forces, so that the planet neither sticks in the same
place nor matches the speed of the Sun’s rotation.

(©) Any more than a magnet, when it attracts iron by its action, becomes heavy.) This is detected as
false by clear experiments. Weigh separately a piece of iron and a magnet, and add the weights together.
Then let the iron be suspended from the magnet by this invisible force, and let the magnet be fastened to
the scale of a balance, or thrown into it, since its force would pass through the scale, if it were not of iron,
‘You will see that the magnet, while it attracts and holds the iron by its action, will weigh the same as the
two did previously, when they were separate from each other.

() It conveys the heavenly influences right into our bodies.| Not, that is, because the heavenly in-
fluences need some matter to carry them to us; for Aristotle is wrong when he states that air is needed to
transport the sensation of the solar body to the eye, as I have shown in the Optics.* Rather, the less mat-
ter the light meets in the course of its journey, the less itis impeded in its passage. The meaning of these
words, then, is as follows: just as bodies do not impede the heavenly influences from penetrating into our
inner parts, so also powers capable of producing motion do not require intermediate bodies, by which to
take hold of the bodies of planets which are to be moved as if by chains or bars. I chose to make rather
too bold a play with the word “air.” What is a sphere or a heaven? What but air? And what is air? What
but an immaterial emanation of the body, which imparts motion to the planets, as it turns in its gyration?
But, laying aside the play on words, let us concede, that our “air” is a material body, through which
magnetic, moving, heat-producing, light-producing, and similar powers can pass, so that it is a vapor not
totally different in kind from air, but rather distinguished by degrees of thickness from the expanses of
air which surround it

62 loan, Keprert

Alind de Mercurio monitum.
ess Li v p magis mirabcre, cum promiferim , velleme cor-

7273 poribus ipfis in{cribere Planctas , cur Mercurium non
REQ Gaacdeoinferipferim: fed pails fim cumin circuloali-
ee

quo vitra orbem infcriptilemad quadrati Oftaedrici am-
@ plicudinem cxpatiati. Nam fupra cap. 13. & 14.pro577-
numero orbisin{cripti vfurpaui 707. numerum circuliin-
{cripti quadrato. Caufam dicam. Primum, quia cius 4 Sole digrcffiolon-
giorminime pati potuittamanguftos carceres: deinde quia & Oftae-
dron inter corpora,& motus Mercurij inter Planetas peculiare quid , &
communcinuicemhabent. Nam in folo O&aedro fuperangulum erc-
&o vfuvenit, vequadratum diredislatcribus viam aliquam méftretam-
plioricirculo,quam eft orbisin{criptus,per medium tranfeundi.Id quod.
innulloaliocorpore quomodocunque voluto vfa venit. Semper enim
tranfuerfa per medium & impediraincedent latera.

In hoc fchemate quatuor liner extreme
funt quatuor perpendiculares totidem plano-
rumin Odaedro. RIT V funteorum piano-
rum centra,determinantia amplitudinem orbis

S infcripti,de quo hic vides Circulum maxirnum,
Qui orbis fi incelligacur volui faper punétisad
XH, duos angulos figura, reperietin P Qua-
drante a polis circumcirea amplitudinem ali-
quam maicrem quam eft OT, yel O P emidia-

meter orbis,nempe OQ. Difterentiacius ct PQ. Ertantacftlatitudo
circuli , qui vitra orbem excurrens, inftar Horizontisalicuius in {phara
armillari,per medium Odtaedri tranfirepoteft. Qenim & $ {unt media
punéta duorum laterum, proinde & proxima orbi.

Quomode fi animatus quidam planeta permedium Ogtaedrum.
currerc iuberctuz,& angulosduos pro polis,amplitudinem infcripti pro
curriculo obferuaresnon hercle mirum,fi inuicacus ila amplitudine, vbi
nulleilli meta obftarent per totum ambitum, exorbitaret aliquando,ve
Phacthonille,tantifper, dum repellerctur aboccurrentilatere., Quod
periocum dixi,id {erio aiunt Attitices euenire Mercurio.Cum enim c¢-
teriomnesin fingulis rcuolutionibus defcribant ciu(dem amplitudinis
citculos(quantum enim ab vna parce difcedunt,tantit exaltcra viz parte
acceduntadSolem) (1) folus Mercurius ab Artificibus obrinuit, yra-
liquando maiorem, aliquandominorem circulum defcribere diceretur:

idque priuilegium merum haberet. Dicunt enim
z | illumaccedere& recedere 4 Cétro fui orbis O per

° Yaz lincam rectam Y Z; vbi femidiameter O Y longe

minorem Circulum defcribit,quam OZ.Nam c¢-

Cerasinaqualitates omnes cum alijs equaliter for-

titus



You will wonder all the more, since I have promised that I intend to inscribe the
planets within the actual solids, why I have not inscribed Mercury within the oc-
tahedron, but have allowed it to diverge to the full breadth of the square in the
octahedron in a circle outside the inscribed sphere. For above, in Chapters 13 and
14, I have taken instead of 577, the value for the inscribed sphere, 707, the value
for the circle inscribed in the square. I will explain the reason. First, it is because
its further deviation from the Sun could in no way permit such narrow confines;
and secondly because the octahedron among the solids and the motion of Mer-
cury among the planets have a feature which is peculiar and shared by both. For
only in the case of an octahedron standing on a vertex does it happen that the
square built from the perpendicular edges shows a path by which to go round in a
circle wider than the inscribed sphere. That occurs in the case of no other solid
however it is turned. For edges always cross the path and block it.?

In this diagram the four outer lines are four perpendiculars of the same number
of faces on the octahedron. RITV are the centers of the faces, which determine
the size of the inscribed sphere of which you here see a great circle. If that sphere
is understood to be turned on the points at X and H, two vertices of the figure,
there will be traced out at P, a quadrant from the poles, a width all round which is
greater than OI or OP the semidiameter of the sphere, that is, OQ. The difference
is PQ. And that is the size of the circle which can run round outside the sphere,
like a horizon on an armillary sphere, and go round inside the octahedron. For Q
and S are the midpoints of two edges, and consequently the nearest points to the
sphere.

Thus if some sentient planet were ordered to follow a path inside the oc-
tahedron, and to take two vertices as the poles of its orbit, and the breadth of the
inscribed sphere as the limit of its track, heavens! it would not be surprising if at-
tracted by such a breadth, where no boundary markers obstructed it round the
entire circuit, it should sometimes leave its orbit, as Phaethon did, up to the point
where it was driven back by finding an edge in its way. Although I have said this
jokingly, the practitioners say seriously that this happens to Mercury, For
whereas all the rest in their individual revolutions describe circles of the same size
(for if they move outwards on one side, on the other side of their path they move
in towards the Sun by the same amount), (1) Mercury alone has persuaded the
practitioners that he should be said to describe a circle which is sometimes larger,
sometimes smaller, and that he should have that privilege without penalty. For
they say that he moves towards and away from the center of his orbit O along the
straight line YZ, where the semidiameter OY describes a far smaller circle than
OZ.* For he has been allotted his share of the other irregularities equally with
the rest, and he has not exchanged any of them for this departure from his orbit.


titus cfs nullamque cum hac cxorbitatione commutavit. (2) Ercum
cxtcrorum ecccntrotetes omnes, fi non proportion: liter, fictamen de-
crefeant; veminorisfemper minor fiteccentricitas:folus Mercuriusim-
manem habet,nempedecuplum Vencris,cumipfi ve inferiori minus et-
1am deberecur. Quare etfitllam inequalitatem priuatam nondum cum
hac civeuli ab orbe differentia conciliauerim, nec ea furtaffe conciliari
poflit,ve prodita eft ab Artificibus,ad amuffim : Nihilominus ego nédu-
bito,quin creator ad figure huius prefcriptum in motibus Mercurio tri-
buédis refpexerit. Quo diuinior magis magify; mihi & Aftronomiaé&
Copernici placita,& hecipfa 5. corpora videntur.

G) Querantalij,quivoluerint,caterarum etiam eccétricitacum cau-
fas ex fuisquafque corporibus.Cum cnim neqs hz exorbitationes a Deo
temete & finécaufa tance fingulis Planctis indultz fint: non defperanda
eft neq; harumcaufarum inuettigatio.

Porro vevarictas Mercurijad Ogtaedron accommodetur , fic agi
poffer. Sumcretur proportioeccentr. ¥ad diftantiam mediam 4 © pro
certa,ve quia in Copernico dittantia(ficut vides in tab. V.cap.15.) longif
fima eft 488. breu:llima 231. media igitur erit 360. & craffities tota 257.
Heciam craflities cotrigeretur proportionaliter,vt quia circulusO @ae-
dn pro 488. numero Copernicilargitur non plus 47 4-crgo craflitieserit
in hac proportione 250.& media correcta diftantiz 349. Iam vide, quid
crbisin O@acdroad:mitrat,{cil.387. Differentiaigitur inter 387.altidi-
mam orbis,& 349.mediam eft 38. & duplum 76. craihities orbis ad modi
catcrorum,maior quidem adhuc quam Veneris, fed tamen non itaim-
manis.R cliqua differentiammteraltiffimam orbis 387-8 aleflimam circu-
li 47 4.qux clt 87.debetur peculiariexorbitationi Mercurij_ Hoc Saxe
ge, an abijciendum, anconciliandum cum vankr» forma motuumin
¥,an noua motuum ratio confticuenda,confiderent Artifices.Nec enim
ita bene explorati funt crrores huius fideris , vt cius orbis correétione
non ¢geat.

In Caput X VII. Nota Audonis.

«) Solus Mereuriusobtinuit.] Quale it illud, quod artificespeculisriter adferibuns Mere

curio,redtiuspetesex Ptolemeo ipfo,exque Purbachij & Majilins Theoricis: dinigue quomodo
Coperinicusillud duplici via, qui. {ibiipfe non [atifecie) informam fuarum bypotbefium tranflule=
rit, feipfiom tarnen confuderit, plus aliquid praflans (per fos motus triangulationis alicuins mules)
qieamex Proleieofibipropofuerat expriniendumn: id totum, nec adco neceffarium eft hoc loco expli=
.avicums fit d opinionibus homsinum,nonde veritate rerum; ¢> fiquid veulsterdici pore, reCtins al
lorfiun reiitur. tureesion, hoc ft,quod Mercurins fact enormem Eccentricitatem circuli fuia Sole
quem crrcilum Prolemaus Epicyclum , ego cecentricum dico , quodque in ilo etiam eccentric mout-
tur ineqsaliter,ad propartiouem eccentricitatis, Ex bisprincipiis , & ex eccentricitate Telliris, quo
modoconftaca fitphantafia lla duplicisin Mercucrioperigai,g> fic motus quafitriangularixid expli
cabicur tu denzanftratione motuann Merctwrifynec plane pratereofumnmam retin Epit. Aftr Lib.6.Suf
ficit hoc loco,ittnd monere , non ffe buins fingularitatis Mercurialiscaufam aliquam Archetypicams
ex O-sedro;coqite falf-an huins capitis Hypothefin : iucundifimam tamien recordationem buius E~
pichirematie vt app-reat, quibus ignorantie gradibusad Aftronomis ftientiam & conflitutionem

afenderi,
(2) Er

Chapter XVII

(2) And whereas the eccentricities of the others all decrease, if not in proportion to
their size, at any rate in such a way that the smaller orbit always has a smaller ec-
centricity, Mercury alone has a huge one, in fact ten times that of Venus,
although as it is the lower planet it ought also to have had a smaller eccentricity.
For this reason although I have not yet linked this special irregularity with the dif-
ference between the orbit and the sphere, and perhaps it is impossible to link
them, as has been claimed by the practitioners, precisely, nevertheless I have no
doubt that the Creator oberved the pattern of this figure in allocating motions to
Mercury. All the more, and ever yet more divine do astronomy, and the theories
of Copernicus, and these very five solids, seem to me.

(3) Let those who wish seek the reasons for the other eccentricities also in the
appropriate solid in each case. For since such large departures from their orbits
have not been conceded to the individual planets by God at random and without a
reason, neither should we despair of finding out even those reasons.

Further, adjustment of the variation of Mercury to the octahedron could be
achieved at this point in the following way. The ratio of the eccentricity of Mer-
cury to its mean distance from the Sun would be taken as certain, that is, since in
Copernicus the longest distance (as you see in Plate V,* Chapter 15) is 488, the
shortest 231, the mean distance will therefore be 360, and the total thickness 257.
Now this thickness would be corrected by proportion, that is, since the circle of
the octahedron instead of Copernicus’s value, 488, concedes no more than 474,
therefore the thickness will be 250 in that ratio, and the corrected mean distance
349. Now consider what the sphere in the octahedon permits, that is to say, 387.
Then the difference between 387, the highest level of the sphere, and 349, the
mean, is 38, and twice that is 76, which is the thickness of the sphere reckoned in
the same way as the others, still larger indeed than that of Venus, but yet not so
huge. The remaining difference between the highest level of the sphere, 387, and
the highest level of the circle, 474, which is 87, is due to the peculiar departure of
Mercury from its orbit. Whether this attempt should be rejected, or reconciled
with the pattern of motions assumed in the hypothesis for Mercury, or whether a
new system should be established for the motions, let the practitioners examine.
For the deviations of this star are not so well investigated that its orbit does not
need correction.


(1) Mercury alone has persuaded.| The nature of what the practitioners ascribe as a peculiarity to
Mercury you can more properly find in Ptolemy himself, and in the Theories of Peurbach and Maestlin,
and lastly in the way in which Copernicus incorporated it into the pattern of his hypotheses by two
methods (because he was not satisfied about it himself) yet confounded himself, providing something
more (by these motions which emulate a triangular arrangement) than he had planned to draw from
Ptolemy. It is not so essential for this whole matter to be explained at this point, as it concerns the opin-
ions of men, not the truth of things; and if anything can usefully be said, it is more properly dismissed to
somewhere else. For the fact of the matter is that there is an enormous eccentricity in Mercury's circle

circle which Ptolemy calls the epicycle, but I call the eccentric, and that on that circle it
also moves non-uniformly in proportion to the eccentricity. From these principles and from the eccen-
tricity of the Earth, the way in which the fantasy of a double perigee of Mercury and thus of its motion’s
being, so to speak, triangular, was concocted, will be explained in the derivation of the motions of Mer-
cury; and I certainly do not omit to mention the summary of the matter in the Epitome of Astronomy,
Book VI. Itis sufficient at this point to make the comment that there is not an archetypal cause of this
singular property of Mercury which arises from the octahedron; and hence the hypothesis of this chapter
is false; yet it is very pleasant to recall to mind this argument for it, so that it is made apparent by what
steps of ignorance I have ascended to the knowledge and establishment of astronomy.


yer(arisadeo vt magna conieéturacontrame fuiffet, ficum nuracr’s Co-
petnici penitus confentiffem.

Eorum autemargumentorum hoc primum efto,quod Prutenicus

calculus non raro incolligendis Planctarum locis talitur. Multa qui-
dem reftaurauit nobis Copernicus in collapfa motuum {cientia: muito-
que noftra quam patrum memotiaspurior eft Aftronomia. Verunta-
men firem ipfam penitusinfpiciamus,facert vtique cogemur, nos abilla
beara & optabili perfeétione haud muito propiusabefle, quam ab hodi-
erna vetusabeft A ftronomia. Longa via elt,& varicambagesad hanc ve-
ritacem. Monftrarunt illam nobis veteres,ngrelli fant maiores noftri,
nosillos anteuertimns , & gradu propiori contiftimus, fed metam non-
dum attigimus. Nonegohxcin Aftronomiz contemptum dico: Eft ali-
qua prodiretenus, finondatur vicre 5 fedidco, ne quis temeregrauius
quid in hancdifcordiam ftatuat,& dum me petit,& hiec quinquc corpo-
125 inipfa fundamenta Aftronomiz infulcet, Ad omnium Artificum
ob{cruationes prouoco : ex quibus videre eft , quanta fepe fit inter
yerum locum, & inter cum, quem calculus indicatdifferentia, quain-
terdum (2) inquibufdam ad fecundum integroram graduum longitu-
dinem excrefcit. Quod cumutafir, expedit mihi nonnihila Coperni-
cinumeris difcedere + & iam porto diligentium obferuatorum iudicio
relinquitar,veriarcus cum coclo propius conueniant, mei ,an Coperai-
cani.

Alreram argumentum,quo differentia huius culpam in ipfas Pra-
tenicastransfero, prabent mihi fufpecte Planetarum Eccentricitatess
quod cotendie , vequamuis nec mei arcus omnino perfeati & certi fine
(licuti fateri cogor)ramen vitium ex contagione Eccentricitatum con-
traxermt.Si corpora faper media planetarum diftatiz fuperficies pha-
ricas {truercntur,ve adem fuperficics circum{cripti corporis centra, &
infcripti angulos tangeret;tum nihil mihi ret effet cum orbium craflitic,
quam requirunt viz Planetarum Eccentricz.

G) Cumautemillud ficri non potucrit,& nondum fimiliter caufa
Eccentricitatum,ve & differentiarum, explorata fit;oportuic me orbium
spiffitudines 4 Copetnico,tanquam certas mutuari 3 quastamen non
certiffimas effe in confelfoeft. “Quamuis enim omnis celeftium mo-
tuum hiftoria lubiico eft aditu, perdiuturnas , & difficiles obferuatio-
nes 5 precipue tamen hoc in conftituendis Eccentricitatibus & locis
Apogzorum apparet. Solaris (vel tcrreftris ) Eccentricitas omnium
reétiffimehabéredebebats. Nam & viciniffima ftellarum eft Tcllusno-
bisincolis, (4) & paucioribus quam caccre motibus vchitur. Inmun-
do vero per interieéta corpora ftrucndo,fupra cap.XV. vidimus,quan-
tumafferatmomentumad omnes fpharasartandasaut laxandas folius
¥egvioxs Junaris appofitio,vel exemptio, qui valde exigua porniiculater-
reftris orbiscraffiticm excedit. (5) Hicigiturorbis, qué certiffime di-
menfumhabere oportebat , & pofle verifimile crats hic, inquam,vide,in
quanta verfecur difficultace apud Copetnicum qui ipfe lib.3. Reuol.cap.
2o0.queritur, (6) quod per minima quedara G vix apprehen ibilia magna va-
tiocinart cog imur,quod interdum fub uno diuerfitatis ferup.5.vel 6.g7 pratcreanty
& modicus error in immenfum fofe prepaget. Quanto peius igitur habe-

I bunt

Chapter XVIII

and the Prutenic Tables are at fault; so that there would be a large query against
me if I agreed completely with Copernicus’s values.

Now of those arguments let this be the first, that the Prutenic calculation is not
seldom wrong in determining the positions of planets. There are indeed many
things which Copernicus repaired for us in our ruined knowledge of the motions,
and our astronomy is much purer than our fathers remember. However, if we
thoroughly examine the facts of the matter, we are certainly obliged to confess that
‘we are not much nearer to that blessed and desirable state of perfection than the
ancient astronomy was to the modern. The way to the truth of the matter is long
and has many windings. The ancients have shown it to us; our predecessors have
started on it; we go on ahead of them, and stand on a closer level, but we have not
yet reached the goal. I do not say this to show contempt for Astronomy —

“You can get somewhere, if you can’t get further”! —

but to prevent anyone rashly putting a more serious construction on this disagree-
ment, and while aiming at me, and the 5 solids, scoffing at the very foundations
of astronomy. I appeal to the observations of all the practitioners. From them it
can be seen how great a difference there often is between the true position and
that which calculation indicates, amounting sometimes (2) in certain cases to as
much as two complete degrees in longitude. In that case, it is helpful for me to
depart to some extent from Copernicus’s values; and from this point on, it is left
to the decision of careful observations which angles agree more closely with the
heaven, mine or those of Copernicus.

The second argument by which I transfer the blame for the difference to the actual
Prutenic Tables is provided for me by the suspect eccentricities? of the planets. This
tends to show that although my angles are not completely correct and certain (as ] am
obliged to confess) yet they have contracted the fault by infection from the eccen-
tricities. If the solids were constructed over the spherical surfaces of the mean distance
of the planets, so that the same surface touched the centers of the faces of the cir-
cumscribed solid and the vertices of the inscribed solid, then I should not be troubled
by the thickness of the spheres which the eccentric paths of the planets require.

(3) However, since that has not been possible, and neither the cause of the eccen-
tricities, nor that of their differences, has yet been investigated, I had to borrow the
thicknesses of the spheres from Copernicus as if they were certain; yet it is admitted
that they are not entirely certain. For though the approach to the whole history of the
heavenly motions is slippery, and requires lengthy and difficult observations, yet it is
particularly apparent that in establishing the eccentricities and the positions of the
apogees, the solar (or terrestrial) eccentricity should be the most accurately known of
all. For the Earth is both the nearest of the stars to us who inhabit it, and travels (4)
with fewer motions than the others. Now in constructing the universe by the inter-
polation of the solids, we have seen above in Chapter 15 what a great effect on the
narrowing or widening of all the spheres the addition or removal of only the tiny
heaven of the Moon has, which goes outside the thickness of the terrestrial sphere by a
very tiny little fraction. (5) Then, in the case of this sphere, of which we ought to have,
and probably could have, an absolutely certain measurement, consider, I say, what
great difficulty is found over it in Copernicus, who himself in Book III, Chapter 20 of
the De Revolutionibus complains (6) that “we are forced to work out from very small
and almost imperceptible quantities large quantities (the errors in) which sometimes
exceed 5 or 6 degrees for a difference of a single minute, and a small error is immense-
ly magnified.’ How much worse off, then, will be the thicknesses of the spheres

66 loan, Keprert

bunt fpifficudines orbium & remotiorum a nobis,& qui pluribus motuit
yatietatibus funt obnoxi i}-Quod fi aut orbiumilla iz" certiflime explo-
rata,autcaufe faltem probabiles patefa& fuerint,curtanta fingulis at-
uibutafint4 Conditore: (7) tum cgo fpondco me produéturum ex his
corporibus atcus peromnia motibus confonos.Sic enim exiftimo,quic-
quid hanc proportionem cactorum inuentam adhuc impediat, quo mi-
nus ad cxatam motuum cognitionem veniatur: (8) idomneineccen-
tricitatum vitia conferendum;quibus fublatis, (9) magno adiumento
Artificibus futura puto folidahac quinque; ad correétionem motuum
quam paflim meditantur non pauci.

Vehocillis fpondcam de eccentricitatibus,mouit me & hoc,quod
(10) _vbique de minori particula,quam eft 77s orbisintegrum contro-
uertitur.Eripe namque omnibus fex orbibus fua ix" nota,autdupla fin-
gulis attribuesvidebis mundum & aes4a?apio’s omnes in immenfum il-
lic confidere & augeri,hicdiftrahi & deminui. Veitaveritas internihil
& duplum confiftat, neque metucndum fic, ne nimiam habeat Artifex
licentiam eccentricitates mutadi ; fi quis illas his figutis aptare conctur.
Arquefichzcaltera ratio eft, que me dedifcordiainter meos & Coper-
nici numeros excufare poteft.

Tertiam mihi prabencipfi numeri Prutenicarum ctiamnum craf-
fiynec itaexpreffi,vtnon poffit aliquido bonacum venia vel femiffe gra-
dus abiis ditcedi. Rheinholdus quidem in Prutcnicis omnia dihgentif-
fimedifpofuit. Sednolimaliquis hac {pecie {crupuloficatis inefcatus,
craffiufculos numerosin A ftronomia faftidiatsrem exactius cenfeat. Il-
Ja fummiviri minuta & {crupulofa curaaut cft propter certitudiné cal-
culi, autnon neceffaria in partibusnumerorum, ipfos vero totos nume-
ros, quos tam {crupulofe diduxit,¢ Copernico excerpfir, ficutiillos re-
perit.

Acipfe quidem Copernicus quam humanus fitin rccipiendis quae
libufcunque numeris qui quadamtenus ex voro obucniunt,& ad inftitu-
tum faciunt:id experietur diligens Copernicile&tor. Numeros qui per
diuerfas operationes vi demonftrationis penitus conuenire debcbant,
non repudiat ,quamuis difcrepentaliquot {crupulis. Obfcruationes in
VValtero , in Ptolemzo &alibificlegir, vtijseo commodioribus vta-
turad extruendum calculum, vndeintempore horas, in arcubus qua-
drantes graduum & aimplius interdum negligere vel mutare nulla illi
rchgio. Alicubi, vc in murata eccentricitate Martis & Vencris, finus
etiam difcrepantes a veritate acceptat , tantum ideo,quia parumper ad
€0S, quos optat, digitumintendunt. Multaquz ex ipfius confelsione

emendanda fuiffent, integra & fincera ex Ptolemzo depromit, muta~
tis cxteris fimilibus ; atque 4js poftea fundamenta nouz Aftronomiz:
extruit. Quorumomnium mihi plurima documenta dedit Mzftlinus:
qua breuitatis caufa mitcoafcribere. Atque adco in reprehenfionem in-
currere iure videretur ; nifi confulto feciffet , eo quod praftarct, im-
perfeétam quodammode habere Aftronomiam , quam penitusnullam.
jam ciufmodi quidem difficuleates occurrent, dum fidera currét : quas
faperare, & nonimpeditumad conftitutionem fcientizcum minimo
damno afpirare, veaufusecft Copernicus,id viri fortis eft; ignaui fubrer-
fugere,


Chapter XVIII

which are both farther away from us, and subject to more variations in their motions.
But either if these thicknesses of the spheres were investigated with complete certain-
ty, or at least if the probable reasons why such large thicknesses were allocated by the
Creator to each of them were revealed, (7) then I pledge that I should produce from
the solids angles which would agree with the motions in all cases. For it is my opinion
that after the discovery of this proportion in the heavens (8) everything which still
prevents us from attaining exact knowledge of the motions is to be attributed to errors
in the eccentric ; and if those were removed, (9) I think that the five solids would
be of great assistance to the practitioners for the correction of the motions, which not
a few of them in various places are contemplating.

Another thing which impelled me to make that pledge to them on the eccentricities
is that (10) everywhere the controversy is about a minor part, which is less than the
complete thickness of the sphere. For take away their known thicknesses from all the
six spheres, or grant double the thickness to each of them: you will see that there is an
immense contraction of the universe and an increase in all the equations in the former
case, or swelling of the universe and diminution of the equations in the latter. Thus
the truth stands between nothing and double, and we need not fear that the practi-
tioner will have too much license to alter the eccentricities, if he should try to make
them fit these figures. And so this is a second line of argument which can excuse me
for the disagreement between my values and those of Copernicus.

Athird is afforded to me by the actual values of the Prutenic Tables, which are even
now rough and not so refined that one cannot sometimes pardonably depart from
them by even half a degree. Rheinhold indeed set everything out with great care in the
Prutenic Tables. But I should not like anyone to be enticed by this kind of pedantry
into disdaining rather rough values in astronomy. Let him consider the point in more
detail. The minute and pedantic precision of that great man is either appropriate on
account of the accuracy of the calculation, or inappropriate in the fractional parts:
but the actual whole numbers which he so pedantically divided he took from Coper-
nicus just as he found them,

Indeed the human failings of Copernicus himself in accepting any sort of figures
which suit him up to a point and help his case will be found out by the careful
reader of Copernicus. He does not repudiate values which though derived in dif-
ferent ways ought according to a theoretical proof to have agreed exactly, even
though they differ by a few minutes. He selects observations in Walter,‘ in
Ptolemy and elsewhere in such a way as to make all the more convenient use of
them in building up his calculation, so that he has no scruple sometimes in neglect-
ing or altering hours in time, quarters of degrees in angles, or more. In some
places, as in altering the eccentricity of Mars and Venus, he even accepts sines
which are at variance with the truth, simply because they point a little towards
those which he wants. Many things which ought on his own admission to have been
corrected he takes complete and as they are from Ptolemy, though he has altered
others like them; and on these he later builds the foundations of the new
astronomy. Of all this Maestlin has given me a great many examples, which I omit
from this account in the interests of brevity. And consequently he would rightly
seem to incur criticism, if he had not done it deliberately, because it was better to
have an astronomy with some imperfections than none at all. For difficulties of
that kind will occur while the stars run their courses. To overcome them, and
unhampered to aspire to the establishment of knowledge with the least possible
detriment, as Copernicus dared to do, is the part of a brave man. It is for a lazy


fayere,timidi defperare,& omnemhanccuramabijecre Quemadmo-
dumm & ipfe Copernicus hc modo recenfita e>druare dereneque dif.
fimulatsneque cum pudore fatetur. Exemplo Prolemai & vetorii fe mu-
nit,dtfiicultate obferuandi excufat ,atque vbique alijs exemplo prait,in

rxclarorum inuentorum confirmat.one minutulos hofce defeétus c6-
remnendisquod nifi fictumantea fuiffetsaunquam Prolemausillam p*-
paw avvlake, Copernicus t+ a/eAryedy libros, Rheinholdus Prutenicas
nobis edidiffet.

Neg; nullamexcufationem mihi quarto loco fuppeditat illa Maft-
linitabulain cap. XV. inferca. Copernico,cum eccentricitates Planeta-
rum a Prolemxe mutuaretur, nihilminus , quam de hac diuina caclo-
1um proportione fuboluit 5 venon iniuria vehementer quis miretur, i-
pfum tam propead cam accelfilfe; neque fore puravie » veneceffitas ali~
quando cogeret inquircre diftantias a Sole, & #94" loca.Qui idmirum
igiturfiin hacad viuum refectione, & «vet mundi muleadeprehen-
dantur rudia,cum artifexad minima nonrefpexcrit?_ Quafiin parua pi-
Qura, que vixintegram faciemad {enfum exprimic, fi quis oculi aut pu-
pill veram proportionem quarat,cum falli neceffeeft. Neglexitenim
hanc prétor ob exilitacem,contentus fi, guzfunt cuidentiora >quodam-
modoreprefentarct. Sic adhanc ¢véavewquamiuis optima ratione ac-
cellerim , cogente mevidemonftrationis, & conditionerci propofita:
nolimtamen, ve quis fibi perfuadeat,abfolure certidimos numeros {ein-
deretuliffe. Ficrinamque poteft, vthzcipfa refectio erroris vlterioris,
caufafuerit. Ecce non leuiaindicia. Cauffam, curmutencur Eccen-
tricitates Martis & Venceris, Copernicus in mutationem terrenz con-
fert. Non igitur muratur veracorum a Sole Eccentricitass Demonttra-
tionemadoculum habesin tabula. Quod fiita eft, oportebat Eccentri-
citates 4 terra, quz Prolemai feculo, & quz noftro fuerunt »codem de-
dacere atque cx vtrifque eandem 4 Sole Eccentricitatem concludere.
Atquicalculum contule, videbis hocnon ove parcrat,ficri. Difcrepan-
tes cnim inuicem prouenient etiam #040 Eccentricitates. Idem delo-
cis é2zA‘ev didtum efto, quiahacmutuo connexa funt: Atque hacy-
num eft. , '

Deindc facile colligitur exafpeétu tabulz, cum inzqualiter proce-
dant, & @?iae & datyes, magnam inde fucceffu feculorum extituram
caxereomirey diuerfiratem. Hodie Saturni & Telluris abfides prope
coniun@x fant,quareintcgra Telluris Eccentricitate minor eft Sacurni
acentroorbisterrefttis,quam 4 Sole,diftantia. Vbi quadrate diftiterint,
aqualis crit veraq & 4 © & 4 Terra, crefcetnempe Copernicofua Eccé-
tricitas Saturnia vique dum opponenturinuicem Sacurai &Tellurisab-
fides. Quemad cuentum etfi mundus non durabitscamen fi perfectacf-
{ct Aftronomia , tales debebathypothefes vfurpare, quz quafi zterno
mundo fafficerent. Atquinihil horum monet neq; Copcrnicus,neque
Rheinholdus. Né igicur perfestiffimi funt corum numeri,neq; integras
planceard (phcras nobis explicant,quibus illos feros motusaccidere pot

{cincelligamus. Hec & huufmodi fimiliacé menénihil conturbarent,
atqi ego hereréinops cofilij,quafi qui Ce rotulas es


Chapter XVII

man to shirk it, for a coward to give up hope and to reject all this trouble. Hence
even Copernicus himself neither tries to hide his own failings, which have just been
recounted, nor shows shame in admitting them. He arms himself with the example
of Ptolemy and the ancients; he excuses himself by the difficulty of observing; and
everywhere he sets a precedent for others of scorning these petty little short-
comings in the process of establishing splendid discoveries. If that had not been
done previously, Ptolemy would never have brought forth for us his A/magest,
Copernicus his books On the Revolutions, Rheinhold his Prutenic Tables.

Also a not inconsiderable excuse, in the fourth place, is provided for me by
Maestlin’s table inserted in Chapter 15. So it is for Copernicus, since he borrowed
the eccentricities of the planets from Ptolemy, even though he had an inkling of the
divine proportion of the heavens; so that there may rightly be great wonder at his
having approached so near to it; and he did not suppose that he would ever be com-
pelled by necessity to inquire into the distances from the Sun, and the positions of
the aphelia. Then why is it surprising, if in this pruning to the live wood, and
analysis of the universe, many awkwardnesses are detected, since the craftsman did
not consider details? Thus in a miniature, which scarcely enables us to see the whole
face, anyone who looks for the true proportion of eye or pupil must necessarily be
disappointed. For the painter neglected that on account of its minuteness, content
if somehow he portrayed the more obvious points. So although I have approached
this analysis by the finest logical process, governed by the force of the demonstra-
tion and the requirements of the premises, yet I should not want anyone to persuade
himself that absolutely certain values have been inferred by this means. For it can
happen that this very pruning of error is a cause of further error. Consider the
following weighty evidence. The cause of the changes in the eccentricities of Mars
and Venus is ascribed by Copernicus to the change in the Earth’s, Therefore there is
no change in their true eccentricity from the Sun. You have a visual demonstration
in the plate. If that is the case, their eccentricities from the Earth, as they were in
Ptolemy's time and in our own, ought to have led to the same conclusion, and the
same eccentricity from the Sun ought to have resulted from both. But refer to the
calculation: you will see that that does not come about, as it should have done. For
there will also turn out to be discrepancies in the eccentricities with respect to the
Sun. The same must be said of the positions of the aphelia, because these are
mutually interdependent; and this is one and the same thing.

Further, one easily gathers from a glance at the plate that when both the aphelia
and the apogees advance irregularly, with the lapse of time the effect of that will be
a great variation in the eccentricities. Today the apsides of Saturn and the Earth are
nearly in conjunction, so that the distance of Saturn from the center of the Earth’s
sphere is less than that from the Sun by the whole of the Earth’s eccentricity. When
they are a quarter of a circle apart, the distance from the Sun and the Earth will be
equal and indeed according to Copernicus Saturn’s own eccentricity will increase
until the apsides of Saturn and the Earth are in opposition to each other. Although
the universe will not endure until that comes about, yet if astronomy were correct, it
ought to adopt hypotheses which would be satisfactory if the universe were eternal.
But neither Copernicus nor Rheinhold tells us any of these things. Consequently
their values are not absolutely correct, and they do not set out for us the complete
spheres of the planets, to show us that those motions can occur in the end.

As these and similar points disturbed me considerably, and I was hesitating in
doubt of how to proceed, like a man who does not know how to put together

68 loan. Keprert

nem redigerencfcitsMzftlinus me confolatus,imo dehortatus eft ab his
fubtilitatibus: Non poffe nos,aiebat, omnes naturz thefauros exhauri-
re, nonmoucndum cffe malum bene conditum, &tolerandam potius,
atque fuftentandam leuaminibus quibufdam hanc velutirupturam hu-
mani corporis, quam vetam exquifita anatome conijciatur rger in prz-
fentiffimum vite periculum. Proferebatmihicxemplum Rhetici,cura-
que ciusad ynguem mez fimiliter curiofam, &increpantem pro fe Co-
pernicum. Epiftolaeft Rhetict Ephemeridi 15si-pratfixa, que quianon
paflim eft obuia,& totum hoc caput multis locis mirifice iuuat,precipua
inde procolophone huic capiti fubiungam.SicigiturRheticusad leéto-
rem inter cxtera.Suas antem(Copcrnicus)exguifitiones mediocres sno nimas
cffevoluit. Itag, confulto,non inertia aut radio defatigationis, eas comminutiones
‘vitanit,quas nonnulli etiam affectarunt, G- funt quiexigant , qualiseSt Purbachy
in Eclipfium tabulis fubtilitas. Videas autem quofdam i his omnemcuram pone-
re, ut plane ferupulofe loca fiderum ferutentur , qui dur fecundanis, G terttante,
quartants quintanis minutis iahiant jntegras interim partes praterennt , neque
refpiciuat Cy in momentis 750 Qasvoptvan {ape horis , non etiam nunquam diebus
totis aberrant. Hoc nimirumeit quod in fabulis FE fopicis fit ab co, qui inffus bo~
uem amiffamreducere,dum anicults quibufdam captandis fiudet , meque his patitur,
ft boue etiaan ipfo prinatur. Recordor curs & ipfeiuuenil, curiofitate impellebar,c
quaftin penetralia fideruin peruenire cupiebams. Itaque de hac exquifitione interdit
étiam rixabar cums ptine & maximo viro Copernico. Sed alleycumn quidem animi
mei honefta cupiditate delecharetur , molli brachio obiurgare me, gy hortari folebat,
ut manum ctiam de tabula tollere difcerem + Ego,inquit, fiad fextantes, qua funt
Serupula decerm,veritates adducere potero,non minus exultabe animis , quamra-
tione norm... ceperta Pythagoram accepimits. Mirante me,¢y annitendum effead
certiovadicente: huc quidem cum dsfficultate etiam peruentum iri demonftrabat,
cum aliisstum tribses potiffimum de caufis. Harum primam effe aiebat, quod anim-
aducrteret , plerasque obferuationes vcterum fynceras non effe, fed accommodatas
adeam dottrinam motuit,quam fibiipfi nusquifque peculiariter confiituiffet. Itag,
opus cffeattentione Cr induftriafingulsri,vt quibus aut nibil,aut parumadmodum
opinio obfernatoris addidiffet,detraxiffetve,es & corruptis fecernerentur. Secitdam
canfarneffedicebat , fiderum inerrantinm locaaveteribus non vlterius, quamad
Sextantes partinns exquifita:Et fecundura hac tamen pracipue errantium pofitusca-
pioporterespanca exctpicbat in quibus declinatiofideris ab equinoctial annetataré
adinnaret,quod de hac locus i oilers certius conftitui iam poffet. Tertiam caufams
hancmemorabat : Nomhaberenos talesauttores , quales Ptolemaus habuiffet poit
Babylonios & Chaldeos,illalumina artis, Hipparchum,Timocharem, Menelaum,
& cateros, quorum G nos obferuationibus ,ac praceptis nitt acconfidere poffimus.
Se guider mallein its acquiefiere,quorumveritatem profiteri poffet,quam in ambi-
guorurs dubia fubtalivateoftentareingenyjacrimoniam. Haud quiderslongius certe,
~veletiams propius omninoabfuturas fuas indicationes fextite,aut quadrante partis
uniusa veroscuius defedlus,tantur abelfe vt fe panitest,vt magnopere letetur,buc
Uff, longo tempore,ingenti labore,maximacontentione, fiudio »induftria fingu-
Lart,procedere potuiffé. “Mercuriuma quidem, quafi fecundum prouerbinm Gracorit,
relinquebat in medio communcm s quod deillo neque fuo fludio obfernatum effe di-
ceret,ned, ab alits fe accepiffé,quomagnopere adtuuari,aut quod onanino probare pof-
Set. Me quidem multa monens , Jubsjciens, Pracipiens, in primis hortabatur, vt
Srellarum

Chapter XVIII

again the dismantled wheels of a machine, Maestlin consoled me, or rather
dissuaded me from such minute precision. “We cannot,” he said, “exhaust all the
treasuries of Nature; the deeply seated flaw cannot be removed; and we must
rather tolerate, and endure by some palliatives, this, so to speak, injury to the
human body, than throw the sick man by so radical an operation into immediate
danger of his life.” He pointed out to me the example of Rheticus, and his atten-
tion to every last detail, as laborious as mine, and in itself a criticism of Coper-
nicus. There is a letter of Rheticus prefixed to his Ephemeris for the year 1551;
and as it is not to be found everywhere, and gives wonderful support to the whole
of this chapter in many places, I shall append its main points as a tailpiece to this
chapter.* This, then, among other things, is what Rheticus says to the reader.

“However he (Copernicus) wanted his investigation to be moderately thorough but not
excessive. So it was on purpose, and not from laziness or distaste for sustained effort, that
he avoided the minuteness of detail which several have even striven for, and some demand,
such as Peurbach’s precision in his tables of eclipses. However you can see certain people
giving all their attention to examining the positions of the stars with absolute pedantry, and
while they peer at seconds, and third, fourth, or fifth minutes,* meanwhile overlooking
complete degrees, and not considering them; and in the precise times of phenomena they
are often out by hours, and sometimes even by days. This is exactly what the man in
Aesop’s fables did when he was told to bring back a lost ox, but in his eagerness to catch
some little birds he both failed to take them and missed the ox itself. I remember when I
myself was also driven by a youthful inquisitiveness and desire to penetrate, as it were, the
inner fastnesses of the stars. Consequently on the score of precision I sometimes even quar-
relled with that great and good man Copernicus. But he, since indeed he was delighted by
my mind’s creditable desire, used to upbraid me with a gentle hand, and encourage me to
learn to lift up my eyes from the page. ‘If,’ said he, ‘I can get as close to the truth as sixths of
a degree, which are ten minutes, I shall be no less glad at heart than we have been told
Pythagoras was when he discovered the right-angle theorem.’ I was surprised, and said that
we should strive for greater accuracy; but he showed that even that point would only be
reached with difficulty, for three reasons in particular, among others. Of these he said that
the first was that he noticed many observations of the ancients were not genuine but were
adjusted to fit the particular theory of the motions which each had decided for himself.

Particular care and attention were therefore needed to separate those in which the opinion
of the observer had added or subtracted nothing, or very little, from the corrupt ones. He
said that the second reason was that the positions of the fixed stars had not been in-
vestigated by the ancients with greater accuracy than sixths of degrees, though it was chiefly
by reference to them that the positions of the planets had to be determined. He made an ex-
ception for a few cases in which noting the declination of the star from the equator had
helped matters, because the actual position of the star could now be determined more ac-
curately from it. The third reason which he told me was the following: we do not have
authorities such as Ptolemy had after the Babylonians and Chaldeans, those ornaments of
the science, Hipparchus, Timochares, Menelaus, and the rest, on whose observations and
injunctions we also could depend and rely. He himself preferred silent acceptance of points
for the truth of which he could vouch to a display of the quickness of his wits over ques-
tionable precision in uncertainties. Certainly his own indications would be no more, or even
less, than a sixth or a quarter of a degree from the truth; and he was so far from regretting
that deficiency that he was extremely pleased to have been able to make so much progress
after a long time, vast toil, a very great struggle, assiduity, and particular application. Mer-

cury indeed, as if according to the Greek proverb, he left open to all comers, as he said that

no observation had been made of it either by his own assiduity, nor had any been obtained

from others, which was of great assistance, or which he could completely corroborate. In-
deed in giving me many admonitions, injunctions, and instructions, first and foremost he


fellarum incrrantinm obfernationi operam darem , illarum potifrmunt que fn fi-
guifero apparent quod cum bis errantium congreffius notari poffent, Ge. Hagtenus
ex epiftola Rhetici ea,que ad rem fuére. Quid tu iam, amice Leétot,de
Copernico fentis? Sidchocnegotio fuiffetmonitus, atquedeprehen-
diffet,quam propeabfit ab co cum fuis rationibus, quid putas non tenta-
curus furffet , quem laborem non fum pfiffet, ve corpora cum fuis orbibus
conciliarct? Atque hoc fi daretur, qui confenfus,qua perfeétio non fpe-
randaeffee. Quain re quid alij ,quidipfe Maftlinusaliquando, fauente
Deo,praftiturus fit, com pus docebit. Intereanolim , quis temere contra
me pronunciets& xquo animo hanc litis dilationem ferat.

Notz Auctoris.

(1) ATH nonrantum incertum eft, vtrorum vitio.] Fifi verim elf, Prutenicas pecca-

re,cum alias,turnettsn in Proftbaphercfibus Orbis annuisporipirmatamen caufa, no bains
tamtum rei, qicod interuall Orbrun nonexaéte quadrant ad proportionesquingue corportim Geo
anetricessfed etiam abies naioris reigquod filicet Planctariorbes babent tantas finguli, tainque dif-
forentes Eccentricitates, vtritfyae nrquatm rei caufeedt m archetypo exornationis motuum, fecum=
dum rationes Haemonicsserb aun non poffent exch: proportions figurales flare iuxta proportiones
Harmonicas, is,vt magis ad rationes materte decinantibus, derogariparum aliquid yt
proyortioncs Iisraonice insta laciun haberent lle quidem in{paciis muundi yifle vero inter motus
per fpacia. Vide banc Oraatim oraatifunnm,lib.V Marmon. cap. 1X. Prop.a XLV Lin XLIXad
longum,

(2) Inquibufdamad z.incegrorum gr. Imoin Marte tresin Venere quinquegradus
in tranfucrfian , in Mercurio 10. vel x.gradus ( fietiam dei locssvbi Planeta bic viders nequit, ex
bypothefi Thzoite Mereserg ame conjiscuca licee aliquid affirmare)certis Orbiuin locis,in errore funt,
apud Pratenicas.

(3) Cumautemillud fierinon potuerit.] Centra planorumn figura circumferipte,
anguli figurainficipte, non porvarsnnteffeconinnctiin boc archetypo mundi, Caufa ditt edt fupe-
riortbus. Nisnisims eniin confiderent Orbes:fierent maiores Proftbapbare{es Orbis magni apud fingu-
os quantos non obferuannus, Ergo uit refpiciendum ad diflantias planetarum a Sole non mediocres,
fed apbeliann duorum interioris  cpevibeliam exteriors; idcft, ad Eccentricitates planetarium, qua
anti aphlisnn C> Peribeliamt formant. Atqui ficad incerta vefpiciebaru-nondum enim erat co
gaita Eccentricitatian castfa,cur tanta offer penes fingulos Planetas Eccentricitas; cxr tanta differen
tha,ctty Satiernus, lupiter, meitiocres haberent, Mars, Mercurius maxinnss, Tellus, Venus, minimas,
Ighoratacauf,quantitatennignorariecefferat a prions remittebar ad nudas obferuationes,

(4) Etpaucioribus quam cater motibus.] Ita quidem tenet Prolemeus, Cex ile
Cepernicis.Sol cnint (feu Terra ) non tantum Epicyclo caret , fed etiam Aquante, ve illiputabant.
At fecunrduum rciveritatem , in inotu illo tranflationis circa Solem fimilis 6 Terra vnicuique reli-
quoruin Plaictarnin in onsnibus; vt demonftratuin e&t a me in Comment. Martis,parte tertia: & Ex
pit. Afir.ib.7.

(5) Hicigicar Orbis,quem certiffime.] Hic Orbis Prolemao Solis , Copernico Terra
Trutenicis Annuns di®us,

(6) Quod per minima quadam, } Hee Copernicé qierela potifinnun attingit loca
Apogcoruint jus loca nihil attinent boc negotiuns proportions Orbiuamn ) noneadern eff de Eccentri-
catatibits Teague non putts, fed elins babent ipfe Orb fpifitudines.

(7) Tum ego fpondeo me produdturum.] Audaciam ecce fponfionissfuffuleam dif-
fultate conditionis hic propofite side tasinen Cr feclicstatemsexploratafunt a mequantitates Eccen-
tricttatunaex Ob feruationibus Brahe parefaite in Harmonicis ca [a Eccentricitatum fingularum:
G ecceproductosarmquidemese (lis. figurts,fed potysimmum ex can fisEccentricitatum(Harmoniie)
nncusper onttaaatibusconfonas.

1 8)Id

Chapter XVIII

exhorted me to attend to the observation of the fixed stars, especially those which are to be
seen in the zodiac, since the conjunctions of those with the planets could be noted, etc.”

This is the end of the part of Rheticus’s letter which was relevant. What is your
opinion about Copernicus now, dear reader? If he had been told about this
undertaking, and had understood how close it is to his own thinking, what do you
suppose he would not have attempted, what toil would he not have undertaken,
to reconcile the solids with his spheres? And if that were achieved, what agree-
ment, what accuracy could not be hoped for? In this matter what others, what
Maestlin himelf some day, with God’s favor, will produce for us, time will show.
Meanwhile I should not wish anyone to pronounce against me hastily, and this
postponement of judgment should be accepted without dismay.


(I) Yet it is not only uncertain which of the two is responsible. Though it is true that there are
mistakes in the Prutenic Tables in various places including the equations of the annual orbit, yet the
chief cause not only of the fact that the intervals between the orbits do not exactly square with the
‘geometrical proportions of the five solids, but also of a more important fact, which is that the individual
planetary orbits have such large and such different eccentricities, the cause I say of both facts is in the
archetype of the display of the motions according to the harmonic ratios. There, since the exact propor-
tions of the figures could not stand alongside the harmonic proportions, it was necessary for the former,
as leaning more towards the arguments from the material side, to be moderated somewhat, so that the
harmonic proportions might find a place beside them, the former indeed in the spaces of the universe, the
latter however among the motions through the spaces. See this display most elegantly displayed in Book
V of the Harmonice, Chapter 9, Propositions 46 to 49 throughout

(2) In certain cases to as much as two complete degrees.| Rather in the case of Mars three, in the case
of Venus five degrees in longitude, in the case of Mercury 10 or 11 degrees (if I may make a statement,
from the hypothesis established by me for the theory of Mercury, even about those positions in which the
planet cannot be seen here) is the amount of the error in the Prutenic Tables at certain positions on the
orbits.

(3) However since that has not been possible.) The centers of the faces of the circumscribed figure and
the vertices of the inscribed figure could not have been linked in this archetype of the universe. The
reason has been stated in what precedes. For too much consideration would be given to the orbits: the
equations for the Great Orbit would become too great in particular instances, of a size which we cannot
observe. It was therefore necessary to have regard not to the mean distances of the planets from the Sun,
but to the distance at aphelion of the inner of the two, and at perihelion of the outer: that is, to the eccen-
tricities of the planets, which regulate the distances, at aphelion and at perihelion. But I was consequently
having regard to what was uncertain; for it was not yet known what the cause of the eccentricities was;
why the eccentricity was so great in the case of particular planets; why the difference was so great; why
Saturn and Jupiter had intermediate eccentricities, Mars and Mercury the greatest, and the Earth and
Venus the smallest. As the cause was unknown, it was inevitable that I did not know the amount a priori,
and I was driven back to the bare observations.

(4) With fewer motions than the other.] This indeed is what Ptolemy holds, and following him Coper-
nicus. For the Sun (or the Earth) not only has no epicycle, but also no equant, as they thought. But ac-
cording to the truth of the matter, in its motion of translation round the Sun the Earth is similar in all
respects to each of the remaining planets, as has been shown by me in my Commentaries on Mars, Part 3,
and my Epitome of Astronomy, Book VII.

(3) Then, in the case of this sphere,...an absolutely certain.| This is the orbit called in Ptolemy the
Sun’s, in Copernicus the Earth’s, and in the Prutenic Tables annual.

(6) That “....from very small.” This complaint of Copernicus chiefly applies to the positions of the
apogees (which do not affect this business of the proportion of the spheres at all). It does not apply in the
same way to the eccentricities. Therefore the thicknesses of the spheres are not in a worse state, but
better.

(7) Then I pledge that I should produce.) You can see the audacity of this pledge, beset with the dif-
ficulty of the condition here proposed. However notice also how fortunate it was, The amounts of the ec-
centricities have been investigated by me from the observations of Brahe; the causes of the eccentricities
have been made clear in the Harmonice; and you can see that arcs which agree with the motions in all
respects have been inferred, not indeed from the five figures alone, but chiefly from the causes of the ec-
centricities (the harmonies.)

70 loan. Keprert

(8) IdomneinEtcentricitatum vitia, ] Landabisopinor etiam puerulum trim,
prafumentem anime pugnam cungigantibus.Non enim onsnes Aftronomia neut,imo minima illerit
‘ars funt ex vitiofis Eccentricitatibusfingulorum.De Solis vel Terre Eccenricitate pot dicetur.

(9) Magnoadiumento futura folida hxc quinquead correctionem motui.]
‘Nullocquidem,ne minimo quidem ; quia non formant Orbes, necpreferibunt metas Eccentriciratit
‘Sed vbiprinsinuente fuerint Eccentricitates, vt wim, ex Obferuationibus Brshes:tam denique le-
cum babe inguifitiocanfarum , feu rd dch ex hisquingue figuris, &iunétis proportionibus Har-
smonicis,

(10) Vbique de minori particula,quam eft miz2s Orbis,contcouertitur.] Cum
evimHarnioniarum fitaliqua copia , electefuerunt pro fingulis bigis Planctarumt vicinoram, que
quantitate quamproxime refponderent proportionibus barum quingue figuraruan.

Defingulorum in fpeite Planetarum refidua difcordia.

CBSE AS. £c igiturin genere fuere , qu caufam meam releuare
£5) (SB pofllunt. Nuncin fpecievideamus, ecquid excufariam.
a ES] plius poffit. Initium a Saturno fumamus. Atque cius qui-
£2 dem émoipah magna facta eftacccllios{ed qug tamen dif-
Czewt ferentiam profthapharefeos caufata eft non maiorem
SF 41{crupulis. Nam ficucingens cus diftancia facilimam
errori caufam prxbetin obferuatione; ficerror in diftantia quamuislu-
culentus exiguam & opinionc minorem efficitin 7729 «9 «pied diuerfita-
tem. Ertamen nequchuius fideris motus certiffime diméfos effe Aftro-
nomos,vel fola praterita hyeme cernereerat.Nam die +, Noucmb.an-
no1594-Saturnus vifus eft exaéte inter Ceruiccm & cor Leonis, vbi cfle
debebatfecundumealculum die}; Oaob.praterita. Differentialong.
37-{crup.plus minus. Quod fi hanc quantitatem non excedat eius a Co-
pernico difcordia spoSadapicws , correéta modo diftantias exiftiment
Aftronomi fibi abunde fatisfaétum.,

In Iouc nihil iure defiderari poteft. Nam exiguam habet differen-
tiam;atque minorem fextante gradus.

Quod autem ctiam in Marte femifsis gradusabundat,nihil mirum,
necme moucr; mouctid potius,maiorem non effediuerfitatem. Tefta~
tur enim in prafatione Ephemeridis ad annum1577. Mattlinusfide-
rishuius crrores a calculo intra duorum graduum anguftias cogi non
poffe.

Tamad inferiores $ & ¥ quod attinet,, erfi prz fuperioribus nonni-
hilcommoditatis habere videnturspropterca,quod cx elongatione ma-
xima faciliuseft, quam ex #xpovx42 obferuatione, ipforum orbes dime-
tiri,ipfatamen obferuandi via mihi fafpedta cft. Quamuis reétius A ftro-
nomis hoc aftimandum relinquo ; nempe vtrum non in his planetis
(1) vaporum denfirate & phyfica parallaxi, quam nec Solnec Lunaef-
fugit,incerdum fallantur. Certe Meftlinus in Difputatione de Eclipfi-
bus , thefis8.de Vencreaffirmat , quod nonrarovifa fucrit cius 4 Sole
prope horizontem diftantia notabiliter minor yera. Quanto magisid
de Mer-


Chapter XIX

(8) Everything. . 10 errors in the eccentricities. You would praise, I think, even a little boy of three
years old who had the spirit to take on a battle with giants. For not all the blemishes of astronomy, in-
deed only the smallest part of them, are due to erroneous eccentricities of particular planets. The eccen-
tricity of the Sun or the Earth will be spoken of later.

(9) The five solids would be of great assistance for the correction of the motions.] Of no assistance, in
fact, not even the smallest, becduse they do not regulate the spheres, nor prescribe the limits of the eccen-
tricities. But now that the eccentricities have already been found, as knowledge “that,” from the observa-
tions of Brahe, at last there is room for a search for causes, or knowledge “why,” from these five figures
and the linked harmonic proportions.

(10) Everywhere the controversy is about a minor part which is less than the thickness of the
sphere.) For since there is an abundance of harmonies, for the individual couples of neighboring planets
have been chosen those which would correspond as nearly as possible quantitatively with the proportions
of these five figures.


Those, then, were the general arguments which may save my case. Now let us see
what further defense can be made in individual cases. Let us start with Saturn.
Now a great increase has been made in its distance; but this has been made the
reason for a difference in the equation not greater than 41 minutes. For just as its
vast distance provides a very easy cause of error in observation, so an error in the
distance even if it is considerable produces a tiny, and less than expected variation
in the equation. Yet the fact that astronomers have not very accurately measured
the motions of this star was easily to be perceived even in the passage of a single
winter. For on the 2nd/12th November in the year 1594 Saturn appeared exactly
between the neck and the heart of Leo, where it should have been according to
calculation on the 21st/31st of the previous October. The difference in longitude
is 37 minutes more or less. But if that amount were not exceeded by the discrepan-
cy between Copernicus and its equation, the correction now having been made in
the distance, the astronomers would think they had given thorough satisfaction.

In the case of Jupiter nothing can rightly be desired, for it has a tiny difference,
less than a sixth of a degree.

That there is also half a degree too much in the case of Mars, however, is not at
all surprising, and does not influence me. I am influenced rather by the fact that
the variation is not greater. For Maestlin bears witness in the preface to his
Ephemeris for the year 1577 that the irregularities of this star cannot be confined
within the limits of two degrees."

Now as far as the inferior planets Mercury and Venus are concerned, although
compared with the superior planets they seem to have considerable convenience,
because it is easier to measure their orbits from the maximum elongation than
from an observation at opposition, yet the actual way of observing seems to me
suspect. However I leave one point to the astronomers to evaluate: that is,
whether in the case of these planets they are not sometimes led astray (1) by the
density of the atmosphere and the physical parallax, which are not escaped by the
Sun or the Moon either. Certainly Maestlin in his Disputation on Eclipses, in
thesis 58, asserts of Venus that its distance from the Sun near the horizon has not
infrequently seemed to be noticeably less than the truth.? That can be said all the

Mystertvm CosmoGraraicve ”

de Mercurio dici poterit, qui ferc femper fub (olis radijs chs S quamuis
interdum emergat: nunquam tamen , nifi prope horizoncem per inter-
iectam exhalationum copiam noftro fe vifui prefentat. Ex quamuis
Veneri opitulentur fixe , fimul & prope apparentes : Mercurius ta-
men frequentiusinculpamance , quiipfe rarocernitur , & rarius fixae
prope ipfum. Cumgque hxc hodie accidant ; credibile eft & vereri-
bus quantifcunque Artificibus accidere potuiffe. Nam quod Leéto-
rem de co non monent , idipfum fufpicionem de horum Planerarum
dimenfionibus vitiofis auget. Hoc enim indicio eft ; nec animaduer-
fam ipfis nee correétum elle, fi quid cx co vitij extitic. Quareinleétio-
ne veterumn imprimis {peétandum cfle puto,vtrum fingularum obferua-
tionum, quaallegantur, inftrumenta & modi huic crrori obnoxij effe
potucrint.

Deinde noniniuria metuo , vemultaadhucin ratione hypothe-
fium his duobus Planctisrelidtaincertafint. Copernicus (vt colligitur
cx modo pofita Rhetici, & infra ex Meftlini epiftola ) plus Peolemai
placita quam obferuationum neceffitatem fequutus eftincmendandis
theorijs. Quain requominus reprehendipoficr, Rheticus in fua nar-
ratione effecits vbi monct , religiofilime veterum vettigijs inharen-
dum,nec facile quid mutandum  donecobferuationum extremanecef-
fitasvrgeat. Quod igituradco exquifitz obferuationes haberi non pof-
fent, ea fortafle fatis magna caufa fuir Artifici prudentiffimo , prater
accommodationem ad re placica nihil viceriusin Planctas hofceten-
tandis

Quod igitur in Venere magnam videsarcuum diuerfitatem eius

rei culpaminter cetera, quzin genere pramifi (que te probe meminif-
fevclim ) etiam in hac modoallegata offendicula confer ; & magnitudi-
nem difcordi equanimitate tua, fi bene fingula perpendifti, facile
faperabis. Qua in re magno tibi folatio erit , quod numerus Coper-
nicanusmediuseft inter arcus exinterpofica, & ex omiffa Luna prod-
cuntes. Nam fi orbem magnum fyftemate Lung farcias : Icofaedron
Venerem longius A tcrradimouct, atque Copernicus prodidit; fin ex-
cmpta Luna tenuiorem efficias orbem magnum : figura Vencrem ni.
mium propeadmittit , maioremque , quam eftin Copernico,effe pati-
tur. Quarealiquid minus Lunarem iuuare poterit,fi tenendusCoper-
nicus eft.

De Mercurio verotantumiam dium eft, dicique amplius po-
teft, ve exiftimemte,Lectorxque, Wee tae etiam deeffer, con-
cofurum,atqueexcufaturum. (2) Nequemihidignavideturciusmo-
tus diuerfitas,de qua magnam litem moucam. Quamuis meliusfegerit,
quam Venus;facit enim vniustantum gradus differentiam,quod mitum
eftsadco nunquam non fallacieft ingenio.Certe vnus hiccft, qui Aftro-
logorum famam maxime proftituic, & meteororum rationem omnem
turbar.

G3) Et in ventis quidem pradicendis ( quos certiffime concitat,
quoticfcunque locis eftidoncis) fepe adeo conftanti numero dierum.
aberrats ve parum abfir, quin cum cius in Ephemeride vitiofe prodicum.
circulum corrigere poffim Iraque fiquem Aftronomum cernerem ni-

mium


Chapter XIX

more of Mercury, which is almost always close to the rays of the Sun; and
although it sometimes emerges, yet it never presents itself to our sight except near
the horizon with a quantity of vapors interposed. And although Venus is suc-
cored by the fixed stars, which appear at the same time and close to it, yet Mer-
cury more frequently remains at fault, as it is rarely to be seen itself, and the fixed
stars are more rarely to be seen near it. And since that happens today, we may
believe that it could have happened to the ancient practitioners, however great.
For the very fact that they do net comment on it increases the suspicion that the
measurements of these pianets may be faulty. For a sign of that is their failure to
mention or correct any fault which resulted from it. Consequently in reading the
ancients I think the first thing to look at is whether in particular observations the
instruments which are mentioned and their methods could have been liable to this
error.

Secondly it is not unfair of me to fear that in the case of these two planets much
has still been left uncertain in the reasoning of the hypotheses. Copernicus (as
may be gathered from the letter of Rheticus just quoted, and that of Maestlin
below) followed the beliefs of Ptolemy more than the requirements of the obser-
vations in correcting the theory of the inferior planets. On this point Rheticus
managed to defend him from criticism in his Narratio, where he remarks that we
should adhere scrupulously to the path marked out by the ancients, and alter
nothing lightly, until driven to it by the unavoidable requirements of the observa-
tions. Thus the fact that such refined observations were impossible was perhaps a
great enough reason for that most careful practitioner to attempt nothing more
‘on these planets beyond fitting them to his beliefs.

Thus for the great variation of angles which you see in the case of Venus, at-
tribute the blame, among the other things which I have already stated (which I
should like you to remember thoroughly) to those minor stumbling blocks just
mentioned; and you will easily rise above a discrepancy of that size without
disturbing your calm of mind, if you have considered the individual cases well.?
In this connection it will be a great consolation to you that the Copernican value
is half way between the angles resulting from the interposition of the Moon and
the omission of the Moon. For if you stuff the Great Orbit with the system of the
Moon, the icosahedron moves Venus further from the Earth than Copernicus
reported; but if you leave out the Moon and make the Great Orbit thinner, the
figure lets Venus too near, and allows the orbit to be greater than it is in Coper-
nicus. Consequently something smaller than the Moon will help matters, if
Copernicus is to be retained.‘

About Mercury indeed so much has already been said—and more can be
said—that I think, friendly reader, if anything further were still missing, you
would put up with it, and excuse it. (2) Nor do I think that the variation in its mo-
tion is worth stirring up a great dispute about. However, it conducts itself better
than Venus, for it produces a difference of only one degree, which is remarkable,
as its nature is never unambiguous. Certainly this is the one planet which most of
all disgraces the reputation of the astrologers, and confounds the whole theory of
things on high.

(3) And indeed in predicting winds (which it certainly stirs up, whenever it isin
suitable positions) it is often off its course by such a constant number of days that
at such times I can very nearly correct its circle, which is wrongly published in the
Ephemeris. Thus if I were to see any astronomer devoting himself too intensely to

Hi nu-
meri
fantin
Tab .V.
que cft
eap.ts.
ad ¥.

yh loan. Keprent

mium folicite rimandis planctahuius crroribus incumbere, illum ego
monerem, vetempusilludreétius collocaret, & Tellurem ,atque hare
ambientem Lunam, ¢#)7#2» fidus, quarum illam pedibus , hanc ocu-
lis proximeatcingimus, hac, inquam fidera potius {peculetur , que-
qucin corum monbus inque Eclipfibusadhuc peccamus, limee s tum
demum opera ad Mercurium transferat. Interea fivenia digni fant
errores circa Telluris & Luna motus , multomagis id merebuntur ez-
roresin Mercurio , qui &remorioranobis eft, & fere femper fub So!e
latet.

Atquchicrurfum ve priore capite, coronidisloco epiftolx partem
afcribam , quam Meftlinus ad me mifitsidque duabus de caufis, prima,
quia dere ncceflariace monet; altera, quiacaputhoc pailim confirmat.
Sicille:

Tam mirabilis e5t Aercurius, vt parum abfuerit , quin etiam me fifet-

liffet. Nec miruan , quia etiam Copernico & Kheinbolde admodum molefium
fusffe, antmaduerto. Copernicus hoc de ferpfo fatetur , Multis ( inquit tb. 5.
¢ap.30. ) ambagibus & labor:bus nos torfit hoc fidus , vt cius motus
ferutarcmur. Vade praterquam quod nullas fuas proprias recitat obfernatio-
nes in § habitas, feda Bernbardo VValtero Noribergico mutuatur : etiam ina-
pogeiipfivslocoftaruendo , fibinonconflat. Namquers (cap.r6.)in primis An-
tonini annis ,circaannum CHRISTI 140.4uxta Ptolematobferuationcs, inues
nit in xo.grad. ts O* fub fielaro orbeia 183.¢rad.r0.ferup.d primaflcllaV seun~
dem s83.grad.20.ferup..cap.29.)repoutt ad z1.annum Polen Philadelpht per
indeacfi hoc apogeun intra 400. annos intermedios fab phara fisarum illa-
rumimmotiom quicuiffets cum tamen ( cap.30.4a fine) 63.annis per unui gradum
motum fuilfe pfivideatur ; addit autem: fimodo aqualts fuerit. Rheinheldumin
aifdem difficultaribus hafiffe, calculus Pratenicarum sabularum prodit , guoargui-
ir, Bheinboldums locum apogaei huts ad tempus illad Philadelphiaffurspfiffe eur
dem quidem cum Copernico,vid.r83.g7.20.ferup.aprima flella V. At ad Ptolemai
tempuesilladin locum longealienuan & manzfeftis Ptolemes obfernationsbus &Co-
pernici refamptiontbus,cadit. Ibi enim locus exus computatur non 183..0.n¢C 10.97.
of fed 188.97.50.fer fib or be flellato,& 15. gr.30.fer.2--Ideog, numeriillimet ad
Ptolemsiqudem feculitaccommod.ti fant, non autem vt cateri, ‘per omniscalculs
Tabslarum Prutenicarum, fed Ptolemei obfernationibus conueninnt , eas enim
Copernicss quoque Cr retinuit, Cy feguutis eit, atquecofdem inde numerosprodu-
xit. Udd noftram autem , fine Copernici atatem numeros hofce computare non
volta , propterca quod ¥ longe aly fierent , propter Ecccntricitatem orbis magni
diminutam ; & quod apud Copernicum nullis recentioribus obfiruationibus in-
ucfligati Cy comprobatifant. Optarem autem (quemadmodum me coram dixiffe
meminiffe potes) Copernic dimenfionum harum fundamenta non antiquas fed
nouas obfernationcs affumpfiffe. Grande enim Cy immane poftulatum illudeHt(ib.s.
cap.30.f0l.169.b.lin.7.4 fine) cum,concedendurm inguit,putamss, commenfiratio-
nes circulorum manfiffe a Ptolemao etiamnunt. (4) Namipfaterrena Eccentri-
cits deminuta altos numeros poftulat. Necenim verum eit quod Rheticusinnare
ratione dicit,quod in Mercurio nulla quoque, ficut in Ioue, fentiatur Eccentricite
tis nntatio, nam non fimiliter fal Apoged lates fuo Apogeo chaudit. Hucaccedit,
quod Ptolemaice obferuationes fatis craffe Cr partiles funt.quas omnino pracifiori-
bus corrigere oportebat.Sed de his iam frufira conqucri licet.In tno autem propofite,
fin


Chapter XIX

scrutiny of the deviations of this planet, I should advise him to dispose of his time
more fittingly, and that it should rather be the Earth, and the Moon which circles
it, the clearest star—the former of which we touch most nearly with our feet, the
latter with our eyes—these stars, | say, which he should examine, and that he
should refine the mistakes which we still make in their motions and in the eclipses.
Only after that should he transfer his effort to Mercury. Meanwhile if the errors
in connection with the motions of the Earth and the Moon are pardonable, the er-
rors in the case of Mercury deserve pardon all the more, as it is both more distant
from us, and almost always hidden in the rays of the Sun.

‘And here again as in the previous chapter I shall append as a tailpiece part of a
letter which Maestlin sent me, and for two reasons: the first that it communicates
an important point to you, the second that it confirms this chapter throughout. It
runs as follows.*

“So remarkable is Mercury that it very nearly defeated me as well, and not surprisingly,
as I notice that it was also very troublesome even to Copernicus and Rheinhold. Copernicus
admits this about himself: ‘This star tormented me’ (he says in Book V, Chapter 30) ‘with
many twistings and toilings, in trying to explore its motions.’ Thus apart from the fact that
he reports no observations of his own made on Mercury, but borrows from Bernhard
Walter of Nuremberg, he also contradicts himself in determining the position of its apogee.
For although (Chapter 26) in the first years of Antoninus, about A.D. 140, according to
Ptolemy's observations, he found it was in 10° of Libra, and on the celestial sphere in
183°20 from the first star of Aries, he places it (Chapter 29) in 183°20’ at the 2Ist year of
Ptolemy Philadelphus, just as if the apogee of Mercury had remained motionless against
the sphere of the fixed stars in the intervening 400 years —yet he thinks (Chapter 30, at the
end) that its motion had been one degree in 63 years, adding however ‘assuming it was
regular.’ That Rheinhold was caught up in the same difficulties is shown by the reckoning
of the Prussian (Prutenic) Tables, from which it appears that Rheinhold assumed the posi-
tion of the apogee at the time of Philadelphus to be the same as did Copernicus, that is
183°20' from the first star of Aries; but at the time of Ptolemy it falls into a completely dif-
ferent position from the well-known observations of Ptolemy and Copernicus’s restate-
ments of them. For there the position is computed not as 183°20' and 10° of Libra, but as
188°50’ (i.e., on the celestial sphere), and 15°30 of Libra. Consequently my values have
been adjusted for the time of Ptolemy, but not, as have the rest throughout, by the reckon-
ing of the Prutenic Tables, but in conformity with the observations of Ptolemy, for Coper-
nicus also retained and followed them.* However, I did not want to compute these values
for our own time, or that of Copernicus, because they would be far different, on account of
the decreased eccentricity of the Great Orbit, and because in Copernicus they have not been
checked and verified by any more recent observations. However, I should prefer (as I can
remember my saying in your presence) Copernicus to have taken as the bases of these
measurements not ancient but fresh observations. For it is a huge and mighty assumption
(Book V, Chapter 30, page 169b, 7 lines from the end) when he says, ‘I think it must be ac-
cepted that the dimensions of the circles have remained the same from Ptolemy until now,’
(4) as the very decrease of the Earth’s eccentricity demands different values. For Rheticus’s
statement in his Narratio? that in the case of Mercury as well, just as in the case of Jupiter,
no alteration is perceptible in the eccentricity, is not true because the relationship of the
Sun’s apogee to Mercury’s apogee does not mask the alteration in the same way. In addi-
tion, the Ptolemaic observations are rather crude and isolated, and should be entirely cor-
rected by more accurate ones. But on this subject at present we may complain in vain.*
However, in the case of your scheme, if these values agree to any extent at all, you should
consider that you have performed your task outstandingly, and warmly


These fig-
ures are in
Plate V,
Ch. 15,
under
Mercury.
MystertvM CosMoOGRat dic? 3

[inumerihivteunquetibirefpondeant,te putes officio tnocgregre funitum, tibique
guemadmodum Copernicits apnd Rheticum in epiftola , vebcmenter gratuleris (5)
certifiima fpefrets,propediem fore, vt occafione horit, que ate ingentofiffime fant
inuenta catera quoque, que iam adbuc dubisfunt co Aftronomurum catumnon
parnzn torquent planiffima fint future,

Notz Au@ons.

(VY Apotum denficare& phyficaporallasi.] Rectianeflelarum spelt Tycho Brae

beus,qui banc doctrina aftronomicepartem,conflituit excoluitqueslib.Progymnafmatum,
quiex co tempore prodigt inlucem,quam etiam partemfecs Aftronomie Partis Optica ante 15. annos
editasaucsig, in Epit.Aftr lib.1.afol 52.

(2) Neque mihi digna vidctureius motus diucrfitas.] Itacreditum eft hucnfque
de Mercurio; nec nego,magnam off: verorum etiam eins motunm dierfitatem , fed que quantitatis
654, non frais feuprincipiorumn vt hackenus dacebsmiur ; his nism principiisillenibil dffert & a
teria,

(3) Exinventis quidem pradicendis.] Sequebarid temporiscommunem opiniones,
Mercuriucn ventosin fecie concitare,pracateris Planetic, At memultoriim annorum docuit expe-
ruentia ston efediftrabictas mietationum aure formasinter Planetas; fed generaliter incitari Natu=
ram fablenavemab aff talus binorum,rel 3 flationibus fingulorumyt ita exfudet vapores aut ful~
moscx montibus & offcinis fubterraneis , qui vapores G fusni , velin pluuiae, vel innines, vel
chazinata, vel fabvsina , vel grandincs, velventosdigencrent , pro circumftantiis locorum Cr tem=
porn, Veitrcerte magit, vel nungiam, vel rarifcime funt foll : plunia omnis ante fe ventos agit,
am promcaan ingruct impetu aitssCy aumplariinuan furunt ventiid indiciumedt bumide confistu-
Hosts aint. ct enna in vontanis pluit,wnde venti fpirant,aut nix ibi foluicur, aut vapor bumidus
impetie furfism Latus alsbi in guttascogitur , abbi aflaanssn fipernum frigusimpingitur refilieque,
gue qutidcin tissu lenisaaure gemfiselt, cm ebullit vapor ex aliquo mente, repercutiturque & dee
fist 1 omnes ctrcumcinca plagss, ESE pbi omnis aer per rotas Continentcsextcnfus principio motus
dato in montanisomniit altipimis,in fluxu confliturtur. Ita omnis rentus,ab omnib. promifiue cau=

Sis, vel uncfiigarionibus natura concttari potcst, nce folum incufare pote Mercuri , ortus Ven=
forum.

(4) Nam ipfaterrena Eccentricitasdiminuta.] Supra didlum,id non efeprobabi-
Iegnce ta acctoratas vetertan Obfernationesad hoc probandumrequifits, vt demonftratio eficia~
tur neceffaria. ktaque ainple:tor 1xioma Copernicibic pofitum , Concedendum fe.commen-
furationes ciculorum mantille. Id enim fuader calinatura , & inductio & Planetis ce-
tenis.

(5) Certiilima fpe fretus, propedicm fore] Itatuncille folebat has didtis anima.
re fpeque curs, qui etl quoadtinpus, peexcilit, necenim propedicm ef, quod viginti quatuor
aiinas fo quatier tands it tamu {pet fice compost fadlus per mcum Opus Harmonicum,



Chapter XIX

congratulate yourself like Copernicus according to Rheticus® in the letter, (5) relying on the
sure hope that the day will soon come when by means of your own brilliant discoveries
those points which are still doubtful, and which torment the company of astronomers con-
siderably, will be made manifest.”


(1) By the density of the atmosphere and the physical parallax.) Tycho Brahe calls it the refractions of
the stars, and established this part of the discipline of astronomy and elaborated it in his book Progym-
nasmata, which has brought it into the light from that time on. I have also made it part of my Optical
Part of Astronomy, published 17 years ago, and have augmented it in my Epitome of Astronomy, Book
I, from p. 52 on.

(2) Nor do I think that the variation in its motion is worth.) This has been the belief about Mercury up
until now; and I do not deny that the variation in its true motions is also great. However, it is a variation
in amount, not in form or in principles, as we have been stating up to the present; for in principles it does
not differ at all from the rest.

(3) And indeed in predicting winds.| 1 was following at this time the common opinion that Mercury
stirs up winds as a class, more than the other planets. But the experience of many years has taught me
that the forms of changes in the atmosphere are not allocated among the planets, but that in general
sublunar Nature is stirred by the aspects of pairs of planets, or the stationary points of individual planets,
to discharge vapors or fumes from the mountains and underground workings; and these vapors and
fumes degenerate either into rains, or into snows, or shooting stars, or lightnings, or hails, or winds, ac-
cording to the circumstances of place or time. Certainly great winds are either never or very seldom unac-
companied. All rain drives winds before it, as soon as, driven by its onrush, it sets to; and when the winds
age most, that is a sign that the character of the year is wet. For either itis raining in the mountain coun-
try, from which the winds blow, or the snow is melting there, or a wet vapor carried up by its onrush is in
some places forced into drops, in other places as it surges up strikes against the cold of the upper region,
and recoils. This indeed is the genesis of even a gentle breeze, when a vapor boils out from some moun-
tain, and rebounds, and flows down on all the regions round about. There are places where all the air ex-
tending over whole continents, when movement has been started in the highest mountain country of all,
is set in a state of flux. Thus every wind can be stirred up indiscriminately by all causes or searchings of
Nature; and the origin of winds cannot be blamed on Mercury alone.

(4) As the very decrease of the Earth's eccentricity.) It has been said above that this is not probable,
and the observations of the ancients required to prove it are not so accurate as to establish the demonstra-
tion which is necessary. I therefore adopt the axiom assumed by Copernicus on this point, in other
words, “It must be conceded that the dimensions of the circles have remained the same.” For that is sup-
ported by the nature of the heaven and induction from the other planets.

(3) Relying on the sure hope that the day will soon come.| In this way he used to stimulate my
endeavors with his sayings and hope, although as far as time was concerned his hope was disappointed,
for the day does not come soon; it follows twenty-four years later.

However his hope has at last been fulfilled by my work Harmonice.

714 Ioan, Keprert

(1) Que fit proportio motuum adorbes.

T ave haétenus quidem expeditum eft argumen-
J cum illud , quo ego plurimum roboris afferri puto
nouatis hypothefibus: deméftratumque,quod pro-
portione quinque regularium corporum vrantur
moive orbium in hypothclibus Copernici. Vi-
‘a deamus modo, vtrum altero ctiam argumento ¢x
3 motibusdeducto poffint & noux hypothefes, & ha
ip{e orbium dimenfiones Copernicanz confirma.
ri, atquein proportione motuum ad dmsiuaG certiorratio ex Coperni-
co,quam ex viitatis hypothefibus,haberi. Quain re dism amplitudines or-
binm proximas Copernicants ex motuum meazéensis temporibus bene cognitis ex=
iruo, faue facihs Vranic, pulcherrimo conatui; tuus iam honos agitur.
Primumomnes optant ; vequo longius quilibet orbisabelt a me~
dio,tantotardiori motu incedat. Nihilenim rationi magis eft confenta-
neum,telte Arift.lib.2.de Ceelocap.10. quam. nals rcyu pyreday nie ines
xuvjots wisdwsinac, Quo loco ctfi Philofophusalienam affertabinftituto
noftro rationemalteram, fcilicet mpedimentum ab occurfatione per-
niciffimi primi mobilis: tamen &altera ratione proincadhuc 5 & tora
fententia contra Prolemaum,cétraque {cipfum militar, Placctillinam.
que,motus xqualitacem d motoribusin omnes orbes venire; inaequalita-
temreditusab orbibusipfis caufari: ve, Saturni quidem qualibet parti-
culatam fitvelox,quam cftinftma Lunz fphzra,vi motionis equaliss(ed.
illiiam accidat , veamplius natta fpacium, cum non citatiorfitcateris,
tardius redeat. Atqui viliori hac xqualitace Philofophus in veterum tra+
ditione potirinon potuits quiancceflecrat, vetribus Planctisinzqua-
lium orbium, Soli, Vencri, Mercutio xquales reditus tribucrent, atq; fic
femper fupcriorem in orbe fuo citatiorem efficerentinferiori.In Coper-
nico prima frontetalis offert (cle proportio. Nam fex orbium mobilium
femper quianguftior cft,citius redit. Mercuri) namque curfastrimeftris
eft, Vencris felquiogto menfium, Terrg annuus, Mattis bimus, louis duo-
decim, Sacurnitrigintaannorum. Verum fiad calculos reuoccs ita ve
quanta cit proportio motus Saturniad ambitum orbis, fiue ad diftantia
(cadem enim eft proportio citculorum,quafemidiametrorum)tantam,
ctiam facias proportionem cxtcrorum motuum cuiuique ad faum or-
bem ; deprehendesciufmodi fimplicem proportionem non habere lo-
cum.Cuius rei cape hanctabellam indicem.

f Dies



So far the argument by which I think a great deal of strength has been added to
the novel hypotheses has run smoothly, and it has been shown that the distances
of the orbits in the hypotheses of Copernicus use the ratios of the five regular
solids. Let us now see whether also from a second argument drawn from the mo-
tions both the new hypotheses and the Copernican dimensions of the orbits
themselves can be verified, and for the ratio of the motions to the distances a
more accurate account can be obtained from Copernicus than from the custom-
ary hypotheses. In this affair, during my deduction of the sizes of the orbits, com-
ing very close to the Copernican values, from the well-known periodic times of
the motions, be gracious, kindly Urania, to this splendid endeavor: your good
name is involved now.

First, everybody wants each planet to proceed with a slower motion the further
its distance from the center. For nothing is more reasonable, witness Aristotle, De
Caelo, Book II, Chapter 10,* than that “the motions of each should be in propor-
tion to the distances.” In that passage although the Philosopher is adducing one
line of reasoning which is alien to our scheme, that is, resistance to the influence
of the first moving sphere, which is the fastest moving, yet by another line of
reasoning he is now fighting on my side, and by his whole notion against
Ptolemy, and against himself. For he believes that equality of motion is imparted
to all the orbits by their movers; but he takes the pretext for the inequality of the
times in which they revolve from the orbits themselves. Thus each particle of
Saturn is indeed as fast-moving as the lowest sphere of the Moon, by the force of
their equal motion; but in fact as the former occupies a wider space, since it is no
swifter than the rest, the result is that it revolves in a longer time. Yet the
Philosopher could not achieve this rather paltry equality in the tradition of the
ancients, because it was necessary for them to attribute equal times of revolution
to three planets with unequal orbits, the Sun, Venus, and Mercury, and so they
made the superior planet always swifter in its orbit than the inferior. In Coper-
nicus such a ratio is apparent at first sight. For of the six moving orbits the nar-
rower one always revolves faster. For Mercury passes round in three months,
Venus in eight and a half months,? the Earth in a year, Mars in two years, Jupiter
in twelve, and Saturn in thirty. Indeed if you compare it with the calculations,
making the ratio of the motion of each of the other planets to its sphere the same
as the ratio of the motion of Saturn to the circumference of its orbit, or to its
distance (for the proportion of circles is always the same as that of their radii),
you will discover that there is no room for a simple proportion of that kind. Of
that fact take the following table as evidence.

Mysrtertym CosmoGrapaicyve. 75

+—| | x |
ster.
! —-| Dies fer. = Terra |
ii a = |
% | 6159 4332 37 | ——— | Dies fer. |] —-__ ¥
1785 182 686 59] — Dies fcr. =
ree. | iz 4 853 4)2 [535 ay Di-s fer.
91 844 606} 325 26259 farg 42 }|——-] —
ledas Lose fie [os ns 87 58
Hic capica columetlarum continent dies & dictum ferupula,

quibus fuperinicripti Plancta ub orbe Scellato faas periodos com-
plene: fequentes numeri indicant;quantum dieruin quam proxime de-
beatur interiori Planet , cadcm proportione ad crbem , qua veitur
ille,quicttin capite columetic. Vides igitur, veram periodum femper
minoremedle, quam eftilla, queilliaceeibuiturad fimilicudinem fape-
rioris,
Tnteri

Aramen motuum binorum ed inuicem,non qudemeadan,
fimilis tamca femper cft p-oporuo, qua mterdiftanti

Dues fer.

10759 12 faccipisear Ginus{ If 403 | Atfifuperioris [26 5-2
37 2 ( totusi000. Ete gm 159 caste diftina } ga avo

59 oF Cin ca quantitate terre 532 C fir1090. ch Sperre 658
vy tere | mozus pexiodi- % 615 j intctioris in Zaz
42 2 jeus $392 J Copeinico ¥ 500

Hic vide miht inmotibus medijs, farcerto cognitis idquelonge
ptinsarque de certa diftantiarum ratione Copernicus cogitaret , vide,
inquam,eandem diuerfitatem, que inter ipfas eft diftantias , ex seSa-
aera per Copernicum , & ex quinque corporibus per meextruétas:
nque fecus g minima , inde fecus $, ¥%, Terram,& maxima fecus $+
verinque fecus 34 & ¥ xqualis penesitem & fecusterram, & ¢- Igitur vel
iam ftacim fatis explorata eft Copernico de mundo veteri victoria.

Quod fitamen pracifius ctiam ad veritatem accedere , & propor-
vonum aqualitatem vilam {perarevelimus: duorum alcerum ftatuen-
dum eft:aut (2) Mosrices animas,quo funta Sole remotiores,hoc cf-
feimbccilliores : aut, (3) vnam effe motricemanimam in orbiumo-
mnium centro, fcilicctin Soles que, vt quodlibet corpus eft vicinius,
itavchementiusincitet; inremotoribus propter clongarionem & at-

enuationem virtutis quodammodolanguefeat. Sicutigicor fons Lu-

cisin Soleeft , & principium circuliin loco Solis, {cilicet in centro 5 ita
nunc vita, motus & anima mundiin cundem Solem recidit; veita fixa-
rum fitquics, Planetarumactus fecaidi motuum; Solis aftns ipfe primus:
quiincomparabiliter nobilior cf adtibus fecundis in rebus omnibus; nd
{ecusatque Sol ipfe & {pecici pulchritudine,& virturis etficacia  & lucis
Ko: fplen-


Chapter XX

Saturn

Days Sixtieths Jupiter
Satum 1075912 Days Sixtieths Mars
Jupiter 6159 4332-37 Days Sixtieths Earth
Mars 1785 1282 686 59 Days Sixtieths Venus
Earth 1174 843 452 365 15 Days Sixtieths Mercury
Venus a4 606 325 262 30-224 42_—_—Daays Sintieths
Mercury 434 312 167 135, 1s 8758

Here the heads of the columns contain the days and sixtieths of days in which the
planets shown above them complete their periods against the Sphere of the Stars.
The numbers which follow show how many days, as nearly as possible, are due to
the inferior planet, in the same ratio to its orbit as is taken by the one at the head
of the column. You see, then, that the true period is always less than that which is
appropriate for it by comparison with the superior planet.

Nevertheless, between pairs of motions there is, not indeed the same, but a
similar ratio to that between the distances.*

days sixtieths

for for But if the mean for
10759 12 Saturn the wholesineis Jupiter. 403. distanceof the Jupiter $72
For 4332 37 Jupiter taken as 1000 Mars 159 superior planet is. Mars 290
if 686 59 Mars units, the pro- Earth 532 1000 units, that Earth 658
for 3651S. Earth —portionate period. Venus 615. of the inferior Venus 719
224 42 Venus ic motion will be. Mercury 392 accordingto Mercury $00

Copernicus is

Here please note that in the mean motions, which are accurately enough
known, and that long before Copernicus thought about an accurate reckoning of
the distance, note, I say, the same discrepancy as that between the distances
deduced from the equations according to Copernicus and from the five solids ac-
cording to me: in both cases it is smallest with Mars, then with Mercury, Jupiter,
and Earth, and greatest with Venus; in both cases the discrepancy with Jupiter
and Mercury is almost equal, and similarly with the Earth and Venus. Hence the
victory of Copernicus over the ancient universe is straight away sufficiently
confirmed.

But if, nevertheless, we wish to make an even more exact approach to the truth,
and to hope for any regularity in the ratios, one of two conclusions must be
reached: either (2) the moving souls are weaker the further they are from the Sun;
or, there is (3) a single moving soul in the center of all the spheres, that is, in the
Sun, and it impels each body more strongly in proportion to how near it is. In the
more distant ones on account of their remoteness and the weakening of its power,
it becomes faint, so to speak. Thus, just as the source of light is in the Sun, and
the origin of the circle is at the position of the Sun, which is at the center, so in
this case the life, the motion and the soul of the universe are assigned to that same
Sun; so that to the fixed stars belongs rest, to the planets the secondary impulses
of motions, but to the Sun the primary impulse. In the same way the Sun far ex-
cels all others in the beauty of his appearance, and the effectiveness of his power,

96 loan. Keprerti

{plendore cxtcris omnibus longe praftat. Hiciam longerettius in So-
Jem competuntilla nobilia epitheta, Cor mundi, Rex, Imperator ftella-
rum, Deus vifibilis, & reliqua. (4) Sedhuius matcrizx nobilitaslonge
aliud cempuslocumquerequirit,& iamantea fat clare apparetex Nat-
ratione Rhetici.

Tam autem de modo conftituendx huius qnafite proportionis
nobis cogitandum eft. Supra vifum eft,fifola orbisampheudo faceretad
augendum rempus "e442: quod moruum & diftantiarum mediarum
cadem differentia futura fuiflet. Qua nempe proportio 88. dierum pe-
riodicorum Mercutij, ad 225. dies Vencris: cadem forer femidiametti
orbis Mercurralis ad Veneriam. Tamverocommifeet f huicmocuum
proportioni debilitasmotricisanimz in remotiori. Difpiciendum igi-
tur,cum hacdebilitate ve comparatum fit. Ponamus igitur , id quod yal-
deverifimile eft, (5) cadem ratione motum a Sole difpenfari, qua lu-
cem.Lucis autem ex centro prorogatz debilicatio qua proportione fiat,
docent Optici. Namquancum lucis eftin paruocirculo, tantundem et-
iam lucisfiue radiorum folarium eft in magno. Hinc cum ficin paruo fti-
patior , in magno tenuior, menfura huius attenuationis cx ipfa citculo-
rum proportione petenda erit, idque tam in luce, quam in motrice vir-
tute. Quare quantoamplior Venus Mercurio, canto iftius, quam illius
motus fortior, fine citatior, fiue pernicior, fiue vigentior , feu quocun-
queverboremexprimere placer. Atquanto orbisorbe amplior , tan-
to plus temporis etiam requiritad ambitum, etfi verinque fitaqualis vis
mocus.Ergohinc fequitur,ynam clongationem Planctg aSole maiorem
bis faceread augendam periodum: (6) &contra,incrementum periodi
duplum effe ad 47474 differentiam.

Dimidium igiturincrementi additum periodo minori, exhibere
debet proportionem veram diftantiarum , fic veaggregatum fit vedi-
ftantia fuperioris,&, fimplex minor periodus reprafentetinferioris, f
licet Planeta fui diftantiam in eadem quantitate. Exemplum, ¥ motus
petiodicus eft 88.feredicrum, Veneris 224. cnm beffe ferme,differentia
136.8 bes,dimidium 68. & pars tertia. Hociundtum cum 88. cfficit 156.8¢
trientem. Ergo ve 88.ad 156.cum tertia, fic femidiameter circuli Mercu-
rialis medij ad mediam Vencris. Hocmodo fin fingulis opereris, acque
prouenientes binasdiftantias per numeros finuum explices, fic ve fem-
per fuperioris femidiameter fit finus totus:


o 274 290

prouenict femidiameter orbis, cerezooae Aceftin Copern. $658
ha 7624 719

uF 5639 usoo

(7) Propius, vt vides, ad vericatem acceffimus. Etfivero dubito,
an demonftratiua methodo,quod theoremainftituerat,praxisifta diui.
feditferentiz affequuta fuerit per omnia: ramen non omnino nihilin
hifcenumerislatére, credereme iubet alia numerandi methodus, qua
ad cofdem numeros reuoluar. Quia enim probabil eft, fortitudincm

motus


Chapter XX

and the brilliance of his light. Consequently the Sun has a far better claim to such
noble epithets as heart of the universe, king, emperor of the stars, visible God,
and so on. (4) But the nobility of this theme demands a far different time and
place, and is already clearly apparent from the Narratio of Rheticus.*

Now, however, we must consider the means of establishing this ratio which we
require. It has been seen above that if only the breadth of the sphere contributed
to increasing the periodic time, there would have been the same difference be-
tween the motions and the mean distances. That is to say, the ratio of the 88 days
of the period of Mercury to the 225 days of Venus would be the same as that of
the radius of the sphere of Mercury to that of Venus. As it is, however, this ratio
of the motions is compounded with the weakness of the moving spirit in the more
distant planet. Therefore we must also discover what its relationship is with this
weakness. Let us suppose, then, as is highly probable, that (5) motion is dispensed
by the Sun in the same proportion as light. Now the ratio in which light spreading
out from a center is weakened is stated by the opticians.* For the amount of light
in a small circle is the same as the amount of light or of the solar rays in the great
one. Hence, as it is more concentrated in the small circle, and more thinly spread
in the great one, the measure of this thinning out must be sought in the actual
ratio of the circles, both for light and for the moving power. Therefore in propor-
tion as Venus is wider than Mercury, so Mercury’s motion is stronger, or swifter,
or brisker, or more vigorous than that of Venus, or whatever word is chosen to
express the fact. But in proportion as one orbit is wider than another, it also re-
quires more time to go round it, although the force of the motion is equal in both
cases. Hence it follows that one excess in the distance of a planet from the Sun
acts twice over in increasing the period: (6) and conversely, the increase in the
period is double the difference in the distances.”

Therefore, adding half the increase to the smaller period should show the true
ratio of the distances:* the sum is proportional to the distance of the superior
planet, and the simple lesser period represents the distance of the inferior, that is,
of its own planet, in the same proportion. For example: the periodic motion of
Mercury takes about 88 days, that of Venus about 224% days. The difference is
136% days, and half that is 68%. Adding that to 88 makes 156. Then as 88 is to
15614, so the radius of the mean circle of Mercury is to the mean distance of
Venus. If you operate in this manner in the individual cases, and set out the
resulting pairs of distances by sines, in such a way that the radius of the superior
planet in each case is the whole sine:

then the for Jupiter 314 S72
resulting Mars 214 But in 290
radius of Earth 694 Copernicus 658
the orbit Venus 762 it is 119
will be Mercury 563 500

We have arrived, (7) as you see, closer to the truth. * Although indeed I am
doubtful whether by the demonstrative method this procedure of halving the dif-
ference has in all respects achieved what the theorem had proposed, yet I am led
to suppose that there is some significance lurking in these values by another
method of calculating which will bring me round to the same values. For as it is
probable that the strength of motion is proportionate to the distances, it is also

Mystrriva Cosmocrariuicyvm. 7

motus cum diftantiis efle in proportione 5 crit & hoe probabile, quod
qauibet Planeta, quantum fuperatfuperiorem fortitudine motus, tan-
tim fupercturin d:ftantia. Efto :gitur, exempti gratia, Martis & diftan-
ua Svirtusynitas. Igitur quota particula virtuas Martiz Tellus Marte
tortior eft coram diftantia Martie particulam amittet.. Hoc facile fit
petregulam Falli : poso namque radium Telluris ad Martium effe ve
oy 4.ad 1000.Ergo,inquio,fi amplitudo circuli per 1090. notata peram-
bulacura vi motrice Martia 687. diebus: perambulabitur eadem vi Mar-
tia,circulus minor, per 694. notatus, dicbus 477. lam quiacertum eft
terre circvitum cflenon 477. fed 365. dierum : pergo per regulam in-
uerfam fie: dies 477.confamerentura fim plici vi Marcias quantum devi
Marea confumit circuitum, 3/5. cum quadrante dierum pereundem
ambitum, quem Mars conficeret 477. dicbus?. Nam dubiumnon eft,
quin fortior virtus requiratur quam cft Martia. Proucnit igitur fupra
integram vim Martiam adhue ,366 pars ciufdem virtutis.Eccantum 1 el-
Jus Marte fortior cft : debet igitur & tanto propior effe Solisnempe fi
Mars per r00c.a Solereccilit (diftantia enim fuperioris femper eft inte-
grum quid ) Tellus per 306 carum partium propicrert: & tubtradto fu-
petion 306. abinfertur roco debet prouenire numerus initio pofitus,
videlicct 694.fivera fuicilla pofitiosfin falla foretsergo operarctis {ecun-
dum prxcepta re gul,& eliceres veram pefitionem.

Vides hocaltero theoremate proucnire non alios,quam fi periores
numeross vnde certum eft duo ifta theoremata forma quidem differres
fed reuera coincidere, & niti codem fundamento, qued tamen quo pa-
to fiat, nucttigare ha€tenus munquam potui.

Nota Audtoris.

(1) (Ve fit proportio motuum ad Oubes.] Hac ef propria materialibri lV. Epite-

Qh tranffianpta inde in ib.V Harmonicorum, Nam illius libri cap. 11. bac ipfa qua-
fio enodatur,c inter fundainenta affionitur , quibusdemonftratur, motus Planctarumextremos
continent proportion:bus Harmonicis. Etfivero m hoc capite nonduin affequutus fiom, quod quere-
bamsplerayjetamen adbibita principia, que mibi iam tum natura rerum videbantur confentanea,
cortafinta, Cr totis bis 25. annie vtilifinta funexpertus: prsfertim in Cominentarite de motibus
Martuspartelv.

(2) Motricesanimas.] Quasmullas efeprobaui in Comutent.Martis.

(3) Vnameilemotticem Animam.]| Sipro voce Anima,vocem, Vim, fulftituasha-
bis ipfifumuin principium,ex quo Phyjica caleflisin Comment. Martis eft conflituta,@y ib IV Epi-
toines Aftrexculta, Olim eniin,canfasn moucntem Planetas abfolute Animam effecredebam,quip-
peimbutusdogmatibus, C. Scaligeri,de Motricibus intelligentiis, Atcum perpenderem,hanccau-
{fem mo:riceandebilitaricum diftantia lumen Solisetiamn attenuari cu diflantia 4 Sole shine con
slifis’ su bancoffecorporcim aliquid , finan proprie faltem equiuoces ficut lumen dicimaus effeali-
gird corporcum, td eit fpecican 2 corpore delapfum, fed immmateriatam,

(4) Sed huius materi nobilitas longealiudrempuslocumque.] Nimirumle-
can inuenie in Comment. Martis anno 160 9 editisindetranffuniptacht{umma ret, Cf repetita in
Epi. aftrontiblv,

(5) Eadem ratione motum a Sole.] Hacommia fine vBa mutatione valent etiamin
Comment. Martie.

(6) Etcuntra,incrementum periodiduplum. } Hicerrorincipit. Hoc enitt non

K Hiden


Chapter XX

probable that any planet will be exceeded in distance by the one superior to it by
the same amount as it exceeds it in strength of motion. Then, for example, let
both the distance and the power of Mars be unity. Then the Earth will lose the
same fraction of Mars’s distance as the fraction of Mars’s power by which it is
stronger than Mars. This is easily found by the rule of false assumption. For I
take the ratio of the Earth’s radius to that of Mars to be as 694 to 1000.
Therefore, I say, if the width of the circle, denoted by 1000, is traversed by the
moving force of Mars in 687 days, the lesser circle, denoted by 694, will be
traversed by that same force of Mars in 477 days. Asin fact it is accurately known
that the Earth’s circuit is not in 477 days but in 365, I proceed by the inverse rule
as follows. 477 days would be taken up by the force of Mars on its own, What
multiple of the force of Mars takes up 365% days for the same passage which
Mars would complete in 477 days? For a stronger power than that of Mars is un-
doubtedly required. The result is, then, a further 306/1000 part of the same
power over and above the complete force of Mars. Now this is the amount by
which the Earth is stronger than Mars. Then it must be nearer to the Sun by the
same amount. That is, if Mars is 1000 units away from the Sun (for the distance
of the superior planet is always a round number), the Earth will be nearer by 306 of
the same units; and on subtracting from the 1000 for the superior planet 306 for
the distance from the inferior, the result should be the value assumed at the start,
namely 694, if that assumption was true. But if it was false, then you would
operate as the rule directs, and extract the true assumption.

You see that from this alternative theorem the values which result are no dif-
ferent from those above.’ Hence it is certain that both theorems differ indeed in
form, but in actuality are equivalent, and rest on the same basis. (8) However, by
what means that comes about I have never so far been able to discover.


(1) Whar the ratio of the motions 10 the orbits is.) This is the proper subject matter of Book IV of the
Epitome, transferred from there to Book V of the Harmonice. For in Chapter 3 of that book this very
question is unraveled, and it is included among the basic assumptions by which it is demonstrated that the
extreme motions of the planets are defined by the harmonic proportions. Although in fact in this chapter
Thad not yet attained what I was seeking, yet a number of principles were introduced which then seemed
to me in agreement with the nature of things, and quite certain, and which I have found very useful
throughout the last 25 years, especially in the Commentaries on the motions of Mars, Part 1V.%°

(2) The moving souls. Of which | have proved there are none in the Commentaries on Mars.

(3) There is a single moving soul.) \f for the word “soul” you substitute the word “force,” you have the
very same principle on which the Celestial Physics" is established in the Commentaries on Mars, and
elaborated in Book IV of the Epitome of Astronomy. For once I believed that the cause which moves the
planets was precisely a soul, as I was of course imbued with the doctrines of J.C. Scaliger on moving in-
telligences, But when I pondered that this moving cause grows weaker with distance, and that the Sun’s
light also grows thinner with distance from the Sun, from that 1 concluded, that this force is something,
corporeal, that is, an emanation which a body emits, but an immaterial one.

(4) But the nobility of this theme (demands) a far different time and place. Naturally it finds a place
in the Commentaries on Mars published in the year 1609, and a summary of the matter was transferred
from there and repeated in the Epitome of Astronomy, Book IV.

(5) Motion (is dispensed) by the Sun in the same proportion.] All this is also valid without any altera-
tion in the Commentaries on Mars.** :

(6) And conversely, the increase in the period is double.| Here the mistake begins. For this is not the


efeinsin texcbriscontingit C horvefcit) aberranerim 3 propofite,volenseandem virtutum propor
sone probareyqute «(fet diftantiaruans cunt tainen yartutuns hic preportionem minorem flatuamy
Marts fl. 12.00. Terre 1306.diftantiarum maiorem, Martis1000. Terre69 4. Fuifervero
‘rague proportio inon arithmctice, fed Geometrice media
Ninnis muta de hoc proceffs fepeliendusenina st non errans tantum,, fed fictiam plane legi-~
time piecedats qitia proportio pertodorum noneSt dupla proportionis diftantiarumn mediarum,, fed
poifecispione & abfolusifime ciufilem fofiuialtera:hocelt, fiquarantur radicescubiceex Planeta~
inn temporibus periodicis vt 687. Gr 3654. Grha raduces muultiplicentur quadrate : tunc in
agutaciatis bis numeris ine$t certifima proportio femidiametrerion Orbiuan.Perfiet vero poffint ope-
rationcsifte facile, vel per Tabul.sm Cuborusm Clas, qus adiecta est cius Geometria Practice, vel
lauge facalius per Logartthiaos Neperi Baronis Scoti fic: Prolongentur noflyé numeri pro necefitate
Gcominditatesytfine 687 0 9.6 36525.nec im fequemur fummam fubtilitatem: Logarithm
corms funtt ex Canone Nepert 375.43-C° 1007 1s circater,
Horum partestertis font 12514.G 3357 2+

Irharum dupla, illari befis 25029. & 6714.4.qusexbibent interfinus,numeros hofce 7 7858.
€51097. Inter hosest proportio orbiuin Matis G> Tellurie. Traifpon.ttur enim proportioin al
nuineros,&> fiat t 5109 7 utd 100000.fic 77858.ad 152375. qiveplune eft quantitas mediocris
diflantiz Martis,qualiuin Terra a Sole diftat 100000.

Caufein cur non fit dupla proportioperiadaruan, adproportioncm Orbinon, (od faltem fefqui-
alterasisuencesexplicatamin tpit.Affr lib.g fol.s30.

Hocigitur alscruns c& praflantigionum quidem fecretum auébarg loco nuncaccedat Myfle-
risbifee Cofsangrapbicis: quo in vulgusenuencatolubet nune inter fos, tamn Theologos,quam Phi=
Wyfiphosclans voce ad cenficram doginatis Aviftarchics conuocave: Cattendite viri Reliziofif-
fimisProfiindiffimindodiffimi:

Si verum dicit Ptolemeus demotucorporum Mundanorum, C difpofitione
Orbium: tunc nulls eit conftans Gy identica per omnes Planetas proportio Motuum, >
fou perisdicorum temporumad Orbes. 2

Siverum dicit Tycho Brabeus,Solem quidem fe centrum Planetarum gun. ??
gues velutiquingue Epicyclorim: Terram vero offe centrum orbes Solis ,veTerra??
puicfcente, Solcicumeat,portans cy luxans fj fterna totum Planetarinm: tunc ei ??
\juidem eadein proportio periadicorum temporum ad orbes per onines Planetas s fei-??
‘cet proportio pertodorum, (verbicanfa,Solis Gr Martis) eit (efquialtera propor-??
trons orbinm fuorum, fed motus nd ab eodem centro difpenfatur, Motts enim quin-??
gue planetarum circa Solem difpenfatur a Sole,motus vero Solis circa terran difpe-??
itur a terra;at fic Sol planctarum,Terra vero Solis motor conftituitur. a

Si denique verum dicit Ariftarchus Solem ffecentrum G- quingue Planeta-””
riorurs Orbiuna, Cefextictiam,gui Tellurem vebit, vt Sole quicfeente, Tellusine?
ne Planetas cateroscirca Solem vehatur ; tite binorum quorumeungue Planetarum ”
axbesinter fe proportionem talem habent,que duas tertias complettatur proportio.”?
isperiodorusn,velyproportio periodorum eit perfectiffimme fofquialtera proportio- .
nts orbiums ch motus tam Telluris quam caterorim quingueex vaico  fonte Solaris i
corporis difpenfatur.

Hic aula plane eit exceptio , proportioe munitifimeex vtroque lateresex , ,
porte quidecnfenfiss atteftatur Aftronomorum obferuationes quotidiane, cum omni ,,
jobsilitate fiarex parteverorationts,a[lipulatur nobis Avift.in gencralib.in fpecie ve ,,
14 canfefuppetunt entdentifvime, pofita fpecic inimateriata corports Solariscur pro- ,,
pertiodebeat ofe,necfimpla,nec dupla fed plane fefquialtera:can[a tt fuppetiie cur Sok ,,
potins Torre vt Planetarumcetcrorum,guam Terra Solis moter effe pofitsdenique , ,
viscurale rationts lumen diGkat digniorems Gy magis Archetypicane effe peci¢m Ope- ,

rum Dea



Chapter XX

left hand with his right hand unawares in the dark and is scared), I have wandered away from my inten-
tion, as 1 was meaning to prove that the ratio of the powers was the same as that of the distances, whereas
There establish that the ratio of the powers is smaller, that is 1000 for Mars:1306 for the Earth, and that
of the distances greater, 1000 for Mars:694 for the Earth. Now the ratio would have been the same in
each case if Thad taken not the arithmetic, but the geometrical mean.

Thave said too much about this process; for it should be buried not only as mistaken but even if it were
plainly carried out legitimately, because the ratio of the periods is not the square of the ratio of the mean
distances, but quite perfectly and precisely the 3/2th power of that ratio. That is, if the cube roots of the
periodic times of the planets are found, such as 687 and 365%, and these cube roots are squared, then in
these squares the ratio is exactly that of the radii of the orbits. These operations can in fact easily be car-
ried out either by Clavius's Table of Cubes, which is appended to his Practical Geometry, or much more
‘easily by the logarithms of Napier the Scots baron, as follows: let our numbers be lengthened, for necessi-
ty and convenience, to 68700 and 36525. We shall not now aim for the greatest accuracy. Their
logarithms, from Napier’s table, are 37543 and 100715, approximately

‘The third parts of these are 12514 and 33572; and twice these, two thirds of the former numbers, is
25029 and 67144. The numbers shown for these in the sine tables are 77858 and 51097.'5 The ratio of
these is the ratio between the orbits of Mars and the Earth. For if the ratio is converted into other
numbers, it turns out that 51097:100000 as 77858:152373, which is clearly the amount of the mean
distance of Mars, in units in which the distance of the Earth from the Sun is 100000.

The reason why the ratio of the periods is not as the square of that of the orbits, but in fact as the 3/2th
power, you will find explained in the Epitome of Astronomy, Book IV, page 530.

This, then, is another and an outstanding secret which now comes as an addition to these Secrets of the
Universe; and now that it has been announced publicly, it is our pleasure to call together both theologians
and philosophers one and all with uplifted voice to pass judgment on the Aristarchan doctrine. Attend,
most religious, profound, and learned men.

“If what Ptolemy says about the motion of earthly bodies and the arrangement of the orbits is true,
then there is no ratio of the motions or of the periodic times to the orbits which is permanent and con-
stant for all the planets.

“If it is true as Tycho Brahe says that the Sun is indeed the center of the five planets, as if of five
epicycles, but the Earth is the center of the Sun’s orbit, so that with the Earth at rest the Sun goes round,
carrying and illuminating the whole planetary system, then the ratio of the periodic times to the orbits is,
indeed the same for all the planets, that is the ratio of the periods (for instance, of the Sun and Mars) is as
the 3/2th power of the ratio of their orbits, but the motion is not controiled by the same center. For the
motion of the five planets round the Sun is controlled by the Sun, whereas the motion of the Sun round
the Earth is controlled by the Earth; and in this way the Sun is established as mover of the planets, but the
Earth as mover of the Sun.

“Lastly, if it is true as Aristarchus says that the Sun is the center both of the five planetary orbits and
also of the sixth, which carries the Earth, so that with the Sun at rest the Earth is carried round the Sun
among the other planets, then the orbits of each pair of planets are in the same ratio to each other as the
2/3rds power of the ratio of their periods, or the ratio of the periods is quite precisely as the 3/2th power
of the ratio of the orbits, and the motion both of the Earth and of the other five is controlled from the
single source of the solar body.

“In this case there is plainly no exception, the ratio is completely secured on both sides. On the side of
the senses the daily observations of the astronomers attest it with all their accuracy, and on the side of,
reason Aristarchus agrees with us in general, and in particular very clear reasons are available, assuming,
the immaterial emanation of the solar body, why the ratio should not be either simple, nor as the square,
but plainly as the 3/2th power; and also reasons are available why the Sun can rather be the mover of the
Earth as of the other planets, than the Earth the mover of the Sun, Lastly the natural light of reason

30 loan. Kerrert

>» rum Dei, fi motus omnes abuno fonte fluant,quam fi plerique quidem ab vaville
>» fonte fontis vero ipfius ab alio ignobiliore fonte.

a» Accedat vero formatio ipfa proportionis orbinm feorfims ante metus falta,
>> per quingue figuras G per Harmontas. Nam fi Brahens verum dicitilocum ifts
3» aonhabent , nifi afcrto circulo aliquo Telluris tater orbis Martis dr Veneris per
jo isnaginationems circtamdudto: & Deus non reiipfins, fed imaginationis potiuscu-
»» rambabuit diforquens opus ipfum Mundanum , vt opcrisimaginatio pulchra efé
>» poffet: cumtamen infinite alia fimiles imaginarie fpecies, (vt flationum Gretr
>» gradationum ) careant talsornatu: at fiverum ditt Ariftarchis; tuncornatus ifle
a» iauenitur inresSpecies vero imaginaria omnes nulla cxcepta,permittuntur. necelfin
>» tatibuslegum opticarum,

ap Hifceperpenfis perovos aquos dogmatum cenfores foresmec hofles vos gefluros

oraatus Operum diuinoruns exquifiti[umi.Valete.

(0) Quidex defettu colligendum.

ETE Ic igitur hoc alecrum argumentum habet: quo proba-
i Ce cum eft Ariftotelis auctoricate, potiores cle nouas hypo-
HENS 2 thefes,proptereaquod pereasmonus duplicinomine, &

virtutis intentione,, &celeritate reditus fiant proportio-
nalesé=juae Copernicanis, quod invetcrum de mun-
do traditione ficri nullo paéto potuit.Atque hxc quidem
huius de motu tractatusintentio fola debebateffe. Verum non difficile
mihi eft conijcere sextituros, qui optauerine, yt hancvitimam opufculi
pattem omififfem. Ecenim (dicent) fiveram per corpora proportio-
nem ccelorum conftituiffes : veique motusillamconfirmarent. Veri-
tasenim i{cipfanon diffidet. Atqui vides ipfe,K ep LER E quantum in.
ter fediffidcant motus & corpora, hoc eft diftantia verinque extrudtz.
Quarenudum hoftilatus obijcis , imo teipfum feris, ncc opus alicno iu-
guleregladio.

Hisigitur verefpondeam, primum inucrto rationem, & ipforum,
imo omnium appello iudicium & confcientiam ; vtrum argumentum
putent verifimiliuseffe, num alterum de corporibus, an hocde motu.
Neque mihi probabile eft, quenquam aliter dicturum,quam hanc mo-
tuumad orbes accommodationem admodum concinnam cflz, atque
admurabile Dei opificis xereey4e. Proinde fi altcrutri argumento fides
habendatit, huic pr corporibus, aftipulaturos, ranquam rei magis eui-
denti; quamuis numeri adhucaliquantum a Copernicanis difcrepent.
Quod fiobtinui Lestoris confeffione , vtar pro confirmatione corpo-
rum, & excufatione difcordiz illus, ve qua multis partibus minoreft,
quam hacin motudiffonantia, Nam fi Leétorhie propter concinnita-
tem inuenti magnum crrorem libenter diflimulats paruum illicerro-
rem longe facilius tolerabit. Diuerfitas cnimilla penes corpora, cal-

culum


Chapter XXI

declares that it is a more worthy and archetypal emanation of the works of God, if all the motions flow
from one source, than if most indeed flow from that one source but those of the source itself from
another, more ignoble source.

“To this is added the actual design of the proportion of the orbits which was made separately before
the motions from the five figures and the harmonies. For if what Brahe says is true, there is no place for
these things, except by the introduction of some circle for the Earth drawn round in the imagination be-
tween the orbits of Mars and Venus, and unless God paid attention to imagination rather than to reality,
distorting the earthly work itself, so that the imagined work could be beautiful, whereas an infinite
number of other similar imaginary appearances (such as the stations and retrogressions) lack such a
display; but if what Aristarchus says is true, then that display is found in the reality, while all the imagi-
nary appearances, without exception, are permitted by the requirements of the laws of optics.

“After pondering these things I hope you will judge the doctrines fairly, and not conduct yourselves as
enemies of the most excellent display of the divine works,

“J, Kepler.”


That, then, is the position of this alternative argument. It has been proved by it
on the authority of Aristotle that the new hypotheses are preferable, because by
them the motions are under two headings, both from the extent of the power, and
the speed of revolution, made proportional to the Copernican distances, which
could not be done by any means within the tradition of the ancients on the
universe. That indeed should have been the sole intention of this treatise on mo-
tion. Moreover it is not difficult for me to guess that there will be those who wish
I had omitted this last part of my little work. “For” (they will say) “if you had
established the true proportions of the heavens by means of the solids, the mo-
tions would assuredly confirm it. For the truth does not disagree with itself. But
you see for yourself, Kepler, the extent to which the motions and the solids, that
is, the distances based on them on each side, disagree with each other. So you are
exposing your flank to the enemy, or rather you are striking at yourself, and there
is no need for you to be slaughtered by someone else’s sword.”

To answer them, then, I first invert the argument, and call on their judgment
and fairness to say which argument they think more probable, the other about the
solids, or this one about the motion. It seems to me unlikely that anyone will give
any other answer than that this fitting of the motions to the spheres is very neat, a
wonderful piece of handiwork by God the craftsman. Consequently, if one or
other argument must be accepted, they will assent to the second argument rather
than to the one from the solids, as being the more obviously acceptable, even
though the values still have a slight discrepancy from the Copernican ones. But if
I have obtained the reader’s agreement, I shall use it to reinforce the solids, and to
excuse the discrepancy in them, since it is much smaller than the conflict in the
motion. For if the reader in the latter case willingly overlooks a large error
because the discovery fits so neatly, he will far more easily tolerate the small error
in the former. For the difference with respect to the solids does not disturb
astronomical calculation in the lea: ut the difference with respect to the mo-
tions has a little greater effect. This is the first point: the blow has been parried.


culum Aftronomicum nihil admodum turbat: ifta vero pencs motus
paulo quid maius infert. A tg hoc primum eftspiaga nempe repofita.

Deinde (2) cumcorporadifientiant Amotibus, ve vere mihi ob:}-
citursfaceri vtique cogor,altcrutros in crrore verfari. Veruntamen erro-
réita dcmontftrari pole exiftimo, (3) vencutruminuentum/neque de
motuum,neque de orbium proportione) penitus rclinquere neceffe fit,
Verum autem inuentorum in culpa fit, ex fuperioribus facile eft coijce-
re.Primum diftantia motoriz longius a Copernicanistecedunt, quam
figurales. Deinde,timororias cum Copernicanis conferas,fingulas cum
fingulis,defectusa afcribas, videbis aliquam defectuum cum ipfisnume-
ee adco cum corporibus cognationem, praterquam in Mercurio.

cce:

oe eee Diffcrentiz

2 572 $74 eo Cubus.
ya 290 274 16 Tetraedron.
o Tene 658 690 26 Dodecacdron.

Tere ¢ 719 762 43 Icofaedron.
z ¥ 500 $63 He 63 O@acdron.
vel 559 m4

Plus fcilicet in quatuor , minusin quinto. Namex quatuor, bina
fempcr corpora funtfimilia,quincum folitarium eft. Deinde Mercuri,
veett varius,in ordinem redige, & cogita, debere aliquid altius media or-
bis {pifficudine pro media diftantia cenferi, (4) rantum nempe,quantus
eftorbis Oétacdri, (quod fupraaudiuifti media fpiffitudine amplus effe)
&obtinebit promedia diftantia 559.non 500. Ericigitur hic ordociusnu-
merorum 3 ¥559 1563 | + 4-Ecccin } %& 3 ¥ differentias minores , fc. 2.
4.in @ terra, terra 2 maiores, fc. 26.43. ficucinterieéta corpora illic Cu-
bus & Odtaedron, hic Dodecaedron & Icofaedron funt fimilia. Eta-
nimaduerte, quodillic, vbi magna differentia cit infcriptorum & circ
{eriprorum, paruaeftdifferentia diftantiarum:: viciffim vbi propemo-
dum xquales afcripti, magno inceruallo diffident diftantiz motoriz 4
Copernicanis. :

Cum igiturin defeétu hoc fic qurdam aqualitas, & vero nihilor-
dinatum fortuitoaccidat:ideo cogitandum numcros hofcead veritatem
quidé alluderesnondum tamen cam penitus affecutos. (5) Népe in ipfo
theoremateadhuclimari quid potcfts aurtheorema quidé reéte haber,
(6} fedcius fenfum neutra opcratio affecuta cft. Quod quamuis initio
ftatim fufpicari potui,noluitamen,Leétorem hac occafione, &veluti fti-
mulo pluratentandi,carere. (7) Quid fi namquealiquandodiem illum
videamus,quoan:bohzcinuentaconciliataerunt? (8) Quid fihinc ra-
tio eccentricitatum elici poflit? Nam quo pertinacius retincam etiam.
hocde motibus theorema,illud inter caterain caufa elt,quod vnius mo-
toria d.ftanzizadalteram proportio, nunquam 4 toto orbe Copetnica-
no aberrar, fed femper ad aliquid d.gitum intendit, quod pertinet ad or-
bium (pifficudinem. Eitq; in hoc, quod mirari poffis aliqua etiam xqua-
litas. Quam vt videas, explico ubi ordinemdiftantiarum motoriarum
in partibus, quarum media Tcllurisremotio eft 1000. & appono diftan-
tias Copetnicanas:

L (9) Coper-

a

Chapter XXI

Secondly, (2) since the solids disagree with the motions, an objection which is
truly made against me, | am certainly forced to admit that one or the other is sub-
ject to error. Nevertheless I think the error can be explained in such a way (3) that it
is not necessary to relinquish either discovery altogether (about the ratio of either
the motions or the spheres). However it is easy to conjecture from the foregoing.
which of the discoveries is at fault. First, the distances according to the motions are
further away from the Copernican distances than those according to the figures.
Secondly, if you compare the distances according to the motions with the Coper-
nican distances individually, and make a list of the discrepancies, you will see that
the discrepancies are related to the actual values, and therefore to the solids, except
in the case of Mercury. Take note:

Copernican Distances
distances from motion Differences

Saturn Jupiter 52 374 +2 Cube
Jupiter Mars 290 274 -16 Tetrahedron
Mars Earth 658 694 +36 Dodecahedron
Earth Venus n9 762 +43 Ieosahedron
Venus Mercury 500 563 +63 Octahedron

or


Plainly the difference is positive in four cases, negative in the fifth. For among
the four, all the pairs of bodies are alike; the fifth is on its own. Next bring Mercury
back into the pattern, as it varies from it, and consider that some height greater
than halfway through the thickness of the sphere ought to be taken instead of the
mean distance, (4) that is, the radius of the inscribed sphere of the octahedron
(which as you have heard above extends beyond halfway through the thickness),
and it will achieve for its mean distance 559, not 500. Therefore the pattern for its
values will be:

Venus Mercury 559 563 +4

Notice that for Saturn and Jupiter, and for Venus and Mercury, the differences
are smaller, that is 2 and 4; for Mars and the Earth, and the Earth and Venus, they
are larger, that is 36 and 43. Similarly the solids interposed, the cube and oc-
tahedron in the former cases, the dodecahedron and icosahedron in the latter, are
alike. And observe, that in the former, where the difference between the inscribed
and circumscribed spheres is great, the difference in the distances is small; but on
the other hand where the related spheres are almost equal, the distances according
to the motions differ by a wide margin from the Copernican ones.

Since, then, there is a certain regularity in this deficiency, and indeed no pattern
occurs accidentally, we must therefore consider that these values hint at the truth,
but have not yet completely achieved it. (5) That is to say, there is something in the
theorem itself which can still be improved, or else the actual theorem is correct, (6)
but neither procedure has carried through its intention. Although I could suspect
that at the start, yet I did not want the reader to be without this opportunity, and
so to speak stimulus for making further attempts. (7) For what if at some future
time we should see the day when both these discoveries are reconciled? (8) What if
the rationale of the eccentricities can be deduced from it? For among the reasons
which make me cling more tenaciously to this theorem about the motions is the
fact that the ratio of one distance, according to the motions, to another never
strays outside the complete Copernican sphere, but always points to something
which relates to the thickness of the spheres. In this fact, at which you can wonder,
there is also a regularity. To show it to you, I set out for youa table of the distances
according to the motions, in units in which the mean distance of the Earth is 1000,
and I append the Copernican distances:

82 foan.. Keprerr
(9) Copernici Motorie

Chapter XXI

Summa 9987

Greatest distance

(9) Copernican From motions

Media 4 9154 9163
Ima 8341 ye 1000 ad 577
fic 9163 ads290
Summa 5492 proximus 5261
Media % 5246 5261
Ima joooa vt 1000 ad 333
ficasooo ad 1666
Summa 1648b proximus1648b
Media o 1520 1440
Ima 1393¢ yt 1000 ad 795
fice 1393 ad 1107
Sum.terre1042 terre 02d proximus = tlo2
Med.fim- 1000 cum 1000 1000 d
Ina _plicis.858 ¢ D 858 vt 1000 ad 795
ficeos8 ad 762
Summa 74th proximus 762 f
Media $ 719 762£
Ima 696 yt 1000 ad 577
fic 741 ad 4298
Summa 489 proximus 74rh
Media ¥ 360 429g
Ima 231

Equalitashxc eft,quod in remotis 4 terraad medias diftantias pro-
ximeacceditur:in vicinis Marte & Venere, motoria diftantia verings vi-
cinior eft terrz,quam Copernicana media.

Vides ctia nufquam,ncc excludiloco fue corpus,neq; ordiné turba-
1i,{edad minimi,hiatum tanciiinter medias diftatias patére, qui corpus
recipiat. Vt fi quis maxime mototias hafce pro optime deméftratis acce-
ptare velit(quo de dubitacur tamé)is (10) moda fortaflis interpofitionis
corpori tollat, interpofitionéipfam no collat. Fere n.indicantmotoriz,
quafi (11) duocxteriora fimilia fimiliter inter medias interfint, duo inte-
nora fimiliainter media &extrema,népe dodecaedron ab ima Martis ad
media Terre,Icofaedron a media Terrzad fumma Veneris. Tetraedron.
vero ét fuis fruacur priuilegijs,atq; inter veraq; extremAinterfic. Verihee
omnia {uo loco céfeantur népecxincertis extru€tanumeris motoriari,
necin alii fing,qua ve extimulétur alij ad c6ciliationé:ad qua via praiui.

In Cap. XXI. Notz Auctoris.

() (CY Videx defedtucolligendum.} superuatua iam porroelt bec coniedtatio. Veraenim
roportione inuenta in qu defeelusplane nulus,quid midi opuseSfalfedefettu?

(2) Cum corpora diffentiant’ motibus.] Quianec corpora fu figure,foleformant

interuala Planctarit nce motutun talisin individuo eff proportio.Ita vtrumng, in ervore verfabatur.

(3) Vencurrum inuentum penitusrelinquere cogamur.] Conciliatafuntinter
felibro 5.Harmonicorum, ; : ;

(4) Tantumnempe, quantus eft orbis OGtacdri.] Pofito orbe peribelio oka

i


Mean distance of Saturn 9164 9163

Least distance 8341 1000: $77
as 9163: 5290

Greatest distance 5492 closest 5261

Mean distance of Jupiter 5246 5261

Least distance 50000 1000: 333
asa 5000 : 1666

Greatest distance 16485 closest 16480

Mean distance of Mars 1520 1440

Least distance 1393¢ 1000: 795
asc 1393: 1107

Greatest 1042 of Earth 1102¢ closest 11020

Mean distance 1000 with 1000 1000

Least distance 958e Moon 898 1000: 795
ase 958: 762

Greatest distance 74th closest 762f

Mean distance of Venus 19 62h

Least distance 696 1000: 577
as 741: 429g

Greatest distance 439 closest. 741A

Mean distance of Mercury 360 429g

Least distance 21

The regularity is this, that in the cases of those far from the Earth the values are
very close to the mean distances: in the cases of its neighbors Mars and Venus, the
distance according to the motions is for both planets closer to the Earth than the
Copernican mean distance.

Also you see that never is a solid shut out of its position, or the arrangement
disturbed, but at the least, a large enough gap is open between the mean distances to
accept the solid. So that anyone who is willing to accept these values according to
the motions particularly as very well established (about which, however, there is
doubt), (10) may perhaps discard the method of interposing the solids, but not the
interpolation itself. For that is almost implied by the distances according to the mo-
tions, as if (11) the two outer similar solids were interposed in a similar way between
the mean distances, the two inner similar solids between the mean and the extreme
distance, in other words, the dodecahedron from the least distance of Mars to the
mean distance of the Earth, the icosahedron from the mean distance of the Earth to
the greatest distance of Venus. The tetrahedron indeed enjoys its own privileges,
and is interposed between two extreme distances.? Yet all these points should be
assessed at their proper value, namely as having been based on the inaccurate values
of the distances derived from the motions, and for the sole purpose of stimulating
others to reconcile them. On that road I have been the forerunner.


(1) What is to be inferred from the deficiency.| This conjecture is from now on completely pointless.
For as I have found the true proportion, in which there is plainly no deficiency, what need have I for a
false deficiency?

(2) Since the solids disagree with the motions. Because neither do the solids or the figures alone
regulate the intervals between the planets, nor is there any such proportion between the motions depend-
ing on the individual case. Thus both were erroneous.

(3) That it is not necessary t0 relinquish either discovery altogether.) They have been reconciled with
each other in Book V of the Harmonice.

(4) That is, the radius of the inscribed sphere of the octahedron.| If the distance of the perihelion of
the orbit of Venus, in which the octahedron is inscribed, is taken as 1000 units, the distance of the centers

MysterrvM CosmMoGRAPHICyM. 83

ui Otedron inferibatur, partium 10.0 0.centra O@aedri diflabunt x centro fyflematis partibus
$59. Mercury fiamana diftantia ex Copernsco promatur 7 23.media 500. itaque punctum , vb_
serminantur partes 555. in pfofpacio, feu fpifitudine orbis at nun in medio, fed inter medium
oo. fuminum 723.

(5) Nempeinipfocheoremate.[ Hocnimirum limandum erat , Proportionem alte-
ram fe alterins non duplain,fed fefquialteram.

(6) Sedeius{enfum neutra.] Veclarum feci priori cap.in annotationibus.

(7) Quid fi namquealiquando diemillum videamus.] Vidinus pat 22.annos,
S gaifi fmus, faltemego puto G> Maflinus,c> plarimi alii qui lib.s. Harmon funt lecturi parti~
cipeserunt gaudin,

(8) Quid fihincratio Eccentricisarum.] Ita fomniabam de veritate, opinor bono
Deoinfpirante.Elicitacst,non hinc quidem fedex Harmoniis,ratio Escenticitatum, fed tamen me~
diante boc inuento,nec illud ante fieripotuit quam hoc cmndatuan haberetur. Namlib.s- Harmon,
Cap.3. ponitur inter prisicipia demonftrationis hec fefquialteraproportio,

(9) Copernici famma &c.] Pro hisvon perfedtisinteruallisex Copernico babes Harm.
Lib.5.perfectifinna ex Aftronomiaper Obferuationes Braheanas reflaurata,

(10) Modum fortaffisinterpofitionis corporum tollar. ] Rurfion fommiabam de
veritate.Vide cmendatium modum hun lib.s.Hutrin.cap.9.Prop.46.47.48-49+

(11) Duo exteriora fimilia fimiliter.] Cubus exteriorum e O@taedron interiorum
sltime,imiliver id ef, penctratiue interfint , at noninvermcdias diftantias nimium hoc. Duo
verointeriora , Dodecacdron C> Tcofaedron, fimnilia,rurfum (imiliter,id ef defedtine, at non inter
extremam & median, rurfusn boc nonin ef : Tetraedron vero omnino {ao truitar etiam hie
privilegio intereftqueincerextremas diftantias: imam louis fianmamn Martis, Hoc fic effede-
bere,desnonftraui propofitionibus iam allegatis.

Catere errantistin nusnerorum ad veritatem allufiones,quaspafimn allego, fortuite funt,nee
digne,qua excutiantur;iucunda tainen mibirecogaitusquia monent quibusme.udris, quorum pas
rietuin patlpatione,per tcncbras ignorantie ad pellucens otiuns veritatis deuenerim,


Planeta cur fuper equantis centro equaliter moueatur.

ax Iv 1c18 1 modo,Leétor,etiam imperfecta cognofce- .
3) re , quo minus metuo, tcvltimam hanc & frigidam ca-
taftrophen explofarum. Vitimo autem referre volui,cat
quia vitimo locohabeostum quia cum motibuscohgret,
necexpediri fine XX. capite poteft,quamuis ad 14. pro-
pric pertincat, vtiibi monitus es.

Cum hanc figuralem caclerum proportionem Matftlini cenfura Vide Tab.
fabicciffem:is me de fuperiorum cpieycljstmonuit,quosCopernicuslo-| V.Cap.

coxquantium introduxit,quiquc duplo maiorem efficiant otbi fpifficu- xIy.
dinem,quam Planctwafcenfas defcenfufque requirit. Etinintcrioribus
quidem alij motus func, quibus Planetaad omnem illius epicyclialtitu-
dinem euehitur,adomnem cius humilitatem defcendit,vndein illis pro
eccentrepicyclo eccentrus eccentri a Copernico affumptus eft : in
Mercurio vero peculiaris quedam diameter , per quam accedit &re-
cedit a Sole , fimiliter longe remotius a Sole interdum exporrigitur,
quam Stellavnquam. Exiftimauitigitur , eam orbibus relinquendam
efi fpiflitudinem , qua motibus demonftrandis fufficiat. Cui re-
2 fpondi,

Chapter XXII

of the octahedron from the center of the system will be $59 units, while the greatest distance of Mercury
according to Copernicus comes out at 723 units, the mean distance at 500. Thus the point to which $59
Units extend is within the space, or thickness, of the orbit; yet not at the mean distance, but in between
the mean distance of $00 units and the greatest, 723 units.

(8) That is to say ... in the theorem itself.] Obviously the improvement which should have been made
was that the ratio of the one to the other is not the square, but the 3/2th power.

(6) But neither . .. its intention.] As I have made clear in the notes on the previous chapter,

(7) What if at some future time we should see the day.| We have seen it, 22 years later, and we have re-
joiced; at least I have, and I believe both Maestlin and a great many others, who are going to read Book V
of the Harmonice, will share my joy.

(8) What if the rationale of the eccentricities (can be deduced) from it?| In this way I dreamed of the
truth, I think inspired by the good Lord, The rationale of the eccentricities was deduced, not indeed from
this, but from the harmonies, yet with this discovery as the intermedium; and the deduction could not
have been made before the discovery had been amended. For in Book V of the Harmonice, Chapter 3,
this 3/2th power of the ratio is taken among the bases of the demonstration,

(9) Copernican greatest (distance), etc.) Instead of these imprecise intervals taken from Copernicus
you can find them in the Harmonice, Book V, taken absolutely precise from astronomy restored by
means of the Brahean Observations.*

(10) May perhaps discard the method of interposing the solids.| Again | was dreaming of the truth.
See this method emended in Book V of the Harmonice, Chapter 9, Propositions 46, 47, 48, and 49.

(UD The two outer similar solids (were interposed) in a similar way. The cube is interposed within the
last of the outer, and the octahedron within the last of the inner, in a similar way, that is, so that they
penetrate, but not ber ween the mean distances; that is too much. On the other hand the two inner similar
solids, the dodecahedron and icosahedron, are again interposed in «similar way, that is, so that they fall
short, but not between an extreme and the mean: again that is too much. The tetrahedron, however, here
also absolutely enjoys its own privilege, and is interposed between extreme distances, the least of Jupiter,
the ercatest of Mars. 1 have shown that this must be so in the propositions already quoted.

The remaining hints at the truth which are offered by erroneous values, and which I quote everywhere,
are fortuitous, but do not deserve to be deleted; yet I enjoy recognizing them, because they tell me by
what meanders, and by feeling along what walls through the darkness of ignorance, I have reached the
shining gateway of truth,


You have just learnt, reader, to take cognizance even of imperfect ideas, and so I
am less afraid that you will jeer off the stage this last feeble dénouement.
However, I wanted to keep it till last, both because I discovered it last in order,
and also because it is connected with the motions and cannot be expounded
without Chapter 20, though it properly belongs to 14, as you were informed
there.

As I had submitted this proportion of the heavens based on the figures to
Maestlin’s criticism, he mentioned to me the epicycles of the superior planets,
which Copernicus introduced in place of equants, and which make the thickness
of the sphere twice as great as the upward and downward movement of the planet
requires. And in the case of the inferior planets indeed, there are other motions,
by which the planet is lifted up to the full height of its epicycle, and goes down to
its lowest level; so in their cases instead of an epicycle on an eccentric, an eccentric
on an eccentric was postulated by Copernicus. In fact in the case of Mercury a
certain special diameter, along which it moves towards and away from the Sun, is
similarly extended to a much greater distance from the Sun than the planet ever is.
He therefore considered that a thickness sufficient for deriving the motions
should be left for the spheres. To that I have replied, first, that the whole under-
taking should be abandoned, if the spheres were made twice as fat, for the equa-


Plate IV,
Chap. 14
84 Ioan Keprerr
refpondi,primum,deferendum efletotum negotium,fiduplo craffiores
fiantorbes:nam nimid meFa¢eupéczew ademptumiri : Deinde nihil de-
cederenobilitati miraculofz huius machinationis,fi modo viz ipfz;pla-
netarum defcriptx globulis,rctincant hanc proportionein; quibufcun-
queilliagitencur orbib.magnis an paruis. Ecaddidi, qnx cap.16.habes,
de materia figurarum,que nulla fit;atqsinde non abfardum cffe,corpo-
racum orbibus codem loco includere. Imo vero vel fine orbibus hac viz
inaqualitatem defendipoffe.In qua fententia video Nobilem & excel-
lenuff: Mathematicum Tychonem Brahe, Danum,verfari. Caufamta-
men & modumhzc noftradifertiusindicant. (1) Nempeficadem fic
caufa tarditatis & velocitatis in fingulorum orbibus, qu fupra cap.20.
fuitin vniuerfomundo,hoc modo : Via Planet eccentrica,tarda fupe-
rius c(t,infcrius velox. Adhoc enim demonftranda affumpta (2) Coper-
nicoepicyclia,Ptolemio xquantes. Defcribatur igitur concentricus x-
qualis via Planetaria eccentricx; cuius motus yndiquaque aqualis erit,
quia xqualiter ab origine motus diltac. Ergo in medietate vie eccétrice
fupra concentricéi eminent tardior crit Planeta, quialongius a Sole re-
cedit,& a virtue debiliori mouctur:inrcliquacelerior,quia Soli vicini-
or, & in fortiori virtute. Atg; hanc variationcm motus non fecus per cir-
ccllum demonftrari , ac fiverein co circello Planctamouerctur xquali
motu,cuiibct facile eft colligere.Habes caufam tarditatis huius,videa-
mus niic & men{uram:a fit fons animz mo-
uentis,{c.Sol.a centrum viz ErGH, qui Pla-
neta,{edinzquali paflu,incedit,sp fitveBA,
& cp ciusdimidium. Ciigicurer ficremo-
tior ab a,quam No quantitate aB : coucnie-
bac ve Planctain eF tamtarduseffet,acfidu-
plolongiusab a receffiffer,quatitate fc. av,
& fuper centro Dcurrerct. Etecontra,cum
HG fitproptoripfia quamp Q, eademaB
quantitate,conucniebar,vtPlanctain cHta
velox effet,acfiduplo propiusad a acceffif-
fenimirum itidem quantitate aD.Vtrobigs
ergotantundem cft,acfi fuper p centroincederet. (3) Supra enim cap.
zo.camotuumadorbes fuit proportio. Quarecogita, qui bilociduz
caule pertotum circulumconcurrerunt, eas hicinuerlas & permixtas
effe. Mlic orbis ciufdem integerambitus maior & remotior period au-
xit,& minor ati propior diminuit: Hic autem circuli Nor Q&EFGHE-
quales funt,& huius pars altcrareimotior, altcrapropioreft cétro 4 So-
hi. Quapropter motrix virtusin 4 agicin EF, & incu >tanquam planeta
illic cffetin 1x hicin LM, Veriufque autem, tarditatisillius, & velocita-
tis huius communis menfura inuenitur in p. Teaque Planctain EF GH via
progredicns,tardus veloxque,nec non mediocris circa R & $ fitsperinde
tanquam in 1KLM,fuper p centri aqualiter iret. 14 vide artifices,qui pe-
nitus idé ftatuerunt. Népe Ptolemzus p centr zquantis, & B centrum
viz planetariz fecit. Copetn.vero circa c centrum,medium inter p & B,
eccentrum eccentri vel eccétrepicyclum circumducie. Ei ergo fit,vt via
planctafic qua proxime EF cH,fed motus xqualitas, ficut ipfius orbis in-
termedijinter EFGH 81K LM circa c,ita plancta circa p,reguletur.

(4) Cane


Chapter XXII

tion to be subtracted would be too large; second, there is no detraction from the
miraculous nobility of this mechanism if only the paths themselves traced out by
the planets’ globes retain this proportion, by whatever spheres they are impelled,
large or small. And I have added what you have found in Chapter 16, on the
material of the figures, which is non-existent; and hence that it is not absurd to in-
clude the solids in the same location as the spheres, or rather that even without
the spheres this irregularity in their path can be defended.

I see that the noble and excellent Danish mathematician Tycho Brahe is of the
same opinion. However, the reason and the means are shown more clearly by
these writings of our own, (1) that is, if the cause of the retardation and accelera-
tion is the same for the spheres of the individual planets as it was above in
Chapter 20 for the whole universe, in the following manner. The path of the
planet is eccentric, and it is slower when it is further out, and swift when it is fur-
ther in. For it was to explain this that (2) Copernicus postulated epicycles,
Ptolemy equants. Then describe a concentric circle equal to the eccentric path of
the planet, of which the motion will be equal at all points, since it is equally dis-
tant from the source of the motion. Therefore at the middle part of the eccentric
path where it projects above the concentric circle, the planet will be slower,
because it moves further away from the Sun and is moved by a weaker power; and
in the remaining part it will be faster, because it is closer to the Sun and subject to
a stronger power. And it is easy for anyone to understand that this variation in
motion can be explained by a little circle exactly as if the planet were really mov-
ing on that circle with a uniform motion. You know the cause of this slowness; let
us now look at the measure of it as well. Let A be the source of this moving spirit,
namely the Sun; let B be the center of the path EFGH, on which the planet
moves, but at an irregular pace; and let BD be as BA, and CB half of it. Then
since EF is further away from A than NO by the extent of AB, it turns out that
the planet on EF would be slow by the same amount as if it had moved twice as
far from A, that is, by the extent of AD, and its course was about D as center.
And on the other hand, since HG is nearer to A than PQ, again by the extent of
AB, it turns out that the planet on GH would be fast by the same amount as if it
had moved twice as near to A, naturally by the extent of AD once more. Then in
both cases it is just the same as if the planet were moving about D as center. (3)
For in Chapter 20 above, that was the ratio of the motions to the orbits. Con-
sider, therefore, that the two causes which in the former position act in combina-
tion all round the circle, are in the latter position inverted and have different ef-
fects.? In the former case the complete circuit of a given planet had an increased
period when it was larger and further away, and a diminished period when it was
smaller and nearer: in the latter case the circles NOPQ and EFGH are equal, and
one part of the latter is further away from the center, A, the Sun, and the other
part is nearer to it. Consequently the motive power at A acts along EF, and along
GH, as if the planets were on IK in the former case, on LM in the latter. However
the common measure for each case, of the retardation in the former and of the
acceleration in the latter, is found at D. Thus a planet passing along the path
EFGH becomes slow and fast, and also of medium speed in the region of R and S,
exactly as if it were moving uniformly on IKLM, about D as center.* Now notice
the practitioners, who came to exactly the same conclusion. Ptolemy of course
made D the center of the equant, and B the center of the planetary path. Coper-
nicus, to be sure, constructed an eccentric on an eccentric, or an epicycle on an ec-
centric about C as center, halfway between D and B. For him therefore the path
of the planet is as nearly as possible EFGH, but the uniformity of the motion of
the planet is referred to D as center, just as that of the orbit intermediate between
EFGH and IKLM isto Cas center.

Mysterivm CosmoGrapHicya, 85

(4) Caufam habes, cur equantis centrum parte tertia eccenttici-
tatistotius 4 centro eccentricidiftct. (5) Nempe mundus totus anima
plenus cito, quz rapiar, quicquid adipifcitur ftellarum fiuecometarum,
idqueca peinicitate, quam requirit locia Sole diftantia & ibi fortuitudo
vircutis. Deindeelto in quolibet Plancta peculiarisanima,cuius remigio
ftellaafcendatin fuo ambicu:Et orbibus remotis eadem fequentur,

Atquchxcde Aquate,vbi legerinealiqui,fciogeftient. Namfimi-
santur Aftronomi Prolemaum indemonftratam fampfifl/hanc eadem
menfuram centri Aquantis:multo magis iam mirabunturquidam,fuif-
fecaufam huiusrci,nequetamen deca Prolemxo fuboluiffe, cum ipfam
rem ita,vti habcr,fumerets& quafidiuino nucu cecus ad locum debitum
peruenircr.

Sedtamen cosadmonitos velim, nihil effe ex omni parte beatum.
(6)Namin Venere & Mercurio ifta tarditas & velocitas non ad plane-
tea Soledigreffionem, fed ad folum Terre motumaccommodatur, Et
fi quis huic rei pratexat diuerfam motus conditionem a motu fuperiord:
quam deniquc in (7) Terra annuo motu caufam alferer? Is cnimneque
apud Prolemaum,neque apud Copernicum AEquante indiguit. Qua-
re & hecincertalis fub Aftronomo iudice pendeat.

In Cap. XXII. Nota Auétoris.

(YR TEmne ficadem fir eaufa.] Sique caufeeffcit,ve Saturnns altus fit tardior Touebumi
LS lirics file vicimiors adem effciatyvt Saturnusaltus Cr apogens, fc tardior fipfo perigeo
Cehuimals, Cer Geverinfquerc:ell slongatio Planete d Sole edtlinca, maior vel minor quis longe
diflans a Sole verjatur in virtute Solari tenniore é imbecilliore, .
(2) Copernico Ep:cyclia,Prolemao.equantes.] Quam aquipollentiam hypothe-
fi

dacit in Comment. Martispart.t. ‘ _

(3) Supracnim Cap.XX.camotuumad Orbes.] Hot verain annotationibuse~
amend atanius,Non dupla erat poiodorunn,c fic tarditatumproporto ad proportionem orbium (ed
fofgnialeerafaltem, Atin Plancta vnins motibusexfaleapparentibus, Apbeli & pevibilio regnat
proportiodiftantiarsan; pracifedupla,in motibus pfisdiurnis, vtfiunt arcus eccentricorum, proporti
apfiginadijlantiarum fimpla,vide Comment. Martis part.; Gr 4. Canfam dinerficatisenidentifii
mam babct lib. 4 Epit. Affron.fol.s33. ; :

(4) Caufam habes,curequantis centrum parte tertia.] Hoc de Copernico verum
eBteuti C centrum edt equantis fet poriuseccentrieccentrici, B centrum vie Plancte, & ipfius AC
parstertia BC. Atin Prolemeoratioe alia. senim D eff centramaquantit, B Eccentrici,quare
ipfius AD femigu et BD. ' oo

(5) Nempe mundus totusanimaplenus.] Rurfum proanima intellige Solisfpeciem
immateriatam , extenfamvt lumen: & halcbishicbresibus verbisfummam mee phyficecceletis,
traditamin Comment. Martis part.3. 4.@ repetitam lib.g.Epit.Aftron, .

(6) Namin Vencre& Mercurio.] Nibilopusexceptione:verovverius eftetiam de ¥e-
nore & Mercurio. Nam quod Copernicus aliquashorum Planctarum inequtalitates alligat ad me-
tum orbis annui,id deerroree®. a ;

(7) Terre annuus motus xquantenonindiguit.} Apud Prolemeum quidem oo
Copernicions, Atego in Comment. Martie, pracipuoruns libri membrorum hoc pum fect, & velut
aangularcm lapidem in funnd.mento pofuisiniochauem Aftronomie mscrito app laizqued ex ipfis mo~
tibus Martisliquida demonftrani, feu Solis feu Terrs motum annuum regulari circa alienum centria
equanticifqie eccentricatatcm orbite,dimnidinm folum babere , Eccentricitatis a aultoribuscre=
dite,

idesitaque, Led dio, libelle hoe femina fpanfaclfeomaninon & fingulorum, queex eo
Videsitag or fludiofe, i femine fparfa eff LA a pee


Chapter XXII

(4) You now know the reason why the distance of the center of the equant from
the center of the eccentric is one-third of the whole eccentricity.4 (5) Then,
naturally, let the whole universe be full of a spirit which whirls along any stars or
comets it reaches, and that with the speed which is required by the distance from
the Sun of their positions and the strength of its power there. Next let there be in
each planet a peculiar spirit, by the impulsion of which the star goes up in its cir-
cuit; and even without the spheres the same results will follow.

Anyone who reads this passage on the equant will, | know, rejoice. For if the
astronomers are surprised that Ptolemy assumed this same measure of the center
of the equant without proof, some people will now be all the more surprised that
there was an explanation for it, and Ptolemy did not suspect it, since he assumed
the fact to be as it is, and as if by divine guidance’ arrived blind at the proper
destination.

Yet I should like them to take warning that nothing is pleasing in every way. (6)
For in the cases of Venus and Mercury this slowness and quickness fits in, not
with the planet’s distance from the Sun, but only with the Earth’s motion. And if
anyone elaborates this question with a law of their motion different from that of
the superior planets, what explanation will he eventually put forward for (7) the
annual motion of the Earth? For it did not need an equant either in Ptolemy’s
theory or in Copernicus’s. Consequently, this is also a doubtful case awaiting the
judgment of astronomy.


(1) That is, if the cause. . is the same.] If any cause has the effect that Saturn when itis high is slower
than Jupiter when it is lower and closer to the Sun, the same cause would have the effect that Saturn
when itis high and at apogee would be slower than it is itself at perigee and low. The cause of both effects
is the greater or smaller distance of the planet from the Sun in a straight line, because when it is far dis-
tant from the Sun the power of the Sun which it experiences is thinner and weaker.*

2) Copernicus (postulated) epicycles, Ptolemy equants.] 1 have explained this equivalence of the
hypotheses in my Commentaries on Mars, Part 1

(3) For in Chapter 20 above that (was the ratio) of the motions of the orbits.] However we have
emended it in the notes, The ratio of the periods, and so of the stownesses, was not the square of the ratio
of the orbits, but in fact the 3/2th power. But in the apparent motions of a single planet as seen from the
Sun, at aphelion and perihelion, the reigning ratio is precisely the square of that of the distances: in the
daily motions themselves, as they are arcs of the eccentrics, the actual ratio of the distances is simple. See
the Commentaries on Mars, Parts Ill and IV. The quite obvious cause of the difference is found in Book
IV of the Epitome of Astronomy, page 533.

(4) You now know the reason why (the distance of) the center of the equant . .. is one third.) This is
true for Copernicus, for whom C is the center of the equant, or rather of the eccentric on an eccentric, B
the center of the path of the planet, and BC one-third of AC. But in Ptolemy the reasoning is different.
For to him D is the center of the equant, B of the eccentric, so that BD is half AD.

(3) Then, naturally, (let) the whole universe (be) full of a spirit.| Again, instead of “spirit” understand
the immaterial emanation of the Sun, spreading out like light: and you will have here in a few words a
summary of my celestial physics, expounded in my Commentaries on Mars, Parts II and IV, and
repeated in Book IV of the Epitome of Astronomy.

(6) For in the cases of Venus and Mercury.} There is no need of the exception: indeed it is even more
true of Venus and Mercury. For in Copernicus’s linking of certain irregularities of the planets to the mo-
tion of the annual orbit is where the error lies.

(7) The annual motion of the Earth... did not need an equant.] In Ptolemy's theory indeed and in
Copernicus’s. But in my Commentaries on Mars | have made this one of the chief features of the book,
and have laid it like a cornerstone at the foundation. Indeed I deservedly called the key to astronomy the
fact, which I have demonstrated clearly from the actual motions of Mars, that the annual motion either
of the Sun or of the Earth is controlled by a different center from the equant, and that the eccentricity of
its orbit is only half the eccentricity believed by the authorities.”

You see, then, assiduous reader, that in this book there were scattered the seeds of each and every one

86 Toan. Keprerr

temporein Aftronomia nova > valgo abfurda ex certifimis Brabei obferuationibus& me conffitura
G demonftrata funtiitag, iro teiocuan muni ib. 4. Harm.de meis Imaginibus ex Procli Paradi=
_gunatibvs dclapfs non inuqua cenfure flage.ararum,

De initio ¢> fine Mundi Aftronomico € anno Platontco.

Sa Os T cpulas, poft faftidium ex facuritate, veniamusad.
PE) ESE 3 | bellaria. Problemata duo pononobilia. Primumeftde
: SAS , principio motussalterum define. (1) Certenonteme-
s@A| re Deus infticuirmotus, fed ab vno quodam certo prin-
% cipio & illuftri ftellarum coniunétione, & ininitio Zo-

paesh..2:%1")} diaci,quod creator per inclinationem Telluris domici-
]ij noftri effinxit,quia ommia propter hominé. (2) Annusigitur Chriftt
1595. fireferaturin 5572. mundi( quicommuniter & a probatiflimis 5557-
cenfetur) venict creatio in illuftrem conftellationemin principio V-
Nam anno primo affumptinumeri, die A prilis 27. Iuliano retro compu-
tato, feria prima,qui dics Creationisomnium eft , horavndecimameri-
dici Borufliz,que eft fexta vefpertina in India,talis exhibetur cecli facies
a Prurenicocalculo.




Motus & ¢& & paulifper morare,aut promoue,& venientin loca

cognata, & forte inowr-ad D. Scaligermale Nouiluniumvule. Nam
Luna in poteftatem nottis condita,noéte vtiq: prima fulfic. Verifimilius
initium calculus multis retro porroque annis non fuppeditat. (3) Sed
firationcs fequamur , oportct hoc initium, © in verfante, quarere,
nempe hac ceeli facie.

Ssaekuesehs

PP hp passes

Vulchocyeterum auétoritas, Mundumin Autumno creatum, 8¢
ratio ipfaex Copernico,vt Tellus fab codem initio ftet,quoreliqui.Ap-
parebunt igitur fuperiores in V’, inferiores & @in =, Luna cumcirca
terram fit neque in V, neque in -& competit , ne turbet numerum

terna-

Chapter XXIII

of the things which since that time in this new and, to the masses, absurd astronomy I have established.
and demonstrated from the thoroughly exact observations of Brahe; and I therefore hope that you will
not lash with an unfair judgment my joke in Book IV of the Harmonicet about my Images, which the
Paradigms of Proclus let fall


After the feasting, after the weariness of repletion, let us come to the dessert. I pose
two noble problems. The first concerns the start of motion, the other its end. (1)
Certainly God did not start the motions at random, but from some single definite
starting point, some illustrious conjunction of stars, and at the beginning of the
zodiac, which the Creator formed according to the inclination of the Earth, our
dwelling, since everything is for the sake of Man. (2) Then if the year of Christ 1595
is taken as the year 5572 of the universe (though it is generally and by very sound
authority reckoned as 5557), the creation will come to an illustrious combination of
stars at the start of Aries. For, if we take that number of years, in the first year, on
the 27th April by the Julian calendar, counted backwards, on the first day of the
week, which is the day of the creation of all things, at the eleventh hour at the Prus-
sian meridian, which is the sixth hour of the evening in India, this is the appearance
of the heaven calculated according to the Prutenic Tables.
Sun 3° Aries
Moon 3° Libra
Saturn 15° Aries
Jupiter 10° Aries
Mars 24° Gemini
Venus 10° Taurus
Mercury 3° Aries
Ascending node 18° Virgo
Delay the motions of Mars, Venus, and the ascending node a little, or hasten
them, and they will come to associated positions, and perhaps the ascending node
will be in 0° of Libra with the Moon. Scaliger prefers the New Moon, wrongly. For
the Moon was created to have power in the night, and undoubtedly shone in the
first night. Calculation for many years backwards and forwards does not afford a
more likely beginning. (3) But if we follow reason, we should look for a beginning,
with the Sun in Libra, that is with the following appearance of the heaven.
Saturn 0° Aries
Jupiter 0° Aries
Mars 0° Aries
Descending node 0° Aries
Moon 0° Capricorn
Ascending node 0° Libra
Venus 0° Libra
Mercury 0° Libra
Sun 0° Libra
The conclusion of the ancient authorities is that the universe was created in the
autumn, and the inference from Copernicus is that the Earth is located at the same
starting point as the other planets. Therefore the superior planets will appear in
Aries, the inferior planets and the Sun in Libra, and the Moon, since it is a satellite
of the Earth, does not belong either in Aries or in Libra, in case it should upset the
threefold number of the superior and inferior planets. When the Sun is setting (for

88 loan. Keprerr Myst. Cosmocer.

Lupitcr, Fellus,verfus Capricornumn, Luna, Mars, Venus verfus Cancrit.1n Mercurio abiidant gradus
aliquot fed qui conficanspoffint cius aqiatione maxima ablative, fi modo fatis cognituse8t eins me-
tus medius, vtiuon per hasus conredtionem confunantur, InVenercetiam abundat aliquid , quod e-
quatione toi non potest Feria fecundacit Firmamenti, fenexpanfionisinter aquas Cr aquas, quafi
Orbes feu Planeta , per hancexpanfionem ire infi,flatinn in ipfo ortu expanfi,ceperintires feriavero
quarta demum exornatum caluinextimum fixie, G Sol, & Luna,co¢.vltima manu impofita,

(5) Finem motui nullum cum ratione ftatui.] Dagan innitebatur buie vtprima-
Yiofundainento :quod mster Orbes cacleftes fit proportio, ila que et Orbinm Geometricorumn cuinfti-
betsex quingue figuris.tllarum enim quattuor.proportiones funt inefabiles, fou vt hic cum vulgo ap-
pellata rationales, Lane vero fundamentun hoc rcfutanims: quia proportio calftiun erbium nom
itex foluquingue figuris, Quaritur,quad iam porro de boc dogmatetenendum, & nun detur alt=
qiiaperfects Apocatt.xftufis motunm omnium? Dico, quamuis hoc furdamento fubruto, nullamta~
men dart Apocataftafin. Idprobabo.Certum igitur ef, fi proportions faltem periodicorum tempo-
non font offabites dart Sorona(@ sacry:fieneffabilesnondari. Lameffabitesdentur anineffabiles, fic
diiudicandum, Ommesinotunm Apogeorum C Perigaorum proportiones,tam binorum, quam fin-
gcloruin, fant effcbiles; fin enim defuumptaex Harmoniis, Grille fiat onmes effabiles , vt & Con-
‘ania & concinnis ni{erusentia interualla onmia.Ieaquclib.V-Harmonicorum cap 1X pro.X LPI.
Osnnes hi motus {iis numeris exprejie C effati funt s Nemerienim alli prac fi fant intelligendi. lam
vero pervodicovitin temporum inter fe proportioeSteadem quatitate, que elt Cr motusin mediorum,
Motus vero medii participant demedio arithinetico inter extremas ,aphelitim & peribelinen 5 quod
mncdin est inter effabileshostcrmiinos, effsbile: participant cde medio inter eofdems Geometric
At ustereffabiles ter ninos,nonest femper effabile meditn Geometricim, Sunt igitur motusplaneta~
yumi medi ineffabiles,c> incommenfiurabiles motibus extremis Planetarum omnisn,V ide Harmon,
1b.17.Cuap.1X.Prop.XLVU1, Cuan autem 4 priori null.fit ratio , que formet motes medios , fed cum
refilzantfinguliex fis motibusextremis:non erunt medi motus ne inter fe quidem conmmenfurabiles,
allan enim ordinatumyecffabilitasscafiexiflerefolet. Quareneg, periodi temportrs inter fe com=
amenfurabileserwnt, Nulla igueur data perfecta motuum Apacataftafis,qua pro fine motu formali,
fistrationals haberi pofie.

Habes igitur, Lector, examen Libelli met , cui titulus 4 Myfterio Cofinographico,promiffian
ante annosX.in Comm. Martis Part.III.Verum ante Harmonicorsn editionent locus hic examint
non fuit. Quare fine commientationi impofito,conuertamnr ad hyrantan,qui libra claudit,


Tunune,amiceLedor,finem omnium horum ne obliuifcare, qui eft , Cognitio,admira-
tic & veneratio SapientiffimiOpificis.Nihil enim eft ab oculis ad menten,a vifuad contempla.
ticnem,a cusfu afjedtabili ad profundiflimum Creatoris confilium proceffffe fi hie quie(eere
velis;& non vnv impetn,totaque animi deuotione furfum in Creatoris notitiam , amorem cul
tumgueefferare. Quare cafta mente , & grato animo mecuim perfe@iffimi operis atchitecto fe-

Conclusion

over; but they can be disposed of by its maximum subtractive equation, if only its mean motion is adequate-
ly known, so that they are not used up in correcting it. In the case of Venus there is also something over,
which cannot be removed by the equation. The second day of the week is that of the Firmament, and of its
spreading out between the waters and the waters; as if the orbits or planets, given the order to move by this
spreading out, began to move immediately at the origin of this spreading out. The fourth day of the week,
however, is that of the embellishment of the outermost heaven with the fixed stars, and the setting of the
‘Sun and Moon, etc., in it as the finishing touch.

() Thave not established any end to the motion by argument.| The doctrine leant on the following fact
as its chief foundation: that the ratio between the celestial spheres is that between any of the geometrical
spheres derived from the five figures. For four of those ratios are inexpressible, or as I have called them in
the common way, irrational. Now, however, we have refuted this basis, because the ratios of the celestial
spheres are not derived solely from the five figures. The question is, what belief must we hold about this
doctrine from now on, and is some exact return of all the motions to their starting pcint to be found? I say
that although this foundation has collapsed under us, yet no such return isto be found. I shall prove it. It is,
certain, then, that if the ratios of the periodic times at least are expressible, areturn to the starting point isto
bbe found: if they are inexpressible, it is not. Now whether they are expressible or inexpressible must be
decided in the following way. All the ratios of the motions of the apogees and perigees, both in pairs and
singly, are expressible; for they have been taken from the harmonies, and they are all expressible, as all the
intervals are either melodic or subsidiary to melodic intervals.’ Thus in Book V of the Harmonice, Chapter
9, Proposition 48, all these motions have been set out and expressed by their own numbers. For those
numbers are to be understood as precise. Now in fact the ratio of the periodic times to each other is the same
quantity as that of the mean motions. However, the mean motions are formed from the arithmetic mean
between the extremes, the aphelion and the perihelion, and that mean between these expressible terms is ex-
pressible, They are also formed from the geometric mean between the same terms. But the geometric mean
between expressible terms is not always expressible. Therefore the mean motions of the planets are inex-
pressible, and incommensurable with the extreme motions of all the planets. See the Harmonice, Book V,
‘Chapter 9, Proposition 48, However, since a priori there sno proportion which controlsthemean motions,
but they spring individually from their own extreme motions, the mean motions will not be commensur..ble
‘even among themselves; for no regular property, such as expressibility, normally exists by accident.
Therefore no exact return of the motions to their starting point is to be found, which can be taken as an end
to the motions in accordance with form and reason.

You now have, then, reader, the revision of my litle book, entitled The Secret of the Universe, which was
promised ten years ago in Part III of my Commentaries on Mars. However, before the publication of the
Harmonice, there was no room for this revision. Therefore, having made an end of the commentary, let us
turn to the hymn which closes the book.*


Now, friendly reader, do not forget the end of all this, which is the conception, admiration and veneration
of the Most Wise Maker. For it is nothing to have progressed from the eyes to the mind, from sight to con-
templation, from the visible motion to the Creator's most profound plan, if youare willingto rest there, and
do not soar in a single bound and with complete dedication of spirit to knowledge, love, and worship of the
Creator. Therefore with pure mind and thankful spirit sing with me the following hymn to the Architect of
this most perfect work:

Great God. Creator of the Universe,

Great Builder of the Universe, what plea

quentem Hymnum accine.

OVA Sator Mundisnoftriamgue aterna potefias,

Quanta tuaci omnem terrarum fama perorbe?
Glorta quantatuacit:Cali qus diditafupra
Mania concuffisvolas admirabilis lis,
“Aguuicitpuer G fpretofatur-vbere,balbie
Teditante firuit valida argumentalabells
Argumenta,guilustumidus confunditur hoflie
Contemptorg, tui, contemptor iuris Cp aqui
Alt ego, quo eredam fpaciofo Numen inorbe:
Sujpiciam attonitue vafti molimina cali,
Magniopus Artificiesvalide miraculadextras
Quang, vei fidersos normis diftinxeris orbes,
Quosintra medina Lucisg, animeg, Minifler
Qua lege aterni eurjus moderesur habenas,
Quas capiae-variatavices,quos Lunalapores,
Sparjirisimmenfoquamplurima Sidera campo.


Maxine raundi Opifex, quate vatione coegit
Paruus,inops,bumilis, tangy exigue Incolaglebs
Adamides rerum curas agitare fuarum?
Refpicis immeritum,vehis in fublime, Deorure
‘Tantum non genus et,zantos largirishonoresy
Magnificumgcapus cingisdiademate,Regem
Conftituisg; juper manuxm monumentatuarum,
Qued fiupracaput eft magnes cum motibua orbt
Subijs ingento:quicquid Tellure creatur,
Natur operispecnnatgy arisfumantibus apticon
Queg, habitansfiluasreliqguarum feclaferarum,
Queda, gensa, votucres lenibusferit acrapenni,
Quigg maris sradtus tranant & flumina, pifeets
Onae inbes premereimperio , dexirag, potenti.
Joka [ator Mundi,noftriimg, eterna pattfias
Quanta tun ti omanem terrarum faraapererbiat

‘And our eternal power, how great thy fame
In every corner of the whole wide world!

How great thy glory, which flies wondrously
‘Above the far-flung ramparts of the heavens
‘With rushing wings! The babe salutes it, spurning
‘The breast, replete, and with his halting lips
Bears powerful witness —witness which confounds
The haughty enemy, who shows contempt

For thee, and shows contempt for law and justice.
Yet, to believe thy Godhead is within

This spacious sphere, let me look up astonished
At thy achievement of this mighty heaven,

‘The work of the great Craftsman, miracles

Of thy strong hand; see how thou hast marked out
The five-fold pattern of the starry spheres,
Dispensing light and spirit from their midst;

See by what law thou dost control the reins

Of their eternal course; see how the Moon

Varies her path, her toils, how many stars

‘Thy hand has scattered over that boundless field.

Of the poor, humble, small inhabitant

OF this so tiny plot compelled thy care

For his harsh troubles? Yet thou dost look down
On his unworthiness, carry him up

‘On high, a litle lower than the Gods,

Bestow great honors on him, crown his head
Nobly with diadem, appoint him king.

(Over the tokens of thy handiwork

‘Thou makest all that is above his head,

‘The great spheres with their motions, bow before
His genius. All creatures of the Earth,

‘The herds bred for his works, and fitted for

‘The smoking altars, and the generation

(Of wild beasts which remain to dwell in woods,
‘The birds, which with light feathers strike the air,
‘The fish, which swim through rivers and through seas,
(Over all these by thy command he rules

By his dominion and his strong right hand.

Great God, Creator of the Universe,

‘And our eternal power, how great thy fame
In every corner of the whole wide world!

blank page


PLATEI. Showing the order of the moving celestial spheres, and

at the same time the true proportion of their size according to their

mean distances, also the angles of the corrections for the same on
the Earth’s Great Orbit, according to Copernicus’s theory.

In the center or near it is the Sun, motionless.

EF the smallest circle round the Syn is that of Mercury, which returns in about
88 days.

That is followed by CD the circle of Venus, which revolves about the same Sun
in 22474 days.

AB the circle which follows is that of the Earth, which revolves in 365% days.
It is called the Great Orbit because it has many applications.

Round the Earth is the little orbit, like an epicycle, of the Sphere of the Moon,
at A, which returns with the same motion in the space of a year along with the
Earth to the same Fixed Star. But its own revolution referred to the Sun occupies
29% days.

After that is the orbit of Mars, GH, which completes one passage under the
Fixed Stars, or referred to the Sun, in 687 days.

That is enclosed after a large gap by the Sphere of Jupiter, IK, which has a
round of 4332 days plus about %.

LM, the furthest and largest circle, is that of Saturn:
10,759¥% days.

The Fixed Stars, however, are still higher by an interval so immeasurable that in
comparison the gap between the Sun and the Earth is insensible. Also they at the
edge, like the Sun in the center, are completely motionless.

Angle TGV, or arc TV, is the correction, or parallax, of the Earth’s Great Orbit
with respect to the Sphere of Mars.

Similarly PIN is the parallax of the same Great Orbit with respect to the Sphere
of Jupiter, and PLN or RLS, or arc RS, with respect to the Sphere of Saturn.

Also XAY, or arc XY is the parallax of the Sphere of Venus; and in the same
way ZAAB, or ZAB, is the parallax of the Sphere of Mars with respect to the
Great Orbit.

its periodic time is

PLATE II. Showing the order of the celestial spheres, and in
each case the proportion of the orbits and epicycles, and the
angles or arcs of the equations for the same, according to their
mean distances, in accordance with the theory of the ancients.

In the center is the Earth, which alone is motionless.
The innermost small orbit round the Earth represents the Sphere of the Moon,
of which the motion is monthly.
(continued next page)

Annotations to the Plates

(Plate H continued)

The next round that is the orbit of Mercury; it is followed by that of Venus and
after it is the Sphere of the Sun, which all go round in an annual revolution. The
orbits of the other three, the superior planets, Mars, Jupiter, and Saturn, as well
as the Sphere of the Fixed Stars, are indicated by arcs, which anyone can com-
plete by describing the whole of them about the Earth as the center.

The orbit of Mars makes a turn in two years.

That of Jupiter requires twelve years, as nearly as possible, and that of Saturn
about thirty years. The Fixed Stars complete a period in 49,000 years, according
to the tenets of the Alfonsine tables. The amounts of the equations for each of
them (except the Moon) produced by the epicycles on the concentic circle at their
mean distances are shown by the arcs intercepted by straight lines drawn from the
Earth and touching each of the epicycles, that number of degrees being added.

To follow Chapter 1, page 85.

PLATE III. Showing the dimensions of the spheres of the
planets, and their separations according to the five regular
geometrical solids.

(left column)

You are wondering, spectator, at the work of Kepler, a diagram of heaven which

you never saw before. For the five solids of Euclid tell the distance between the

orbits of the planets. How well the doctrine which Copernicus once declared

agrees, the author’s work now reveals to you. Naturally the author has shown.

himself grateful for his great benefaction to the Duke of Teck, not without praise.
Drawn by
Christopher Leibfried.
Tiibingen: 1597.

(right column)

@ Sphere of Saturn.

B Cube, the first regular geometrical solid, showing distance from Sphere of
Saturn to Jupiter.

y Sphere of Jupiter.

§ Tetrahedron or pyramid, touching Sphere of Jupiter outside, and Mars inside,
producing greatest distance between planets.

€ Sphere of Mars.

§ Dodecahedron, the third solid, showing distance from Sphere of Mars up to
the Great Sphere which carries the Earth along with the Moon.

‘n Great Sphere of the Earth.

6 Icosahedron indicating true distance from Earth’s Great Sphere to Sphere of
Venus.

t Sphere of Venus.

k Octahedron, showing distance from Sphere of Venus to Sphere of Mercury.
% Sphere of Mercury.

# The Sun, the unmoving midpoint or center of the Universe.

Plate to be placed at page 101


Annotations to the Plates

PLATE IY. Showing the true breadth of the heavenly spheres,
and of the intervening spaces, according to the calculations and
theory of Copernicus.

‘At Chap. 14,
page 157.

The outermost circle represents the zodiac on the sphere of the stars, and is
drawn about the center of the universe or of the Earth’s orbit, or even from the
Earth’s globe, as the whole of the Earth’s orbit is insensibly small compared with
it.

A System of Saturn, drawn concentrically about G, the center of the Earth's
orbit.

B System of Jupiter.

C That of Mars.

D Circle or path of the center of the Earth’s globe, drawn concentrically about
the center G, together with the Moon’s small sphere shown on it in two places.
The two dotted circular lines mark the thickness of the Earth’s sphere with the
Moon included.

E Two small circles delineating the thickness of the system of Venus, within
which the whole of the variations of its motions are performed.

F The space between the two small circles, within which the whole of the varia-
tions of the motions of the star Mercury are performed.

G The center of all things, and near it the Solar body.

The circle passing through O and P (of which only two arcs appear here) is the
epicycle on the eccentric of Saturn.

The curved line through Q, through the perigee of the epicycle situated at the
apogee O of the eccentric, and through the apogee of the same epicycle at the
perigee P of the eccentric, is the eccentric path of the planet. It is not in fact a cir-
cle, though it does not sensibly differ from a circular line,

HI. is the thickness included between the two concentric circles which the ec-
centric path of Saturn occupies.

The curved line or virtual circle passing through M, through the apogee of the
epicycle at O, and through the perigee of the same epicycle at P, is the eccentric
which Ptolemy calls the equant.

KL is the thickness intercepted by the two dotted circles, which is required by
the whole epicycle and the aforesaid equant. The planet, however, never goes out-
side H or inside I. The remaining spheres are understood to be marked off by
similar circles in each case, though they are omitted here in case the multiplicity of
lines should obscure the point rather than make it clear. Therefore in the cases of
Jupiter and Mars it is sufficient to draw their eccentric path, and the two concen-
tric circles which contain it, and in the other cases only the concentric circles.

Intermediate spaces. R: position of the Cube. S: of the Tetrahedron
T: of the Dodecahedron. V: of the Icosahedron. of the Octahedron,
Z is the space between Saturn and the fixed stars, which is virtually infinite.

Annotations to the Plates

PLATE V. Showing the positions of the centers of the
eccentric spheres of the universe according to the theory of
Copernicus, and the values in the Prutenic Tables.

fover left diagram) fover right diagram)
AL the time of Ptolemy, about AL the time of Copernicus, about
140 A.D. At Chapter 15, p. 161. 1525. A.D. At Chapter 15. p. 161.

At A is the Sun, the center of the universe. The very small circle at B is the cir-
cle of the eccentricity of the Earth’s Great Orbit. The center of the eccentric Great
Orbit stood in the time of Ptolemy at its edge or in a position further from the
Sun, but in the time of Copernicus in a nearer position. That is, the eccentricity of
the Great Orbit was in the former case nearly at its maximum; in the latter case
almost at minimum. The former instance may be seen in the earlier, or left-hand
diagram, the latter in the later, or right-hand diagram.

AB in the earlier diagram is 4170, where the semidiameter of the Great Orbit is
100,000. Hence the greatest separation of the Earth from the Sun is 104,170 and
the least is 95,830. But in the other diagram the eccentricity, which is almost at its
minimum, is 32,195.

AC is the small circle of the eccentricity of Venus. Its semidiameter (where the
semidiameter of the Great Orbit is 100,000 units) is 1040, and BC (in the right-
hand figure), the eccentricity of the center of the small circle from the center of
the Great Orbit B, is 3120. But AC, the eccentricity of the same center from the
Sun A, is 1262. Hence the maximum distance of Venus from the Sun is 74,232
units, and the minimum is 69,628.

D is the center of the small circle of eccentricity of Mercury. Its semidiameter,
in the same units as above, is 2114, and its eccentricity from the center of the
Great Orbit, DB, is 734514; but DA, its eccentricity from the Sun, is 10,270.
Whence the maximum distance of Mercury from the Sun is found to be 48,114,
and the minimum 23,3452.

E is the center of the smail circle of eccentricity of Mars. Its semidiameter is
76022, and BE, its eccentricity from the center of the Great Orbit, is 22,8072.
But AE, its eccentricity from the Sun, is 20,342. Whence the maximum distance
of Mars from the Sun is 164,780, the minimum 139,300.

Fis the center of the small circle of eccentricity of Jupiter. Its semidiameter is
12,000, and BF, its eccentricity from B, is 36,000. But AF from the Sun is 36,656.
The maximum distance of Jupiter from the Sun is 549,256, the minimum 499,944.

G is the center of the small circle of eccentricity of Saturn. Its semidiameter is
26,075. BG is 78,225, and AG, its eccentricity from the Sun, is 82,290, The max-
imum separation of Saturn from the Sun is 998,740, and the minimum 834, 160.

The straight line HBT is the equator with respect to the Earth; but IAS is the
equator with respect to the Sun. Thus the straight line NB@ is the line of the
solstices with respect to the Earth, and MAy with respect to the Sun.

(continued next page)


(Plate V continued)

Saturn
Jupiter
At Mars
apogee Venus
Mercury
Sun

Saturn
Jupiter
At Mars
aphelion Venus
Mercury
Earth

Annotations to the Plates

At time of Ptolemy

Scorpio
Virgo
Cancer
Taurus
Libra
Cancer

At time of Ptolemy


Scorpio
Virgo

Leo
Capricorn
Libra
Sagittarius

Copernicus



Sagittarius
Libra

Leo
Gemini
Scorpio
Cancer

Copernicus


Sagittarius
Libra
Virgo
Aquarius
Sagittarius
Capricorn
blank page



1. The verse on the verso of the title page is an epigram attributed to Ptolemy
with Kepler’s translation of it. In the first edition, the translation only is given on
the title page itself. On the sources for the epigram and other Latin translations,
see F. Seck, Johannes Kepler als Dichter, Internationales Kepler-Symposium (see
Introduction, note 4), pp. 427-450, especially p. 440 and pp. 449-450.

2. This is an allusion to the conical columns set in the ground at the limits of
the Roman Circus to serve as turning posts in the chariot races.

3. Kepler follows the Vulgate. The references here are to the King James ver-
sion (with the Vulgate locations in parentheses where these differ). Ps. 19 (18),
v. 1; Ps. 8, v. 3 (4); Ps. 147 (146), vv. 4-5; Ps. 148, v. 1 and v. 3.

4. Ovid, Fasti 1, 297-298.

5. In the last two years of his life the Emperor Charles V retired to apartments
adjacent to the monastery of San Geromino de Yuste in Estremadura, where he
died on 21 September 1558. As curator of his clocks Charles invited to
Estremadura Giovanni (Juanelo) Turriano of Cremona, who had accepted a com-
mission to construct for him a clock with planetarium. This took three and a half
years to construct and was still incomplete when Charles died. From Yuste
Juanelo went on to Toledo where he achieved fame by his construction of the
aqueduct. On Juanelo Turriano (Lannellus Turrianus Cremonensis) and his
works, see Ambrosio de Morales, Las Antigiiedades de las ciudades de Espana,
f. 91r-f. 93v. (This work is published as part of Florian de Ocampo, De la
Coronica de Espana, continued by Ambrosio de Morales, Alcala de Henares,
1574-1586, vol. 3). See also William Stirling, The Cloister Life of the Emperor
Charles the Fifth, 2nd edition London, 1853, pp. 60, 97-98 and 268-270.

6. These letters are printed in KGW 13, nos. 73, 69, 96 and 92 respectively.
There is a facsimile and a German translation of Galileo’s letter in Walther
Gerlach and Martha List, Johannes Kepler, Dokumente zu Leben und Werk,
Munich, 1971, pp. 70-73.

7, Seneca, Naturales questiones, vii, 31.

8. According to custom, Kepler expected to receive an honorarium for the
dedication. This was granted (250 Gulden) in 1600, when he evidently found it
useful in moving his family to Prague. (See KGW 19, p. 29).


1, Kepler knew the edition of Euclid’s Elementa by Francois de Fois, Comte de
Candale (Paris, 1566 and 1578), which contains extensive commentaries on the
five regular solids. He cites this edition in the Mysterium cosmographicum,
chapter 13.

2. According to Kepler’s testimony in this passage, Maestlin was openly com-
mitted to the Copernican system, teaching this system and discussing its advan-
tages over that of Ptolemy in his lectures to students. In his Epitome astronomiae
(Heidelberg, 1582; Tiibingen, 1588, 1593, 1597, 1598, 1610 and 1614) Maestlin
restricted his treatment to the Ptolemaic system, though in the later editions he
added some remarks supporting the Copernican system. His position is made
quite clear in his letter to Friedrich von Wiirttemberg recommending the work of
Kepler, where he explains that, while the familiar ancient hypothesis was easier

Commentary Notes

for beginners to understand, and therefore more suitable in an elementary text-
book, all practitioners (artifices) agreed with the demonstrations of Copernicus.
(KGW 13, p. 68. Cf. Introduction, p. 22.)

3. As we remarked in the introduction, this comment indicates that Kepler
made his first comparison of the two systems before reading the Narratio prima.
In Graz, however, he made use of both the Narratio prima and De revolutionibus
itself.

4. Virgil, Aeneid, 4, 175.

5. The analogy is to be seen in terms of the Christian interpretation of the
Platonic doctrine of participation — symbolized in the idea of the Book of Nature
—whereby God is revealed through the creation. Kepler’s promised amplification
of the symbolism of the Trinity is given in the Epitome astronomiae copernicanae
(KGW 7, p. 47, p, 51 and p. 258.). See also W. Petri, Die betrachtende Kreatur im
trinitarischen Kosmos, in Kepler Festschrift 1971 (see Introduction, note 4),
pp. 64-98.

6. The inspiration for this hypothesis probably came from Plato’s construction
of the World-Soul in the Timaeus, 35B-36A, from the geometric series 1, 2, 4, 8
and 1, 3, 9, 27.

7, The idea of postulating invisible bodies in the search for what we may call
archetypal causes (a term used by Kepler himself in the Harmonice mundi and
the Epitome astronomiae copernicanae) had been introduced by the
Pythagoreans. As related by Aristotle (Metaphysics, 986 a 10-15), since they con-
sidered 10 to be the essence of the numerical system, they asserted that 10 bodies
must revolve in the heavens, “and there being only nine that are visible, they make
the counter-earth [an invisible planet revolving round the central fire in opposi-
tion to the earth] the tenth.”

8. See E. Rosen, Three Copernican treatises, New York, 1971, p. 147. As
Kepler explained to Maestlin in his letter of 3 October 1595 (KGW 13, p. 34), the
principles underlying the world’s construction were to be sought in geometrical
relations and not in pure numbers, whose properties were accidental.

9. This hypothesis may have been inspired by reading Regiomontanus, De
triangulis omnimodis, where the sine function is introduced. For a facsimile of
this work with English translation, see Regiomontanus on triangles, translated by
Barnabas Hughes, Madison, 1967.

10. Here can be seen the beginnings of Kepler’s physical theory in which the
moving virtues of the individual planets will be replaced by a single moving virtue
located in the sun.

11. Chance or Providence, in leading Kepler to illustrate the pattern of Jupiter-
Saturn conjunctions in a certain way, simply provided the initial insight for the
invention of the polyhedral hypothesis. As he explains, it was only when he
recognized the need for a pattern of 3-dimensional figures (after unsuccessful
trials with polygons) that he achieved his goal.

12. An allusion to Virgil, Aeneid, 10, 652.

13. Horace, Ars poetica, 388.

14, Cicero, De amicitia (Laelius), 23.

15. This is an allusion to Terence, Heauton timorumenos, 4, 3, 41.

16. See Epitome, Book 1, part 5 (KGW 7, pp. 80-100). The original disputation
is not extant.

17, See Astronomia nova (KGW 3, pp. 22-24) and Epitome, Book 4, part 2,
especially chapter 5 (KGW 7, pp. 312-316).

18. Epitome, Book 1, part 2 (KGW 7, p. 47 and p. 51) and Book 4, part 1

19. Kepler explains the distinction between numeri numerantes (counting
numbers) and numeri numerati (counted numbers) in Harmonice mundi, Book 5,


Commentary Notes

appendix (KGW 6, p. 370). The former are abstract numbers (whose properties
are accidental), the latter concrete numbers or numbers embodied in real things;
that is, for Kepler, numbers embodied in geometrical objects such as regular
polygons and the regular and semi-regular solids.


1. In his report to the University, transmitted to Kepler by Hafenreffer (KGW
13, pp. 86-87), Maestlin recommended the addition of a preface explaining the
Copernican system (KGW 13, pp. 84-86). Kepler responded by suggesting the ad-
dition of an extract from either Copernicus himself or Rheticus, but Hafenreffer
did not favor this idea. By placing the new material in this chapter, Kepler reveals
his acceptance of Hafenreffer’s specific advice, given in a further letter (KGW 13,
p. 90), to insert a brief description of the Copernican system, illustrated with
diagrams, “after the preface to the reader.”

2. Hafenreffer’s letter to Kepler of 12 April 1598 (KGW 13, p. 203: 35-37)
seems to indicate that Kepler had his approval to retain this brief statement con-
cerning the reconciliation of the Copernican hypothesis with the Bible. On the
Lutheran attitude to the Copernican hypothesis, see E. Rosen, Kepler and the
Lutheran attitude towards Copernicanism, in Johannes Kepler, Werk und
Leistung, Linz 1971, pp. 137-158 and R. S. Westman, The Melanchthon circle,
Rheticus and the Wittenberg interpretation of the Copernican theory, Isis, 66

3. Later, in the preface to his Rudolphine tables, which effect an improvement
of nearly two orders of magnitude in the prediction of the planetary positions,
Kepler had to acknowledge that Reinhold’s Prutenic tables, based on Copernicus,
were no more accurate than the A/fonsine tables. See J. Kepler, Preface to the
Rudolphine tables, translated by O. Gingerich and W. Walderman, Quarterly
Journal of the Royal Astronomical Society, 13 (1972), 367.

4. See E. Rosen, Three Copernican treatises, pp. 136-153. Introducing his prin-
cipal arguments in support of the Copernican system, Rheticus remarks, “there is
something divine in the circumstance that a sure understanding of celestial
phenomena must depend on the regular and uniform motions of the terrestrial
globe alone.”

5. The true diurnal appearances are demonstrated from opposite premises
(namely the immobile and rotating earth) because the true cause of the ap-
pearances is a difference with respect to motion between the heavens and the
earth. This difference (or relativity) is here the genus and the immobile and
rotating earth two species of it. Properties inferred from the genus (that is, from
the nature of a subject) are called by Aristotle per se or essential. Although the
same properties may be inferred from a species, it is clearly the genus and not the
species which is the cause or logical basis of these properties. On the kat’ auto or
per se rule, see Aristotle, Posterior analytics, 73 a 21-74 a 44. See also R. S.
Westman, Kepler’s theory of hypothesis and the realist dilemma, in Interna-
tionales Kepler-Symposium (see Introduction, note 4), pp. 32-33.

6. For Kepler, the quality of simplicity provides the ground for the physical
truth of the Copernican hypothesis. Since nature loves simplicity, physical truth is
to be sought in choosing what is simple or natural in preference to what is con-
trived or seemingly miraculous. For example, the Copernican hypothesis explains
the agreement of the retrogressions with the position and apparent motion of the
sun, whereas in the old hypotheses, such coincidences could only provoke
astonishment.

Commentary Notes

6. See Genesis, chapter 1, vv. 6-7.

7. Psalm 147 (146 in the Vulgate), v. 4: “He telleth the number of the stars; he
calleth them by their names.”

8. Kepler argues against the infinity of the universe in De stella nova (KGW 1,
pp. 251-257) and Epitome ustronomiae copernicanue (KGW 7, pp. 45-46). See
also A. Koyré, From the closed world to the infinite universe, Baltimore, 1957,
pp. 58-87.

9. Kepler’s @ priori reasons, as we have remarked, remained hypothetical.
Writing to Herwart on 12 July 1600 (KGW 14, p. 130) Kepler states that “these @
priori speculations must not contradict manifest experience but rather be in agree-
ment with it.”

10. Plato’s association of the solids with the elements is given in Timaeus,
53C-56C, and also in Campanus de Novara’s commentary on Euclid’s E/ementa,
which Kepler cites in the Mysterium cosmographicum, chapter 13. From De
placitis philosophorum, believed at that time to be authentic Plutarch, Kepler
learnt that Plato had imitated Pythagoras in associating the solids with the
elements. Kepler saw the doctrine of Pythagoras as an attempt to solve the
mystery of the cosmos that necessarily failed because Pythagoras had no
knowledge of the true, Copernican system.

11. In both the first and second editions, the heavens are assigned the
icosahedron instead of the dodecahedron. Kepler correctly assigns the
dodecahedron to the heavens in the Harmonice mundi (KGW 6, p. 79, diagram).

12. Here is an illustration of the part played by the empirical data in the con-
struction of hypotheses. Writing to Fabricius on 4 July 1603, Kepler emphasizes
that a hypothesis “is built upon and confirmed by observations” (KGW 14,
p. 412). In the following chapters Kepler will establish with a priori reaons the
order of the solids suggested by the observations.

13. In the first edition Kepler dedicated this diagram to the Duke, who had
authorized the construction of a model in the form of a “Kredenzbecher.” See
Kepler’s letter to the Duke of 17 February 1596 and the marginal note in the
Duke’s hand (KGW 13, pp. 50-51). This idea developed into a design for a
planetarium but the technical difficulties proved to be too great for the crafts-
men. See F. D. Prager, Kepler als Erfinder, in Internationales Kepler-Symposium
(see Introduction, note 4), pp. 386-392. According to Maestlin (KGW 13, p. 151),
Kepler's dedication of this key diagram to the Duke had deterred the theologians
in Tabingen from expressing their criticism openly. _ .

ETC. CONSECRATA,” was omitted in the second edition. Also the diagram
itself was inverted laterally. (A facsimile of the diagram with the dedication, as it
appeared in the first edition, is shown as a frontispiece).

14, These annotations were added to the first edition at the suggestion of
Maestlin (see KGW 13, p. 85).


1. Simple because three faces is the minimum number needed to form a solid
angle.

2. This geocentrism of importance (as we may call it) in the Copernican
universe of Kepler may be compared with the heliocentrism of importance, which
attached special significance to the median position of the sun between the earth
and the fixed stars in the geocentric universe of Renaissance Platonism. See, for
example, M. Ficino, Théologie platonicienne, translated by R. Marcel, Paris.
1964-1970, Book 18, chapter 3, p. 191.


Commentary Notes

1. The Monobiblos to which Simplicius refers is a lost work of Ptolemy (in one
book) entitled Peri Diastaseos (On Dimension). Simplicius writes: “The ad-
mirable Ptolemy in his monobiblos On Dimension showed well that there are not
more than three dimensions; for it is necessary for dimensions to be determinate,
determinate dimensions are found along perpendicular straight lines, and not
more than three mutually perpendicular lines can be found, two of them deter-
mining a plane and the third measuring depth; therefore, if another were added
after the third dimension, it would be completely without measure and indeter-
minate.” Simplicius, Jn Aristotelis de caelo commentaria (Commentaria in
Aristotelem Graeca, vol. 7), edited by J. L. Heiberg, Berlin, 1894, p. 9, lines


1. By physics Kepler here means physical astrology, and the vires naturales of
the planets are their astrological powers.

2. For example, a pentagonal section of the icosahedron is related to the vertex
directly above its center in the same way as the leaves of a plant and its umbilicus,
a projection standing in the middle.

3. See Virgil, Aeneid, 4, 569.

4. What Kepler seems to mean is that when the octahedron is rotated about an
axis joining two opposite vertices, four successive edges (forming a square) move
entirely in the plane through the center of the solid and perpendicular to the axis.
The disposition of these edges therefore facilitates their smooth movement. By
contrast, however other solids are rotated, all the edges are inclined (some one
way and some another) to the plane through the center of the solid and perpen-
dicular to the axis of rotation, so that they rotate awkwardly.

5. See Ptolemy, Tetrabiblos, Book 1, chapters 4-7 (Loeb edition, 1971). For
editions of Ptolemy's Harmonica, see U. Klein, Johannes Keplers Bemithungen
um die Harmonieschriften des Ptolemaios und Porphyrios, in Johannes Kepler,
Werk und Leistung, Linz, 1971, pp. 51-60. Kepler himself made a Latin transla-
tion of Book 3 of Ptolemy’s Harmonica, which was first published in the nine-
teenth century by C. Frisch, Kepleri opera omnia, Frankfurt and Erlangen,
1858-1871, vol 5, pp. 335-412. However, Kepler’s notes on Ptolemy’s Harmonica
were published by him as an appendix to Book V of the Harmonice mundi.

6. On Kepler’s astrology, see F, Hammer, Die Astrologie des Johannes Kepler,
Sudhoffs Archiv, 55 (1971), 113-135 and G. Simon, Kepler’s astrology: the direc-
tion of a reform, Vistas in Astronomy, 18 (1975), 439-448.


1. See notes for Original Preface to the Reader, note 19.


1. Kepler means that the solid is laid out flat in such a way that the octahedron-
square is opened out in a straight line.

2. Following Plato, Kepler assigns to mathematical forms an existence prior to
sense objects. This view, which underlies his central theme of a divine harmony
based on geometry, is expounded in detail in the Harmonice mundi (Book 4,

Commentary Notes

chapter 1), where he quotes a long extract from Proclus’s commentary on Euclid’s
Elementa in support.

3. Aristotle, De caelo, 291 b 28 - 292 a9.

4. See Epitome astronomiae copernicanae, Book 1, part 2. On Kepler's ideas
concerning infinity see also W. Petri (/oc. cit., notes for Original Preface to the
Reader, note 5), pp. 69-72, but note that, on p. 72, line 6, the first word should be
‘endlich’. See also A. Koyré (/oc. cit., notes for chapter 2, note 8).

5. Aristotle, De caelo, 287 b 22-32.

6. Aristotle, De caelo, 284 b 6 - 286 a 2.

7. See notes for chapter 1, note 19.


1. The aspects are the angular separations of planets on the celestial sphere cor-
responding to certain fractional parts of the circle. Traditional astrology, as
represented by the teachings of Ptolemy in the Tetrabiblos, recognized five
aspects, namely conjunction and opposition (0° and 180°), trine, quartile and
sextile (120°, 90° and 60°). In his definitive account of the origin of the aspects,
given in the Harmonice mundi, Book 4, chapter 5, Kepler added a number of
others, notably the quintile and biquintile (72° and 144°), to the set of effective
astrological aspects. See KGW 6, pp. 250-251.

2. Kepler here uses the nomenclature of the hexachordal system, which had
been used by singers since the time of Guido d’Arezzo in the eleventh century to
keep their tonal bearings when sight-reading. In this system, the vocal range,
starting on the G shown in Kepler’s diagrams, was represented by a set of overlap.
ping hexachords, each consisting of six diatonic notes, with a single semitone (be-
tween the middle pair), called ut, re, mi, fa, sol, la. These names were taken from
the opening syllables of six lines of a Latin hymn. The hexachords were classified
as hard, natural or soft according to whether they contained B, no B or B flat.
Thus the lowest hexachord, starting on G, is hard; the next, starting on C, is
natural, and the third, starting on F, is soft. The note described by Kepler as F
(fa, ut) is the F below middle C, which is fa of the second hexachord and ut of the
third. Similarly C (sol, fa, ut) is middle C, which is sol, fa and ut respectively of
the third, fourth and fifth hexachords.

Kepler extended the characterization of the thirds as ‘hard’ (dura) and ‘soft’
(nollis), to be found in the medieval hexachord system, also to the sixths. In the
sixteenth century, Gioseffo Zarlino used the terms major and minor in relation to
both intervals and these terms were adopted in Germany by Johannes Lippius.
Although Kepler occasionally uses the terms major and minor, he generally de-
scribes thirds and sixths as hard and soft. In the Harmonice mundi (Book 3,
chapter 5, KGW 6, p. 135), Kepler explains the origin of the terminology. The
minor, he remarks, sounds softer and more soothing (mollior et blandior) to the
ear, while the major sounds hard or harsh (dura sive aspera).

On the concept of tonality in Kepler’s music theory (which is definitively set out
in Harmonice mundi, Book 3), see M. Dickreiter, Dur und Moll in Keplers
Musiktheorie, in Johannes Kepler, Werk und Leistung, Linz, 1971, pp. 41-50; M.
Dickreiter, Der Musiktheoretiker Johannes Kepler, Bern and Munich, 1973,
especially pp. 160-187; D. P. Walker, Kepler’s celestial music, Journal of the
Warburg and Courtauld Institutes, 30 (1967), 228-250.

3. Thus, a minor third added to a major sixth, or a major third added to a
minor sixth, produces an octave; a minor third added to a major third produces a
perfect fifth; a minor sixth added to a major sixth produces an octave plus a
perfect fifth.

4. Each edge of the octahedron is a chord of a quadrant of a great circle of the
circumscribing sphere.


Commentary Notes

5. By B flat Kepler means the minor third, the base note G being understood.
6. Consider the diagrams:

In the first, the right angle is constructed from a trine and a sextile; in the second,
‘wo quartiles.

frome ‘imperfect harmony B flat again means the minor third with base note G.

8. Here Kepler implies that effective aspects only arise from regular polygons

i ellate in a plane. ;
Soa neemomaneticie a prospectus of books he intended to edit and
publish, among them the Harmonica of Ptolemy with the commentary of Por-
phyry, but his early death prevented him from carrying out this scheme. The pro-
spectus is reproduced in E. Zinner, Leben und Wirken des Johannes Miiller von
Kénigsberg gennant Regiomontanus, Munich, 1938, Tafel 26. An imperfect Latin
translation of Ptolemy’s Harmonica by Antonius Gogavinus had been published
in Venice in 1562. Preparatory to the writing of his Harmonice mundi, Kepler
made a study of this Latin version and also of manuscript copies of the Greek
texts of both Ptolemy and Porphyry, loaned to him by Herwart (KGW 14, p. 137
and KGW 15, p. 408). On Kepler’s study of Ptolemy and Porphyry, see U. Klein
(loc. cit., notes for chapter 9, note 5), pp. 51-60. The texts of Ptolemy's Har-
monica and Porphyry’s commentary, together with a German translation of the
Harmonica, have been published by I. ae ee hégskolas Arsskrift, 36

0) No 1, 38 (1932) No 2 and 40 (193: lo 1.

Ceo ee varietate, Basel 1557, Book 17, chapter 98, p. 680.

11. In De stella nova, as Kepler remarks in a letter to Herwart (KGW 15, p.
453), he had rejected almost all of judicial astrology, except for the aspects.
Kepler clarified his views on astrology in two German works, Antwort auf Réslini
Discurs (KGW 4, pp. 99-144) and Tertius Interveniens (KGW 4, pp. 145-258),
where he took a middle course between the astrologer Réslin and the anti-
astrologer Feselius. Kepler’s definitive account of the efficacy of the aspects and
their relation to the musical harmonies is given in Harmonice mundi, Book 4,
chapters 6 and 7. Whereas the division of the zodiac into twelve signs was ar-
bitrary, so that these signs (and similarly, the twelve houses tied to the observer's
horizon) could have no natural effects, the aspects or configurations reflected the
divine harmony, and in addition to producing meteorological effects by a kind of
resonance with the earth-soul, could also invoke an instinctive response from the
soul of the newly-born infant, whose disposition was therefore influenced, to
some extent, by the birth-constellation; this was the justification for horoscopes.
For an account of Kepler’s “Persénlichkeitslehre,” see B. Sticker, Johannes
Kepler - homo iste, in Internationales Kepler-Symposium (see Introduction, note
4), pp. 463-467.

12. See notes for chapter 11, note 2.

Commentary Notes

13, See Epitome astronomiae copernicanae, Book 4, part 2, chapter 6 (KGW 7,
p. 316).

14, Kepler explains this terminology in the Harmonice mundi, Book 3, chapter
5. The Greeks called the octave 6vamaow» (Diapason). The expression
Diapason epidiapente or 61a névre émt ua macaw means ‘the octave over the
fifth’ or ‘the fifth over the octave’.

15. As Kepler explains, the practical musician uses tempered intervals. See, for
example, Harmonice mundi, Book 3, chapter 9. See also M. Dickreiter. Der
Musiktheoretiker Johannes Kepler, Bern and Munich, 1973, p. 158.

16. Following Euclid, Kepler describes incommensurable lines whose squares
are commensurable as ‘expressible’ (effabiles). The actual Greek term is pntiy
duvduer (potentially expressible). For the opposite (inexpressible) Euclid uses
the term @doyor , translated by Kepler as ‘ineffabiles’. In the Harmonice mundi
(Book 1, definition 15) Kepler warns the reader of the ambiguity contained in the
usual Latin translations of Goyor as ‘irrational’. In the cases of the equilateral
triangle, square and regular hexagon, the ratios of the sides to the radius of the
circumscribing circle are respectively V3:1, 72:1 and 1:1, so that the sides are ‘ex-
pressible’ (they can be constructed using the ruler-and-compasses construction
of square roots). In the case of the hexagon, the sides are also, of course, com-
mensurable.

17. That is, neither 1:11 nor 11:12 and neither 5:7 nor 7:12 represent con-
sonances.

18. See Harmonice mundi, Book 4, chapter 5, proposition 14.

19. By correlating the aspects with harmonic intervals extending over several
octaves, Kepler was able to show that the octile, trioctile, decile and tridecile also
represented musical consonances. While the consonances within an octave are
represented by the complement of the fraction of the circle defining the aspect,
consonances greater than an octave are represented by the fraction itself. For ex.
ample, the trioctile represents the minor sixth (5:8) and also the combined interval
of a fourth and an octave (3:8). In the case of the octile itself, the ratio 7:8 does
not represent a consonance, but the ratio 1:8 corresponds to three octaves, which
is, gf gourse, a consonance. See Harmonice mundi, Book 4, chapter 6 (KGW 6,
p. 261).


21. For example, although regular dodecagons do not tessellate, a suitable
combination of regular dodecagons and equilateral triangles will tessellate.

22. In his explanation of the efficacy of the aspects, Kepler made use of a
reciprocal figure placed at the center, such that the angle between adjacent sides
was equal to the angle subtended at the center of the circle by a side of the cir-
cumferential polygon. Thus the central figure was formed from the angle between
the light rays marking the termini of the aspect. When the soul recognized the
harmony of the light rays, it was at first concerned with the central figure. But as
the efficacy of the aspect was primary, and the way in which it was perceived by
the soul a secondary consideration, the circumferential polygon had greater im-
portance See Harmonice mundi, Book 4, chapter 5, proposition 6 (KGW 6,
p. 247).

23. The melodic intervals (concinna) are differences between pairs of neighbor-
ing consonant intervals smaller than an octave. See Harmonice mundi, Book 3,
chapter 4 (KGW 6, p. 128). Kepler remarks that Ptolemy considered the thirds
and sixths not to be consonances, but divided the interval between ut and fa into
two intervals, each held to be melodic, whereas singers recognized three melodic
intervals between these notes: ut, re, mi, fa. See Harmonice mundi, Book 3, in-
troduction (KGW 6, p. 99).


Commentary Notes

24. Konrad Dasypodius appended the pseudo-Euclidean Harmonica to his edi-
tion of Euclid (Strasbourg, 1571). On the relation between Kepler and
Dasypodius, see H. Balmer, Keplers Beziehungen zu Jost Biirgi und anderen
Schweizern, in Johannes Kepler, Werk und Leistung, Linz, 1971, pp. 123-124.


1. Here Kepler expresses quite clearly that his @ priori reasons were only prob-
able and needed to be tested against the empircal data.

2. Euclid, Book 15, proposition 13. The Latin translation of Campanus de
Novara from the Arabic was first printed in 1482. Kepler possessed the edition
published in Basel in 1537, a reprint of the edition prepared by Jacques Lefevre
d’Etaples and published in Paris in 1514.

3. The expression sinus totus (whole sine) means sin 90°, here taken as 1000
units. The sine of any arc was the perpendicular from one extremity to the
diameter through the other extremity. Taking the sine of the quadrant, that is sin
90° or the radius of the circle, as any convenient number of units, the sines of
other arcs could be expressed in terms of these units, without the introduction of
fractions.

4. See notes for Original Preface to the Reader, note 1.

5. In his calculation of the greatest distance of Mercury according to the
polyhedral hypothesis, Kepler used the value 707, that is, the radius of the circle
inscribed in the square formed by four middle edges of the octahedron, instead of
the radius of the inscribed sphere.


1. The distances on which the comparison given in this table is based are
measured from the center of the earth’s orbit, except in the cases of the ratios
Mars-earth and earth-Venus, where the greatest and least distances of the earth
are measured from the sun, so that the earth’s sphere is given a thickness in ac-
cordance with the eccentricity of the orbit. Thus the distances used in the calcula-
tion are those given by Kepler in the first column of the table on page 162, except
in the case of the earth, where the distances used are those given in the second col-
umn. In the case of Saturn-Jupiter, there is a slight arithmetical error, for the
Copernican data imply a ratio of 1000:631, which is nearer the value 1000:630
used by Kepler in his letter to Maestlin of 2 August 1595 (KGW 13, p. 28).

2. The values used here by Kepler for the radii of the inner and outer surfaces
of the earth’s sphere are those given in the fourth column of the table on page
162, so that the radius of the lunar orbit is taken to be 3' 36".

3. Here we see the idea of simplicity (again in the sense of a preference for the
natural over the seemingly miraculous) used as a justification for the hypothesis.
For the agreement of the hypothesis with the empirical data would be unthinkable
were it not a consequence of God’s plan of creation.

‘4. Epitome astronomiae copernicanae (KGW 7, p. 280). Archetypal reasons led
Kepler to equate the ratio of the distances of the sun and moon from the earth to
the ratio of the lunar distance to the radius of the earth. Observational techniques
had not thus far permitted a more accurate determination.

Commentary Notes

1, Kepler here took the step which converted the Copernican system into a
truly heliocentric system. Again the principle of simplicity provided justification
for the step. For Kepler’s innovation brought the earth’s orbit into line with those
of the other planets.

2. See Introduction, pp. 19-20.

3. Plate V. The basis of these figures, prepared by Maestlin, is the represen-
tation of planetary motion called by Copernicus eccentric-on-eccentric (see De
revolutionibus, Book 5, chapter 4). The point A represents the sun, B the center
of the earth’s orbit and the lines BC, BD, BE, BF and BG the eccentricities of the
respective planetary orbits as defined by Copernicus; that is, they are the eccen-
tricities of the deferent in the epicycle-on-eccentric representation or three
quarters of the eccentricities of the equant in the simple eccentric representation.
The directions of these lines in the two diagrams show the positions of the lines of
apsides in the times of Ptolemy and Copernicus respectively. Despite his reference
to dextrae figurae in introducing his values for the planetary distances, Maestlin
calculates these on the basis of the data for the time of Ptolemy. The diagrams
show, for each planet, the small eccentric, a circle of radius /% €, where € is the
eccentricity as defined above; it may be noted that, if 2e represents the eccentrici-
ty of the equant, then ¢= 3/2e. The planet moves on a large eccentric of radius
a (not shown in the diagram) whose center moves on the small eccentric in such a
way that (except in the case of Mercury, for which a more complicated combina-
tion of circles is needed) the greatest and least distances of the planet from the
center of the earth’s orbit are respectively a +74 € anda — % e. (For a detailed
description of the various representations, see the Appendix.)

4. In Kepler’s table the numbers are given in sexagesimal form, the radius of
the earth’s eccentric being taken as 1°. There are mistakes of various kinds, and in
particular, discrepancies between the distances given by Kepler in the first column
(which may be compared with those he gave in his first detailed communication
of the polyhedral hypothesis to Maestlin, KGW 13, p. 44) and those implicit in
the data accompanying Maestlin’s diagrams. The principal reason for these dif-
ferences is that, whereas Maestlin derived his data by new calculations from the
Prutenic tables (see KGW 13, p. 65), Kepler simply accepted the values given by
Copernicus. In the first column (giving the distances from the center of the earth’s
eccentric), the values for Saturn, Jupiter and Mars are taken from De revolu-
tionibus, Book 5, chapters 9, 14 and 19 respectively, though in the case of Mars,
Kepler gave the greatest distance correctly as 1° 39 56”, correcting a misprint in
Copernicus, where this distance is given as 1° 38’ 57”. In the case of Venus,
although Maestlin’s value of the mean distance agrees with that of Copernicus,
the consequent values of 0° 44’ 25” and 0° 41’ 55” for the greatest and least
distances from the center of the earth’s orbit differ from those given by Kepler,
because Kepler has inadvertently calculated a+4/3 € instead of a+2/3 e.
Again, in the case of Mercury, Kepler’s values differ appreciably from those of
Maestlin. Both Maestlin and Kepler calculate the distances according to the for-
mula (which applies only to Mercury):

(radius of large eccentric) +( € + radius of small eccentric).
Using Maestlin’s values, this gives 35730 + (7345 + 211414), where the radius of
the earth’s eccentric is taken as 100,000. This is equivalent, in the sexagesimal
notation of Kepler's table, to 0° 27' 7” and 0° 15’ 46" for the greatest and least
distances respectively. (Cf. KGW 1, p. 145). The Mercury theory is complicated,
however, by a variation in the radius of the eccentric according to the position of
the earth in relation to the line of apsides (see De revolutionibus, Book 5,
chapters 25 and 27 and also the Appendix). Whereas Maestlin used the minimum


Commentary Notes

value of 35730 for the radius of the eccentric, Kepler used the maximum value of
39530, and this accounts for the differences. In the case of the sun, Kepler’s
values for the greatest and least distances are simply the values given by Coper-
nicus (De revolutionibus, Book 3, chapter 21. Cf. KGW 1, p. 92) for the eccen-
tricities of the earth’s orbit in the time of Ptolemy and in the sixteenth century.
Copernicus had supposed that the eccentricity oscillated in the same period as the
obliquity of the ecliptic. (See notes for chapter 1, note 12.)

In the second column Kepler gives the greatest and least distances of the planets
from the sun, Here he relies on the values of Maestlin, though there is a slight
discrepancy in the case of Venus (Maestlin’s value for the greatest distance being
0° 44’ 32”) and a larger difference in the case of Mercury. For Mercury Kepler
used the values originally communicated to him by Maestlin but revised without
Kepler’s knowledge during the printing of the Mysterium cosmographicum (see
Introduction, p. 20).

The method of calculation of the distances from the sun is described by
Maestlin in a letter of 11 April 1596, where he discusses the distances of Mercury
(KGW 13, p. 78). The lines AC, AD,.. .are calculated from the triangles ACB,
ADB... .using the known values of BC, BD,. . and the angles at B (derived from
the known directions of the apogees). Clearly, the angles at A (or the directions of
the aphelions) can also be calculated. In the letter, Maestlin calculates the
distances of Mercury on the basis that, in the time of Ptolemy, the apogee of Mer-
cury was in 15° 30' of Libra (computed from the Prutenic tables). However, the
distances given in his table are based on Ptolemy’s own position of 10° of Libra
for the apogee of Mercury. (See notes for chapter 19, note 6). Indeed, all the
distances in Maestlin’s tables (accompanying the diagrams) are based on the data
for the time of Ptolemy. Apart from the case of Jupiter, where the eccentricity in
relation to the sun is nearer 36600 than the value of 36656 given by Maestlin, his
solutions of the triangles (giving the eccentricities in relation to the sun and the
directions of the aphelions) are in almost perfect agreement with the data. There
are, however, major errors in the calculation of the distances of Venus and
Saturn. In the case of Venus, the greatest and least distances from the sun are
taken to be

(greatest distance from
center of earth’s orbit) + (AC — radius of small eccentric)

instead of

(mean distance from
center of earth’s orbit) + (AC — radius of small eccentric)

The greatest distance should be 72152 or 0° 43' 17” and the least distance 71708 or
0° 43’ 1”. In the case of Saturn, the radius of the small eccentric has been
neglected. When this is taken into account, the greatest and least distances
become 9° 43’ 36” and 8° 36’ 7” respectively. For the greatest and least distances of
the earth from the sun, Kepler takes the values accompanying the diagrams, cor-
responding to an eccentricity of 4170. There is a misprint in the table of apogees
and aphelions. The direction of BAL in the time of Ptolemy should be in Gemini
(not Cancer). (Cf. KGW 13, p. 78.) In calculating the directions of aphelion in the
time of Copernicus (of which no application is made), Maestlin took into account
the changes in the eccentricities that had been noted by Copernicus.

In the third column, Kepler calculates the distances according to the polyhedral
hypothesis. He starts with the earth’s sphere, whose inner and outer surfaces have
radii equal to the least and greatest distances of the earth given in the second col-
umn. The outer surface of the earth’s sphere is taken as the inscribed sphere of the

Commentary Notes

dodecahedron and the circumscribed sphere of this dodecahedron then becomes
the inner surface of the sphere of Mars. The radius of the outer sphere of Mars
(that is, the theoretical greatest distance of Mars) is then calculated from the
known radius of the inner surface, using the ratio of distances of Mars given in
the second column. This process is continued upwards to Saturn and downwards
to Mercury. The values of the fourth column are calculated in the same way as
those of the third, except that the thickness of the earth’s sphere is increased to in-
clude the moon’s orbit.

Kepler’s results in the third and fourth columns reflect the errors in those of the
second. A significant factor contributing to the errors was no doubt the fact that
Kepler was not himself able to check the manuscript being prepared for the
printer. This was undertaken by Maestlin in Tiibingen, who therefore had respon-
sibility for the final corrections. Kepler made no attempt to correct these errors in
the second edition. (See Introduction, p. 29).

The true relation between the polyhedral hypothesis and the Copernican data
may be seen in Table I of the Introduction, where the distances given by Kepler in
the second, third and fourth columns of his table are compared with the corrected
distances (in parentheses).

5. Kepler’s values are compared with the corrected angles (in parentheses) in
Table II of the Introduction. The angle 29° 19’ given by Kepler for Mercury does
not correspond to his value for the greatest distance of this planet from the sun.
For agreement with this distance (0° 29 19’), the angle should be 29° 15’. It seems
likely that Kepler has mistakenly written the ‘distance’ in place of the angle.

6. Astronomia nova, chapter 6.


1. Kepler’s reason for the existence of the lunar sphere would seem less con-
vincing after the discovery of the satellites of Jupiter, though he makes no com-
ment on this point in the second edition.

2. Kepler makes clear that the polyhedra and spheres are purely geometrical
concepts without material reality.

3. Diogenes Laertius, De vitis philosophorum, ii, 8. For Kepler’s quotation
from Plutarch’s De facie in orbe lunae, xvii, see KGW 2, p. 203.

4. Kepler's conversation with Galileo’s sidereal messenger, translated by E.
Rosen, New York, 19

5. See Astronomia nova, introduction and chapters 33-34,

6. Kepler here uses the expression inertia materiae to denote the inclination of
matter to remain at rest. Although the concept is embodied by Kepler in the ax-
iom “Every corporeal substance, as corporeal, will rest in any place in which it is
found isolated, outside the reach of bodies of the same kind,” printed in the in-
troduction to the Astronomia nova, it is in the Epitome astronomiae coper-
nicanae, Book 1, part 5 and Book 4, parts 2 and 3, that he introduces the term ‘in-
ertia’. For example, Kepler attributes to the planets a “natural and material
resistance or inertia to leaving a place, once occupied.” (KGW 7, p. 333. Cf.
p. 339). See I. B. Cohen, Dynamics, the key to the new science of the seventeenth
century, Acta historiae rerum naturalium necnon technicarum, Special Issue No.
3 (Prague, 1967), pp. 83-100. Cf. I. B. Cohen, Kepler’s century, Vistas in
astronomy, 18 (1975), 3-36, especially 21-22.

7. See, for example, Epitome astronomiae copernicanae, Book 4, (KGW 7,
p. 332).

8. See chapter 1, appendix (KGW 2, pp. 39-42).


Commentary Notes

1. The orbit of Mercury has a large eccentricity compared with those of the
other planets. But Kepler will later admit (see his note 1 in the second edition) that
the archetypal reason for this peculiarity of Mercury is not to be found in the
octahedron.

2. Kepler’s diagram is visually misleading. The figure XQHS represents the dot-
ted rhombus in the diagram of the eee shown below.

Nee
Na
3. For a description of the Mercury theory of Copernicus see the Appendix.
What Kepler here describes as a variation in the radius of the large eccentric is
represented by Copernicus as an oscillation along the diameter of an epicycle.
4, The values for the greatest and least distances of Mercury used here by
Kepler are those communicated to him by Maestlin in the letter of 11 April 1596
(KGW 13, p. 37, lines 37 and 43) and not the revised values accompanying
Maestlin’s diagrams. (Cf. notes for chapter 15, note 4 and notes for chapter 19,
note 6). By taking the values 387 and 474 for the radii of the inscribed sphere and
the circle inscribed in the octahedron-square, Kepler implies that the radius of the
circumscribed sphere (in other words, the least distance of Venus) is 670, which is
the value according to the polyhedral hypothesis (and Kepler’s calculation) when
the moon's orbit is included in the earth’s sphere (p. 162, line 11, column 4).


1. Quotation from Horace, Epistles, I, i, 32.

2. Shortly after his first visit to Tycho Brahe, Kepler explains to Herwart that
one of the principal aims of his visit was to obtain more accurate values of the ec-
centricities, in order to confirm the polyhedral hypothesis (KGW 14, p. 128).
Kepler's chief difficulty at this time was that he did not know the archetypal
causes of the eccentricities and their differences.

3. Kepler here paraphrases the text of Copernicus.

4, Bernhard Walther, patron of Regiomontanus, was a distinguished observer.
His observations were first published in 1544,

5. The whole preface, of which only an extract is given here, may be found in
L. Prowe, Nicolaus Coppernicus, Berlin, 1883-1884, vol. 2, pp. 387-396.

6. The third, fourth and fifth minutes are the corresponding sexagesimal frac-
tions of a degree. For example, in his Prutenic tables, Reinhold gives the distance
Commentary Notes

tT, —
mean 4(T, +T;) in his previous formula £2 = T+ 4M =7), or equivalently

t= ACL*TD) by the geometric mean J"(TiT,), so that the formula becomes

rt?
ry?
1618, He announced it in the Harmonice mundi, Book 5, chapter 3 and gave a
physical explanation in the Epitome astronomiae copernicanae (KGW 7, p. 306).
Two new factors are introduced —the resistance arising from the bulk of the
planet and the capacity of the planet to assimilate the solar force—which combine
with the effects of the length of the path and the weakening of the solar force, to
produce the harmonic law. Kepler took the bulk or quantity of matter propor-
tional to Vr and the volume (measuring, on the analogy of a water-mill, the
capacity to assimilate the solar force) proportional to r. While there was some
observational evidence for the second relation, Kepler had to rely on archetypal
causes for the first (KGW 7, pp. 283-284). On Kepler’s harmonic law, see O.
Gingerich, The origins of Kepler’s third law, Vistas in astronomy, 18 (1975), 600
and R. Haase, Marginalien zum 3. Keplerschen Gesetz, Kepler Festschrift 1971,
Regensburg, 1971, pp. 159-165.

14. Harmonice mundi, Book 3, Digressio politica (KGW 6, p. 188). (Cf. notes
to chapter 1, note 13.) See also A. Nitschke, Keplers Staats — und Rechtslehre, in
Internationales Kepler-Symposium (see Introduction, note 4), pp. 409-424.

15. Napier’s tables are tables of logarithms of natural sines and therefore
needed to be used in conjunction with a table of sines.

3f2
= ©. Kepler discovered his third or harmonic law 27> = +E on 15 May


1. The value 559 is obtained by adding half the eccentricity to the mean
distance 500 (see p. 174, lines 15-25), taking the eccentricity as half the difference
between the greatest and least distances of Mercury given in column 1 of the table
on p. 162. The two mean distances 500 and 559 correspond roughly to the max-
imum and minimum radii of the large eccentric of Mercury. (See notes for
chapter 17, note 3.)

2. The distances in the first column are in agreement with those given by
Kepler in the second column of his table on p. 162, except in the case of Mer-
cury, where he uses the values communicated by Maestlin in the letter of 11
April 1596 (see notes for chapter 20, note 4). In the second column, Kepler
calculates the mean distances in accordance with his formula relating distances
and periodic times (see notes for chapter 20, note 7). In the last column, Kepler
compares the ratios of the distances of neighboring planets with those predicted
by the polyhedral hypothesis, seeking to show the polyhedral hypothesis in the
best light by using the mean distances (that is, neglecting the thickness of the
spheres) whenever these provide a closer fit than the extreme distances. Starting
with the mean distance of Saturn according to the formula relating distances and
periodic times, this distance is reduced in the ratio 577 to 1000, the relation be-
tween the inscribed and circumscribed spheres of the cube, to obtain the value
5290, which is found to correspond approximately to the mean distance of
Jupiter. While the method of comparison in the case of the superior planets is
fairly clear, the treatment of the inferior planets seems confused, and there is in
fact an error, evidently arising out of Maestlin’s failure to comprehend Kepler's
intention (KGW 13, p. 109). As Kepler explains to Maestlin (KGW 13, p. 117),
the comparison of the distances of Venus and Mercury starts with the mean


Commentary Notes

distance of Mercury according to the formula relating distances and periodic
times. Then this distance 429 is increased in the ratio 1000 to 577, the relation be-
tween the inscribed and circumscribed spheres of the octahedron, to obtain the
value 741, which is found to correspond to the greatest distance of Venus accord-
ing to the Copernican data.

3. The statements concerning the outer solids (and also the tetrahedron be-
tween Mars and Jupiter) are directly supported by the figures in the right hand
column. The statements concerning the inner solids should probably be inter-
preted as follows. Although the figures show that these solids could lie between
the extreme distances according to the Copernican data, the spaces between the
earth and the two adjacent planets would be smaller according to the (more
reliable) distances computed from the motions, but the reduction would be less
than the earth’s eccentricity, so that there would still be room for the solids be-
tween the mean distance of the earth and the least distance of Mars on the one
hand, and between the mean distance of the earth and the greatest distance of
Venus on the other. (Cf. Introduction, pp. 28-29 and Table III.)

4. Harmonice mundi, Book 5, chapter 4 (KWG 6, p. 309).


1. See Maestlin’s letter of 27 February 1596 (KGW 13, pp. 54-55). The epicycle-
on-eccentric representation is a geometrically equivalent transformation of the
eccentric-on-eccentric representation illustrated in Maestlin’s diagrams. As is evi-
dent from the diagram of the epicycle-on-eccentric representation (see Appendix,
fig. 4), the thickness of the sphere needed to accommodate the epicycle (assumed
real) is twice as great as the thickness that would suffice to allow the variation in
the distance of the planet, and hence twice the thickness that would be required in
the eccentric-on-eccentric representation. (Cf. Introduction, p. 20.)

2. Maestlin modified Kepler’s text in this passage to remove an error that
Kepler had overlooked. Explaining the point ina letter to Kepler, Maestlin (KGW
13, pp. 109-111) quotes Kepler’s original, where the point C, in the Copernican
representation, is wrongly identified with the center of the Ptolemaic equant. In
fact, AC = three-quarters of the eccentricity of the equant.

3. Kepler’s reasoning, in which the two causes of difference in the periodic
times of separate planets—namely, the length of the path and the weakening of
the solar force in proportion to the distance from the sun —are applied to the mo-
tion in a single orbit, may be interpreted as follows. Taking r to be the radius of
the eccentric EFGH, the distance of the planet from the sun A when in apogee is
r+e, where e=AB (that is, half the eccentricity of the equant). In accordance
with Kepler’s formula, two separate planets, moving at distances r and r+e from
the sun respectively, would have periodic times T, and T; given by

tHe STATA) oy more imply, 22° = B,
Since the mean angular velocities w, and «w would be inversely proportional to

w

. Now interpreting «and w2 as the

angular velocities, about the sun A, of a single planet in its mean distance and
apogee respectively, Kepler infers that the planet moves in its path EFGH as if it
were moving uniformly in the equant IKLM. This conclusion is a generalization
to the whole orbit of a result that has been established only in the neighborhood
of the apsides.

4. By the whole eccentricity, Kepler means AC (= 3/2AB), the eccentricity of
the deferent in the Copernican epicycle-on-eccentric representation. In this

the periodic times, we may write "2°

Commentary Notes

representation (fig. 1), the planet P moves uniformly on the epicycle, while the
epicycle center moves uniformly on the deferent, center C. The angular velocity in
the epicycle (relative to the radius vector of the deferent) is equal to the angular
velocity of the epicycle center moving in the deferent. In the geometrically
equivalent Copernican eccentric-on-eccentric representation (fig. 2), the planet P
moves uniformly on a large eccentric whose center moves uniformly on a small
eccentric with center C and radius BC, so that the center of the eccentric is at B
when the planet is in the apsides. In both representations, the true anomaly
v = a —2esin a + e?sin 2a , where e=AB and the radii of the deferent (fig.
1) and eccentric (fig. 2) are taken as 1. It follows that, to a first approximation,
the angular velocity of the planet about A when in apogee is (1 — 2e) w, where w
is the mean angular velocity. This is clearly equivalent to an angular velocity of w
about the point D, the center of the Ptolemaic equant, since AD =2e. As in the
case of the Ptolemaic representation, Kepler has only verified that his physical
theory is consistent with the Copernican representation in the neighborhood of
the apsides.

Fig. 1 Fig. 2

It should be noted that, in these diagrams, the letters A, B, C and D represent the
similarly designated points in Kepler’s own diagram (p. 216). '
5. The term nutus employed here by Kepler, is the Latin equivalent of Aris-
totle’s pont) , treated extensively by Simplicius, for whom it meant the endeavor
of a body to remain in its natural place, or to return to this place when displaced
from it. By the sixteenth century, however, nutus came also to be identified with
impetus. For example, in the commentary of Henri de Monanthueil on the Ques-
tiones mechanicae (Monantholius, Aristotelis mechanica, Paris, 1599, p. 108),
the term is used in connection with any endeavor, whether a natural inclination or
an impetus in the sense of Philoponus, or even a combination of the two. A nutus
arising either from an external force or an internal volition is described by


Commentary Notes

Monantholius as non naturalis, while a nutus inherent in the nature of a body is
described as naturalis. Here Kepler uses the term figuratively to mean impetus in
the sense of a divinely inspired volition. See H. M. Nobis, Ropé und Nutus in
Keplers Astronomie, Kepler Festschrift 1971 (see Introduction, note 4),
pp. 244-265.

6. Kepler’s hypothesis of the inverse-distance relation for the solar force
pushing a planet along its orbit was the first step in the path that eventually led
him to the area law. See E. J. Aiton, Kepler’s second law of planetary motion,
Isis, 60 (1969), 75-90.

7, Astronomia nova, chapter 28. See C. Wilson, Kepler’s derivation of the
elliptical path, Jsis, 59 (1968), 5-7.

8. Harmonice mundi, Book 4, chapter 7 (KGW 6, p. 264).


1. The ‘upper apsis’ is the ‘apogee’.

2, The terms ‘head’ and ‘tail’ refer respectively to the ascending and descending
nodes and are derived from the mythological explanation of eclipses, found with
variations in ancient India, China and Islam, according to which a dragon, with
its head and tail twisted round the nodes, swallowed the sun and moon whenever
the opportunity occurred. It was appropriate that the moon should have been
placed initially at its greatest distance from the nodes, so that there would be no
danger of an eclipse during the first night. For a description of the sources and
variations of this mythological explanation of eclipses, see W. Hartner, Oriens,
occidens, Hildesheim, 1968, pp. 268-286. The terms caput Draconis and cauda
Draconis for the ascending and descending nodes ( avapifdtav and Karapipdt ww
ouvBeouce ) are defined in the Prutenic tables (see notes for chapter 18, note 6),

. 38b.

3, The Platonic Year (or World Year) is described in the Timaeus, 39D, as the
interval which elapses before all the planets return simultaneously to their starting
points, See also the commentary on this passage by Proclus, Commentaire sur le
Timee, translated by A. J. Festugiére, Paris, 1966-1968, vol. 4, pp. 118-122.

4. Copernicus, De revolutionibus, Book '1, chapter 10 and Pliny, Historia
naturalis, ii, 1

5. The intervals subsidiary to melodic intervals are the differences of melodic
intervals (as the melodic intervals themselves are differences of consonances).
Although not exactly melodic, these intervals —diesis, comma and limma—find
application in melodic modulation. See Harmonice mundi, Book 3, chapter 4
«kaw 6, p. 132-133). See also M. Dickreiter (see notes for chapter 12 , note 15),
p. 153.

6. This hymn is a paraphrase of Psalm 8 into which Kepler has worked a
reference to the five Platonic solids. See F. Seck, Johannes Kepler als Dichter, in
Internationales Kepler-Symposium (see Introduction, note 4), pp. 427-450,
especially p. 431 and p. 443.

blank page



Ptolemy found that a simple eccentric sufficed to represent the apparent motion
of the sun about the earth. For the representation of the motions of the superior
planets he introduced the device known as the equant. Copernicus rejected the
equant as inconsistent with the principles of astronomy and found:that the mo-
tions of all the planets except Mercury could be represented by two geometrically
equivalent constructions, which may be described as eccentric-on-eccentric and
epicycle-on-eccentric, respectively. Mercury required a more complicated com-
bination of circles. Maestlin based his calculation of the distances of the planets
from the sun on Copernicus’s Mercury theory and eccentric-on-eccentric
representations for the other planets. Kepler took his planetary distances directly
from Copernicus and sought a physical basis for the Ptolemaic equant.
The Eccentric

Let E (fig. 3) be the center of the earth’s orbit and C a point on the line of ap-
sides of the planet, such that ED = 2e, where 2e is the eccentricity as defined by
Ptolemy; that is, the eccentricity of the center of uniform motion. Then the ec-
centric, with center D, is taken to be the path of the planet.

The Equant

In the case of the superior planets Ptolemy found that the planet moved not on
the eccentric with center D (fig. 2) but on an equal eccentric (the deferent) with
center C, where EC=CD=e. Thus the eccentricity of the center of equal
distances C is half the eccentricity of the center of equal angular motion D, so
that the eccentricity may be said to be bisected. In this representation, the eccen-
tric circle with center D is known as the equant circle and its center as the equant
point. Both the circle and its center are often referred to simply as the equant.

Eccentric-on-eccentric

This representation is called by Copernicus eccentri eccentrus, eccentricus ec-
centrici and eccentreccentricus. 9

Let E (fig. 3) be the center of the earth’s orbit and C a point on the line of ap-
sides of the planet such that EC= € where € is three-quarters of the eccentricity
of the planetary orbit considered as a simple eccentric (i.e., three-quarters of the
eccentricity of the equant in the Ptolemaic theory). With center C and radius 4 €
construct the small eccentric FG. Then with center F and radius a construct the
large eccentric LM.

Suppose that initially the planet is at L. As the planet moves uniformly on the
large eccentric, the center F moves along the small eccentric in the same sense and
with twice the angular velocity. It follows that, when the planet is in apogee at L,
the center of the large eccentric is at F and EL=a+ % €; when the planet is in
perigee at M, the center of the large eccentric is again at Fand EM=a—% e. But
when the planet is in the mean distances, the center of the large eccentric is at G.
The path of the planet is nearly circular. (Cf. p. 252, fig 2.)

In this representation, one-quarter of the eccentricity is assigned to the small
eccentric and three-quarters to the large eccentric. By this distribution of the ec-
centricity, Copernicus was able to approximate the Ptolemaic theory of the in-
equality of the planet's motion without having to depart from the principle of
uniform circular motion by postulating an equant.

Fig. 5

Appendix



Abbreviations

K G W = Johannes Kepler, Gesammelte Werke, Munich, 1937—.

KFR = E, Preuss (ed.) Kepler Festschrift 1971, Regensburg, 1971.

KLC = Johannes Kepler, Werk und Leistung, Linz, 1971.

KSW =F. Krafft, K. Meyer, B. Sticker (eds.) internationales Kepler-
Symposium, Weil der Stadt 1971, Hildesheim, 1973.

Works by Kepler

Joannis Kepleri astronomi opera omnia, edited by C. Frisch, with commentary in
Latin, Frankfurt-Erlangen, 1858-1871.

Johannes Kepler, Gesammelte Werke, edited by Walther von Dyck, Max Caspar,
Franz Hammer and Martha List, with commentary in German, Munich,

. Das Weltgeheimnis, translated with commentary by Max Caspar, Augsburg,
1923 and Munich-Berlin, 1936.
. Neue Astronomie, translated with commentary by Max Caspar, Munich-
Berlin, 1929.
.. Weltharmonik, translated with commentary by Max Caspar, Munich-Berlin,
1939, reprinted Darmstadt, 1967.
. Conversation with Galileo’s sidereal messenger, translated with commentary
by E. Rosen, New York, 1965.
. The six-cornered snowflake, translated by C. Hardie, Oxford, 1966.
.L'Etrenne ou la neige sexangulaire, translated by R. Halleux, Paris, 1975.
| Somnium, translated with commentary by E. Rosen, Madison-London, 1967.
. .Selbstzeugnisse, edited with introduction by Franz Hammer, translated by
Esther Hammer, Stuttgart-Bad Cannstatt, 1971.

Johannes Kepler in seinen Briefen, selections in translation, edited by Max

Caspar and Walther von Dyck, Munich-Berlin, 1930.

Bibliographical Works

Max Caspar, Bibliographia Kepleriana, Munich 1936; second edition, revised by
Martha List, Munich, 1968.

Martha List, Bibliographia Kepleriana, 1967-1975, in Vistas in astronomy, 18

Other Works.

E. J. Aiton, Kepler’s second law of planetary motion, Isis, 60 (1969), 75-90.
. Johannes Kepler and the Astronomy without hypotheses, Japanese Studies in
the history of science, 14 (1975), 49-71.
. Johannes Kepler in the light of recent research, History of science, 14 (1976),
. Johannes Kepler and the ‘Mysterium Cosmographicum,’ Sudhoffs Archiv, 61
Aristotle, De caelo (Loeb Classical Library).
-Posterior analytics (Loeb Classical Library).
H. Balmer, Keplers Beziehungen zu Jost Biirgi und anderen Schweizern, in KLC.
Carola Baumgardt, Johannes Kepler: life and letters, New York, 1951.
A. Beer and P. Beer (eds.), Kepler. Four hundred years. A special volume of
Vistas in astronomy (vol. 18) containing the Proceedings of Conferences held
in honor of Johannes Kepler, 1975.

Bibliography

Tycho Brahe, Opera omnia, edited by J. L. E. Dreyer, Copenhagen, 1913-1929.

Ruth Breitsohl-Klepser, Heiliger ist mir die Wahrheit Johannes Kepler, edited by
Martha List, Stuttgart, 1976.

G. Buchdahl, Methodological aspects of Kepler’s theory of refraction, in KSW.

K. H. Burmeister, Georg Joachim Rheticus, Wiesbaden, 1967-1968.

M. Caspar, Kepler, translated by C. Doris Hellman, London and New York,

I. B. Cohen, Dynamics, the key to the new science of the seventeenth century,
Acta historiae rerum naturalium necnon technicarum, Special Issue No. 3
(Prague, 1967), pp. 83-100.

..-Kepler’s century, Vistas in astronomy, 18 (1975), 3-36.

N. Copernicus, De revolutionibus orbium coelestium, facsimile of Kepler’s copy,
with introduction by Johannes Miiller, New York and London, 1965.

..On the revolutions of the heavenly spheres, translated with introduction and
notes by A. M. Duncan, London, Vancouver and New York, 1976.
...On the revolutions, translated by Edward Rosen, Warsaw and London, 1978.
ficolai Cusae Cardinalis Opera, Paris, 1514; reprinted, Frankfurt am Main,

M. Dickreiter, Dur und Moll in Keplers Musiktheorie, in KLC.

..-Der Musiktheoretiker Johannes Kepler, Bern and Munich, 1973.

P. Duhem, Le systéme du monde, Paris, 1913-1959.

M. Ficino, Théologie platonicienne, translated by R. Marcel, Paris, 1964-1970.

Walther Gerlach and Martha List, Johannes Kepler, Dokumente zu Leben und
Werk, Munich, 1971.

O. Gingerich, Johannes Kepler and the new astronomy, Quarterly Journal of the
Royal Astronomical Society, 13 (1972), 346-373.

.--The origins of Kepler’s third law, Vistas in astronomy, 18 (1975), 595-601.

...Kepler’s treatment of redundant observations, in KSW.

A. Grafton, Michael Maestlin’s account of Copernican planetary theory, Pro-
ceedings of the American Philosophical Society, 117 (1973), 523-550.

R. Haase, Marginalien zum 3. Keplerschen Gesetz, in KFR.

F. Hammer, Die Astrologie des Johannes Kepler, Sudhoffs Archiv, 55 (1971),

W. Hartner, Oriens, occidens, Hildesheim, 1968.

C. Doris Hellman, The comet of 1577: its place in the history of astronomy, New
York, 1971.

R. Hooykaas, Humanisme, science et réforme: Pierre de la Ramée, Leiden, 1968.

J. Hiibner, Die Theologie Johannes Keplers zwischen Orthodoxie und Natur-
wissenschaft, Tiibingen, 1975.

H. Hugonnard-Roche, E. Rosen and J. P. Verdet, Introductions a l'astronomie
de Copernic, Paris, 1975.

Johannes Kepler — Werk und Leistung, Katalog der Ausstellung Linz 19 Juni bis
29 August 1971, Linz, 1971.

Kepler und Tiibingen, Tiibingen Kataloge Nummer 13, published by the
Kulturamt der Stadt Tiibingen, 1971.

U. Klein, Johannes Keplers Bemihungen um die Harmonieschriften des
Ptolemaios und Porphyrios, in KLC.

A. Koestler, The Sleepwalkers, London, 1968.

A. Koyré, From the closed world to the infinite universe, Baltimore, 1957.

.- The astronomical revolution, translated by R. E. W. Maddison, Paris, Lon-
don and New York, 1973.

F. Krafft, K. Meyer and B. Sticker (eds.), Internationales Kepler-Symposium
Weil der Stadt 1971 (= Arbor scientiarum, Reihe A, Band 1), Hildesheim,


Bibliography

F. Krafft, Physikalische Realitat oder mathematische Hypothese? Philosophia
naturalis, 14 (1973), 243-275.

-Johannes Keplers Beitrag zur Himmelsphysik, in KSW.

. Maestlin, Ephemerides novae ab annos 1577 ad annum 1590, Tiibingen, 1580.

: . Disputatio de eclipsibus solis et lunae, Tibingen, 1596.

D. Mahnke, Unendliche Sphdre und Allmittelpunkt, Halle 1937; reprinted
Stuttgart-Bad Cannstatt, 1966.

J. Mittelstrass, Die Rettung der Phinomene. Ursprung und Geschichte eines an-
tiken Forschungsprinzips, Berlin, 1962.

.--Neuzeit und Aufklérung, Berlin, 1970.

- Methodological elements of Keplerian astronomy, Studies in history and

Philosophy of science, 3 (1972), 203-232.
P, Moesgaard, The 1717 Egyptian years and the Copernican theory of preces-
sion, Centaurus, 13 (1968), 120-138.


H. Monantholius, Aristotelis mechanica Graeca, emendate, Latina facta, & com-
mentariis illustrata, Paris, 1599.

A. Nitschke, Keplers Staats- und Rechtslehre, in KSW.

H. M. Nobis, Ropé und nutus in Keplers Astronomie, in KFR.

Florian de Ocampo, De /a Cordnica de Espatia, continued by Ambrosio de
Morales, Alcala de Henares, 1574-1586.

W. Petri, Die betrachtende Kreatur in trinitarischen Kosmos, in KFR.

Plato, Republic (Loeb Classical Library).

.. . Timaeus (Loeb Classical Library).

Timaeus Platonis, sive de universitate, interpretibus M. Tullio Cicerone &
Chalcidio, una cum eius docta explanatione, Paris, 1563.

Plutarch, De facie quae in orbe lunae apparet (Moralia, vol. 12, Loeb Classical
Library).

F. D. Prager, Kepler als Erfinder, in KSW.

E. Preuss (ed.), Kepler Festschrift 1971 (= Acta Albertina Ratisbonensia, Band
32), Regensburg, 1971.

Proclus, Commentaire sur le Timée, translated by A. J. Festugiére, Paris,

L. Prowe, Nicolaus Coppernicus, Berlin, 1883-1884,

Ptolemy, Tetrabiblos (Loeb Classical Library).

..-Handbuch der Astronomie, transiation of the Almagest by K. Manitius,

reprinted Leipzig, 1963.
. R. Ravetz, Astronomy and cosmology in the achievement of Nicolaus Coper-
nicus, Warsaw, 1965.

. Reinhold, Prutenicae tabulae coelestium motuum, Wittenberg, 1585.

. Rosen, Three Copernican treatises, New York, 1971.

.-Kepler and the Lutheran attitude to the Copernican hypothesis, in KLC.

- L. Russell, Kepler and scientific method, Vistas in astronomy, 18 (1975),


G. Simon, Kepler’s astrology: the direction of a reform, Vistas in astronomy, 18

Simplicius, /n Aristotelis de caelo commentaria (= Commentaria in Aristotelem
Graeca, vol. 7), edited by J. L. Heiberg, Berlin, 1894.

B. Sticker, Johannes Kepler — homo iste, in KSW.

N. Swerdlow, The derivation of the first draft of Copernicus’s planetary theory: a
translation of the Commentariolus with commentary, Proceedings of the
American Philosophical Society, 117 (1973), 423-512.

D. P. Walker, Kepler’s celestial music, Journal of the Warburg and Courtauld In-
stitutes, 30 (1967), 228-250.

Oo fie)

Bibliography

R. S. Westman, Kepler’s theory of hypothesis and the realist dilemma, in KSW.

...The Melanchthon circle, Rheticus and the Wittenberg interpretation of the
Copernican theory, Isis, 66 (1975), 165-193.

. The comet and the cosmos: Kepler, Mastlin and the Copernican hypothesis, in
J. Dobrzycki (ed.), The reception of Copernicus’ heliocentric theory, Dor-
drecht, 1972.

C. Wilson, Kepler’s derivation of the elliptical path, Isis, 59 (1968), 5-25.

---How did Kepler discover his first two laws? Scientific American, 226 (1972),


H. A. Wolfson, The problem of the souls of the spheres from the Byzantine com-
mentaries on Aristotle through the Arabs and St. Thomas to Kepler, Dumbar-
ton Oaks papers, No. 16, Washington, 1962, pp. 65-93.

E, Zinner, Leben und Wirken des Johannes Miiller von KOnigsberg genannt
Regiomontanus, Munich, 1938.




accidental proof, 75

Aiton, E.J., 7, 9, 30, 31, 253, 259

Alfonso, 22, 75

Alfonsine tables, 83, 235, 236

‘Ambrosio de Morales, 233, 261

America, 51

Anaxagoras, 169

anima movers, 18, 22, 23, 27, 199, 203. See
also solar force

a posteriori derivation, 17, 77, 97

a priori reasons (or archetypal causes), 8, 9,
236-40, 247; confirmed by experience, 24;
for eccentricities, 187, 247; for har-
monies, 141; hypothetical nature of, 24,
237, 238, 243; for number, dimensions
and arrangements of planetary orbits, 9,
18, 21, 236; for order of polyhedra, 24,
238; for peculiarity of Mercury, 25, 26,
175, 247; for planetary distances, 20;
source in geometrical relations, 19

Aristarchus, 207

Aristotle, 21, 31, 77, 95, 109, 125, 127, 129,
169, 197, 234-36, 240, 250, 259; per se
rule, 235

artifices, 13. See also practitioners

aspects, astrological, 25, 135, 137, 145, 147,
195, 240-42; efficacy of, 25, 145, 195,
241, 242; origin of, 147, 241; properties
of, 137, 241; relation to musical har-
monies, 25, 135, 137, 241, 242

astrology, 24, 25, 171, 195, 239; influences
of the heavens, 171, 195; natural powers
of the planets, 24, 115-19, 195, 239

atmospheric refraction, 195, 248

Averroes, 169

Bacon, R., 15

Balmer, H., 243, 259

Baumgardt, C., 9, 259

Birkenmajer, L.A., 14

Bodin, J., 236

Brahe, Tycho, 10, 20-27, 39, 45, 59, 61, 77,

Buchdahl, G., 31, 260

Burmeister, K.H., 260


c

‘Campanus de Novara, 149-51, 238, 243

Cardanus, 241

Caspar, M., 13, 259, 260

Charles V, 57, 61, 233

Cicero, 23, 45, 69, 93, 193, 234, 237

Cohen, 1.B., 10, 247, 260

colures, solsticial and equinoctial, 83, 236

comets, 10, 87

Copernicus, N., 17-30, 49, 75-107, 155-59,

Copernican system, 7-29, 49, 59-69, 75,
227-31, 233-38, 255-58; distances of plan-
ets in, 8, 19-28, 157-59, 177-79, 197,
209-11; explanatory power of, 17, 235;
motion of the earth in, 17, 235; perfect
numbers in, 10; physical truth of, 235;
reconciliation of, with the Bible, 9, 19-22,
75, 85, 235; representation of orbits in,
20, 255-58; unknown to Pythagoras, 238;
Wittenberg interpretation of, 235

corpus, 14, 92, 93

creation, idea of, 17-24, 49, 53, 63, 93-99,

Crombie, A.C., 15

Crusius, M., 20, 30

Cusanus, 19, 23, 93, 237, 260


Dasypodius, C., 147, 243

demonstro and demonstratio, 14

Dickreiter, M., 240, 242, 243, 260

Diogenes Laertius, 169, 246

diurnal motion, 85, 91, 236

Dreyer, J.L.E., 237

Duhem, P., 236, 260

Duke of Wiirttemberg, 19, 22, 233, 238;
dedication of Plate III to, 22, 238

Duncan, A.M., 7, 30, 260


earth, position of, 24, 238
eclipses, mythological explanation of, 253
eccentric, 79-89, 175, 237, 244, 247, 255
Index

eccentricities, 20-27, 79-91, 159-61, 175-95,
221, 244-48, 251, 255; archetypal causes
of, 27, 187, 215, 247; errors in, 26, 181,

eccentric-on-eccentric, 20, 215, 244, 251,

eighth sphere, 83, 236; trepidation of, 83

epicycle, 29, 79-89, 165, 175, 187, 207, 215,

epicycle-on-eccentric, 20, 215, 244, 251,

equant, 22, 29, 187, 215-19, 244, 251, 252,
255; physical explanation of, 215-19

equant-type theories, 237

equinoxes, precession of, 83, 91, 236

Euclid, 8, 71, 73, 97, 101, 149-51, 233, 238,

expressible lines, 143, 225, 242

exorno, 14

Feselius, P., 241

Ficino, M., 238, 260

Field, J.V., 7, 10

final and efficient causes, 23, 27, 29, 30

fixed stars, 95, 97, 103, 109, 139; number,
size and position of, 97, 103; sphere of,

Fois, Frangois de, 233

Frisch, O., 239, 259


Galileo, 7-10, 59, 71, 169, 233

Gerlach, W.., 233, 260

Gingerich, O., 9, 235, 237, 250, 260

Grafton, A., 30, 260

gravity, 26, 167, 171

Breat conjunctions of Jupiter and Saturn,

great orbit (orbis magnus), 73, 83, 87, 89,
161, 187, 191, 193, 236; thickness of, 191

Grosseteste, R., 15

Gruppenbach, G., 18, 19, 21

Haase, R., 250, 260

Hafenreffer, M., 21, 22, 235
Halleux, R., 259

Hammer, Esther, 30, 259
Hammer, Franz, 30, 239, 259, 260
Hardie, C., 10, 259

Hartner, W., 253, 260

Heerbrand, J., 23

Hellman, C.D., 236, 237, 260

Herwart von Hohenburg, 21, 25, 237, 238,

hexachords, 240; classification of, 240;
system of, 240

Hipparchus, 89, 185

Hooykaas, R., 31, 260

Horace, 13, 234, 247

Hiibner, J., 30, 260

Hugonnard-Roche, H., 260

inertia, Kepler’s concept of, 171, 246

inexpressible lines, 145, 225, 242

infinity of the universe, 223, 238, 240

invisible planets, 63, 234

irrational, 137, 139, 145, 223, 225, 242; am-
biguity of the term, 145, 242


Kepler, J., 7-10, 17-31, 63, 171, 209, 235-50,
255, 259-62; account of star polyhedra, 7,
10; and astrology, 24, 25, 239; astro-
nomical method of, 9, 22, 24, 30, 238;
concept of geometrical harmony, 19, 24,
25, 239; concept of inertia, 171, 246; con-
cept of tonality, 240; Latin style, 13; on
the law of intensity of light, 18, 19, 249;
political theory of, 236, 250; and
theology, 17-21, 63; works by, 7, 9, 10,

Klein, U., 239, 241, 260

Koestler, A., 260

Koyré, A., 10, 238, 240, 260

Krafft, F., 31, 249, 259-61


law of intensity of light, 18, 19, 201, 250

laws of planetary motion, 7, 9, 18, 29, 31,
205, 207, 215, 219, 250, 253; archetypal
explanation of harmonic law, 207, 250;
harmonic law, 9, 205, 207, 215, 219, 250

Lear, J., 7, 10

Lefevre d’Etaples, J., 243

List, M., 9, 30, 233, 259, 260

logarithms, Napier’s, 207

Lucretius, 15


Index


Maestlin, M., 17-29, $9, 63, 69, 79, 81, 161,
244-50, 255, 261; atmospheric refraction
observed by, 189, 248; on comet of 1577,
236; Copernican system accepted by, 17,
22, 236; distances of planets calculated
by, 19, 20, 25, 161, 183, 244, 245; letter
on Mercury, 20, 25, 193, 250; reports on
Kepler's work by, 19-21, 235

magnetism, 91, 171

Mahnke, D., 237, 261

Mercury, 25, 26, 173, 175, 191-95, 243, 247,
250, 256-58; archetypal reason for pecu-
liarity of, 25, 26, 247; astrological influ-
ence of, 191, 195; and the circle in the
octahedron-square, 25, 173, 175, 243,
247; Copernican theory of, 256-58; large
eccentricity of, 175, 247; letter of
Maestlin concerning, 20, 25, 193, 250

Mittelstrass, J., 31, 261

Moesgaard, K.P., 236, 261

Monantholius, H., 252, 253, 261

moon, similar in nature to the earth, 24, 26,

motions of the planets, 18, 20, 27-30, 41,
250; correspondence with harmonic
ratios, 187; number, extent and times of
retrogressions, 75, 235; periodic times of,
197, 199, 205, 207, 223, 225, 249; physi-
cal cause of, 18, 27, 28, 29, 30, 199-203,
234; in relation to the distances, 20, 28,
63, 197-213, 249, 250; in relation to the
zodiac, 159-63

mundus, 14

musical harmony, 19, 25, 27, 131-47, 225
239-42, 251-53; archetypal cause of, 141,
145; classification of concords, 133, 135,
143, 145, 240; consonances, 251, 252; dif-
ferences of melodic intervals, 225, 252;
melodic intervals, 141, 225, 242, 253; re-
lation with astrological aspects, 25, 242;
tempered intervals, 242; terminology,
240, 242; tonality, 240


Neoplatonism, 15

Newton, I., 9

Nicholas of Cusa. See Cusanus
ninth sphere, 83, 91, 236
Nitschke, A., 236, 250, 261
Nobis, H.M., 253, 261

numbers, 10, 19, 24, 49, 65, 71, 73, 109,
119, 121, 139, 234; counted (numerati),
24, 73, 234; counting (numerantes), 71,
139, 234; nobility (perfection) of, 10, 65,
71, 73, 119, 121; perfection (nobility) of
number three, 109; source of nobility, 24,

mutus, 250, 253

oO

orbis, 14, 15

order and arrangement of polyhedra, 24,
107-15, 123-29, 238; a priori reasons for,
238; empirical evidence for, 238

per se rule, 77, 235

Peurbach, G., 175, 185

Petri, W., 234, 240, 261

Philoponus, 257

planets, 7-9, 18, 24, 63, 115-19, 237, 239;
astrological properties of, 24, 115-19,
239; imaginary, 63-65, 234; number,
order and size of orbits of, 8, 9, 63; posi-
tion of orbits in relation to the sun, 237.
See also laws of planetary motion; mo-
tions of the planets

Plato, 8, 23, 30, 31, 43, 61, 63, 93, 97, 221,
234, 236, 238, 239, 261; construction of
the world-soul, 234

Platonic Year, 221-23, 253

Pliny, 253

Plutarch, 169, 238, 246, 261

polyhedra, the five regular, 9, 17-28, 49, 61-
234, 238-40, 253; association with ele-
ments, 99, 238; circumscribed and in-
scribed spheres of, 21, 23, 240; classifica-
tion of, 24, 105, 113, 115, 121, 149-53;
properties of, 26, 101-21, 149-53, 239;
relation with musical harmonies, 143

polyhedral hypothesis, 19-31, 69, 71, 97, 99,
209-15, 234-38, 243-53; astronomical
proof of, 155; compared with distances
derived from motions, 28, 209, 250, 251;
disagreement with the motions, 209-15;
discrepancies concerning individual
planets, 189-93; errors in calculation, 25,
26, 31, 244-48; order and arrangement of
polyhedra in, 24, 107-15, 123-29, 238; re-
lation with musical harmonies, 25, 169;
tests of, 19-28, 99, 157-63, 177, 179, 189,


Index

Pontano, G.G., 249

Porphyry, 24, 137, 147, 239, 241

practitioners, 22, 173-81, 191, 217, 234

Prager, F.D., 238, 261

Preuss, E., 259, 261

Proclus, 221, 240, 253, 261

prosthaphaeresis (equation), 15, 26, 81,

Prowe, L., 247, 261

Prutenic tables, 19, 20, 26, 161, 177-93, 221,
230, 235, 244-48, 253; unreliability of,

Ptolemy, 8, 9, 17, 20, 22, 24, 29, 37, 63, 75-
252, 255, 261; astrology and harmony,
24, 119, 135, 145, 147, 239-42; on dimen-
sion, 239; epigram ascribed to, 37, 23:
equant, 22, 29, 215-19, 244, 251, 252,
255; system of, 8, 17, 63, 79, 83, 233, 236

Pythagoras, 8, 49, 53, 61, 63, 99, 121, 149,


quantitas (quantity), 15, 72, 93, 95, 127,
131, 137; shapes, number and extension
types of, 95

Ramus, P., 29-31

Ravetz, J.R., 236, 261

Regiomontanus, 137, 234, 241

Reinhold, E., 181, 183, 193, 235, 247, 261

retrogressions, cause of, 87

Rheticus, G.J., 10, 13, 17, 20, 21, 31, 33
201, 235, 236, 248, 250; letter of (con-
cerning errors in Copernicus), 185

Rosen, E., 7, 10, 14, 30, 31, 234-36, 246,

Réslin, H., 119, 137, 241

Royal Way (via regia), 91, 129, 237

Rudolph, Emperor, 35, 45, $7, 61

Rudolphine tables, 45, 235

Russell, J.L., 31, 261

s

Scaliger, J.C., $1, 203, 221
Seck, F., 233, 253

Seneca, 61, 233

Simon, G., 239, 261


Simplicius, 31, 109, 239, 253, 261
solar force, 25, 65, 199-207, 234, 249, 250,
solids, perfect. See polyhedra
species immateriata, 15, 169, 203, 253
sphere, properties of, 101
spheres, planetary, 8, 10, 14, 15, 18-29, 49,
51, 61, 67, 79, 99, 103, 105, 149, 155-71, i
179-87, 215, 225, 246, 249-51; interpola-
tion of the polyhedra, 8, 28, 29, 157; with
lumps, 169; nature of, 167, 171; not con-
tiguous, 155; not real, 10, 20, 26, 91, 103,
159, 246; number, order and magnitude
of, 18, 20, 21, 23, 1575 souls of, 249;
thickness of, 20, 26, 155-65, 179, 181,
Stadius, G., 17, 63
Sticker, B., 30, 241, 259-61
Stirling, W., 233
Swerdlow, N., 261
symbolism, 19, 23, 24, 93, 95, 234, 237; of '
the curved and the straight, 19, 23, 24,
93, 95, 237; of the Trinity, 19, 23, 93, 95,

Tampach, G., 29

telescope, 8, 10, 169

tenth sphere, 85, 236

Terence, 234

Turriano of Cremona, J., 57, 233

v

vicarious hypothesis, 237
Virgil, 63, 234, 239

w

Walden, J.H., 9
Walderman, W., 235

Walker, D.P., 240, 261 1
Wallis, C.G., 9

Walther of Nuremberg, B., 193

Westman, R.S., 31, 235, 236, 262

Wilson, C., 31, 253, 262

Wolfson, H.A., 249, 262

Index

Zarlino, G., 240

Zinner, E., 241, 262

zodiac, 24, 25, 83, 125, 129, 131, 137, 139,
159-67, 187, 223, 241; division of, 25,
131, 223, 241; origin of, 24, 125, 129;
position of planets on, 159-63


Maristella Lorch and A. Kent Hieatt

Michael Mahoney

Mary J. Gregor and Robert E, Anchor

Aloysius Patrick Martinich; with an inquiry into
Hobbes’s Theory of Language, Speech, and Reasoning
by Isabel C. Hungerland and George R. Vick

Gordon Treash

Eitenne Bonnot de Condillac, LA LOGIQUE (LOGIC).
W.R. Albury


Martin Goodman

E.J. Aiton and A.M. Duncan


Marsilio Ficino, DE VITA. Carol Kaske



Clyde Miller





an
w
a
a
Ze
2g 2
es Oo
a
Po r)
avo z
ES nig
wet BS
5 On
S et
8 Sh
2 od
= ae
r=) Bs
se
ie OF
ee}
idee


Ny fy

Wy
Wi


Le Vis


ie

eae fs
Ze ap




i Sa es :
Keren mi vars pis, SPECTATOR, obymp. z
: ntea. gue menguam unre figura ' Fb

) Ma neterum diffantia quanta.
Stang qu

we infer

Orbes; Paclde a quingia

SE . “ docent
Quai bene con uem aE quot do
J nzevs olim_

| Tradidit , Autorn minc |\
i 4G monS Prat opin!)
Scilcee® that Fant se mits

: nere RES

e )
| en TECCIACO non une

Liu de BE.


Cobra Daphorig Lis}

Li Sried. FF

gma Copen,


cE

: i
“Bcd bas “Titbinge Georgiin Grippen bach AG.


oc. tae 6
subacz ‘Prismsim Corpien veg ilenre Geormet rite.
P. Cobet Fr ymin pin wast ey aie. stale
kK Share B. is
- “Fetme drow Sue Bre mise. axteriar He,
rn Betingent ites maim
inter fr fam Cau
Bade cactien Veep Riphins, ew
Drbagnion ‘ohm eltrine cis Dane “fee
st Diba egg Ee
0. Jerardron ah arbe Magne <fSpeiam Pues

zen: difdantian, madicans

% here ¥ es
(Soe Bleditim se (Ba trimm Viuery

enn obile. J

baat oy a\end{document}
