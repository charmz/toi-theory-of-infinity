\documentclass{article}
\usepackage[utf8]{inputenc}
\begin{document}

HARMONIES OF THE
WORLD
BY
JOHANNES KEPLER
TRANSLATED BY CHARLES GLENN WALLIS

1939


Harmonies Of The World By Johannes Kepler.
This edition was created and published by Global Grey
©GlobalGrey 2019

globalgreyebooks.com


CONTENTS
Proem. The Harmonies Of The World
1. Concerning The Five Regular Solid Figures
2. On The Kinship Between The Harmonic Ratios And The Five Regular
Figures
3. A Summary Of Astronomical Doctrine Necessary For Speculation Into
The Celestial Harmonies
4. In What Things Having To Do With The Planetary Movements Have
The Harmonic Consonances Been Expressed By The Creator, And In
What Way?
5. In The Ratios Of The Planetary Movements...
6. In The Extreme Planetary Movements The Musical Modes Or Tones
Have Somehow Been Expressed
7. The Universal Consonances Of All Six Planets, Like Common FourPart Counterpoint, Can Exist
8. In The Celestial Harmonies Which Planet Sings Soprano, Which Alto,
Which Tenor, And Which Bass?
9. The Genesis Of The Eccentricities In The Single Planets From The
Procurement Of The Consonances Between Their Movements
10. Epilogue Concerning The Sun, By Way Of Conjecture


1

PROEM. THE HARMONIES OF THE WORLD
by Johannes Kepler
Concerning the very perfect harmony of the celestial movements, and
the genesis of eccentricities and the semidiameters, and the periodic
times from the same.
After the model of the most correct astronomical doctrine of today, and
the hypothesis not only of Copernicus but also of Tycho Brahe, whereof
either hypotheses are today publicly accepted as most true, and the
Ptolemaic as outmoded.
I commence a sacred discourse, a most true hymn to God the Founder,
and I judge it to be piety, not to sacrifice many hecatombs of bulls to
Him and to burn incense of innumerable perfumes and cassia, but first
to learn myself, and afterwards to teach others too, how great He is in
wisdom, how great in power, and of what sort in goodness. For to wish
to adorn in every way possible the things that should receive
adornment and to envy no thing its goods—this I put down as the sign
of the greatest goodness, and in this respect I praise Him as good that
in the heights of His wisdom He finds everything whereby each thing
may be adorned to the utmost and that He can do by his unconquerable
power all that he has decreed.
GALEN, on the Use of Parts. Book III
PROEM
[268] As regards that which I prophesied two and twenty years ago
(especially that the five regular solids are found between the celestial
spheres), as regards that of which I was firmly persuaded in my own
mind before I had seen Ptolemy's Harmonies, as regards that which I
promised my friends in the title of this fifth book before I was sure of the
thing itself, that which, sixteen years ago, in a published statement, I
insisted must be investigated, for the sake of which I spent the best part
of my life in astronomical speculations, visited Tycho Brahe, [269] and
took up residence at Prague: finally, as God the Best and Greatest, Who
had inspired my mind and aroused my great desire, prolonged my life


2

and strength of mind and furnished the other means through the
liberality of the two Emperors and the nobles of this province of Austriaon-the-Anisana: after I had discharged my astronomical duties as much
as sufficed, finally, I say, I brought it to light and found it to be truer than
I had even hoped, and I discovered among the celestial movements the
full nature of harmony, in its due measure, together with all its parts
unfolded in Book III—not in that mode wherein I had conceived it in my
mind (this is not last in my joy) but in a very different mode which is also
very excellent and very perfect. There took place in this intervening time,
wherein the very laborious reconstruction of the movements held me in
suspense, an extraordinary augmentation of my desire and incentive for
the job, a reading of the Harmonies of Ptolemy, which had been sent to
me in manuscript by John George Herward, Chancellor of Bavaria, a
very distinguished man and of a nature to advance philosophy and every
type of learning. There, beyond my expectations and with the greatest
wonder, I found approximately the whole third book given over to the
same consideration of celestial harmony, fifteen hundred years ago. But
indeed astronomy was far from being of age as yet; and Ptolemy, in an
unfortunate attempt, could make others subject to despair, as being one
who, like Scipio in Cicero, seemed to have recited a pleasant Pythagorean
dream rather than to have aided philosophy. But both the crudeness of
the ancient philosophy and this exact agreement in our meditations,
down to the last hair, over an interval of fifteen centuries, greatly
strengthened me in getting on with the job. For what need is there of
many men? The very nature of things, in order to reveal herself to
mankind, was at work in the different interpreters of different ages, and
was the finger of God—to use the Hebrew expression; and here, in the
minds of two men, who had wholly given themselves up to the
contemplation of nature, there was the same conception as to the
configuration of the world, although neither had been the other's guide
in taking this route. But now since the first light eight months ago, since
broad day three months ago, and since the sun of my wonderful
speculation has shone fully a very few days ago: nothing holds me back. I
am free to give myself up to the sacred madness, I am free to taunt
mortals with the frank confession that I am stealing the golden vessels of
the Egyptians, in order to build of them a temple for my God, far from
the territory of Egypt. If you pardon me, I shall rejoice; if you are
enraged, I shall bear up. The die is cast, and I am writing the book—


3

whether to be read by my contemporaries or by posterity matters not.
Let it await its reader for a hundred years, if God Himself has been ready
for His contemplator for six thousand years.
The chapters of this book are as follows:
1. Concerning the five regular solid figures.
2. On the kinship between them and the harmonic ratios.
3. Summary of astronomical doctrine necessary for speculation into the
celestial harmonies.
4. In what things pertaining to the planetary movements the simple
consonances have been expressed and that all those consonances which
are present in song are found in the heavens.
5. That the clefs of the musical scale, or pitches of the system, and the
genera of consonances, the major and the minor, are expressed in certain
movements.
6. That the single musical Tones or Modes are somehow expressed by the
single planets.
7. That the counterpoints or universal harmonies of all the planets can
exist and be different from one another.
8. That four kinds of voice are expressed in the planets: soprano,
contralto, tenor, and bass.
9. Demonstration that in order to secure this harmonic arrangement,
those very planetary eccentricities which any planet has as its own, and
no others, had to be set up.
10. Epilogue concerning the sun, by way of very fertile conjectures.
Before taking up these questions, it is my wish to impress upon my
readers the very exhortation of Timaeus, a pagan philosopher, who was
going to speak on the same things: it should be learned by Christians
with the greatest admiration, and shame too, if they do not imitate him:
Ἀλλ᾽ ὦ Σὼκρατες, τοῦτο γε δὴ πντες ὅσοι καὶ κατὰ βραχὺ σωφροσύνης
μετέχουσιν, ἐπὶ πασῇ ὁρμῇ καὶ σμίκροῦ καὶ μεγάλου πράγματος θεὸν ἀει
που καοῦσιν. ἡμᾶς δὲ τοὺς περὶ τοῦ πὰντος λόγους ποιεῖσθαι πῃ


4

μέλλοντας. . . , εἰ μὴ πανταπασι παραλλάττομεν, ἀνάγκη θεοὺς τε καὶ
θεὰς ἐπικαλουμενους εὔχεσθαι πάντα, κατὰ νοῦν ἐκεῖνοις μἑν μάλιστα,
ἑπομένως δέ ἠμῖν εἰπεῖν. For truly, Socrates, since all who have the least
particle of intelligence always invoke God whenever they enter upon
any business, whether light or arduous; so too, unless we have clearly
strayed away from all sound reason, we who intend to have a
discussion concerning the universe must of necessity make our sacred
wishes and pray to the Gods and Goddesses with one mind that we may
say such things as will please and be acceptable to them in especial and,
secondly, to you too.


5

1. CONCERNING THE FIVE REGULAR SOLID
FIGURES
[271] It has been said in the second book how the regular plane figures
are fitted together to form solids; there we spoke of the five regular
solids, among others, on account of the plane figures. Nevertheless their
number, five, was there demonstrated; and it was added why they were
designated by the Platonists as the figures of the world, and to what
element any solid was compared on account of what property. But now,
in the anteroom of this book, I must speak again concerning these
figures, on their own account, not on account of the planes, as much as
suffices for the celestial harmonies; the reader will find the rest in
the Epitome of Astronomy, Volume II, Book Iv.
Accordingly, from the Mysterium Cosmographicum, let me here briefly
inculcate the order of the five solids in the world, whereof three are
primary and two secondary. For the cube (1) is the outmost and the most
spacious, because firstborn and having the nature [rationem] of a whole,
in the very form of its generation. There follows the tetrahedron (2), as if
made a part, by cutting up the cube; nevertheless it is primary too, with
a solid trilinear angle, like the cube. Within the tetrahedron is
the dodecahedron (3), the last of primary figures, namely, like a solid
composed of parts of a cube and similar parts of a tetrahedron, i.e., of
irregular tetrahedrons, wherewith the cube inside is roofed over. Next in
order is the icosahedron (4) on account of its similarity, the last of the
secondary figures and having a plurilinear solid angle.
The octahedron (6) is inmost, which is similar to the cube and the first of
the secondary figures and to which as inscriptile the first place is due,
just as the first outside place is due to the cube as circumscriptile.
[272] However, there are as it were two noteworthy weddings of these
figures, made from different classes: the males, the cube and the
dodecahedron, among the primary; the females, the octahedron and the
icosahedron, among the secondary, to which is added one as it were
bachelor or hermaphrodite, the tetrahedron, because it is inscribed in
itself, just as those female solids are inscribed in the males and are as it


6

were subject to them, and have the signs of the feminine sex, opposite
the masculine, namely, angles opposite planes. Moreover, just as the
tetrahedron is the element, bowels, and as it were rib of the male cube,
so the feminine octahedron is the element and part of the tetrahedron in
another way; and thus the tetrahedron mediates in this marriage.

The main difference in these wedlocks or family relationships consists in
the following: the ratio of the cube is rational. For the tetrahedron is one
third of the body of the cube, and the octahedron half of the tetrahedron,
one sixth of the cube; while the ratio of the dodecahedron's wedding
is irrational [ineffabilis] but divine.
The union of these two words commands the reader to be careful as to
their significance. For the word ineffabilis here does not of itself denote
any nobility, as elsewhere in theology and divine things, but denotes an
inferior condition. For in geometry, as was said in the first book, there
are many irrationals, which do not on that account participate in a divine
proportion too. But you must look in the first book for what the divine
ratio, or rather the divine section, is. For in other proportions there are
four terms present; and three, in a continued proportion; but the divine
requires a single relation of terms outside of that of the proportion itself,
namely in such fashion that the two lesser terms, as parts make up the


7

greater term, as a whole. Therefore, as much as is taken away from this
wedding of the dodecahedron on account of its employing an irrational
proportion, is added to it conversely, because its irrationality approaches
the divine. This wedding also comprehends the solid star too, the
generation whereof arises from the continuation of five planes of the
dodecahedron till they all meet in a single point. See its generation in
Book
Lastly, we must note the ratio of the spheres circumscribed around them
to those inscribed in them: in the case of the tetrahedron it is rational,
100,000 : 33,333 or 3 : 1; in the wedding of the cube it is irrational, but
the radius of the inscribed sphere is rational in square, and is itself the
square root of one third the square on the radius [of the circumscribed
sphere], namely 100,000 : 57,735; in the wedding of the dodecahedron,
clearly irrational, 100,000 : 79,465; in the case of the star, 100,000 :
52,573, half the side of the icosahedron or half the distance between two
rays.


8

2. ON THE KINSHIP BETWEEN THE HARMONIC
RATIOS AND THE FIVE REGULAR FIGURES
[273] This kinship [cognatio] is various and manifold; but there are four
degrees of kinship. For either the sign of kinship is taken from the
outward form alone which the figures have, or else ratios which are the
same as the harmonic arise in the construction of the side, or result from
the figures already constructed, taken simply or together; or, lastly, they
are either equal to or approximate the ratios of the spheres of the figure.
In the first degree, the ratios, where the character or greater term is 3,
have kinship with the triangular plane of the tetrahedron, octahedron,
and icosahedron; but where the greater term is 4, with the square plane
of the cube; where 5, with the pentagonal plane of the dodecahedron.
This similitude on the part of the plane can also be extended to the
smaller term of the ratio, so that wherever the number 3 is found as one
term of the continued doubles, that ratio is held to be akin to the three
figures first named: for example, 1 : 3 and 2 : 3 and 4 : 3 and 8 : 3, et
cetera; but where the number is 5, that ratio is absolutely assigned to the
wedding of the dodecahedron : for example, 2 : 5 and 4 : 5 and 8 : 5, and
thus 3 : 5 and 3 : 10 and 6 : 5 and 12 : 5 and 24 : 5. The kinship will be
less probable if the sum of the terms expresses this similitude, as in 2 : 3
the sum of the terms is equal to 5, as if to say that 2 : 3 is akin to the
dodecahedron. The kinship on account of the outward form of the solid
angle is similar: the solid angle is trilinear among the primary figures,
quadrilinear in the octahedron, and quinquelinear in the icosahedron.
And so if one term of the ratio participates in the number 3, the ratio will
be connected with the primary bodies; but if in the number 4, with the
octahedron; and finally, if in the number 5, with the icosahedron. But in
the feminine solids this kinship is more apparent, because the
characteristic figure latent within follows upon the form of the angle: the
tetragon in the octahedron, the pentagon in the icosahedron; and so 3 : 5
would go to the sectioned icosahedron for both reasons.
The second degree of kinship, which is genetic, is to be conceived as
follows: First, some harmonic ratios of numbers are akin to one wedding


9

or family, namely, perfect ratios to the single family of the cube;
conversely, there is the ratio which is never fully expressed in numbers
and cannot be demonstrated by numbers in any other way, except by a
long series of numbers gradually approaching it: this ratio is
called divine, when it is perfect, and it rules in various ways throughout
the dodecahedral wedding. Accordingly, the following consonances begin
to shadow forth that ratio: 1 : 2 and 2 : 3 and 2 : 3 and 5 : 8. For it exists
most imperfectly in 1 : 2, more perfectly in 5 : 8, and still more perfectly
if we add 5 and 8 to make 13 and take 8 as the numerator, if this ratio
has not stopped being harmonic.
Further, in constructing the side of the figure, the diameter of the globe
must be cut; and the octahedron demands its bisection, the cube and the
tetrahedron its trisection, the dodecahedral wedding its quinquesection.
Accordingly, the ratios between the figures are distributed according to
the numbers which express those ratios. But the square on the diameter
is cut too, or the square on the side of the figure is formed from a fixed
part of the diameter. And then the squares on the sides are compared
with the square on the diameter, and they constitute the following ratios
: in the cube 1 : 3, in the tetrahedron 2 : 3, in the octahedron 1 : 2.
Wherefore, if the two ratios are put together, the cubic and the
tetrahedral will give 1 : 2; the cubic and the octahedral, 2 : 3; the
octahedral and the tetrahedral, 3 : 4. The sides in the dodecahedral
wedding are irrational.
Thirdly, the harmonic ratios follow in various ways upon the already
constructed figures. For either the number of the sides of the plane is
compared with the number of lines in the total figure; [274] and the
following ratios arise : in the cube, 4 : 12 or 1 : 3; in the tetrahedron 3 : 6
or 1 : 2; in the octahedron 3 : 12 or 1 : 4; in the dodecahedron 5 : 30 or 1 :
6; in the icosahedron 3 : 30 or 1 : 10. Or else the number of sides of the
plane is compared with the number of planes; then the cube gives 4 : 6 or
2 : 3, the tetrahedron 3 : 4, the octahedron 3 : 8, the dodecahedron 5 : 12,
the icosahedron 3 : 20. Or else the number of sides or angles of the plane
is compared with the number of solid angles, and the cube gives 4 : 8 or 1
: 2, the tetrahedron 3 : 4, the octahedron 3 : 6 or 1 : 2, the dodecahedron
with its consort 5 : 20 or 3 : 12 (i.e., 1 : 4). Or else the number of planes is
compared with the number of solid angles, and the cubic wedding gives 6
: 8 or 3 : 4, the tetrahedron the ratio of equality, the dodecahedral


10

wedding 12 : 20 or 3 : 5. Or else the number of all the sides is compared
with the number of the solid angles, and the cube gives 8 : 12 or 2 : 3, the
tetrahedron 4 : 6 or 2 : 3, and the octahedron 6 : 12 or 1 : 2, the
dodecahedron 20 : 30 or 2 : 3, the icosahedron 12 : 30 or 2 : 5.
Moreover, the bodies too are compared with one another, if the
tetrahedron is stowed away in the cube, the octahedron in the
tetrahedron and cube, by geometrical inscription. The tetrahedron is one
third of the cube, the octahedron half of the tetrahedron, one sixth of the
cube, just as the octahedron, which is inscribed in the globe, is one sixth
of the cube which circumscribes the globe. The ratios of the remaining
bodies are irrational.
The fourth species or degree of kinship is more proper to this work: the
ratio of the spheres inscribed in the figures to the spheres circumscribing
them is sought, and what harmonic ratios approximate them is
calculated. For only in the tetrahedron is the diameter of the inscribed
sphere rational, namely, one third of the circumscribed sphere. But in
the cubic wedding the ratio, which is single there, is as lines which are
rational only in square. For the diameter of the inscribed sphere is to the
diameter of the circumscribed sphere as the square root of the ratio 1 : 3.
And if you compare the ratios with one another, the ratio of the
tetrahedral spheres is the square of the ratio of the cubic spheres. In the
dodecahedral wedding there is again a single ratio, but an irrational one,
slightly greater than 4 : 5. Therefore the ratio of the spheres of the cube
and octahedron is approximated by the following consonances : 1 : 2, as
proximately greater, and
3 : 5, as proximately smaller. But the ratio of the dodecahedral spheres is
approximated by the consonances 4 : 5 and 5 : 6, as proximately smaller,
and 3 : 4 and 5 : 8, as proximately greater.
But if for certain reasons 1 : 2 and 1 : 3 are arrogated to the cube, the
ratio of the spheres of the cube will be to the ratio of the spheres of the
tetrahedron as the consonances 1 : 2 and 1 : 3, which have been ascribed
to the cube, are to 1 : 4 and 1 : 9, which are to be assigned to the
tetrahedron, if this proportion is to be used. For these ratios, too, are as
the squares of those consonances. And because 1 : 9 is not harmonic, 1 :
8 the proximate ratio takes its place in the tetrahedron. But by this
proportion approximately 4 : 5 and 3 : 4 will go with the dodecahedral


11

wedding. For as the ratio of the spheres of the cube is approximately the
cube of the ratio of the dodecahedral, so too the cubic consonances 1 : 2
and 2 : 3 are approximately the cubes of the consonances 4 : 5 and 3 : 4.
For 4 : 5 cubed is 64 : 125, and 1 : 2 is 64 : 128. So 3 : 4 cubed is 27 : 64,
and 1 : 3 is 27 : 81.


12

3. A SUMMARY OF ASTRONOMICAL DOCTRINE
NECESSARY FOR SPECULATION INTO THE
CELESTIAL HARMONIES
First of all, my readers should know that the ancient astronomical
hypotheses of Ptolemy, in the fashion in which they have been unfolded
in the Theoricae of Peurbach and by the other writers of epitomes, are to
be completely removed from this discussion and cast out of [275] the
mind. For they do not convey the true lay out of the bodies of the world
and the polity of the movements.
Although I cannot do otherwise than to put solely Copernicus’ opinion
concerning the world in the place of those hypotheses and, if that were
possible, to persuade everyone of it; but because the thing is still new
among the mass of the intelligentsia [apud vulgus studiosorum], and the
doctrine that the Earth is one of the planets and moves among the stars
around a motionless sun sounds very absurd to the ears of most of them:
therefore those who are shocked by the unfamiliarity of this opinion
should know that these harmonical speculations are possible even with
the hypotheses of Tycho Brahe—because that author holds, in common
with Copernicus, everything else which pertains to the lay out of the
bodies and the tempering of the movements, and transfers solely the
Copernican annual movement of the Earth to the whole system of
planetary spheres and to the sun, which occupies the centre of that
system, in the opinion of both authors. For after this transference of
movement it is nevertheless true that in Brahe the Earth occupies at any
time the same place that Copernicus gives it, if not in the very vast and
measureless region of the fixed stars, at least in the system of the
planetary world. And accordingly, just as he who draws a circle on paper
makes the writing-foot of the compass revolve, while he who fastens the
paper or tablet to a turning lathe draws the same circle on the revolving
tablet with the foot of the compass or stylus motionless; so too, in the
case of Copernicus the Earth, by the real movement of its body, measures
out a circle revolving midway between the circle of Mars on the outside
and that of Venus on the inside; but in the case of Tycho Brahe the whole
planetary system (wherein among the rest the circles of Mars and Venus


13

are found) revolves like a tablet on a lathe and applies to the motionless
Earth, or to the stylus on the lathe, the midspace between the circles of
Mars and Venus; and it comes about from this movement of the system
that the Earth within it, although remaining motionless, marks out the
same circle around the sun and midway between Mars and Venus, which
in Copernicus it marks out by the real movement of its body while the
system is at rest. Therefore, since harmonic speculation considers the
eccentric movements of the planets, as if seen from the sun, you may
easily understand that if any observer were stationed on a sun as much
in motion as you please, nevertheless for him the Earth, although at rest
(as a concession to Brahe), would seem to describe the annual circle
midway between the planets and in an intermediate length of time.
Wherefore, if there is any man of such feeble wit that he cannot grasp the
movement of the earth among the stars, nevertheless he can take
pleasure in the most excellent spectacle of this most divine construction,
if he applies to their image in the sun whatever he hears concerning the
daily movements of the Earth in its eccentric—such an image as Tycho
Brahe exhibits, with the Earth at rest.
And nevertheless the followers of the true Samian philosophy have no
just cause to be jealous of sharing this delightful speculation with such
persons, because their joy will be in many ways more perfect, as due to
the consummate perfection of speculation, if they have accepted the
immobility of the sun and the movement of the earth.
Firstly [I], therefore, let my readers grasp that today it is absolutely
certain among all astronomers that all the planets revolve around the
sun, with the exception of the moon, which alone has the Earth as its
centre: the magnitude of the moon's sphere or orbit is not great enough
for it to be delineated in this diagram in a just ratio to the rest.
Therefore, to the other five planets, a sixth, the Earth, is added, which
traces a sixth circle around the sun, whether by its own proper
movement with the sun at rest, or motionless itself and with the whole
planetary system revolving.
Secondly [II]: It is also certain that all the planets are eccentric, i.e., they
change their distances from sun, in such fashion that in one part of their
circle they become farthest away from the sun, [276] and in the opposite
part they come nearest to the sun. In the accompanying diagram three


14

circles apiece have been drawn for the single planets: none of them
indicate the eccentric route of the planet itself; but the mean circle, such
as BE in the case of Mars, is equal to the eccentric orbit, with respect to
its longer diameter. But the orbit itself, such as AD, touches AF, the
upper of the three, in one place A, and the lower circle CD, in the
opposite place D. The circle GH made with dots and described through
the centre of the sun indicates the route of the sun according to Tycho
Brahe. And if the sun moves on this route, then absolutely all the points
in this whole planetary system here depicted advance upon an equal
route, each upon his own. And with one point of it (namely, the centre of
the sun) stationed at one point of its circle, as here at the lowest,
absolutely each and every point of the system will be stationed at the
lowest part of its circle. However, on account of the smallness of the
space the three circles of Venus unite in one, contrary to my intention.

Thirdly [III]: Let the reader recall from my Mysterium
Cosmographicum, which I published twenty-two years ago, that the


15

number of the planets or circular routes around the sun was taken by the
very wise Founder from the five regular solids, concerning which Euclid,
so many ages ago, wrote his book which is called the Elements in that it
is built up out of a series of propositions. But it has been made clear in
the second book of this work that there cannot be more regular
bodies, i.e., that regular plane figures cannot fit together in a solid more
than five times.
Fourthly [IV]: As regards the ratio of the planetary orbits, the ratio
between two neighbouring planetary orbits is always of such a
magnitude that it is easily apparent that each and every one of them
approaches the single ratio of the spheres of one of the five regular
solids, namely, that of the sphere circumscribing to the sphere inscribed
in the figure. Nevertheless it is not wholly equal, as I once dared to
promise concerning the final perfection of astronomy. For, after
completing the demonstration of the intervals from Brahe's
observations, I discovered the following: if the angles of the cube [277]
are applied to the inmost circle of Saturn, the centres of the planes are
approximately tangent to the middle circle of Jupiter; and if the angles of
the tetrahedron are placed against the inmost circle of Jupiter, the
centres of the planes of the tetrahedron are approximately tangent to the
outmost circle of Mars; thus if the angles of the octahedron are placed
against any circle of Venus (for the total interval between the three has
been very much reduced), the centres of the planes of the octahedron
penetrate and descend deeply within the outmost circle of Mercury, but
nonetheless do not reach as far as the middle circle of Mercury; and
finally, closest of all to the ratios of the dodecahedral and icosahedral
spheres—which ratios are equal to one another—are the ratios or
intervals between the circles of Mars and the Earth, and the Earth and
Venus; and those intervals are similarly equal, if we compute from the
inmost circle of Mars to the middle circle of the Earth, but from the
middle circle of the Earth to the middle circle of Venus. For the middle
distance of the Earth is a mean proportional between the least distance
of Mars and the middle distance of Venus. However, these two ratios
between the planetary circles are still greater than the ratios of those two
pairs of spheres in the figures, in such fashion that the centres of the
dodecahedral planes are not tangent to the outmost circle of the Earth,
and the centres of the icosahedral planes are not tangent to the outmost


16

circle of Venus; nor, however, can this gap be filled by the semidiameter
of the lunar sphere, by adding it, on the upper side, to the greatest
distance of the Earth and subtracting it, on the lower, from the least
distance of the same. But I find a certain other ratio of figures—namely,
if I take the augmented dodecahedron, to which I have given the name of
echinus, (as being fashioned from twelve quinquangular stars and
thereby very close to the five regular solids), if I take it, I say, and place
its twelve points in the inmost circle of Mars, then the sides of the
pentagons, which are the bases of the single rays or points, touch the
middle circle of Venus. In short: the cube and the octahedron, which are
consorts, do not penetrate their planetary spheres at all; the
dodecahedron and the icosahedron, which are consorts, do not wholly
reach to theirs, the tetrahedron exactly touches both: in the first case
there is falling short; in the second, excess; and in the third, equality,
with respect to the planetary intervals.
Wherefore it is clear that the very ratios of the planetary intervals from
the sun have not been taken from the regular solids alone. For the
Creator, who is the very source of geometry and, as Plato wrote,
"practices eternal geometry," does not stray from his own archetype. And
indeed that very thing could be inferred from the fact that all the planets
change their intervals throughout fixed periods of time, in such fashion
that each has two marked intervals from the sun, a greatest and a least;
and a fourfold comparison of the intervals from the sun is possible
between two planets: the comparison can be made between either the
greatest, or the least, or the contrary intervals most remote from one
another, or the contrary intervals nearest together. In this way the
comparisons made two by two between neighbouring planets are twenty
in number, although on the contrary there are only five regular solids.
But it is consonant that if the Creator had any concern for the ratio of the
spheres in general, He would also have had concern for the ratio which
exists between the varying intervals of the single planets specifically and
that the concern is the same in both cases and the one is bound up with
the other. If we ponder that, we will comprehend that for setting up the
diameters and eccentricities conjointly, there is need of more principles,
outside of the five regular solids.
Fifthly [V]: To arrive at the movements between which the consonances
have been set up, once more I impress upon the reader that in


17

the Commentaries on Mars I have demonstrated from the sure
observations of Brahe that daily arcs, which are equal in one and the
same eccentric circle, are not traversed with equal speed; but that these
differing delays in equal parts of the eccentric observe the ratio of their
distances from the sun, the source of movement; and conversely, that if
equal times are assumed, namely, one natural day in both cases, the
corresponding true diurnal arcs [278] of one eccentric orbit have to one
another the ratio which is the inverse of the ratio of the two distances
from the sun. Moreover, I demonstrated at the same time that the
planetary orbit is elliptical and the sun, the source of movement, is at
one of the foci of this ellipse; and so, when the planet has completed a
quarter of its total circuit from its aphelion, then it is exactly at its mean
distance from the sun, midway between its greatest distance at the
aphelion and its least at the perihelion. But from these two axioms it
results that the diurnal mean movement of the planet in its eccentric is
the same as the true diurnal arc of its eccentric at those moments
wherein the planet is at the end of the quadrant of the eccentric
measured from the aphelion, although that true quadrant appears still
smaller than the just quadrant. Furthermore, it follows that the sum of
any two true diurnal eccentric arcs, one of which is at the
same distance from the aphelion that the other is from the perihelion, is
equal to the sum of the two mean diurnal arcs. And as a
consequence, since the ratio of circles is the same as that of the
diameters, the ratio of one mean diurnal arc to the sum of all the mean
and equal arcs in the total circuit is the same as the ratio of the mean
diurnal arc to the sum of all the true eccentric arcs, which are the same
in number but unequal to one another. And those things should first be
known concerning the true diurnal arcs of the eccentric and the true
movements, so that by means of them we may understand the
movements which would be apparent if we were to suppose an eye at the
sun.
Sixthly [VI]: But as regards the arcs which are apparent, as it were, from
the sun, it is known even from the ancient astronomy that, among true
movements which are equal to one another, that movement which is
farther distant from the centre of the world (as being at the aphelion)
will appear smaller to a beholder at that centre, but the movement which
is nearer (as being at the perihelion) will similarly appear greater.


18

Therefore, since moreover the true diurnal arcs at the near distance are
still greater, on account of the faster movement, and still smaller at the
distant aphelion, on account of the slowness of the movement, I
demonstrated in the Commentaries on Mars that the ratio of the
apparent diurnal arcs of one eccentric circle is fairly exactly the inverse
ratio of the squares of their distances from the sun. For example, if the
planet one day when it is at a distance from the sun of 10 parts, in any
measure whatsoever, but on the opposite day, when it is at the
perihelion, of 9 similar parts: it is certain that from the sun its apparent
progress at the aphelion will be to its apparent progress at the perihelion,
as 81 : 100.
But that is true with these provisos: First, that the eccentric arcs should
not be great, lest they partake of distinct distances which are very
different—i.e., lest the distances of their termini from the apsides cause a
perceptible variation; second, that the eccentricity should not be very
great, for the greater its eccentricity (viz., the greater the arc becomes)
the more the angle of its apparent movement increases beyond the
measure of its approach to the sun, by Theorem 8 of Euclid's Optics;
none the less in small arcs even a great distance is of no moment, as I
have remarked in my Optics, Chapter 11. But there is another reason why
I make that admonition. For the eccentric arcs around the mean
anomalies are viewed obliquely from the centre of the sun. This obliquity
subtracts from the magnitude of the apparent movement, since
conversely the arcs around the apsides are presented directly to an eye
stationed as it were at the sun. Therefore, when the eccentricity is very
great, then the eccentricity takes away perceptibly from the ratio of the
movements; if without any diminution we apply the mean diurnal
movement to the mean distance, as if at the mean distance, it would
appear to have the same magnitude which it does have—as will be
apparent below in the case of Mercury. All these things are treated at
greater length in Book V of the Epitome of Copernican Astronomy; but
they have been mentioned here too because they have to do with the very
terms of the celestial consonances, considered in themselves singly and
separately.
Seventhly [VII]: If by chance anyone runs into those diurnal movements
which are apparent [279] to those gazing not as it were from the sun but
from the Earth, with which movements Book VI of the Epitome of


19

Copernican Astronomy deals, he should know that their rationale is
plainly not considered in this business. Nor should it be, since the Earth
is not the source of the planetary movements, nor can it be, since with
respect to deception of sight they degenerate not only into mere quiet or
apparent stations but even into retrogradation, in which way a whole
infinity of ratios is assigned to all the planets, simultaneously and
equally. Therefore, in order that we may hold for certain what sort of
ratios of their own are constituted by the single real eccentric orbits
(although these too are still apparent, as it were to one looking from the
sun, the source of movement), first we must remove from those
movements of their own this image of the adventitious annual movement
common to all five, whether it arises from the movement of the Earth
itself, according to Copernicus, or from the annual movement of the total
system, according to Tycho Brahe, and the winnowed movements proper
to each planet are to be presented to sight.
Eighthly [viii]: So far we have dealt with the different delays or arcs of
one and the same planet. Now we must also deal with the comparison of
the movements of two planets. Here take note of the definitions of the
terms which will be necessary for us. We give the name of nearest
apsides of two planets to the perihelion of the upper and the aphelion of
the lower, notwithstanding that they tend not towards the same region of
the world but towards distinct and perhaps contrary regions. By extreme
movements understand the slowest and the fastest of the whole
planetary circuit; by converging or converse extreme movements, those
which are at the nearest apsides of two planets—namely, at the
perihelion of the upper planet and the aphelion of the lower;
by diverging or diverse, those at the opposite apsides—namely, the
aphelion of the upper and the perihelion of the lower. Therefore again, a
certain part of my Mysterium Cosmographicum, which was suspended
twenty-two years ago, because it was not yet clear, is to be completed and
herein inserted. For after finding the true intervals of the spheres by the
observations of Tycho Brahe and continuous labour and much time, at
last, at last the right ratio of the periodic times to the spheres
though it was late, looked to the unskilled man,
yet looked to him, and, after much time, came,


20

and, if you want the exact time, was conceived mentally on the 8th of
March in this year One Thousand Six Hundred and Eighteen but
unfelicitously submitted to calculation and rejected as false, finally,
summoned back on the 15th of May, with a fresh assault undertaken,
outfought the darkness of my mind by the great proof afforded by my
labor of seventeen years on Brahe's observations and meditation upon it
uniting in one concord, in such fashion that I first believed I was
dreaming and was presupposing the object of my search among the
principles. But it is absolutely certain and exact that the ratio which
exists between the periodic times of any two planets is precisely the
ratio of the 3/2th power of the mean distances, i.e., of the spheres
themselves; provided, however, that the arithmetic mean between both
diameters of the elliptic orbit be slightly less than the longer diameter.
And so if any one take the period, say, of the Earth, which is one year,
and the period of Saturn, which is thirty years, and extract the cube roots
of this ratio and then square the ensuing ratio by squaring the cube
roots, he will have as his numerical products the most just ratio of the
distances of the Earth and Saturn from the sun. 1 For the cube root of 1
is 1, and the square of it is 1; and the cube root of 30 is greater than 3,
and therefore the square of it is greater than 9. And Saturn, at its mean
distance from the sun, is slightly higher [280] than nine times the mean
distance of the Earth from the sun. Further on, in Chapter 9, the use of
this theorem will be necessary for the demonstration of the
eccentricities.
Ninthly [IX]: If now you wish to measure with the same yardstick, so to
speak, the true daily journeys of each planet through the ether, two ratios
are to be compounded—the ratio of the true (not the apparent) diurnal
arcs of the eccentric, and the ratio of the mean intervals of each planet
from the sun (because that is the same as the ratio of the amplitude of
the spheres), i.e., the true diurnal arc of each planet is to be multiplied
by the semidiameter of its sphere: the products will be numbers fitted
for investigating whether or not those journeys are in harmonic ratios.
Tenthly [X]: In order that you may truly know how great any one of these
diurnal journeys appears to be to an eye stationed as it were at the sun,
although this same thing can be got immediately from the astronomy,
1 For in the Commentaries on Mars, chapter 48, page 232, I have proved that this Arithmetic mean is

either the diameter of the circle which is equal in length to the elliptic orbit, or else is very slightly less.


21

nevertheless it will also be manifest if you multiply the ratio of the
journeys by the inverse ratio not of the mean, but of the true intervals
which exist at any position on the eccentrics: multiply the journey of the
upper by the interval of the lower planet from the sun, and conversely
multiply the journey of the lower by the interval of the upper from the
sun.
Eleventhly [XI]: And in the same way, if the apparent movements are
given, at the aphelion of the one and at the perihelion of the other, or
conversely or alternately, the ratios of the distances of the aphelion of the
one to the perihelion of the other may be elicited. But where the mean
movements must be known first, viz., the inverse ratio of the periodic
times, wherefrom the ratio of the spheres is elicited by Article VIII
above: then if the mean proportional between the apparent movement
of either one of its mean movement be taken, this mean proportional is
to the semidiameter of its sphere (which is already known) as the mean
movement is to the distance or interval sought. Let the periodic times of
two planets be 27 and 8. Therefore the ratio of the mean diurnal
movement of the one to the other is 8 : 27. Therefore the semidiameters
of their spheres will be as 9 to 4. For the cube root of 27 is 3, that of 8 is
2, and the squares of these roots, 3 and 2, are 9 and 4. Now let the
apparent aphelial movement of the one be 2 and the perihelial
movement of the other 33⅓. The mean proportionals between the mean
movements 8 and 27 and these apparent ones will be 4 and 30.
Therefore if the mean proportional 4 gives the mean distance of 9 to the
planet, then the mean movement of 8 gives an aphelial distance 18,
which corresponds to the apparent movement 2; and if the other mean
proportional 30 gives the other planet a mean distance of 4, then its
mean movement of 27 will give it a perihelial interval of 3 3/5. I say,
therefore, that the aphelial distance of the former is to the perihelial
distance of the latter as 18 to 3 3/5. Hence it is clear that if the
consonances between the extreme movements of two planets are found
and the periodic times are established for both, the extreme and the
mean distances are necessarily given, wherefore also the eccentricities.
Twelfthly [XII]: It is also possible, from the different extreme
movements of one and the same planet, to find the mean movement. The
mean movement is not exactly the arithmetic mean between the extreme
movements, nor exactly the geometric mean, but it is as much less than


22

the geometric mean as the geometric mean is less than the [arithmetic]
mean between both means. Let the two extreme movements be 8 and 10:
the mean movement will be less than 9, and also less than the square
root of 80 by half the difference between 9 and the square root of 80. In
this way, if the aphelial movement is 20 and the perihelial 24, the mean
movement will be less than 22, even less than the square root of 480 by
half the difference between that root and 22. There is use for this
theorem in what follows.
[281] Thirteenthly [XIII]: From the foregoing the following proposition
is demonstrated, which is going to be very necessary for us: Just as the
ratio of the mean movements of two planets is the inverse ratio of the
3/2th powers of the spheres, so the ratio of two apparent converging
extreme movements always falls short of the ratio of the 3/2th powers of
the intervals corresponding to those extreme movements; and in what
ratio the product of the two ratios of the corresponding intervals to the
two mean intervals or to the semidiameters of the two spheres falls short
of the ratio of the square roots of the spheres, in that ratio does the ratio
of the two extreme converging movements exceed the ratio of the
corresponding intervals; but if that compound ratio were to exceed the
ratio of the square roots of the spheres, then the ratio of the converging
movements would be less than the ratio of their intervals. 2
Let the ratio of the spheres be DH : AE; let the ratio of the mean
movements be HI : EM, the 3/2th power of the inverse of the former.
Let the least interval of the sphere of the first be CG; and the greatest
interval of the sphere of the second be BF; and first
let DH : CG comp. BF : AE be smaller than the ½th power of DH : AE.
And let GHbe the apparent perihelial movement of the upper planet,
and FL the aphelial of the lower, so that they are converging extreme
movements.

2 Kepler always measures the magnitude of a ratio from the greater term to the smaller, rather than

from the antecedent to the consequent, as we do today. For example, as Kepler speaks, 2 : 3 is the
same as 3 : 2, and 3 : 4 is greater than 7 : 8.—C. G. Wallis.


23

I say that
GK : FL = BF : CG
BF3/2 : CG3/2.
For
HI : GK =CG2 : DH2;
and
FL : EM =AE2 : BF2.
Hence
HI : GK comp. FL : EM = CG2 : DH2 comp. AE2 : BF2.
But
CG : DH comp. AE : BF < AE½ : DH½
by a fixed ratio of defect, as was assumed. Therefore too
HI : GK comp. FL : EM AE2/2 : DH2/2
AE : DH
by a ratio of defect which is the square of the former. But by number VIII
HI : EM = AE3/2 : DH3/2.
Therefore let the ratio which is smaller by the total square of the ratio of
defect be divided into the ratio of the 3/2th powers; that is,


24

HI : EM comp. GK : HI comp. EM : FL AE½ : DH½
by the excess squared. But
HI : EM comp. GK : HI comp. EM : FL = GK : FL.
Therefore
GK : FL AE½ : DH½
by the excess squared. But
AE : DH = AE : BF comp. BF :CG comp. CG :DM;
And
CG : DH comp. AE : BF AE½ : DH½
by the simple defect. Therefore
BF : CG AE½ : DH½
by the simple excess. But
GK : FL AE½ : DT½
but by the excess squared. But the excess squared is greater than the
simple excess. Therefore the ratio of the movements GK to FL is greater
than the ratio of the corresponding intervals BF to CG.
In fully the same way, it is demonstrated even contrariwise that if the
planets approach one another in G and F beyond the mean distances
in H and E, in such fashion that the ratio of the mean
distances DH : AE becomes less than DH½ : AE½, then the ratio of the
movements GK : FL becomes less than the ratio of the corresponding
intervals BF: CG. For you need to do nothing more than to change the
words greater to less, > to <, excess to defect, and conversely.
In suitable numbers, because the square root of 4/9 is 2/3; and 5/8 is
even greater than 2/3 by the ratio of excess 15/16; and the square of the
ratio 8 : 9 [282] is the ratio 1600 : 2025, i.e., 64 : 81; and the square of
the ratio 4 : 5 is the ratio 3456 : 5400, i.e., 16 : 25; and finally the 3/2th
power of the ratio 4 : 9 is the ratio 1600 : 5400, i.e., 8 : 27: therefore too
the ratio 2025 : 3456, i.e., 75 : 128, is even greater than 5 : 8, i.e., 75 :
120, by the same ratio of excess (i.e., 120 : 128), 15 : 16; whence 2025 :


25

3456, the ratio of the converging movements, exceeds 5 : 8, the inverse
ratio of the corresponding intervals, by as much as 5 : 8 exceeds 2 : 3, the
square root of the ratio of the spheres. Or, what amounts to the same
thing, the ratio of the two converging intervals is a mean between the
ratio of the square roots of the spheres and the inverse ratio of the
corresponding movements.
Moreover, from this you may understand that the ratio of the diverging
movements is much greater than the ratio of the 3/2th powers of the
spheres, since the ratio of the 3/2th powers is compounded with the
squares of the ratio of the aphelial interval to the mean interval, and that
of the mean to the perihelial.


26

4. IN WHAT THINGS HAVING TO DO WITH THE
PLANETARY MOVEMENTS HAVE THE HARMONIC
CONSONANCES BEEN EXPRESSED BY THE
CREATOR, AND IN WHAT WAY?
Accordingly, if the image of the retrogradation and stations is taken away
and the proper movements of the planets in their real eccentric orbits are
winnowed out, the following distinct things still remain in the planets: 1)
The distances from the sun. 2) The periodic times. 3) The diurnal
eccentric arcs. 4) The diurnal delays in those arcs. 5) The angles at the
sun, and the diurnal arcs apparent to those as it were gazing from the
sun. And again, all of these things, with the exception of the periodic
times, are variable in the total circuit, most variable at the mean
longitudes, but least at the extremes, when, turning away from one
extreme longitude, they begin to return to the opposite. Hence when the
planet is lowest and nearest to the sun and thereby delays the least in
one degree of its eccentric, and conversely in one day traverses the
greatest diurnal arc of its eccentric and appears fastest from the sun:
then its movement remains for some time in this strength
without perceptible variation, until, after passing the perihelion, the
planet gradually begins to depart farther from the sun in a straight line;
at that same time it delays longer in the degrees of its eccentric circle; or,
if you consider the movement of one day, on the following day it goes
forward less and appears even more slow from the sun until it has drawn
close to the highest apsis and made its distance from the sun very great:
for then longest of all does it delay in one degree of its eccentric; or on
the contrary in one day it traverses its least arc and makes a much
smaller apparent movement and the least of its total circuit.
Finally, all these things may be considered either as they exist in any one
planet at different times or as they exist in different planets: whence, by
the assumption of an infinite amount of time, all the affects of the circuit
of one planet can concur in the same moment of time with all the affects
of the circuit of another planet and be compared, and then the total
eccentrics, as compared with one another, have the same ratio as their


27

semidiameters or mean intervals; but the arcs of two eccentrics, which
are similar or designated by the same number [of degrees], nevertheless
have their true lengths unequal in the ratio of their eccentrics. For
example, one degree in the sphere of Saturn is approximately twice as
long as one degree in the sphere of Jupiter. And conversely, the diurnal
arcs of the eccentrics, as expressed in astronomical terms, do not exhibit
the ratio of the true journeys which the globes complete in one day [283]
through the ether, because the single units in the wider circle of the
upper planet denote a quarter part of the journey, but in the narrower
circle of the lower planet a smaller part.
Therefore let us take the second of the things which we have posited,
namely, the periodic times of the planets, which comprehend the sums
made up of all the delays—long, middling, short—in all the degrees of the
total circuit. And we found that from antiquity down to us, the planets
complete their periodic returns around the sun, as follows in the table:

Accordingly, in these periodic times there are no harmonic ratios, as is
easily apparent, if the greater periods are continuously halved, and the
smaller are continuously doubled, so that, by neglecting the intervals of
an octave, we can investigate the intervals which exist within one octave.


28

All the last numbers, as you see, are counter to harmonic ratios and
seem, as it were, irrational. For let 687, the number of days of Mars,
receive as its measure 120, which is the number of the division of the
chord: according to this measure Saturn will have 117 for one sixteenth of
its period, Jupiter less than 95 for one eighth of its period, the earth less
than 64, Venus more than 78 for twice its period, Mercury more than 61
for four times its period. These numbers do not make any harmonic ratio
with 120, but their neighbouring numbers—60, 75, 80, and 96—do. And
so, whereof Saturn has 120, Jupiter has approximately 97, the Earth
more than 65, Venus more than 80, and Mercury less than 63. And
whereof Jupiter has 120, the Earth has less than 81, Venus less than 100,
Mercury less than 78. Likewise, whereof Venus has 120, the Earth has
less than 98, Mercury more than 94. Finally, whereof the Earth has 120,
Mercury has less than 116. But if the free choice of ratios had been
effective here, consonances which are altogether perfect but not
augmented or diminished would have been taken. Accordingly we find
that God the Creator did not wish to introduce harmonic ratios between
the sums of the delays added together to form the periodic times.
[284] And although it is a very probable conjecture (as relying on
geometrical demonstrations and the doctrine concerning the causes of
the planetary movements given in the Commentaries on Mars) that the
bulks of the planetary bodies are in the ratio of the periodic times, so
that the globe of Saturn is about thirty times greater than the globe of the
Earth, Jupiter twelve times, Mars less than two, the Earth one and a half
times greater than the globe of Venus and four times greater than the
globe of Mercury: not therefore will even these ratios of bodies be
harmonic.
But since God has established nothing without geometrical beauty,
which was not bound by some other prior law of necessity, we easily infer
that the periodic times have got their due lengths, and thereby the
mobile bodies too have got their bulks, from something which is prior in
the archetype, in order to express which thing these bulks and periods
have been fashioned to this measure, as they seem disproportionate. But
I have said that the periods are added up from the longest, the middling,
and the slowest delays: accordingly geometrical fitnesses must be found


29

either in these delays or in anything which may be prior to them in the
mind of the Artisan. But the ratios of the delays are bound up with the
ratios of the diurnal arcs, because the arcs have the inverse ratio of the
delays. Again, we have said that the ratios of the delays and intervals of
any one planet are the same. Then, as regards the single planets, there
will be one and the same consideration of the following three: the arcs,
the delays in equal arcs, and the distance of the arcs from the sun or the
intervals. And because all these things are variable in the planets, there
can be no doubt but that, if these things were allotted any geometrical
beauty, then, by the sure design of the highest Artisan, they would have
been received that at their extremes, at the aphelial and perihelial
intervals, not at the mean intervals lying in between. For, given the ratios
of the extreme intervals, there is no need of a plan to fit the intermediate
ratios to a definite number. For they follow of themselves, by the
necessity of planetary movement, from one extreme through all the
intermediates to the other extreme.
Therefore the intervals are as follows, according to the very accurate
observations of Tycho Brahe, by the method given in the Commentaries
on Mars and investigated in very persevering study for seventeen years.
Intervals Compared with Harmonic Ratios 3
3 GENERAL NOTE: Throughout this text Kepler's concinna and inconcinna are translated as

"concordant" and "discordant." Concinna is usually used by Kepler of all intervals whose ratios occur
within the "natural system" or the just intonation of the scale. Inconcinna refers to all ratios that lie
outside of this system of tuning. "Consonant" (consonans) and "dissonant" (dissonans) refer to
qualities which can be applied to intervals within the musical system, in other words to "concords."
"Harmony" (harmonia) is used sometimes in the sense of "concordance" and sometimes in the sense
of "consonance."
Genus durum and genus molle are translated either as "major mode" and "minor mode," or as "major
scale" and "minor scale," or as "major kind" and "minor kind" (of consonances). The use of modus, to
refer to the ecclesiastical modes, occurs only in Chapter 6.
As our present musical terms do not apply strictly to the music of the sixteenth and seventeenth
centuries, a brief explanation of terms here may be useful. This material is taken from Kepler's
Harmonies of the World, Book III.
An octave system in the minor scale (Systema octavae in cantu molli)

In the major scale (In cantu duro)


30

As in all music, these scales can be repeated at one or more octaves above. The ratios would then all be
halved, i.e.,

Various intervals which Kepler considers are:

All these are simple intervals. When one or more octaves are added to any simple intervals the
resultant interval is a "compound" interval.
1 : 3 equals ½ × ⅔—an octave and a perfect fifth
3 : 32 equals (½)3 × ¾—three octaves and a perfect fourth
1 : 20 equals (½)4 × (16/20)—four octaves and a major third
Concords: All intervals from diesis downward on above list.
Consonances: Minor and major thirds and sixths, perfect fourth, fifth, and octave.
"Adulterine" consonances: sub-minor third, ditone, lesser imperfect fourth and fifth, greater imperfect
fourth and fifth, imperfect minor sixth, greater major sixth.
Dissonances: All other intervals.
Throughout this work Kepler, after the fashion of the theorists of his time, uses the ratios of string
lengths rather than the ratios of vibrations as is usually done today. String lengths are, of course,
inversely proportionate to the vibrations. That is, string lengths 4 : 5 are expressed in vibrations as 5
:4. This accounts for the descending order of the scale, which follows the increasing numerical order.
It is an interesting fact that Kepler's minor and major scales are inversions of each other and hence,
when expressed in ratios of vibrations, are in the opposite order from those in ratios of string lengths:


31

[285] Therefore the extreme intervals of no one planet come near
consonances except those of Mars and Mercury.
But if you compare the extreme intervals of different planets with one
another, some harmonic light begins to shine. For the extreme diverging
intervals of Saturn and Jupiter make slightly more than the octave; and
the converging, a mean between the major and minor sixths. So the
diverging extremes of Jupiter and Mars embrace approximately the
double octave; and the converging, approximately the fifth and the
octave. But the diverging extremes of the Earth and Mars embrace
somewhat more than the major sixth; the converging, an augmented
fourth. In the next couple, the Earth and Venus, there is again the same
augmented fourth between the converging extremes; but we lack any
harmonic ratio between the diverging extremes: for it is less than the
semi-octave (so to speak) i.e., less than the square root of the ratio 2 : 1.
Finally, between the diverging extremes of Venus and Mercury there is a
ratio slightly less than the octave compounded with the minor third;
between the converging there is a slightly augmented fifth.

An arbitrary pitch G is chosen to situate these ratios. This g or "gamma" was usually the lowest tone of
the sixteenth-century musical gamut.
ELLIOTT CARTER, JR.


32

Accordingly, although one interval was somewhat removed from
harmonic ratios, this success was an invitation to advance further. Now
my reasonings were as follows: First, in so far as these intervals are
lengths without movement, they are not fittingly examined for harmonic
ratios, because movement is more properly the subject of consonances,
by reason of speed and slowness. Second, inasmuch as these same
intervals are the diameters of the spheres, it is believable that the ratio of
the five regular solids applied proportionally is more dominant in them,
because the ratio of the geometrical solid bodies to the celestial spheres
(which are everywhere either encompassed by celestial matter, as the
ancients hold, or to be encompassed successively by the accumulation of
many revolutions) is the same as the ratio of the plane figures which may
be inscribed in a circle (these figures engender the consonances) to the
celestial circles of movements and the other regions wherein the
movements take place. Therefore, if we are looking for consonances, we
should look for them not in these intervals in so far as they are the
semidiameters of spheres but in them in so far as they are the measures
of the movements, i.e., in the movements themselves, rather. Absolutely
no other than the mean intervals can be taken as the semi-diameters of
the spheres; but we are here dealing with the extreme intervals.
Accordingly, we are not dealing with the intervals in respect to their
spheres but in respect to their movements.
Accordingly, although for these reasons I had passed on to the
comparison of the extreme movements, at first the ratios of the
movements remained the same in magnitude as those which were
previously the ratios of the intervals, only inverted. Wherefore too,
certain ratios, which are discordant and foreign to harmonies, as before,
have been found between the movements. But once again I judged that
this happened to me deservedly, because I compared with one another
eccentric arcs which are not expressed and numbered by a measure of
the same magnitude but are numbered in degrees and minutes which are
of diverse magnitude in diverse planets, nor do they from our place give
the appearance of being as great as the number of each says, except only
at the centre of the eccentric of each planet, which centre rests upon no
body; and hence it is also unbelievable that there is any sense or natural
instinct in that place in the world which is capable of perceiving this; or,
rather, it was impossible, if I was comparing the eccentric arcs of


33

different planets with respect to their appearance at their centres, which
are different for different planets. But if diverse apparent magnitudes are
compared with one another, they ought to be apparent in one place in
the world in such a way that that which possesses the faculty of
comparing them may be present in that place from which they are all
apparent. Accordingly, I judged that the appearance of these eccentric
arcs should be removed from the mind or else should be formed
differently. But if I removed the appearance and applied my mind to the
diurnal journeys of the planets, I saw that I had to employ the rule which
I gave in Article IX of the preceding chapter.
[286] Accordingly if the diurnal arcs of the eccentric are multiplied by
the mean intervals of the spheres, the following journeys are produced:

Thus Saturn traverses barely one seventh of the journey of Mercury; and
hence, as Aristotle judged consonant with reason in Book II of On the
Heavens, the planet which is nearer the sun always traverses a greater
space than the planet which is farther away—as cannot hold in the
ancient astronomy.
And indeed, if we weigh the thing fairly carefully, it will appear to be not
very probable that the most wise Creator should have established
harmonies between the planetary journeys in especial. For if the ratios of
the journeys are harmonic, all the other affects which the planets have


34

will be necessitated and bound up with the journeys, so that there is no
room elsewhere for establishing harmonies. But whose good will it be to
have harmonies between the journeys, or who will perceive these
harmonies? For there are two things which disclose to us harmonies in
natural things: either light or sound: light apprehended through the eyes
or hidden senses proportioned to the eyes, and sound through the ears.
The mind seizes upon these forms and, whether by instinct (on which
Book IV speaks profusely) or by astronomical or harmonic ratiocination,
discerns the concordant from the discordant. Now there are no sounds in
the heavens, nor is the movement so turbulent that any noise is made by
the rubbing against the ether. Light remains. If light has to teach these
things about the planetary journeys, it will teach either the eyes or a
sensorium analogous to the eyes and situated in a definite place; and it
seems that sense-perception must be present there in order that light of
itself may immediately teach. Therefore there will be sense-perception in
the total world, namely in order that the movements of all the planets
may be presented to sense-perceptions at the same time. For that former
route—from observations through the longest detours of geometry and
arithmetic, through the ratios of the spheres and the other things which
must be learned first, down to the journeys which have been exhibited—
is too long for any natural instinct, for the sake of moving which it seems
reasonable that the harmonies have been introduced.
Therefore with everything reduced to one view, I concluded rightly [287]
that the true journeys of the planets through the ether should be
dismissed, and that we should turn our eyes to the apparent diurnal arcs,
according as they are all apparent from one definite and marked place in
the world—namely, from the solar body itself, the source of movement of
all the planets; and we must see, not how far away from the sun any one
of the planets is, nor how much space it traverses in one day (for that is
something for ratiocination and astronomy, not for instinct), but how
great an angle the diurnal movement of each planet subtends in the solar
body, or how great an arc it seems to traverse in one common circle
described around the sun, such as the ecliptic, in order that these
appearances, which were conveyed to the solar body by virtue of light,
may be able to flow, together with the light, in a straight line into
creatures, which are partakers of this instinct, as in Book IV we said the
figure of the heavens flowed into the foetus by virtue of the rays.


35

Therefore, if you remove from the proper planetary movement the
parallaxes of the annual orbit, which gives them the mere appearances of
stations and retrogradations, Tycho's astronomy teaches that the diurnal
movements of the planets in their orbits (which are apparent as it were
to spectator at the sun) are as shown in the table on the opposite page.
Note that the great eccentricity of Mercury makes the ratio of the
movements differ somewhat from the ratio of the square of the
distances. For if you make the square of the ratio of 100, the mean
distance, to 121, the aphelial distance, be the ratio of the aphelial
movement to the mean movement of 245´32″, then an aphelial
movement of 167 will be produced; and if the square of the ratio of 100
to 79, the perihelial distance, be the ratio of the perihelial to the same
mean movement, then the perihelial movement will become 393; and
both cases are

greater than I have here laid down, because the mean movement at the
mean anomaly, viewed very obliquely, does not appear as great, viz., not
as great as


36

245'32", but about 5' less. Therefore, too, lesser aphelial and perihelial
movements will be elicited. But the aphelial [appears] lesser and the
perihelial greater, on account of theorem 8, Euclid's Optics, as I
remarked in the preceding Chapter, Article VI.
Accordingly, I could mentally presume, even from the ratios of the
diurnal eccentric arcs given above, that there were harmonies and
concordant intervals between these extreme apparent movements of the
single planets, since I saw that everywhere there the square roots of
harmonic ratios were dominant, but knew that the ratio of the apparent
movements was the square of the ratio of the eccentric movements. But
it is possible by experience itself, or without any ratiocination to prove
what is affirmed, as you see [288] in the preceding table. The ratios of
the apparent movements of the single planets approach very close to
harmonies, in such fashion that Saturn and Jupiter embrace slightly
more than the major and minor thirds, Saturn with a ratio of excess of 53
: 54, and Jupiter with one of 54 : 55 or less, namely approximately a
sesquicomma; the Earth, slightly more (namely 137 : 138, or barely a
semicomma) than a semitone; Mars somewhat less (namely 29 : 30,
which approaches 34 : 35 or 35 : 36) than a fifth; Mercury exceeds the
octave by a minor third rather than a whole tone, viz., it is about 38 : 39
(which is about two commas, viz., 34 : 35 or 35 : 36) less than a whole
tone. Venus alone falls short of any of the concords the diesis; for its
ratio is between two and three commas, and it exceeds two thirds of a
diesis, and is about 34 : 35 or 35 : 36, a diesis diminished by a comma.
The moon, too, comes into this consideration. For we find that its hourly
apogeal movement in the quadratures, viz., the slowest of all its
movements, to be 26´26″; its perigeal movement in the syzygies, viz.,
the fastest of all, 35´12″, in which way the perfect fourth is formed very
precisely. For one third of 26´26″ is 8´49″, the quadruple of which is
35´16″. And note that the consonance of the perfect fourth is found
nowhere else between the apparent movements; note also the analogy
between the fourth in consonances and the quarter in the phases. And so
the above things are found in the movements of the single planets.
But in the extreme movements of two planets compared with one
another, the radiant sun of celestial harmonies immediately shines at
first glance, whether you compare the diverging extreme movements or


37

the converging. For the ratio between the diverging movements of
Saturn and Jupiter is exactly the duple or octave; that between the
diverging, slightly more than triple or the octave and the fifth. For one
third of 5´30″ is 1´50″, although Saturn has 1´46″ instead of that.
Accordingly, the planetary movements will differ from a consonance by a
diesis more or less,viz., 26 : 27 or 27 : 28; and with less than one second
acceding at Saturn's aphelion, the excess will be 34 : 35, as great as the
ratio of the extreme movements of Venus. The diverging and converging
movements of Jupiter and Mars are under the sway of the triple octave
and the double octave and a third, but not perfectly. For one eighth of
38´1″ is 4´45″, although Jupiter has 4´30″; and between these numbers
there is still a difference of 18 : 19, which is a mean between the semitone
of 15 : 16 and the diesis of 24 : 25, namely, approximately a perfect
lemma of 128 : 135. 4 Thus one fifth of 26´14″ is 5´15″, although Jupiter
has 5´30″; accordingly in this case the quintuple ratio is diminished in
the ratio of 21 : 22, the augment in the case of the other ratio, viz.,
approximately a diesis of 24 : 25.
The consonance 5 : 24 comes nearer, which compounds a minor instead
of a major third with the double octave. For one fifth of 5´30″ is 1´6″,
which if multiplied by 24 makes 26´24″, does not differ by more than a
semicomma. Mars and the Earth have been allotted the least ratio,
exactly the sesquialteral or perfect fifth: for one third of 57´3″ is 19´1″,
the double of which is 38´2″, which is Mars’ very number, viz., 38´11″.
They have also been allotted the greater ratio of 5 : 12, the octave and
minor third, but more imperfectly. For one twelfth of 61´18″ is 5´6½″,
which if multiplied by 5 gives 25´33″, although instead of that Mars has
26´14″. Accordingly, there is a deficiency of a diminished diesis
approximately, viz., 35 : 36. But the Earth and Venus together have been
allotted 3 : 5 as their greatest consonance and 5 : 8 as their least, the
major and minor sixths, but again not perfectly. For one fifth of 97´37″,
which if multiplied by 3 gives 58´33″, which is greater than the
movement of the Earth in the ratio 34 : 35, which is approximately 35 :
36: by so much do the planetary ratios differ from the harmonic. Thus
one eighth of 94´50″ is 11´51″ +, five times which is 59´16″, which is
approximately equal to the mean movement of the Earth. Wherefore
here the planetary ratio is less than the harmonic [289] in the ratio of 29
4 cf. Footnote to Intervals Compared with Harmonic Ratios, p. 1026.


38

: 30 or 30 : 31, which is again approximately 35 : 36, the diminished
diesis; and thereby this least ratio of these planets approaches the
consonance of the perfect fifth. For one third of 94´50″ is 31´37″, the
double of which is 63´14″, of which the 61´18″ of the perihelial
movement of the Earth falls short in the ratio of 31 : 32, so that the
planetary ratio is exactly a mean between the neighbouring harmonic
ratios. Finally, Venus and Mercury have been allotted the double octave
as their greatest ratio and the major sixth as their least, but not absoluteperfectly. For one fourth of 384´ is 96´0″, although Venus has 94´50″.
Therefore the quadruple adds approximately one comma. Thus one fifth
of 164´ is 32´48″, which if multiplied by 3 gives 98´24″, although Venus
has 97´37″. Therefore the planetary ratio is diminished by
about two thirds of a comma, i.e., 126 : 127.
Accordingly the above consonances have been ascribed to the planets;
nor is there any ratio from among the principal comparisons (viz., of the
converging and diverging extreme movements) which does not approach
so nearly to some consonance that, if strings were tuned in that ratio, the
ears would not easily discern their imperfection--with the exception of
that one excess between Jupiter and Mars.
Moreover, it follows that we shall not stray far away from consonances if
we compare the movements of the same field. For if Saturn's 4 : 5 comp.
53 : 54 are compounded with the intermediate 1 : 2, the product is 2 : 5
comp. 53 : 54, which exists between the aphelial movements of Saturn
and Jupiter. Compound with that Jupiter's 5 : 6 comp. 54 : 55, and the
product is 5 : 12 comp 54 : 55, which exist between the perihelial
movements of Saturn and Jupiter. Thus compound Jupiter's 5 : 6 comp.
54 : 55 with the intermediate ensuing ratio of 5 : 24 comp. 158 : 157, the
product will be 1 : 6 comp. 36 : 35 between the aphelial movements.
Compound the same 5 : 24 comp. 158 : 157 with Mars’ 2 : 3 comp. 30 :
29, and the product will be 5 : 36 comp. 25 : 24 approximately, i.e., 125 :
864 or about 1 : 7, between the perihelial movements. This ratio is still
alone discordant. With 2 : 3 the third ratio among the intermediates,
compound Mars’ 2 : 3 less 29 : 30; the result will be 4 : 9 comp.
30:29, i.e., 40 : 87, another discord between the aphelial movements. If
instead of Mars’ you compound the Earth's 15 : 16 comp. 137 : 138, you
will make 5 : 8 comp. 137 : 138 between the perihelial movements. And if
with the fourth of the intermediates, 5 : 8 comp. 31 : 30, or 2 : 3 comp. 31


39

: 32, you compound the Earth's 15 : 16 comp. 137 : 138, the product will
be approximately 3 : 5 between the aphelial movements of the Earth and
Venus. For one fifth of 94´50″ is 18´58″, the triple of which is 56´54″,
although the Earth has 57´3″. If you compound Venus’ 34 : 35 with the
same ratio, the result will be 5 : 8 between the perihelial movements. For
one eighth of 97´37″ is 12´12″+ which if multiplied by 5 gives 61´1″,
although the Earth has 61´18″. Finally, if with the last of the
intermediate ratios, 3 : 5 comp. 126 : 127 you compound Venus’ 34 : 35,
the result is 3 : 5 comp. 24 : 25, and the interval, compounded of both,
between the aphelial movements, is dissonant. But if you compound
Mercury's 5 : 12 comp. 38 : 39, the double octave or 1 : 4 will be
diminished by approximately a whole diesis, in proportion to the
perihelial movements.
Accordingly, perfect consonances are found: between the converging
movements of Saturn and Jupiter, the octave; between the converging
movements of Jupiter and Mars, the octave and minor third
approximately; between the converging movements of Mars and the
Earth, the fifth; between their perihelial, the minor sixth; between the
extreme converging movements of Venus and Mercury, the major sixth;
between the diverging or even between the perihelial, the double octave:
whence without any loss to an astronomy which has been built, most
subtly of all, upon Brahe's observations, it seems that the residual very
slight discrepancy can be discounted, especially in the movements of
Venus and Mercury.
But you will note that where there is no perfect major consonance, as
between Jupiter and Mars, there alone have I found the placing of the
solid figure to be approximately perfect, since the perihelial distance of
Jupiter is approximately three times the aphelial distance of Mars, in
such fashion that this pair of planets strives after the perfect consonance
in the intervals which it does not have in the movements.
[290] You will note, furthermore, that the major planetary ratio of
Saturn and Jupiter exceeds the harmonic, viz., the triple, by
approximately the same quantity as belongs to Venus; and the common
major ratio of the converging and diverging movements of Mars and the
Earth are diminished by approximately the same. You will note thirdly
that, roughly speaking, in the upper planets the consonances are


40

established between the converging movements, but in the lower planets,
between movements in the same field. And note fourthly that between
the aphelial movements of Saturn and the Earth there are approximately
five octaves; for one thirty-second of 57´3″ is 1´47″, although the
aphelial movement of Saturn is 1´46″.
Furthermore, a great distinction exists between the consonances of the
single planets which have been unfolded and the consonances of the
planets in pairs. For the former cannot exist at the same moment of time,
while the latter absolutely can; because the same planet, moving at its
aphelion, cannot be at the same time at the opposite perihelion too, but
of two planets one can be at its aphelion and the other at its perihelion at
the same moment of time. And so the ratio of plain-song or monody,
which we call choral music and which alone was known to the ancients, 5
to polyphony—called "figured song,"; 6 the invention of the latest
generations—is the same as the ratio of the consonances which the single
planets designate to the consonances of the planets taken together. And
so, further on, in Chapters 5 and 6, the single planets will be compared to
the choral music of the ancients and its properties will be exhibited in
the planetary movements. But in the following chapters, the planets
taken together and the figured modern music will be shown to do similar
things.

5 The choral music of the Greeks was monolinear, everyone singing the same melody together.—E. C.,

Jr.

6 In plain-song all the time values of the notes were approximately equal, while in "figured song" time

values of different lengths were indicated by the notes, which gave composers an opportunity both to
regulate the way different contrapuntal parts joined together and to produce many expressive effects.
Practically all melodies since this time are in "figured song" style.—E. C., Jr.


41

5. IN THE RATIOS OF THE PLANETARY
MOVEMENTS...
WHICH ARE APPARENT AS IT WERE TO SPECTATORS AT THE SUN,
HAVE BEEN EXPRESSED THE PITCHES OF THE SYSTEM, OR
NOTES OF THE MUSICAL SCALE, AND THE MODES OF SONG
[GENERA CANTUS], THE MAJOR AND THE MINOR 7

Therefore by now I have proved by means of numbers gotten on one side
from astronomy and on the other side from harmonics that, taken in
every which way, harmonic ratios hold between these twelve termini or
movements of the six planets revolving around the sun or that they
approximate such ratios within an imperceptible part of least concord.
But just as in Book III in the first chapter, we first built up the single
harmonic consonances separately, and then we joined together all the
consonances—as many as there were—in one common system or musical
scale, or, rather, in one octave of them which embraces the rest in power,
and by means of them we separated the others into their degrees or
pitches [loca] and we did this in such a way that there would be a scale;
so now also, after the discovery of the consonances [harmoniis] which
God Himself has embodied in the world, we must consequently see
whether those single consonances stand so separate that they have no
kinship with the rest, or whether all are in concord with one another.
Notwithstanding it is easy to conclude, without any further inquiry, that
those consonances were fitted together by the highest prudence in such
fashion that they move one another about within one frame, so to speak,
and do not jolt one another out of it; since indeed we see that in such a
manifold comparison of the same terms there is no place where
consonances do not occur. For unless in one scale all the consonances
were fitted to all, it could easily have come about (and it has come about
wherever necessity thus urges it) that many dissonances should exist.
For example, if someone had set up a major sixth between the first and
the second term, and likewise a major third between the second and the
7 See note to Intervals Compared with Harmonic Ratios, p. 1026.


42

third term, without taking the first into account, then he would admit a
dissonance and the discordant interval 12 : 25 between the first and
third.
But come now, let us see whether that which we have already inferred by
reasoning is really found in this way. [291] But let me premise some
cautions, that we may be the less impeded in our progress. First, for the
present, we must conceal those augments or diminutions which are less
than a semitone; for we shall see later on what causes they have. Second,
by continuous doubling or contrary halving of the movements, we shall
bring everything within the range of one octave, on account of the
sameness of consonance in all the octaves.
Accordingly the numbers wherein all the pitches or clefs [loca seu
claves] of the octave system are expressed have been set out in a table in
Book III, Chapter 7 8 , i.e., understand these numbers of the length of
two strings. As a consequence, the speeds of the movements will be in
the inverse ratios.
Now let the planetary movements be compared in terms of parts
continuously halved. Therefore

8 The table is as follows:


43

Now the aphelial movement of Saturn at its slowest—i.e., the slowest
movement—marks G, the lowest pitch in the system with the number
1´46″. Therefore the aphelial movement of the Earth will mark the same
pitch, but five octaves higher, because its number is 1´47″, and who
wants to quarrel about one second in the aphelial movement of Saturn?
But let us take it into account, nevertheless; the difference will not be
greater than 106 : 107, which is less than a comma. If you add 27″, one
quarter of this 1´47″, the sum will be 2´14″, although the perihelial
movement of Saturn has 2´15″; similarly the aphelial movement of
Jupiter, but one octave higher. Accordingly, these two movements mark
the note b, or else are very slightly higher. Take 36″, one third of 1´47″,
and add it to the whole; you will get as a sum 2´23″ for the note c; and
here's the perihelion of Mars of the same magnitude but four octaves
higher. To this same 1´47″ add also 54″, half of it, and the sum will be
2´41″ for the note d; and here the perihelion of Jupiter is at hand, but
one octave higher, for it occupies the nearest number, viz., 2´45″. If you
add two thirds, viz., 1´11″, the sum will be 2´58″; and here's the aphelion
of Venus at 2´58″. Accordingly, it will mark the pitch or the note e, but
five octaves higher. And the perihelial movement of Mercury, which is
3´0″, does not exceed it by much but is seven octaves higher. Finally,
divide the double of 1´47″, viz., 3´34″, into nine parts and subtract one
part of 24″ from the whole; 3´10″ will be left for the note f, which the
3´17″ of the aphelial movement of Mars marks approximately but three
octaves higher; and this number is slightly greater than the just number
and approaches the note f sharp. For if one sixteenth of 3´34″, viz.,
13½″, is subtracted from 3´34″, then 3´20½″ is left, to which 3´17″ is


44

very near. And indeed in music fsharp is often employed in place of f, as
we can see everywhere.
Accordingly all the notes of the major scale [cantus duri] (except the
note a which was not marked by harmonic division, in Book III, Chapter
2) are marked by all the extreme movements of the planets, except the
perihelial movements of Venus and the Earth [292] and the aphelial
movement of Mercury, whose number, 2´34″, approaches the
note c sharp. For subtract from the 2´41″ of d one sixteenth or 10″, and
2´30″ remains for the note c sharp. Thus only the perihelial movement
of Venus and the Earth are missing from this scale, as you may see in the
table.

On the other hand, if the beginning of the scale is made at 2´15″, the
aphelial movement of Saturn, and we must express the note G in those
degrees: then for the note A is 2´32″, which closely approaches the
aphelial movement of Mercury; for the note b flat, 2´42″, which is
approximately the perihelial movement of Jupiter, by the equipollence of
octaves; for the note c, 3´0″, approximately the perihelial movement of
Mercury and Venus; for the note d, 3´23″ and the aphelial movement of


45

Mars is not much graver, viz., 3´17″, so that here the number is about as
much less than its note as previously the same number was greater than
its note; for the note e flat, 3´36″, which the aphelial movement of the
Earth approximates; for the note e, 3´50″, and the perihelial movement
of the Earth is 3´49″; but the aphelial movement of Jupiter again
occupies g. In this way, all the notes except f are expressed within one
octave of the minor scale by most of the aphelial and perihelial
movements of the planets, especially by those which were previously
omitted, as you see in the table.

Previously, however, f sharp was marked and a omitted; now a is
marked, f sharp is omitted; for the harmonic division in Chapter 2 also
omitted the note f.


46

Accordingly, the musical scale or system of one octave with all its
pitches, by means of which natural song 9 is transposed in music, has
been expressed in the heavens by a twofold way and in two as it were
modes of song. There is this sole difference: in our harmonic sectionings
both ways start together from one and the same terminus G; but here, in
the planetary movements, that which was previously b now becomes G in
the minor mode.

For as in music 2160 : 1800, or 6 : 5, so in that system which the heavens
express, 1728 : 1440, namely, also 6 : 5; and so for most of the
remaining, 2160 : 1800, 1620, 1440, 1350, 1080 as 1728 : 1440, 1296,
1152, 1080, 864.
Accordingly you won't wonder any more that a very excellent order of
sounds or pitches in a musical system or scale has been set up by men,
since you see that they are doing nothing else in this business except to
play the apes of God the Creator and to act out, as it were, a certain
drama of the ordination of the celestial movements.
But there still remains another way whereby we may understand the
twofold musical scale in the heavens, where one and the same system but
a twofold tuning [tensio] is embraced, one at the aphelial movement of
Venus, the other at the perihelial, because the variety of movements of
this planet is of the least magnitude, as being such as is comprehended
within the magnitude of the diesis, the least concord. And the aphelial
tuning [tensio], as above, has been given to the aphelial movements of
Saturn, the Earth, Venus, and (relatively speaking) Jupiter, in G, e, b, but
9 Natural song: music in the basic major or minor system without accidentals. E. C., Jr.


47

to the perihelial movements of Mars and (relatively speaking) Saturn
and, as is apparent at first glance, to those of Mercury, in c, e, and b. On
the other hand, the perihelial tuning supplies a pitch even for the
aphelial movements of Mars, Mercury, and (relatively speaking) Jupiter,
but to the perihelial movements of Jupiter, Venus, and (relatively
speaking) Saturn, and to a certain extent to that of the Earth and
indubitably to that of Mercury too. For let us suppose that now not the
aphelial movement of Venus but the 3´3″ of the perihelial gets the pitch
of e; it is approached very closely by the 3´0″ of the perihelial movement
of Mercury, through a double octave, at the end of Chapter 4. But if 18″
or one tenth of this perihelial movement of Venus is subtracted, 2´45″
remains, the perihelion of Jupiter, which occupies the pitch of d; and if
one fifteenth or 12″ is added, the sum will be 3´15″, approximately the
perihelion of Mars which occupies the pitch of f; and thus in b, the
perihelial movement of Saturn and the aphelial movement of Jupiter
have approximately the same tuning. But one eighth, or 23″, if
multiplied by 5, gives 1´55″, which is the perihelial movement of the
Earth; and, although it does not square with the foregoing in the same
scale, as it does not give the interval 5 : 8 below e nor 24 : 25 aboveG,
nevertheless if now the perihelial movement of Venus and so too the
aphelial movement of Mercury, outside of the order, occupy the pitch eflat instead of e, then there the perihelial movement of the Earth will
occupy the pitch of G, and the aphelial movement of Mercury is in
concord, because 1´1″, or one third of 3´3″, if multiplied by 5, gives
5´5″, half of which, or 2´32″, approximates the aphelion of Mercury,
which in this extraordinary adjustment will occupy the pitch of c.
Therefore, all these movements are of the same tuning with respect to
one another; but the perihelial movement of Venus together with the
three (or five) prior movements, viz., in the same harmonic mode,
divides the scale differently from the aphelial movement of the same in
its tuning, viz., in the major mode [denere duro]. Moreover, the
perihelial movement of Venus, together with the two posterior
movements, divides the same scale differently, viz., not into concords
but merely into a different order of concords, namely one which belongs
to the minor mode [generis mollis].
But it is sufficient to have laid before the eyes in this chapter what is the
case casually, but it will be disclosed in Chapter 9 by the most lucid


48

demonstrations why each and every one of these things was made in this
fashion and what the causes were not merely of harmony but even of the
very least discord.


49

6. IN THE EXTREME PLANETARY MOVEMENTS
THE MUSICAL MODES OR TONES HAVE
SOMEHOW BEEN EXPRESSED
[294] This follows from the aforesaid and there is no need of many
words; for the single planets somehow mark the pitches of the system
with their perihelial movement, in so far as it has been appointed to the
single planets to traverse a certain fixed interval in the musical scale
comprehended by the definite notes of it or the pitches of the system,
and beginning at that note or pitch of each planet which in the preceding
chapter fell to the aphelial movement of that planet: G to Saturn and the
Earth, b to Jupiter, which can be transposed higher to G, f-sharp to
Mars,e to Venus, a to Mercury in the higher octave. See the single
movements in the familiar terms of notes. They do not form articulately
the intermediate positions, which you here see filled by notes, as they do
the extremes, because they struggle from one extreme to the opposite not
by leaps and intervals but by a continuum of tunings and actually
traverse all the means (which are potentially infinite)—which cannot be
expressed by me in any other way than by a continuous series of
intermediate notes. Venus remains approximately in unison and does
not equal even the least of the concordant intervals in the difference of
its tension.


50

But the signature of two accidentals (flats) in a common staff and the
formation of the skeletal outline of the octave by the inclusion of a
definite concordant interval are a certain first beginning of the
distinction of Tones or Modes [modorum]. Therefore the musical Modes
have been distributed among the planets. But I know that for the
formation and determination of distinct Modes many things are
requisite, which belong to human song, as containing (a) distinct [order
of] intervals; and so I have used the word somehow.
But the harmonist will be free to choose his opinion as to which Mode
each planet expresses as its own, since the extremes have been assigned
to it here. From among the familiar Modes, I should give to Saturn the
Seventh or Eighth, because if you place its key-note at G, the perihelial
movement ascends to b; to Jupiter, the First or Second Mode, because its
aphelial movement has been fitted to G and its perihelial movement
arrives at b flat; to Mars, the Fifth or Sixth Mode, not only because Mars
comprehends approximately the perfect fifth, which interval is common
to all the Modes, but principally because when it is reduced with the
others to a common system, it attains c with its perihelial movement and
touches f with its aphelial, which is the key-note of the Fifth or Sixth
Mode or Tone; I should give the Third or Fourth Mode to the Earth,
because its movement revolves within a semitone, while the first interval
of those Modes is a semitone; but to Mercury will belong indifferently all
the Modes or Tones on account of the greatness of its range; to Venus,
clearly none on account of the smallness of its range; but on account of
the common system the Third and Fourth Mode, because with reference
to the other planets it occupies e. (The Earth sings MI, FA, MI so that
you may infer even from the syllables that in this our domicile MIsery
and FAmine obtain.) 10

10 See note on hexachordal system.


51

7. THE UNIVERSAL CONSONANCES OF ALL SIX
PLANETS, LIKE COMMON FOUR-PART
COUNTERPOINT, CAN EXIST
[295] But now, Urania, there is need for louder sound while I climb
along the harmonic scale of the celestial movements to higher things
where the true archetype of the fabric of the world is kept hidden. Follow
after, ye modern musicians, and judge the thing according to your arts,
which were unknown to antiquity. Nature, which is never not lavish of
herself, after a lying-in of two thousand years, has finally brought you
forth in these last generations, the first true images of the universe. By
means of your concords of various voices, and through your ears, she has
whispered to the human mind, the favorite daughter of God the Creator,
how she exists in the innermost bosom.
(Shall I have committed a crime if I ask the single composers of this
generation for some artistic motet instead of this epigraph? The Royal
Psalter and the other Holy Books can supply a text suited for this. But
alas for you! No more than six are in concord in the heavens. For the
moon sings here monody separately, like a dog sitting on the Earth.
Compose the melody; I, in order that the book may progress, promise
that I will watch carefully over the six parts. To him who more properly
expresses the celestial music described in this work, Clio will give a
garland, and Urania will betroth Venus his bride.)
It has been unfolded above what harmonic ratios two neighbouring
planets would embrace in their extreme movements. But it happens very
rarely that two, especially the slowest, arrive at their extreme intervals at
the same time; For example, the apsides of Saturn and Jupiter are about
81° apart. Accordingly, while this distance between them measures out
the whole zodiac by definite twenty-year leaps 11 , eight hundred years
pass by, and nonetheless the leap which concludes the eighth century,
11 That is to say, since Saturn and Jupiter have one revolution with respect to one another every twenty

years, they are 81° apart once every twenty years, while the end-positions of this 81° interval traverse
the ecliptic in leaps, so to speak, and coincide with the apsides approximately once in eight hundred
years. C. G. W.


52

does not carry precisely to the very apsides; and if it digresses much
further, another eight hundred years must be awaited, that a more
fortunate leap than that one may be sought; and the whole route must be
repeated as many times as the measure of digression is contained in the
length of one leap. Moreover, the other single pairs of planets have
periods like that, although not so long. But meanwhile there occur also
other consonances of two planets, between movements whereof not both
are extremes but one or both are intermediate; and those consonances
exist as it were in different tunings [tensionibus]. For, because Saturn
tends from G to b, and slightly further, and Jupiter from b to d and
further; therefore between Jupiter and Saturn there can exist the
following consonances, over and above the octave: the major and minor
third and the perfect fourth, either one of the thirds through the tuning
which maintains the amplitude of the remaining one, but the perfect
fourth through the amplitude of a major whole tone. For there will be a
perfect fourth not merely from G of Saturn to cc of Jupiter but also
from A of Saturn todd of Jupiter and through all the intermediates
between the G and A of Saturn and the cc and dd of Jupiter. But the
octave and the perfect fifth exist solely at the points of the apsides. But
Mars, which got a greater interval as its own, received it in order that it
should also make an octave with the upper planets through some
amplitude of tuning. Mercury received an interval great enough for it to
set up almost all the consonances with all the planets within one of its
periods, which is not longer than the space of three months. On the other
hand, the Earth, and Venus much more so, on account of the smallness
of their intervals, limit the consonances, which they form not merely
with the others but with one another in especial, to visible fewness. But if
three planets are to concord in one harmony, many periodic returns are
to be awaited; nevertheless there are many consonances, so that they
may so much the more easily take place, while each nearest consonance
follows after its neighbour, and very often threefold consonances are
seen to exist between Mars, the Earth, and Mercury. But the
consonances of four planets now begin to be scattered throughout
centuries, and those of five planets throughout thousands of years.
But that all six should be in concord [296] has been fenced about by the
longest intervals of time; and I do not know whether it is absolutely
impossible for this to occur twice by precise evolving or whether that


53

points to a certain beginning of time, from which every age of the world
has flowed.

Harmonies of all the Planets, or Universal Harmonies in the Major Mode

But if only one sextuple harmony can occur, or only one notable one
among many, indubitably that could be taken as a sign of the Creation.
Therefore we must ask, in exactly how many forms are the movements of


54

all six planets reduced to one common harmony? The method of inquiry
is as follows: let us begin with the Earth and Venus, because these two
planets do not make more than two consonances and (wherein the cause
of this thing is comprehended) by means of very short intensifications of
the movements.
Therefore let us set up two, as it were, skeletal outlines of harmonies,
each skeletal outline determined by the two extreme numbers wherewith
the limits of the tunings are designated, and let us search out what fits in
with them from the variety of movements granted to each planet.
Saturn joins in this universal consonance with its aphelial movement,
the Earth with its aphelial, Venus approximately with its aphelial; at
highest tuning, Venus joins with its perihelial; at mean tuning, Saturn
joins with its perihelial, Jupiter with its aphelial, Mercury with its
perihelial. So Saturn can join in with two movements, Mars with two,
Mercury with four. But with the rest remaining, the perihelial movement
of Saturn and the aphelial of Jupiter are not allowed. But in their place,
Mars joins in with perihelial movement.
The remaining planets join in with single movements, Mars alone with
two, and Mercury with four.
[297] Accordingly, the second skeletal outline will be that wherein the
other possible consonance, 5 : 8, exists between the Earth and Venus.
Here one eighth of the 94´50″ of the diurnal aphelial movement of
Venus or 11´51″ +, if multiplied by 5, equals the 59´16″ of the movement
of the Earth; and similar parts of the 97´37″ of the perihelial movement
of Venus are equal to the 61´1″ of the movement of the Earth.
Accordingly, the other planets are in concord in the following diurnal
movements:


55

Here again, in the mean tuning Saturn joins in with its perihelial
movement, Jupiter with its aphelial, Mercury with its perihelial. But at
highest tuning approximately the perihelial movement of the Earth joins
in.


56

And here, with the aphelial movement of Jupiter and the perihelial
movement of Saturn removed, the aphelial movement of Mercury is
practically admitted besides the perihelial. The rest remain.
Therefore astronomical experience bears witness that the universal
consonances of all the movements can take place, and in the two modes
[generum], the major and minor, and in both genera of form, or (if I may
say so) in respect to two pitches and in any one of the four cases, with a
certain latitude of tuning and also with a certain variety in the particular
consonances of Saturn, Mars, and Mercury, of each with the rest; and
that is not afforded by the intermediate movements alone, but by all the
extreme movements too, except the aphelial movement of Mars and the
perihelial movement of Jupiter; because since the former
occupies f sharp; and the latter, d Venus, which occupies perpetually the
intermediate e flat or e, does not allow those neighbouring dissonances
in the universal consonance, as she would do if she had space to go
beyond e or e flat. This difficulty is caused by the wedding of the Earth
and Venus, or the male and the female. These two planets divide the
kinds [genera] of consonances into the major and masculine and the
minor and feminine, according as the one spouse has gratified the
other—namely, either the Earth is in its aphelion, as if preserving [298]
its marital dignity and performing works worthy of a man, with Venus
removed and pushed away to her perihelion as to her distaff; or else the
Earth has kindly allowed her to ascend into aphelion or the Earth itself
has descended into its perihelion towards Venus and as it were, into her
embrace, for the sake of pleasure, and has laid aside for a while its shield
and arms and all the works befitting a man; for at that time the
consonance is minor.
But if we command this contradictory Venus to keep quiet, i.e., if we
consider what the consonances not of all but merely of the five remaining
planets can be, excluding the movement of Venus, the Earth still
wanders around its g string and does not ascend a semitone above it.
Accordingly b♭, b, c, d, e♭, and e can be in concord with g, whereupon,
as you see, Jupiter, marking the d string with its perihelial movement, is
brought in. Accordingly, the difficulty about Mars’ aphelial movement
remains. For the aphelial movement of the Earth, which occupies g, does
not allow it on f sharp; but the perihelial movement, as was said above in


57

Chapter V, is in discord with the aphelial movement of Mars by about
half a diesis.

Here at the most grave tuning, Saturn and the Earth join in with their
aphelial movements; at the mean tuning, Saturn with its perihelial and
Jupiter with its aphelial; at the most acute, Jupiter with its perihelial.


58

Here the aphelial movement of Jupiter is not allowed, but at the most
acute tuning Saturn practically joins in with its perihelial movement.
But there can also exist the following harmony of the four planets,
Saturn, Jupiter, Mars, and Mercury, wherein too the aphelial movement
of Mars is present, but it is without latitude of tuning.


59

Accordingly the movements of the heavens are nothing except a certain
everlasting polyphony (intelligible, not audible) with dissonant tunings,
like certain syncopations or cadences (wherewith men imitate these
natural dissonances), which tends towards fixed and prescribed
clauses—the single clauses having six terms (like voices)— and which
marks out and distinguishes the immensity of time with those notes.
Hence it is no longer a surprise that man, the ape of his Creator, should
finally have discovered the art of singing polyphonically [per
concentum], which was unknown to the ancients, namely in order that
he might play the everlastingness of all created time in some short part
of an hour by means of an artistic concord of many voices and that he
might to some extent taste the satisfaction of God the Workman with His
own works, in that very sweet sense of delight elicited from this music
which imitates God.
NOTE: The comparison Kepler draws between the celestial harmonies
and the polyphonic music of his time may be clarified by a simple
example for four voices from—Palestrina, O Crux:


60

As will be observed each of the four voices (as it would also be with the
six to which Kepler refers) moves from one consonant chord to another
while following a graceful melodic line. Sometimes bits of scales or
passing tones are added to give a voice more melodic freedom
expressiveness. For the same reason a voice may remain on the same
note while the other voices change to a new chord. When this becomes a
dissonance (called a syncopation) in the new chord it usually resolves by
moving one step downward to a tone that is consonant with the other
voices. As in this example each section or "clause" ends with a cadence.
E. C., JR.


61

8. IN THE CELESTIAL HARMONIES WHICH PLANET
SINGS SOPRANO, WHICH ALTO, WHICH TENOR,
AND WHICH BASS?
Although these words are applied to human voices, while voices or
sounds do not exist in the heavens, on account of the very great
tranquillity of movements, and not even the subjects in which we find
the consonances are comprehended under the true genus of movement,
since we were considering the movements solely as apparent from the
sun, and finally, although there is no such cause in the heavens, as in
human singing, for requiring a definite number of voices in order to
make consonance (for first there was the number of the six planets
revolving around the sun, from the number of the five intervals taken
from the regular figures, and then afterwards—in the order of nature, not
of time—the congruence of the movements was settled): I do not know
why but nevertheless this wonderful congruence with human song has
such a strong effect upon me that I am compelled to pursue this part of
the comparison, also, even without any solid natural cause. For those
same properties which in Book III, [300] Chapter 16, custom ascribed to
the bass and nature gave legal grounds for so doing are somehow
possessed by Saturn and Jupiter in the heavens; and we find those of the
tenor in Mars, those of the alto are present in the Earth and Venus, and
those of the soprano are possessed by Mercury, if not with equality of
intervals, at least proportionately. For howsoever in the following
chapter the eccentricities of each planet are deduced from their proper
causes and through those eccentricities the intervals proper to the
movements of each, none the less there comes from that the following
wonderful result (I do not know whether it is occasioned by the
procurement and mere tempering of necessities): (1) as the bass is
opposed to the alto, so there are two planets which have the nature of the
alto, two that of the bass, just as in any Mode of song there is one [bass
and one alto] on either side, while there are single representatives of the
other single voices. (2) As the alto is practically supreme in a very narrow
range [in angustiis] on account of necessary and natural causes unfolded
in Book III, so the almost innermost planets, the Earth and Venus, have


62

the narrowest intervals of movements, the Earth not much more than a
semitone, Venus not even a diesis. (3) And as the tenor is free, but none
the less progresses with moderation, so Mars alone—with the single
exception of Mercury—can make the greatest interval, namely a perfect
fifth. (4) And as the bass makes harmonic leaps, so Saturn and Jupiter
have intervals which are harmonic, and in relation to one another pass
from the octave to the octave and perfect fifth. (5) And as the soprano is
the freest, more than all the rest, and likewise the swiftest, so Mercury
can traverse more than an octave in the shortest period. But this is
altogether per accidens; now let us hear the reasons for the
eccentricities.


63

9. THE GENESIS OF THE ECCENTRICITIES IN THE
SINGLE PLANETS FROM THE PROCUREMENT OF
THE CONSONANCES BETWEEN THEIR
MOVEMENTS
Accordingly, since we see that the universal harmonies of all six planets
cannot take place by chance, especially in the case of the extreme
movements, all of which we see concur in the universal harmonies—
except two, which concur in harmonies closest to the universal—and
since much less can it happen by chance that all the pitches of the system
of the octave (as set up in Book III) by means of harmonic divisions are
designated by the extreme planetary movements, but least of all that the
very subtle business of the distinction of the celestial consonances into
two modes, the major and minor, should be the outcome of chance,
without the special attention of the Artisan: accordingly it follows that
the Creator, the source of all wisdom, the everlasting approver of order,
the eternal and superexistent geyser of geometry and harmony, it
follows, I say, that He, the Artisan of the celestial movements Himself,
should have conjoined to the five regular solids the harmonic ratios
arising from the regular plane figures, and out of both classes should
have formed one most perfect archetype of the heavens: in order that in
this archetype, as through the five regular solids the shapes of the
spheres shine through on which the six planets are carried, so too
through the consonances, which are generated from the plane figures,
and deduced from them in Book III, the measures of the eccentricities in
the single planets might be determined so as to proportion the
movements of the planetary bodies; and in order that there should be
one tempering together of the ratios and the consonances, and that the
greater ratios of the spheres should yield somewhat to the lesser ratios of
the eccentricities necessary for procuring the consonances, and
conversely those in especial of the harmonic ratios which had a greater
kinship with each solid figure should be adjusted to the planets— in so
far as that could be effected by means of consonances. And in order that,
finally, in that way both the ratios of the spheres and the eccentricities of


64

the single planets might be born of the archetype simultaneously, while
from the amplitude of the spheres and the bulk of the bodies the periodic
times of the single planets might result.
[301] While I struggle to bring forth this process into the light of human
intellect by means of the elementary form customary with geometers,
may the Author of the heavens be favourable, the Father of intellects, the
Bestower of mortal senses, Himself immortal and superblessed, and may
He prevent the darkness of our mind from bringing forth in this work
anything unworthy of His Majesty, and may He effect that we, the
imitators of God by the help of the Holy Ghost, should rival the
perfection of His works in sanctity of life, for which He choose His
church throughout the Earth and, by the blood of His Son, cleansed it
from sins, and that we should keep at a distance all the discords of
enmity, all contentions, rivalries, anger, quarrels, dissensions, sects,
envy, provocations, and irritations arising through mocking speech and
the other works of the flesh; and that along with myself, all who possess
the spirit of Christ will not only desire but will also strive by deeds to
express and make sure their calling, by spurning all crooked morals of all
kinds which have been veiled and painted over with the cloak of zeal or
of the love of truth or of singular erudition or modesty over against
contentious teachers, or with any other showy garment. Holy Father,
keep us safe in the concord of our love for one another, that we may be
one, just as Thou art one with They Son, Our Lord, and with the Holy
Ghost, and just as through the sweetest bonds of harmonies Thou hast
made all Thy works one; and that from the bringing of Thy people into
concord the body of Thy Church may be built up in the Earth, as Thou
didst erect the heavens themselves out of harmonies.
PRIOR REASONS
I. AXIOM. It is reasonable that, wherever in general it could have been
done, all possible harmonies were due to have been set up between the
extreme movements of the planets taken singly and by twos, in order
that that variety should adorn the world.
II. AXIOM The five intervals between the six spheres to some extent
were due to correspond to the ratio of the geometrical spheres which
inscribe and circumscribe the five regular solids, and in the same order
which is natural to the figures.


65

Concerning this, see Chapter 1 and the Mysterium Cosmographicum and
the Epitome of Copernican Astronomy.
III. PROPOSITION. The intervals between the Earth and Mars, and
between the Earth and Venus, were due to be least, in proportion to their
spheres, and thereby approximately equal; middling and approximately
equal between Saturn and Jupiter, and between Venus and Mercury; but
greatest between Jupiter and Mars.
For by Axiom II, the planets corresponding in position to the figures
which make the least ratio of geometrical spheres ought likewise to make
the least ratio; but those which correspond to the figures of middling
ratio ought to make the greatest; and those which correspond to the
figures of greatest ratio, the greatest. But the order holding between the
figures of the dodecahedron and the icosahedron is the same as that
between the pairs of planets, Mars and the Earth, and the Earth and
Venus, and the order of the cube and octahedron is the same as that of
the pair Saturn and Jupiter and that of the pair Venus and Mercury; and,
finally, the order of the tetrahedron is the same as that of the pair Jupiter
and Mars (see Chapter 3). Therefore, the least ratio will hold between the
planetary spheres first mentioned, while that between Saturn and
Jupiter is approximately equal to that between Venus and Mercury; and,
finally, the greatest between the spheres of Jupiter and Mars.
IV. Axiom. All the planets ought to have their eccentricities diverse, no
less than a movement in latitude, and in proportion to those
eccentricities also their distances from the sun, the source of movement,
diverse.
As the essence of movement consists not in being but in becoming, so
too the form or figure of the region which any planet traverses in its
movement does not become solid immediately from the start but in the
succession of time acquires at last not only length but also breadth and
depth (its perfect ternary of dimensions); and, gradually, thus, by the
interweaving and piling up of many circuits, the form of a concave
sphere comes to be represented—just as out of the silk-worm's thread, by
the interweaving and heaping together of many circles, the cocoon is
built.


66

V. PROPOSITION. Two diverse consonances were to have been
attributed to each pair of neighbouring planets.
For, by Axiom IV, any planet has a longest and a shortest distance from
the sun, wherefore, by Chapter 3, it will have both a slowest movement
and a fastest. Therefore, there are two primary comparisons of the
extreme movements, one of the diverging movements in the two planets,
and the other of the converging. Now it is necessary that they be diverse
from one another, because the ratio of the diverging movements will be
greater, that of the converging, lesser. But, moreover, diverse
consonances had to exist by way of diverse pairs of planets, so that this
variety should make for the adornment of the world—by Axiom I—and
also because the ratios of the intervals between two planets are diverse,
by Proposition III. But to each definite ratio of the spheres there
correspond harmonic ratios, in quantitative kinship, as has been
demonstrated in Chapter 5 of this book.
VI. PROPOSITION. The two least consonances, 4 : 5 and 5 : 6, do not
have a place between two planets.
For
5 : 4=1,000 : 800
and
6 : 5 =1,000 : 833.
But the spheres circumscribed around the dodecahedron and
icosahedron have a greater ratio to the inscribed spheres than 1,000 :
795, etc., and these two ratios indicate the intervals between the nearest
planetary spheres, or the least distances. For in the other regular solids
the spheres are farther distant from one another. But now the ratio of the
movements is even greater than the ratios of the intervals, unless the
ratio of the eccentricities to the spheres is vast —by Article XIII of
Chapter 3. Therefore the least ratio of the movements is greater than 4 :
5 and 5 : 6. Accordingly, these consonances, being hindered by the
regular solids, receive no place among the planets. VII.
PROPOSITION. The consonance of the perfect fourth can have no place
between the converging movements of two planets, unless the ratios of


67

the extreme movements proper to them are, if compounded, more than
a perfect fifth.
For let 3 : 4 be the ratio between the converging movements. And first,
let there be no eccentricity, no ratio of movements proper to the single
planets, but both the converging and the mean movements the same;
then it follows that the corresponding intervals, which by this hypothesis
will be the semi-diameters of the spheres, constitute the 2/3d power of
this ratio, viz., 4480 : 5424 (by Chapter 3). But this ratio is already less
than the ratio of the spheres of any regular figure; and so the whole inner
sphere would be cut by the regular planes of the figure inscribed in any
outer sphere. But this is contrary to Axiom II.
Secondly, let there be some composition of the ratios between the
extreme movements, and let the ratio of the converging movements be 3
: 4 or 75 : 100, but let the ratio of the corresponding intervals be 1,000 :
795, since no regular figure has a lesser ratio of spheres. And because the
inverse ratio of the movements exceeds this ratio of the intervals by the
excess 750 : 795, then if this excess is divided into the ratio 1,000 : 795,
according to the doctrine of Chapter 3, the result will be 9434 : 7950, the
square root of the ratio of the spheres. Therefore the square of this
ratio, viz., 8901 : 6320, i.e., 10,000 : 7,100 is the ratio of the spheres.
Divide this by 1000 : 795, the ratio of the converging intervals, the result
will be 7100 : 7950, about a major whole tone. The' compound of the two
ratios which the mean movements have to the converging movements on
either side must be at least so great, in order that the perfect fourth may
be possible between the converging movements. Accordingly, the
compound ratio of the diverging extreme intervals to the converging
extreme intervals is about the square root of this ratio,i.e., two tones, and
again the converging intervals are the square of this, i.e., more than a
perfect fifth. Accordingly, if the compound of the proper movements of
two neighbouring planets is less than a perfect fifth, a perfect fourth will
not be possible between their converging movements.
VIII. PROPOSITION. The consonances 1 : 2 and 1 : 3, i.e., the octave and
the octave plus a fifth were due to Saturn and Jupiter.
For they are the first and highest of the planets and have obtained the
first figure, the cube, by Chapter 1 of this book; and these consonances
are first in the order of nature and are chief in the two families of figures,


68

the bisectorial or tetragonal and the triangular, by what has been said in
Book I. But that which is chief, the octave 1 : 2, is approximately greater
than the ratio of the spheres of the cube, [303] which is 1 : √3; wherefore
it is fitted to become the lesser ratio of the movements of the planets on
the cube, by Chapter 3, Article XIII; and, as a consequence, 1 : 3 serves as
the greater ratio.
But this is also the same as what follows: for if some consonance is to
some ratio of the spheres of the figures, as the ratio of the movements
apparent from the sun is to the ratio of the mean intervals, such a
consonance will duly be attributed to the movements. But it is natural
that the ratio of the diverging movements should be much greater than
the ratio of the 3/2th powers of the spheres, according to the end of
Chapter 3, i.e., it approaches the square of the ratio of the spheres; and
moreover 1 : 3 is the square of the ratio of the spheres of the cube, which
we call the ratio of 1: √3. Therefore, the ratio of the diverging movements
of Saturn and Jupiter is 1 : 3. (See above, Chapter 2, for many other
kinships of these ratios with the cube.)
IX. PROPOSITION. The private ratios of the extreme movements of
Saturn and Jupiter compounded were due to be approximately 2 : 3, a
perfect fifth.
This follows from the preceding; if the perihelial movement of Jupiter is
triple the aphelial movement of Saturn, and conversely the aphelial
movement of Jupiter is double the perihelial of Saturn, then 1 : 2 and 1 :
3 compounded inversely give 2 : 3.
X. Axiom. When choice is free in other respects, the private ratio of
movements, which is prior in nature or of a more excellent mode or even
which is greater, is due to the higher planet.
XI. PROPOSITION. The ratio of the aphelial movement of Saturn to the
perihelial was due to be 4 : 5, a major third, but that of Jupiter's
movements 5 : 6, a minor third.
For as compounded together they are equivalent to 2 : 3; but 2 : 3 can be
divided harmonically no other way than into 4 : 5 and 5 : 6. Accordingly
God the composer of harmonies divided harmonically the consonance 2 :
3, (by Axiom I) and the harmonic part of it which is greater and of the
more excellent major mode, as masculine, He gave to Saturn the greater


69

and higher planet, and the lesser ratio 5 : 6 to the lower one, Jupiter (by
Axiom X).
XII. PROPOSITION. The great consonance of 1 : 4, the double octave,
was due to Venus and Mercury.
For as the cube is the first of the primary figures, so the octahedron is the
first of the secondary figures, by Chapter 1 of this book. And as the cube
considered geometrically is outer and the octahedron is inner, i.e., the
latter can be inscribed in the former, so also in the world Saturn and
Jupiter are the beginning of the upper and outer planets, or from the
outside; and Mercury and Venus are the beginning of the inner planets,
or from the inside, and the octahedron has been placed between their
circuits: (see Chapter 3). Therefore, from among the consonances, one
which is primary and cognate to the octahedron is due to Venus and
Mercury. Furthermore, from among the consonances, after 1 : 2 and 1 : 3,
there follows in natural order 1 : 4; and that is cognate to 1 : 2, the
consonance of the cube, because it has arisen from the same cut of
figures, viz., the tetragonal, and is commensurable with it, viz., the
double of it; while the octahedron is also akin to, and commensurable
with the cube. Moreover, 1 : 4 is cognate to the octahedron for a special
reason, on account of the number four being in that ratio, while a
quadrangular figure lies concealed in the octahedron and the ratio of its
spheres is said to be 1 : √2.
Accordingly the consonance 1 : 4 is a continued power of this ratio, in the
ratio of the squares, i.e., the 4th power of 1: √2 (see Chapter 2).
Therefore, 1 : 4 was due to Venus and Mercury. And because in the cube
1 : 2 has been made the smaller consonance of the two, since the
outermost position is over against it, in the octahedron there will be 1 : 4,
the greater consonance of the two, as the innermost position is over
against it. But too, this is the reason why 1 : 4 has here been given as the
greater consonance, not as the smaller. 12 For since the ratio of the
spheres of the octahedron is the ratio of 1: √3, then if it is postulated that
the inscription of the octahedron among the planets is perfect (although
it is not perfect, but penetrates Mercury's sphere to some extent-which is
of advantage to us): accordingly, the ratio of the converging movements
12 Smaller (lesser) and greater consonances are equivalent to our modern "more closely spaced" and

"more widely spaced" consonances. E. C., Jr.


70

must be less than the 3/2th powers of 1 : √3 but indeed 1 : 3 is plainly the
square of the ratio 1: √3 and is thus greater than the exact ratio; all the
more then will 1 : 4 be greater than the exact ratio, as greater than 1 : 3.
Therefore, not even the square root of 1 : 4 is allowed between the
converging movements. Accordingly, 1 : 4 cannot be less than the
octahedric; so it will be greater.
Further: 1 : 4 is akin to the octahedric square, where the ratio of the
inscribed and circumscribed circles is 1: √2, just as 1 : 3 is akin to the
cube, where the ratio of the spheres is 1 : √3 . For as 1 : 3 is a power of 1 :
√3, viz., its square, [304] so too here 1 : 4 is a power of 1: √2, viz., twice
its square, i.e., its quadruple power. Wherefore, if 1 : 3 was due to have
been the greater consonance of the cube (by Proposition VII),
accordingly 1 : 4 ought to become the greater consonance of its
octahedron.
XIII. PROPOSITION. The greater consonance of approximately 1 : 8, the
triple octave, and the smaller consonance of 5 : 24, the minor third and
double octave, were due to the extreme movements of Jupiter and Mars.
For the cube has obtained 1 : 2 and 1 : 3, while the ratio of the spheres of
the tetrahedron, which is situated between Jupiter and Mars, called the
triple ratio, is the square of the ratio of the spheres of the cube, which is
called the ratio of 1: √3. Therefore, it was proper that ratios of
movements which are the squares of the cubic ratios should be applied to
the tetrahedron. But of the ratios 1 : 2 and 1 : 3 the following ratios are
the squares: 1 : 4 and 1 : 9. But 1 : 9 is not harmonic, and 1 : 4 has already
been used up in the octahedron. Accordingly, consonances neighbouring
upon these ratios were to have been taken, by Axiom I. But the lesser
ratio 1 : 8 and the greater 1 : 10 are the nearest. Choice between these
ratios is determined by kinship with the tetrahedron, which has nothing
in common with the pentagon, since 1 : 10 is of a pentagonal cut, but the
tetrahedron has greater kinship with 1 : 8 for many reasons (see Chapter
2).
Further, the following also makes for 1 : 8: just as 1 : 3 is the greater
consonance of the cube and 1 : 4 the greater consonance of the
octahedron, because they are powers of the ratios between the spheres of
the figures, so too 1 : 8 was due to be the greater consonance of the
tetrahedron, because as its body is double that of the octahedron


71

inscribed in it, as has been said in Chapter 1, so too the term 8 in the
tetrahedral ratio is double the term 4 in the tetrahedral ratio.
Further, just as 1 : 2 the smaller consonance of the cube, is one octave,
and 1 : 4, the greater consonance of the octahedron, is two octaves, so
already 1 : 8, the greater consonance of the tetrahedron, was due to be
three octaves. Moreover, more octaves were due to the tetrahedron than
to the cube and octahedron, because, since the smaller tetrahedral
consonance is necessarily greater than all the lesser consonances in the
other figures (for the ratio of the tetrahedral spheres is greater than all
the spheres of figures): too the greater tetrahedral consonance was due
to exceed the greater consonances of the others in number of octaves.
Finally, the triple of octave intervals has kinship with the triangular form
of the tetrahedron, and has a certain perfection, as follows: every three is
perfect; since even the octuple, the term [of the triple octave], is the first
cubic number of perfect quantity, namely of three dimensions.
A greater consonance neighbouring upon 1 : 4 or 6 : 24 is 5 : 24, while a
lesser is 6 : 20 or 3 : 10. But again 3 : 10 is of the pentagonal cut, which
has nothing in common with the tetrahedron. But on account of the
numbers 3 and 4 (from which the numbers 12, 24 arise) 5 : 24 has
kinship with the tetrahedron. For we are here neglecting the other lesser
terms, viz., 5 and 3, because their lightest degree of kinship is with
figures, as it is possible to see in Chapter 2. Moreover, the ratio of the
spheres of the tetrahedron is triple; but the ratio of the converging
intervals too ought to be approximately so great, by Axiom ii. By Chapter
3, the ratio of the converging movements approaches the inverse ratio of
the 3/2th powers of the intervals, but the 3/2th power of 3 : 1 is
approximately 1000 : 193. Accordingly, whereof the aphelial movement
of Mars is 1000, the [perihelial] of Jupiter will be slightly greater than
193 but much less than 333, which is one third of 1,000. Accordingly, not
the consonance 10 : 3, i.e., 1,000 : 333, but the consonance 24 : 5, i.e.,
1,000 : 208, takes place between the converging movements of Jupiter
and Mars.
XIV. PROPOSITION. The private ratio of the extreme movements of
Mars was due to be greater than 3 : 4, the perfect fourth, and
approximately 18 : 25.


72

For let there be the exact consonances 5 : 24 and 1 : 8 or 3 : 24, which are
commonly attributed to Jupiter and Mars (Proposition XIII). Compound
inversely 5 : 24, the lesser with 3 : 24, the greater; 3 : 5 results as the
compound of both ratios. But the proper ratio of Jupiter alone has been
found to be 5 : 6, in Proposition xi, above. Then compound this inversely
with the composition 3 : 5, i.e., compound 30 : 25 and 18 : 30; there
results as the proper ratio of Mars 18 : 25, which is greater than 18 : 24
or 3 : 4. But it will become still greater, if, on account of the ensuing
reasons, the common greater consonance 1 : 8 is increased.
XV. PROPOSITION. The consonances 2 : 3, the fifth; 5 : 8, the minor
sixth; and 3 : 5, the major sixth were to have been distributed among the
converging movements of Mars and the Earth, the Earth and Venus,
Venus and Mercury, and in that order.
For the dodecahedron and the icosahedron, the figures interspaced
between Mars, the Earth, and Venus have the least ratio between their
circumscribed and inscribed spheres. [305] Therefore from among
possible consonances the least are due to them, as being cognate for this
reason, and in order that Axiom u may have place. But the least
consonances of all, viz., 5 : 6 and 4 : 5, are not possible, by Proposition
IV. Therefore, the nearest consonances greater than they, viz., 3 : 4 or 2 :
3 or 5 : 8 or 3 : 5 are due to the said figures.
Again, the figure placed between Venus and Mercury, viz., the
octahedron, has the same ratio of its spheres as the cube. But by
Proposition vii, the cube received the octave as the lesser consonance
existing between the converging movements. Therefore, by
proportionality, so great a consonance, viz., 1 : 2, would be due to the
octahedron as the lesser consonance, if no diversity intervened. But the
following diversity intervenes: if compounded together, the private ratios
of the single movements of the cubic planets, viz., Saturn and Jupiter,
did not amount to more than 2 : 3; while, if compounded, the ratios of
the single movements of the octahedral planets, viz., Venus and Mercury
will amount to more than 2 : 3, as is apparent easily, as follows: For, as
the proportion between the cube and octahedron would require if it were
alone, let the lesser octahedral ratio be greater than the ratios here given,
and thereby clearly as great as was the cubic ratio, viz., 1 : 2; but the
greater consonance was 1 : 4, by Proposition XII. Therefore if the lesser


73

consonance 1 : 2 is divided into the one we have just laid down, 1 : 2, still
remains as the compound of the proper movements of Venus and
Mercury; but 1 : 2 is greater than 2 : 3 the compound of the proper
movements of Saturn and Jupiter; and indeed a greater eccentricity
follows upon this greater compound, by Chapter 3, but a lesser ratio of
the converging movements follows upon the greater eccentricity, by the
same Chapter 3. Wherefore by the addition of a greater eccentricity to
the proportion between the cube and the octahedron it comes about that
a lesser ratio than 1 : 2 is also required between the converging
movements of Venus and Mercury. Moreover, it was in keeping with
Axiom I that, with the consonance of the octave given to the planets of
the cube, another consonance which is very near (and by the earlier
demonstration less than 1 : 2) should be joined to the planets of the
octahedron. But 3 : 5 is proximately less than 1 : 2, and as the greatest of
the three it was due to the figure having the greatest ratio of its
spheres, viz., the octahedron. Accordingly, the lesser ratios, 5 : 8 and 2 :
3 or 3 : 4, were left for the icosahedron and dodecahedron, the figures
having a lesser ratio of their spheres.
But these remaining ratios have been distributed between the two
remaining planets, as follows. For as, from among the figures, though of
equal ratios between their spheres, the cube has received the consonance
1 : 2, while the octahedron the lesser consonance 3 : 5, in that the
compound ratio of the private movements of Venus and Mercury
exceeded the compound ratio of the private movements of Saturn and
Jupiter; so also although the dodecahedron has the same ratio of its
spheres as the icosahedron, a lesser ratio was due to it than to the
icosahedron, but very close on account of a similar reason, viz., because
this figure is between the Earth and Mars, which had a great eccentricity
in the foregoing. But Venus and Mercury, as we shall hear in the
following, have the least eccentricities. But since the octahedron has 3 :
5, the icosahedron, whose species are in a lesser ratio, has the next
slightly lesser, viz., 5 : 8; accordingly, either 2 : 3, which remains, or 3 : 4
was left for the dodecahedron, but more likely 2 : 3, as being nearer to
the icosahedral 5 : 8; since they are similar figures.
But 3 : 4 indeed was not possible. For although, in the foregoing, the
private ratio of the extreme movements of Mars was great enough, yet
the Earth—as has already been said and will be made clear in what


74

follows—contributed its own ratio, which was too small for the
compound ratio of both to exceed the perfect fifth. Accordingly,
Proposition vii, 3 : 4 could not have place. And all the more so, because—
as will follow in Proposition XVII—the ratio of the converging intervals
was due to be greater than 1,000 : 795.
XVI. PROPOSITION. The private ratios of movements of Venus and
Mercury, if compounded together, were due to make approximately 5 :
12.
For divide the lesser harmonic ratio attributed in Proposition xv to this
pair jointly into the greater of them, 1 : 4 or 3 : 12, by Proposition XII;
there results 5 : 12, the compound ratio of the private movements of
both. And so the private ratio of the extreme movements of Mercury
alone is less than 5 : 12, the magnitude of the private movement of
Venus. Understand this of these first reasons. For below, by the second
reasons, through the addition of some variation to the joint consonances
of both, it results that only the private ratio of Mercury is perfectly 5 : 12.
XVII. PROPOSITION. The consonance between the diverging
movements of Venus and the Earth could not be less than 5 : 12.
For in the private ratio of its movements Mars alone has received more
than the perfect fourth and more than 18 : 25, by Proposition XIV. But
their lesser consonance is the perfect fifth, [306] by Proposition XV.
Accordingly, the ratio compounded of these two parts is 12 : 25. But its
own private ratio is due to the Earth, by Axiom IV. Therefore, since the
consonance of the diverging movements is made up out of the said three
elements, it will be greater than 12 : 25. But the nearest consonance
greater than 12 : 25, i.e., 60 : 125, is 5 : 12, viz., 60 : 144. Wherefore, if
there is need of a consonance for this greater ratio of the two planets, by
Axiom I, it cannot be less than 60 : 144 or 5 : 12.
Therefore up to now all the remaining pairs of planets have received
their two consonances by necessary reasons; the pair of the Earth and
Venus alone has as yet been allotted only one consonance, 5 : 8, by the
axioms so far employed. Therefore, we must now take a new start and
inquire into its remaining consonance, viz., the greater, or the
consonance of the diverging movements.
POSTERIOR REASONS


75

XVIII. Axiom. The universal consonances of movements were to be
constituted by a tempering of the six movements, especially in the case of
the extreme movements.
This is proved by Axiom I.
XIX. Axiom. The universal consonances had to come out the same within
a certain latitude of movements, namely, in order that they should occur
the more frequently.
For if they had been limited to indivisible points of the movements, it
could have happened that they would never occur, or very rarely.
XX. Axiom. As the most natural division of the kinds [generum] of
consonances is into major and minor, as has been proved in Book 3, so
the universal consonances of both kinds had to be procured between the
extreme movements of the planets.
XXI. Axiom. Diverse species of both kinds of consonances had to be
instituted, so that the beauty of the world might well be composed out of
all possible forms of variety—and by means of the extreme movements,
at least by means of some extreme movements.
By Axiom I.
XXII. PROPOSITION. The extreme movements of the planets had to
designate pitches or strings [chordas] of the octave system, or
notes [claves] of the musical scale.
For the genesis and comparison of consonances beginning from one
common term has generated the musical scale, or the division of the
octave into its pitches or tones [sonos], as has been proved in Book 3.
Accordingly, since varied consonances between the extremes of
movements are required, by Axioms I, XX, and XXI, wherefore the real
division of some celestial system or harmonic scale by the extremes of
movements is required.
XXIII. PROPOSITION. It was necessary for there to be one pair of
planets, between the movements of which no consonances could exist
except the major sixth 3 : 5 and the minor sixth 5 : 8.


76

For since the division into kinds of consonances was necessary, by Axiom
XX, and by means of the extreme movements at the apsides, by XXII,
because solely the extremes, viz., the slowest and the fastest, need the
determination of a manager and orderer, the intermediate tensions come
of themselves, without any special care, with the passage of the planet
from the slowest movement to the fastest: accordingly, this ordering
could not take place otherwise than by having the diesis or 24 : 25
designated by the extremes of the two planetary movements, in that the
kinds of consonances are distinguished by the diesis, as was unfolded in
Book 3.
But the diesis is the difference either between two thirds, 4 : 5 and 5 : 6,
or between two sixths, 3 : 5 and 5 : 8, or between those ratios increased
by one or more octave intervals. But the two thirds, 4 : 5 and 5 : 6, did
not have place between two planets, by Proposition vi, and neither the
thirds nor the sixths increased by the interval of an octave have been
found, except 5 : 12 in the pair of Mars and the Earth, and still not
otherwise than along with the related 2 : 3, and so the intermediate
ratios 5 : 8 and 3 : 5 and 1 : 2 were alike admitted. Therefore, it remains
that the two sixths, 3 : 5 and 5 : 8, were to be given to one pair of planets.
But too the sixths alone were to be granted to the variation of their
movements, in such fashion that they would neither expand their terms
to the proximately greater interval of one octave, 1 : 2, [307] nor contract
them to the narrows of the proximately lesser interval of the fifth, 2 : 3.
For, although it is true that the same two planets, which make a perfect
fifth with their extreme converging movements, can also make sixths and
thus traverse the diesis too, still this would not smell of the singular
providence of the Orderer of movements. For the diesis, the least
interval—which is potentially latent in all the major intervals
comprehended by the extreme movements—is itself at that time
traversed by the intermediate movements varied by continuous tension,
but it is not determined by their extremes, since the part is always less
than the whole, viz., the diesis than the greater interval 3 : 4 which exists
between 2 : 3 and 1 : 2 and which whole would be here assumed to be
determined by the extreme movements.
XXIV. PROPOSITION. The two planets which shift the kind [genus] of
harmony, which is the difference between the private ratios of the
extreme movements, ought to make a diesis, and the private ratio of one


77

ought to be greater than a diesis, and they ought to make one of the
sixths with their aphelial movements and the other with their perihelial.
For, since the extremes of the movements make two consonances
differing by a single diesis, that can take place in three ways. For either
the movement of one planet will remain constant and the movement of
the other will vary by a diesis, or both will vary by half a diesis and make
3 : 5, a major sixth, when the upper is at its aphelion and the lower in its
perihelion, and when they move out of those intervals and advance
towards one another, the upper into its perihelion and the lower into its
aphelion, they make 5 : 8, a minor sixth; or, finally, one varies its
movement from aphelion to perihelion more than the other does, and
there is an excess of one diesis, and thus there is a major sixth between
the two aphelia, and a minor sixth between the two perihelia. But the
first way is not legitimate, for one of these planets would be without
eccentricity, contrary to Axiom IV. The second way was less beautiful
and less expedient; less beautiful, because less harmonic, for the private
ratios of the movements of the two planets would have been out of tune
[inconcinnae], for whatever is less than a diesis is out of tune; moreover
it occasions one single planet to labour under this ill-concordant small
difference—except that indeed it could not take place, because in this
way the extreme movements would have wandered from the pitches of
the system or the notes [clavibus] of the musical scale, contrary to
Proposition xxii. Moreover, it would have been less expedient, because
the sixths would have occurred only at those moments in which the
planets would have been at the contrary apsides; there would have been
no latitude within which these sixths and the universal consonances
related to them could have occurred; accordingly, these universal
consonances would have been very rare, with all the [harmonic]
positions of the planets reduced to the narrow limits of definite and
single points on their orbits, contrary to Axiom xix. Accordingly, the
third way remains: that both of the planets should vary their own private
movements, but one more than the other, by one full diesis at the least.
XXV. PROPOSITION. The higher of the planets which shift the kind of
harmony ought to have the ratio of its private movements less than a
minor whole tone 9 : 10; while the lower, less than a semitone 15 : 16.


78

For they will make 3 : 5 either with their aphelial movements or with
their perihelial, by the foregoing proposition. Not with their perihelial,
for then the ratio of their aphelial movements would be 5 : 8.
Accordingly, the lower planet would have its private ratio one diesis
more than the upper would, by the same foregoing proposition. But that
is contrary to Axiom X. Accordingly, they make 3 : 5 with their aphelial
movements, and with their perihelial 5 : 8, which is 24 : 25 less than the
other. But if the aphelial movements make 3 : 5, a major sixth, therefore,
the aphelial movement of the upper together with the perihelial of the
lower will make more than a major sixth; for the lower planet will
compound directly its full private ratio.
In the same way, if the perihelial movements make 5 : 8, a minor sixth,
the perihelial movement of the upper and the aphelial movement of the
lower will make less than a minor sixth; for the lower planet will
compound inversely its full private ratio. But if the private ratio of the
lower equalled the semitone 15 : 16, then too a perfect fifth could occur
over and above the sixths, because the minor sixth, diminished by a
semitone, because the perfect fifth; but this is contrary to Proposition
XXIII. Accordingly, the lower planet has less than a semitone in its own
interval. And because the private ratio of the upper is one diesis greater
than the private ratio of the lower, but the diesis compounded with the
semitone makes 9 : 10 the minor whole tone.
XXVI. PROPOSITION. On the planets which shift the kind of harmony,
the upper was due to have either a diesis squared, 576 : 625, i.e.,
approximately 12 : 13, as [308] the interval made by its extreme
movements, or the semitone 15 : 16, or something intermediate differing
by the comma 80 : 81 either from the former or the latter; while the
lower planet, either the simple diesis 24 : 25, or the difference between a
semitone and a diesis, which is 125 : 128, i.e., approximately 42 : 43; or,
finally and similarly, something intermediate differing either from the
former or from the latter by the comma 80 : 81, viz., the upper planet
ought to make the diesis squared diminished by a comma, and the lower,
the simple diesis diminished by a comma.
For, by Proposition XXV, the private ratio of the upper ought to be
greater than a diesis, but by the preceding proposition less than the
[minor] whole tone 9 : 10. But indeed the upper planet ought to exceed


79

the lower by one diesis, by Proposition XXIV. And harmonic beauty
persuades us that, even if the private ratios of these planets cannot be
harmonic, on account of their smallness, they should at least be from
among the concordant [ex concinnis] if that is possible, by Axiom I. But
there are only two concords less than 9 : 10, the [minor] whole tone, viz.,
the semitone and the diesis; but they differ from one another not by the
diesis but by some smaller interval, 125 : 128. Accordingly, the upper
cannot have the semitone; nor the lower, the diesis; but either the upper
will have the semitone 15 : 16, and the lower, 125 : 128, i.e., 42 : 43; or
else the lower will have the diesis 24 : 25, but the upper the diesis
squared, approximately 12 : 13. But since the laws of both planets are
equal, therefore, if the nature of the concordant had to be violated in
their private ratios, it had to be violated equally in both, so that the
difference between their private intervals could remain an exact diesis,
which is necessary for distinguishing the kinds of consonances, by
Proposition XXIV. But the nature of the concordant was then violated
equally in both, if the interval whereby the private ratio of the upper
planet fell short of the diesis squared and exceeded the semitone is the
same interval whereby the private ratio of the lower planet fell short of a
simple diesis and exceeded the interval 125 : 128.
Furthermore, this excess or defect was due to be the comma 80 : 81,
because, once more, no other interval was designated by the harmonic
ratios, and in order that the comma might be expressed among the
celestial movements as it is expressed in harmonics, namely, by the mere
excess and defect of the intervals in respect to one another. For in
harmonics the comma distinguishes between major and minor whole
tones and does not appear in any other way.
It remains for us to inquire which ones of the intervals set forth are
preferable —whether the diesis, the simple diesis for the lower planet
and the diesis squared for the upper, or the semitone for the upper and
125 : 128 for the lower. And the dieses win by the following arguments:
For although the semitone has been variously expressed in the musical
scale, yet its allied ratio 125 : 128 has not been expressed. On the other
hand, the diesis has been expressed variously and the diesis squared
somehow, viz., in the resolution of whole tones into dieses, semitones,
and lemmas; for then, as has been said in Book III, Chapter 8, two dieses
proximately succeed one another in two pitches. The other argument is


80

that in the distinction into kinds, the laws of the diesis are proper but not
at all those of the semitone. Accordingly, there had to be greater
consideration of the diesis than of the semitone. It is inferred from
everything that the private ratio of the upper planet ought to be 2916 :
3125 or approximately 14 : 15, and that of the lower, 243 : 250 or
approximately 35 : 36.
It is asked whether the Highest Creative Wisdom has been occupied in
making these tenuous little reckonings. I answer that it is possible that
many reasons are hidden from me, but if the nature of harmony has not
allowed weightier reasons—since we are dealing with ratios which
descend below the magnitude of all concords—it is not absurd that God
has followed even those reasons, wherever they appear tenuous, since He
has ordained nothing without cause. It would be far more absurd to
assert that God has taken at random these magnitudes below the limits
prescribed for them, the minor whole tone; and it is not sufficient to say:
He took them of that magnitude because He chose to do so. For in
geometrical things, which are subject to free choice, God chose nothing
without a geometrical cause of some sort, as is apparent in the edges of
leaves, in the scales of fishes, in the skins of beasts and their spots and
the order of the spots, and similar things.
XXVII. PROPOSITION. The ratio of movements of the Earth and Venus
ought to have been greater than a major sixth between the aphelial
movements; less than a minor sixth between the perihelial movements.
By Axiom XX it was necessary to distinguish the kinds of consonances.
But by Proposition XXIII that could not be done except through the
sixths. Accordingly, since by Proposition XV the Earth and Venus,
planets next to one another and icosahedral, had received the minor
sixth, 5 : 8, it was necessary for the other sixth, 3 : 5, to be assigned to
them, but not between the converging or diverging extremes, but
between the extremes of the same field, one sixth [309] between the
aphelial, and the other between the perihelial, by Proposition XXIV.
Furthermore, the consonance 3 : 5 is cognate to the icosahedron, since
both are of the pentagonal cut. See Chapter 2.
Behold the reason why exact consonances are found between the aphelial
and perihelial movements of these two planets, but not between the
converging, as in the case of the upper planets.


81

XXVIII. PROPOSITION. The private ratio of movements fitting the
Earth was approximately 14 : 15, Venus, approximately 35:36.
For these two planets had to distinguish the kinds of consonances, by the
preceding proposition; therefore, by Proposition XXVI, the Earth as the
higher was due to receive the interval 2916 : 3125, i.e., approximately 14 :
15, but Venus as the lower the interval 243 : 250, i.e., approximately 35 :
36.
Behold the reason why these two planets have such small eccentricities
and, in proportion to them, small intervals or private ratios of the
extreme movements, although nevertheless the next higher planet, Mars,
and the next lower, Mercury, have marked eccentricities and the greatest
of all. And astronomy confirms the truth of this; for in Chapter 4 the
Earth clearly had 14 : 15, but Venus 34 : 35, which astronomical certitude
can barely discern from 35 : 36 in this planet.
XXIX. PROPOSITION. The greater consonance of the movements of
Mars and the Earth, viz., that of the diverging movements, could not be
from among the consonances greater than 5 : 12.
Above, in Proposition XVII, it was not any one of the lesser ratios; but
now it is not any one of the greater ratios either. For the other common
or lesser consonance of these two planets is 2 : 3, when the private ratio
of Mars, which by Proposition XIV exceeds 18 : 25, makes more than 12 :
25, i.e., 60 : 125. Accordingly, compound the private ratio of the Earth 14
: 15, i.e., 56 : 60, by the preceding proposition. The compound ratio is
greater than 56 : 125, which is approximately 4 : 9, viz., slightly greater
than an octave and a major whole tone. But the next greater consonance
than the octave and whole tone is 5 : 12, the octave and minor third.
Note that I do not say that this ratio is neither greater nor smaller than 5
: 12; but I say that if it is necessary for it to be harmonic, no other
consonance will belong to it.
XXX. PROPOSITION. The private ratio of movements of Mercury was
due to be greater than all the other private ratios.
For by Proposition XVI the private movements of Venus and Mercury
compounded together were due to make about 5 : 12. But the private
ratio of Venus, taken separately, is only 243 : 250, i.e., 1458 : 1500. But if


82

it is compounded inversely with 5 : 12, i.e., 625 : 1500, Mercury singly is
left with 625 : 1458, which is greater than an octave and a major whole
tone; although the private ratio of Mars, which is the greatest of all those
among the remaining planets, is less than 2 : 3, i.e., the perfect fifth.
And thereby the private ratios of Venus and Mercury, the lowest planets,
if compounded together, are approximately equal to the compounded
private ratios of the four higher planets, because, as will now be apparent
immediately, the compounded private ratios of Saturn and Jupiter
exceed 2 : 3; those of Mars fall somewhat short of 2 : 3: all compounded,
4 : 9, i.e., 60 : 135. Compound the Earth's 14 : 15, i.e., 56 : 60, the result
will be 56 : 135, which is slightly greater than 5 : 12, which just now was
the compound of the private ratios of Venus and Mercury. But this has
not been sought for nor taken from any separate and singular archetype
of beauty but comes of itself, by the necessity of the causes bound
together by the consonances hitherto established.
XXXI. PROPOSITION. The aphelial movement of the Earth had to
harmonize with the aphelial movement of Saturn, through some certain
number of octaves.
For, by Proposition xviii, it was necessary for there to be universal
consonances, wherefore also there had to be a consonance of Saturn with
the Earth and Venus. But if one of the extreme movements of Saturn had
harmonized with neither of the Earth's and Venus’, this would have been
less harmonic than if both of its extreme movements had harmonized
with these planets, by Axiom I. Therefore both of Saturn's extreme
movements had to harmonize, the aphelial with one of these two planets,
the perihelial with the other, since nothing would hinder, as was the case
with the first planet. Accordingly these consonances will be either
identisonant 13 [identisonae] or diversisonant [diversisonae], i.e., either
of continued double proportion or of some other. But both of them
cannot be of some other proportion, for between the terms 3 : 5 (which
determine the greater consonance between the aphelial movements of
the Earth and Venus, by Proposition XXVII) two harmonic means
cannot be set up; for the sixth cannot be divided into three intervals (see
Book III). Accordingly, Saturn could not, [310] by means of both its
movements, make an octave with the harmonic means between 3 and 5;
13 "Identisonant consonances" are such as 3 : 5, 3 : 10, 3 : 20, etc.


83

but in order that its movements should harmonize with the 3 of the earth
and the 5 of Venus, it is necessary that one of those terms should
harmonize identically, or through a certain number of octaves, with the
others, viz., with one of the said planets. But since the identisonant
consonances are more excellent, they had to be established between the
more excellent extreme movements, viz., between the aphelial, because
too they have the position of a principle on account of the altitude of the
planets and because the Earth and Venus claim as their private ratio
somehow and as a prerogative the consonance 3 : 5, with which as their
greater consonance we are now dealing. For although, by Proposition
XXII, this consonance belongs to the perihelial movement of Venus and
some intermediate movement of the Earth, yet the start is made at the
extreme movements and the intermediate movements come after the
beginnings.
Now, since on one side we have the aphelial movement of Saturn at its
greatest altitude, on the other side the aphelial movement of the Earth
rather than Venus is to be joined with it, because of these two planets
which distinguish the kinds of harmony, the Earth, again, has the greater
altitude. There is also another nearer cause: the posterior reasons—with
which we are now dealing—take away from the prior reasons
but only with respect to minima, and in harmonics that is with respect to
all intervals less than concords. But by the prior reasons the aphelial
movement not of Venus but of the Earth, will approximate the
consonance of some number of octaves to be established with the
aphelial movement of Saturn. For compound together, first, 4 : 5 the
private ratio of Saturn's movements, i.e., from the aphelion to the
perihelial of Saturn (Proposition XI), secondly, the 1 : 2 of the converging
movements of Saturn and Jupiter, i.e., from the perihelion of Saturn to
the aphelion of Jupiter (by Proposition VIII), thirdly, the 1 : 8 of the
diverging movements of Jupiter and Mars, i.e., from the aphelion of
Jupiter to the perihelion of Mars (by Proposition XIV), fourthly, the 2 : 3
of the converging movements of Mars and the Earth, i.e., from the
perihelion of Mars to the aphelion of the Earth (by Proposition XV): you
will find between the aphelion of Saturn and the perihelion of the Earth
the compound ratio 1 : 30, which falls short of 1 : 32, or five octaves, by
only 30 : 32, i.e., 15 : 16 or a semitone. And so, if a semitone, divided into
particles smaller than the least concord, is compounded with these four


84

elements there will be a perfect consonance of five octaves between the
aphelial movements of Saturn and the Earth, which have been set forth.
But in order for the same aphelial movement of Saturn to make some
number of octaves with the aphelial movement of Venus, it would have
been necessary to snatch approximately a whole perfect fourth from the
prior reasons; for if you compound 3 : 5, which exists between the
aphelial movements of the Earth and Venus, with the ratio 1 : 30
compounded of the four prior elements, then as it were from the prior
reasons, 1 : 50 is found between the aphelial movements of Saturn and
Venus: This interval differs from 1 : 32, or five octaves, by 32 : 50, i.e., 16
: 25, which is a perfect fifth and a diesis; and from six octaves, or 1 : 64, it
differs by 50 : 64, i.e., 25 : 32, or a perfect fourth minus a diesis.
Accordingly, an identisonant consonance was due to be established, not
between the aphelial movements of Venus and Saturn but between those
of Venus and the Earth, so that Saturn might keep a diversisonant
consonance with Venus. XXXII. PROPOSITION. In the universal
consonances of planets of the minor scale the exact aphelial movement
of Saturn could not harmonize precisely with the other planets.
For the Earth by its aphelial movement does not concur in the universal
consonance of the minor scale, because the aphelial movements of the
Earth and Venus make the interval 3 : 5, which is of the major scale (by
Proposition XVII). But by its aphelial movement Saturn makes an
identisonant consonance with the aphelial movement of the Earth (by
Proposition XXXI). Therefore, neither does Saturn concur by its aphelial
movement. Nevertheless, in place of the aphelial movement there follows
some faster movement of Saturn, very near to the aphelial, and also in
the minor scale—as was apparent in Chapter 7.
XXXIII. PROPOSITION. The major kind of consonances and musical
scale is akin to the aphelial movements; the minor to the perihelial.
For although a major consonance [dura harmonia] is set up not only
between the aphelial movement of the Earth and the aphelial movement
of Venus but also between the lower aphelial movements and the lower
movements of Venus as far as its perihelion; and, conversely, there is a
minor consonance not merely between the perihelial movement of Venus
and the perihelial of the Earth but also between the higher movements of
Venus as far as the aphelion and the higher movements of the Earth (by


85

Propositions XX and XXIV). Accordingly, the major scale is designated
properly only in the aphelial movements, the minor, only in the
perihelial.
XXXIV. PROPOSITION. The major scale is more akin to the upper of the
two planets, the minor, to the lower.
[311] For, because the major scale is proper to the aphelial movements,
the minor, to the perihelial (by the preceding proposition), while the
aphelial are slower and graver than the perihelial; accordingly, the major
scale is proper to the slower movements, the minor to the faster. But the
upper of the two planets is more akin to the slow movements, the lower,
to the fast, because slowness of the private movement always follows
upon altitude in the world. Therefore, of two planets which adjust
themselves to both modes, the upper is more akin to the major mode of
the scale, the lower, to the minor. Further, the major scale employs the
major intervals 4 : 5 and 3 : 5, and the minor, the minor ones, 5 : 6 and 5
: 8. But, moreover, the upper planet has both a greater sphere and
slower, i.e., greater movements and a lengthier circuit; but those things
which agree greatly on both sides are rather closely united.
XXXV. PROPOSITION. Saturn and the Earth embrace the major scale
more closely Jupiter and Venus, the minor.
For, first, the Earth, as compared with Venus and as designating both
scales along with Venus, is the upper. Accordingly, by the preceding
proposition, the Earth embraces the major scale chiefly; Venus, the
minor. But with its aphelial movement Saturn harmonizes with the
Earth's aphelial movement, through an octave (by Proposition XXXI):
wherefore too (by Proposition XXXIII) Saturn embraces the major scale.
Secondly, by the same proposition, Saturn by means of its aphelial
movement nurtures more the major scale and (by Proposition XXXII)
spits out the minor scale. Accordingly, it is more closely related to the
major scale than to the minor, because the scales are properly designated
by the extreme movements.
Now as regards Jupiter, in comparison with Saturn it is lower; therefore
as the major scale is due to Saturn, so the minor is due to Jupiter, by the
preceding proposition.


86

XXXVI. PROPOSITION. The perihelial movement of Jupiter had to
concord with the perihelial movement of Venus in one scale but not also
in the same consonance; and all the less so, with the perihelial
movement of the Earth.
For, because the minor scale chiefly was due to Jupiter, by the preceding
proposition, while the perihelial movements are more akin to the minor
scale (by Proposition XXX), accordingly, by its perihelial movement
Jupiter had to designate the key of the minor scale, viz., its definite pitch
or key-note [phthongum]. But too the perihelial movements of Venus
and the Earth designate the same scale (by Proposition XXVIII);
therefore the perihelial movement of Jupiter was to be associated with
their perihelial movements in the same tuning, but it could not
constitute a consonance with the perihelial movements of Venus. For,
because (by Proposition VIII) it had to make about 1 : 3 with the aphelial
movement of Saturn, i.e., the note [clavem] d of that system, wherein the
aphelial movement of Saturn strikes the note G, but the aphelial
movement of Venus the note e: accordingly, it approached the
note e within an interval of least consonance. For the least consonance is
5 : 6, but the interval between d and e is much smaller, viz., 9 : 10, a
whole tone. And although in the perihelial tension [tensione] Venus is
raised from the d of the aphelial tension yet this elevation is less than a
diesis, (by Proposition XXVIII). But the diesis (and hence any smaller
interval) if compounded with a minor whole tone does not yet equal 5 : 6
the interval of least consonance. Accordingly, the perihelial movement of
Jupiter could not observe 1 : 3 or thereabouts with the aphelial
movement of Saturn and at the same time harmonize with Venus. Nor
with the Earth. For if the perihelial movement of Jupiter had been
adjusted to the key of the perihelial movement of Venus in the same
tension in such fashion that below the quantity of least concord it should
preserve with the aphelial movement of Saturn the interval 1 : 3, viz., by
differing from the perihelial movement of Venus by a minor whole tone,
9 : 10 or 36 : 40 (besides some octaves) towards the low. Now the
perihelial movement of the Earth differs from the same perihelial
movement of Venus by 5 : 8, i.e., by 25 : 40. And so the perihelial
movements of the Earth and Jupiter differ by 25 : 36, over and above
some number of octaves. But that is not harmonic, because it is the
square of 5 : 6, or a perfect fifth diminished by one diesis.


87

XXXVII. PROPOSITION. It was necessary for an interval equal to the
interval of Venus to accede to the 2 : 3 of the compounded private
consonances of Saturn and Jupiter and to 1 : 3 the great consonance
common to them.
For with its aphelial movement Venus assists in the proper designation
of the major scale; with its perihelial, that of the minor scale, by
Propositions XXVII and XXXIII. But by its aphelial movement Saturn
had to be in concord also with the major scale and thus with the aphelial
movement of Venus, by Proposition XXXV, but Jupiter's perihelial with
the perihelial of Venus, by the preceding proposition. Accordingly, as
great as Venus makes its interval from aphelial to perihelial to be, so
great an interval must also accede to that movement of Jupiter which
makes 1 : 3 with the aphelial movement of Saturn—to the very perihelial
movement of Jupiter. But the consonance of the converging movements
of Jupiter and Saturn is precisely 1 : 2, by Proposition VIII. Accordingly,
if the interval 1 : 2 is divided into the interval [312] greater than 1 : 3,
there results, as the compound of the private ratios of both, something
which is proportionately greater than 2 : 3.
Above, in Proposition XXVI, the private ratio of the movements of Venus
was 243 : 250 or approximately 35 : 36; but in Chapter 4, between the
aphelial movement of Saturn and the perihelial movement of Jupiter
there was found a slightly greater excess beyond 1 : 3, viz., between 26 :
27 and 27 : 28. But the quantity here prescribed is absolutely equalled,
by the addition of a single second to the aphelial movement of Saturn,
and I do not know whether astronomy can discern that difference.
XXXVIII. PROPOSITION. The increment 243 : 250 to 2 : 3, the
compound of the private ratios of Saturn and Jupiter, which was up to
now being established by the prior reasons, was to be distributed among
the planets in such fashion that of it the comma 80 : 81 should accede to
Saturn and the remainder, 19,683 : 20,000 or approximately 62 : 63, to
Jupiter.
It follows from Axiom XIX that this was to have been distributed
between both planets so that each could with some latitude concur in the
universal consonances of the scale akin to itself. But the interval 243 :
250 is smaller than all concords: accordingly no harmonic rules remain
whereby it may be divided into two concordant parts, with the single


88

exception of those of which there was need in the division of 24 : 25, the
diesis, above in Proposition XXVI; namely, in order that it may be
divided into the comma 80 : 81 (which is a primary one of those intervals
which are subordinate to the concordant) and into the remainder 19,683
: 20,000, which is slightly greater than a comma, viz., approximately 62 :
63. But not two but one comma had to be taken away, lest the parts
should become too unequal, since the private ratios of Saturn and
Jupiter are approximately equal (according to Axiom X extended even to
concords and parts smaller than those) and also because the comma is
determined by the intervals of the major whole tone and minor whole
tone, not so two commas. Furthermore, to Saturn the higher and
mightier planet was due not that part which was greater, although
Saturn had the greater private consonance 4 : 5, but that one which is
prior and more beautiful, i.e., more harmonic. For in Axiom X the
consideration of priority and harmonic perfection comes first, and the
consideration of quantity comes last, because there is no beauty in
quantity of itself. Thus the movements of Saturn become 64 : 81, an
adulterine 14 major third, as we have called them in Book III, Chapter 12,
but those of Jupiter, 6,561 : 8,000.
I do not know whether it should be numbered among the causes of the
addition of a comma to Saturn that the extreme intervals of Saturn can
constitute the ratio 8 : 9, the major whole tone, or whether that resulted
without further ado from the preceding causes of the movements.
Accordingly, you here have, in place of a corollary, the reason why, above
in Chapter 4, the intervals of Saturn were found to embrace
approximately a major whole tone.
XXXIX. PROPOSITION. Saturn could not harmonize with its exact
perihelial movement in the universal consonances of the planets of the
major scale, nor Jupiter with its exact aphelial movement.
For since the aphelial movement of Saturn had to harmonize exactly with
the aphelial movements of the Earth and Venus (by Proposition XXXI),
that movement of Saturn which is 4 : 5 or one major third faster than its
aphelial will also harmonize with them. For the aphelial movements of
the Earth and Venus make a major sixth, which, by the demonstrations
of Book iii, is divisible into a perfect fourth and a major third, therefore
14 See footnote to Intervals Compared with Harmonic Ratios, 1026.


89

the movement of Saturn, which is still faster than this movement already
harmonized but none the less below the magnitude of a concordant
interval, will not exactly harmonize. But such a movement is Saturn's
perihelial movement itself, because it differs from its aphelial movement
by more than the interval 4 : 5, viz., one comma or 80 : 81 more (which is
less than the least concord), by Proposition XXXVIII. Accordingly the
perihelial movement of Saturn does not exactly harmonize. But neither
does the aphelial movement of Jupiter do so precisely. For while it does
not harmonize precisely with the perihelial movement of Saturn, it
harmonizes at a distance of a perfect octave (by Proposition VIII),
wherefore, according to what has been said in Book III, it cannot
precisely harmonize.
XL. PROPOSITION. It was necessary to add the lemma of Plato to 1 : 8,
or the triple octave, the joint consonance of the diverging movements of
Jupiter and Mars established by the prior reasons.
For because, by Proposition XXXI, there had to be 1 : 32, i.e., 12 : 384,
between the aphelial movements of Saturn and the Earth, but there had
to be 3 : 2, i.e., 384 : 256, from the aphelion of the Earth to the
perihelion of Mars [313] (by Proposition XV), and from the aphelion of
Saturn to its perihelion, 4 : 5 or 12 : 15 with its increment (by Proposition
XXXVIII); finally, from the perihelion of Saturn to the aphelion of
Jupiter 1 : 2 or 15 : 30 (by Proposition VIII); accordingly, there remains
30 : 256 from the aphelion of Jupiter to the perihelion of Mars, by the
subtraction of the increment of Saturn. But 30 : 256 exceeds 32 : 256 by
the interval 30 : 32, i.e., 15 : 16 or 240 : 256, which is a semitone.
Accordingly, if the increment of Saturn, which (by Proposition XXXVIII)
had to be 80 : 81, i.e., 240 : 243, is compounded inversely with 240 :
243, the result is 243 : 256; but that is the lemma of Plato, 15 viz.,
approximately 19 : 20, see Book III. Accordingly, Plato's lemma had to be
compounded with the 1 : 8.
And so the great ratio of Jupiter and Mars, viz., of the diverging
movements, ought to be 243 : 2,048, which is somehow a mean between
243 : 2,187 and 243 : 1,944, i.e., between 1 : 9 and 1 : 8, whereof
proportionality required the first, above; and a nearer harmonic concord,
the second.
15 Timaeus, 36.


90

XLI. PROPOSITION. The private ratio of the movements of Mars has
necessarily been made the square of the harmonic ratio 5 : 6, viz., 25 :
36.
For, because the ratio of the diverging movements of Jupiter and Mars
had to be 243 : 2,048, i.e., 729 : 6,144, by the preceding proposition, but
that of the converging movements 5 : 24,i.e., 1,280 : 6,144 (by
Proposition XIII), therefore the compound of the private ratios of both
was necessarily 729 : 1,280 or 72,900 : 128,000. But the private ratio of
Jupiter alone had to be 6,561 : 8,000, i.e., 104,976 : 128,000 (by
Proposition XXVIII). Therefore, if the compound ratio of both is divided
by this, the private ratio of Mars will be left as 72,900 : 104,976, i.e., 25 :
36, the square root of which is 5 : 6.
In another fashion, as follows: There is 1 : 32 or 120 : 3,840 from the
aphelial movement of Saturn to the aphelial movement of the Earth, but
from that same movement to the perihelial of Jupiter there is 1 : 3 or 120
: 360, with its increment. But from this to the aphelial movement of
Mars is 5 : 24 or 360 : 1,728. Accordingly, from the aphelial movement of
Mars to the aphelial movement of the Earth, there remains 1,728 : 3,840
minus the increment of the ratio of the diverging movements of Saturn
and Jupiter. But from the same aphelial movement of the Earth to the
perihelial of Mars there is 3 : 2, i.e., 3,840 : 2,500. Therefore between the
aphelial and perihelial movements of Mars there remains the ratio 1,728
: 2,560, i.e., 27 : 40 or 81 : 120, minus the said increment. But 81 : 120 is
a comma less than 80 : 120 or 2 : 3. Therefore, if a comma is taken away
from 2 : 3, and the said increment (which by Proposition XXXVII is
equal to the private ratio of Venus) is taken away too, the private ratio of
Mars is left. But the private ratio of Venus is the diesis diminished by a
comma, by Proposition XXVI. But the comma and the diesis diminished
by a comma make a full diesis or 24 : 25. Therefore if you divide 2 :
3, i.e., 24 : 36 by the diesis 24 : 25, Mars’ private ratio of 25 : 36 is left, as
before, the square root of which, or 5 : 6, goes to the intervals, by
Chapter 3.
Behold again the reason why—above, in Chapter 4—the extreme
intervals of Mars have been found to embrace the harmonic ratio 5 : 6.
XLII. PROPOSITION. The great ratio of Mars and the Earth, or the
common ratio of the diverging movements, has been necessarily made to


91

be 54 : 125, smaller than the consonance 5 : 12 established by the prior
reasons.
For the private ratio of Mars had to be a perfect fifth, from which a diesis
has been taken away, by the preceding proposition. But the common or
minor ratio of the converging movements of Mars and the Earth had to
be a perfect fifth or 2 : 3, by Proposition XV. Finally, the private ratio of
the Earth is the diesis squared, from which a comma is taken away, by
Propositions XXVI and XXVIII. But out of these elements is
compounded the major ratio or that of the diverging movements of Mars
and the Earth—and it is two perfect fifths (or 4 : 9, i.e., 108 : 243) plus a
diesis diminished by a comma, i.e., plus 243 : 250; namely, it is 108 : 250
or 54 : 125, i.e., 608 : 1,500. But this is smaller than 625 : 1,500, i.e., than
5 : 12, in the ratio 602 : 625, which is approximately 36 : 37, smaller than
625 : 1,500, i.e., than 5 : 12, in the ratio 602 : 625, which is
approximately 36 : 37, smaller than the least concord.
XLIII. PROPOSITION. The aphelial movement of Mars could not
harmonize in some universal consonance; nevertheless it was necessary
for it to be in concord to some extent in the scale of the minor mode.
For, because the perihelial movement of Jupiter has the pitch d of acute
tuning in the minor mode, and the consonance 5 : 24 ought to have
existed between that and the aphelial movement of Mars, therefore, the
aphelial movement of Mars occupies the adulterine pitch of the same
acute tuning. I say adulterine for, although in Book III, Chapter 12, the
adulterine consonances were reviewed and deduced from the
composition of systems, certain ones which exist in the simple natural
system were omitted. [314] And so, after the line which ends 81 : 120, the
reader may add: if you divide into it 4 : 5 or 32 : 40, there remains 27 :
32, the subminor sixth, 16 which exists
between d and f or c and e 17 or a and c of even the simple octave. And in
the ensuing table, the following should be in the first line; for 5 : 6 there
is 27 : 32, which is deficient.
From that it is clear that in the natural system the true note [clavem] f,
as regulated by my principles, constitutes a deficient or adulterine minor
16 Here "sixth" (sexta) should probably be "third" (tertia). E. C., Jr.

17 C and e do not produce a subminor third in the "natural system." E. C., Jr.


92

sixth with the note d. Accordingly since between the perihelial
movement of Jupiter set up in the true note d and the aphelial movement
of Mars there is a perfect minor sixth over and above the double octave,
but not the diminished (by Proposition XIII), it follows that with its
aphelial movement Mars designates the pitch which is one comma
higher than the true note f; and so it will concord not absolutely but
merely to a certain extent in this scale. But it does not enter into either
the pure or the adulterine universal harmony. For the perihelial
movement of Venus occupies the pitch of e in this tuning [tensionem].
But there is dissonance between e and f, on account of their nearness.
Therefore, Mars is in discord with the perihelial movement of one of the
planets, viz., Venus. But too it is in discord with the other movements of
Venus; they are diminished by a comma less than a diesis: wherefore,
since there is a semitone and a comma between the perihelial movement
of Venus and the aphelial movement of Mercury, accordingly, between
the aphelion of Venus and the aphelion of Mars there will be a semitone
and a diesis (neglecting the octaves), i.e., a minor whole tone, which is
still a dissonant interval. Now the aphelial movement of Mars concords
to that extent in the scale of the minor mode, but not in that of the
major. For since the aphelial movement of Venus concords with the e of
the major mode, while the aphelial movement of Mars (neglecting the
octaves) has been made a minor whole tone higher than e, then
necessarily the aphelial movement of Mars in this tuning would fall
midway between f and f sharp and would make with g (which in this
tuning would be occupied by the aphelial movement of the Earth) the
plainly discordant interval 25 : 27, viz., a major whole tone diminished
by a diesis.
In the same way, it will be proved that the aphelial movement of Mars is
also in discord with the movements of the Earth. For because it makes a
semitone and comma with the perihelial movement of Venus, i.e., 14 : 15
(by what has been said), but the perihelial movements of the Earth and
Venus make a minor sixth 5 : 8 or 15 : 24 (by Proposition XXVII).
Accordingly, the aphelial movement of Mars together with the perihelial
movement of the Earth (the octaves added to it) will make 14 : 24 or 7 :
12, a discordant interval and one not harmonic, like 7 : 6. For any
interval between 5 : 6 and 8 : 9 is dissonant and discordant, as 6 : 7 in
this case. But no other movement of the Earth can harmonize with the


93

aphelial movement of Mars. For it was said above that it makes the
discordant interval 25 : 27 with the Earth (neglecting the octaves); but all
from 6 : 7 or 24 : 28 to 25 : 27 are smaller than the least harmonic
interval.
XLIV. COROLLARY. Accordingly it is clear from the above Proposition
XLIII concerning Jupiter and Mars, and from Proposition XXXIX
concerning Saturn and Jupiter, and from Proposition XXXVI concerning
Jupiter and the Earth, and from Proposition XXXII concerning Saturn,
why—in Chapter 5, above—it was found that all the extreme movements
of the planets had not been adjusted perfectly to one natural system or
musical scale, and that all those which had been adjusted to a system of
the same tuning did not distinguish the pitches [loca] of that system in a
natural way or effect a purely natural succession of concordant intervals.
For the reasons are prior whereby the single planets came into
possession of their single consonances; those whereby all the planets, of
the universal consonances; and finally, those whereby the universal
consonances of the two modes, the major and the minor: when all those
have been posited, an omniform adjustment to one natural system is
prevented. But if those causes had not necessarily come first, there is no
doubt that either one system and one tuning of it would have embraced
the extreme movements of all the planets; or, if there was need of two
systems for the two modes of song, the major and minor, the very order
of the natural scale would have been expressed not merely in one mode,
the major, but also in the remaining minor mode. Accordingly, here in
Chapter 5, you have the promised causes of the discords through least
intervals and intervals smaller than all concords.
XLV. PROPOSITION. It was necessary for an interval equal to the
interval of Venus to be added to the common major consonance of Venus
and Mercury, the double octave, and also the private consonance of
Mercury, which were established above in Propositions XII and XIII by
the prior reasons, [315] in order that the private ratio of Mercury should
be a perfect 5 : 12 and that thus Mercury should with both its movements
harmonize with the single perihelial movement of Venus.
For, because the aphelial movement of Saturn, the highest and outmost
planet, circumscribed around its regular solid, had to harmonize with the
aphelial movement of the Earth, the highest movement of the Earth,


94

which divides the classes of figures; it follows by the laws of opposites
that the perihelial movement of Mercury as the innermost planet,
inscribed in its figure, the lowest and nearest to the sun, should
harmonize with the perihelial movement of the Earth, with the lowest
movement of the Earth, the common boundary: the former in order to
designate the major mode of consonances, the latter the minor mode, by
Propositions XXXIII and XXXIV. But the perihelial movement of Venus
had to harmonize with the perihelial movement of the Earth in the
consonance 5 : 3, by Proposition XXVII; therefore too the perihelial
movement of Mercury had to be tempered with the perihelial of Venus in
one scale. But by Proposition XII the consonance of the diverging
movements of Venus and Mercury was determined by the prior reasons
to be 1 : 4; therefore, now by these posterior reasons it was to be adjusted
by the accession of the total interval of Venus. Accordingly, not from
further on, from the aphelion, but from the perihelion of Venus to the
perihelion of Mercury there is a perfect double octave. But the
consonance 3 : 5 of the converging movements is perfect, by Proposition
XV. Accordingly if 1 : 4 is divided by 3 : 5, there remains to Mercury
singly the private ratio 5 : 12, perfect too, but not further (by Proposition
xvi, through the prior reasons) diminished by the private ratio of Venus.
Another reason. Just as only Saturn and Jupiter are touched nowhere on
the outside by the dodecahedron and icosahedron wedded together, so
only Mercury is untouched on the inside by these same solids, since they
touch Mars on the inside, the Earth on both sides, and Venus on the
outside. Accordingly, just as something equal to the private ratio of
Venus has been added distributively to the private ratios of movements
of Saturn and Jupiter, which are supported by the cube and tetrahedron;
so now something as great was due to accede to the private ratio of
solitary Mercury, which is comprehended by the associated figures of the
cube and tetrahedron; because, as the octahedron, a single figure among
the secondary figures, does the job of two among the primary, the cube
and tetrahedron (concerning which see Chapter 1), so too among the
lower planets there is one Mercury in place of two of the upper
planets, viz., Saturn and Jupiter.
Thirdly, just as the aphelial movement of the highest planet Saturn had
to harmonize, in some number of octaves, i.e., in the continued double
ratio, 1 : 32, with the aphelial movement of the higher and nearer of the


95

two planets which shift the mode of consonance (by Proposition XXXI);
so, vice versa, the perihelial movement of the lowest planet Mercury,
again through some number of octaves, i.e., in the continued double
ratio, 1 : 4, had to harmonize with the perihelial movement of the lower
and similarly nearer of the two planets which shift the mode of
consonance.
Fourthly, of the three upper planets, Saturn, Jupiter, and Mars, the
single but extreme movements concord with the universal consonances;
accordingly both extreme movements of the single lower planet, viz.,
Mercury, had to concord with the same; for the middle planets, the Earth
and Venus, had to shift the mode of consonances, by Propositions
XXXIII and XXXIV.
Finally, in the three pairs of the upper planets perfect consonances have
been found between the converging movements, but adjusted
[fermentatae] consonances between the diverging movements and
private ratios of the single planets; accordingly, in the two pairs of the
lower planets, conversely, perfect consonances had to be found not
between the converging movements chiefly, nor between the diverging,
but between the movements of the same field. And because two perfect
consonances were due to the Earth and Venus, therefore two perfect
consonances were due to Venus and Mercury also. And the Earth and
Venus had to receive as perfect a consonance between their aphelial
movements as between their perihelial, because they had to shift the
mode of their consonance; but Venus and Mercury, as not shifting the
mode of their consonance, did not also require perfect consonances
between both pairs, the aphelial movements and the perihelial; but there
came in place of the perfect consonance of the aphelial movements, as
being already adjusted the perfect consonance of the converging
movements, so that just as Venus, the higher of the lower planets, has
the least private ratio of all the private ratios of movements (by
Proposition XXVI), and Mercury, the lower of the lower, has received the
greatest ratio of all the private ratios of movements (by Proposition
XXX), so too the private ratio of Venus should be the most imperfect of
all the private ratios or the farthest removed from consonances, while
the private ratio of Mercury should be most perfect of all the private
ratios, i.e., an absolute consonance without adjustment, and that finally
the relations should be everywhere opposite.


96

For He Who is before the ages and on into the ages thus adorned the
great things of His wisdom: nothing excessive, nothing defective, no
room for any censure. How lovely are his works! All things, in twos,
one [316] against one, none lacking its opposite. He has strengthened the
goods—adornment and propriety—of each and every one and established
them in the best reasons, and who will be satiated seeing their glory?
XLVI. Axiom. If the interspacing of the solid figures between the
planetary spheres is free and unhindered by the necessities of antecedent
causes, then it ought to follow to perfection the proportionality of
geometrical inscriptions and circumscriptions, and thereby the
conditions of the ratio of the inscribed to the circumscribed spheres.
For nothing is more reasonable than that physical inscription should
exactly represent the geometrical, as the work, its pattern.
XLVII. PROPOSITION. If the inscription of the regular solids among the
planets was free, the tetrahedron was due to touch with its angles
precisely the perihelial sphere of Jupiter above it, and with centres of its
planes precisely the aphelial sphere of Mars below it. But the cube and
the octahedron, each placing its angles in the perihelial sphere of the
planet above, were due to penetrate the sphere of the inside planet with
the centres of their planes, in such fashion that those centres should turn
within the aphelial and perihelial spheres: on the other hand, the
dodecahedron and icosahedron, grazing with their angles the perihelial
spheres of their planets on the outside, were due not quite to touch with
the centres of their planes the aphelial spheres of their inner planets.
Finally, the dodecahedral echinus, placing its angles in the perihelial
sphere of Mars, was due to come very close to the aphelial sphere of
Venus with the midpoints of its converted sides which interdistinguish
two solid rays.
For the tetrahedron is the middle one of the primary figures, both in
genesis and in situation in the world; accordingly, it was due to remove
equally both regions, that of Jupiter and that of Mars. And because the
cube was above it and outside it, and the dodecahedron was below it and
within it, therefore it was natural that their inscription should strive for
the contrariety wherein the tetrahedron held a mean, and that the one of
them should make an excessive inscription, and the other a
defective, viz., the one should somewhat penetrate the inner sphere, the


97

other not touch it. And because the octahedron is cognate to the cube
and has an equal ratio of spheres, but the icosahedron to the
dodecahedron, accordingly, whatever the cube has of perfection of
inscription, the same was due to the octahedron also, and whatever the
dodecahedron, the same to the icosahedron too. And the situation of the
octahedron's similar to the situation of the cube, but that of the
icosahedron to the situation of the dodecahedron, because as the cube
occupies the one limit to the outside, so the octahedron occupies the
remaining limit to the inside of the world, but the dodecahedron and
icosahedron are midway: accordingly even a similar inscription was
proper, in the case of the dodecahedron, one penetrating the sphere of
the inner planet, in that of the icosahedron, one falling short of it.
But the echinus, which represents the icosahedron with the apexes of its
angles and the dodecahedron with the bases, was due to fill, embrace, or
dispose both regions, that between Mars and the Earth with the
dodecahedron as well as that between the Earth and Venus with the
icosahedron. But the preceding axiom makes clear which of the
opposites was due to which association. For the tetrahedron, which has a
rational inscribed sphere, has been allotted the middle position among
the primary figures and is surrounded on both sides by figures of
incommensurable spheres, whereof the outer is the cube, the inner the
dodecahedron, by Chapter 1 of this book. But this geometrical
quality, viz., the rationality of the inscribed sphere, represents in nature
the perfect inscription of the planetary sphere. Accordingly, the cube and
its allied figure have their inscribed spheres rational only in square, i.e.,
in power alone; accordingly, they ought to represent a semiperfect
inscription, where, even if not the extremity of the planetary sphere, yet
at least something on the inside and rightfully a mean between the
aphelial and perihelial spheres—if that is possible through other reasons
is touched by the centres of the planes of the figures. On the other hand,
the dodecahedron and its allied figure have their inscribed spheres
clearly irrational both in the length of the radius and in the square;
accordingly, they ought to represent a clearly imperfect inscription and
one touching absolutely nothing of the planetary sphere, i.e., falling
short and not reaching as far as the aphelial sphere of the planet with the
centres of its planes.


98

Although the echinus is cognate to the dodecahedron and its allied
figure, nevertheless it has a property similar to the tetrahedron. For the
radius of the sphere inscribed in its inverted sides is indeed
incommensurable with the radius of the circumscribed sphere, but it is,
however, commensurable with the length of the distance between two
neighbouring angles. And so the perfection of the commensurability of
rays is approximately as great as in the tetrahedron; but elsewhere the
imperfection is as great as in the [317] dodecahedron and its allied
figure. Accordingly it is reasonable too that the physical inscription
belonging to it should be neither absolutely tetrahedral nor absolutely
dodecahedral but of an intermediate kind; in order that (because the
tetrahedron was due to touch the extremity of the sphere with its planes,
and the dodecahedron, to fall short of it by a definite interval) this
wedge-shaped figure with the inverted sides should stand between the
icosahedral space and the extremity of the inscribed sphere and should
nearly touch this extremity—if nevertheless this figure was to be
admitted into association with the remaining five, and if its laws could be
allowed, with the laws of the others remaining. Nay, why do I say "could
be allowed"? For they could not do without them. For if an inscription,
which was loose and did not come into contact fitted the dodecahedron,
what else could confine that indefinite looseness within the limits of a
fixed magnitude, except this subsidiary figure cognate to the
dodecahedron and icosahedron, and which comes almost into contact
with its inscribed sphere and does not fall short (if indeed it does fall
short) any more than the tetrahedron exceeds and penetrates —with
which magnitude we shall deal in the following.
This reason for the association of the echinus with the two cognate
figures (viz., in order that the ratio of the spheres of Mars and Venus,
which they had left indefinite, should be made determinate) is rendered
very probable by the fact that 1,000, the semidiameter of the sphere of
the Earth, is found to be practically a mean proportional between the
perihelial sphere of Mars and the aphelial sphere of Venus; as if the
interval, which the echinus assigns to the cognate figures, has been
divided between them as proportionally as possible. XLVIII.
PROPOSITION. The inscription of the regular solid figures between the
planetary spheres was not the work of pure freedom; for with respect


99

to very small magnitudes it was hindered by the consonances
established between the extreme movements.
For, by Axioms I and II, the ratio of the spheres of each figure was not
due to be expressed immediately by itself, but by means of it the
consonances most akin to the ratios of the spheres were first to be sought
and adjusted to the extreme movements.
Then, in order that, by Axioms XVIII and XX, the universal consonances
of the two modes could exist, it was necessary for the greater
consonances of the single pairs to be readjusted somewhat, by means of
the posterior reasons. Accordingly, in order that those things might
stand, and be maintained by their own reasons, intervals were required
which are somewhat discordant with those which arise from the perfect
inscription of figures between the spheres, by the laws of movements
unfolded in Chapter 3. In order that it be proved and made manifest how
much is taken away from the single planets by the consonances
established by their proper reasons; come, let us build up, out of them,
the intervals of the planets from the sun, by a new form of calculation
not previously tried by anyone.
Now there will be three heads to this inquiry: First, from the two extreme
movements of each planet the similar extreme intervals between it and
the sun will be investigated, and by means of them the radius of the
sphere in those dimensions, of the extreme intervals, which are proper to
each planet. Secondly, by means of the same extreme movements, in the
same dimensions for all, the mean movements and their ratio will be
investigated. Thirdly, by means of the ratio of the mean movements
already disclosed, the ratio of the spheres or mean intervals and also one
ratio of the extreme intervals, will be investigated; and the ratio of the
mean intervals will be compared with the ratios of the figures.
As regards the first: we must repeat, from Chapter 3, Article VI, that the
ratio of the extreme movements is the inverse square of the ratio of the
corresponding intervals from the sun. Accordingly, since the ratio of the
squares is the square of the ratio of its sides, therefore, the numbers,
whereby the extreme movements of the single planets are expressed, will
be considered as squares and the extraction of their roots will give the
extreme intervals, whereof it is easy to take the arithmetic mean as the


100

semidiameter of the sphere and the eccentricity. Accordingly the
consonances so far established have prescribed:

For the second of the things proposed, we again have need of Chapter 3,
Article XII, where it was shown that the number which expresses the
movement which is as a mean in the ratio of the extremes is less than
their arithmetic mean, also less than the geometric mean by half the
difference between the geometric and arithmetic means. And because we
are investigating all the mean movements in the same dimensions,
therefore let all the ratios hitherto established between different twos
and also all the private ratios of the single planets be set out in the
measure of the least common divisible. Then let the means be sought:
the arithmetic, by taking half the difference between the extreme
movements of each planet, the geometric, by the multiplication of one
extreme into the other and extracting the square root of the product;
then by subtracting half the difference of the means from the geometric
mean, let the number of the mean movement be constituted in the
private dimensions of each planet, which can easily, by the rule of ratios,
be converted into the common dimensions.
[319] Therefore, from the prescribed consonances, the ratio of the mean
diurnal movements has been found, viz., the ratio between the numbers
of the degrees and minutes of each planet. It is easy to explore how
closely that approaches to astronomy.


101

The third head of things proposed requires Chapter 3, Article VIII. For
when the ratio of the mean diurnal movements of the single planets has
been found, it is possible to find the ratio of the spheres too. For the ratio
of the mean movements is the 3/2th power of the inverse ratio of the
spheres. But, too, the ratio of the cube numbers is the 3/2th power of the
ratio of the squares of those same square roots, given in the table of
Clavius, which he subjoined to his Practical Geometry. Wherefore, if the
numbers of our mean movements (curtailed, if need be, of an equal
number of ciphers) are sought among the cube numbers of that table,
they will indicate on the left, under the heading of the squares, the
numbers of the ratio of the spheres; then the eccentricities ascribed
above to the single planets in the private ratio of the semidiameters of
each may easily be converted by the rule of ratios into dimensions
common to all, so that, by their addition to the semidiameters of the
spheres and subtraction from them, the extreme intervals of the single
planets from the sun may be established. Now we shall give to the
semidiameter of the terrestrial sphere the round number 100,000, as is
the practice in astronomy, and with the following design: because this


102

number or its square or its cube is always made up of mere ciphers; and
so too we shall raise the mean movement of the Earth to the number
10,000,000,000 and by the rule of ratios make the number of the mean
movement of any planet be to the number of the mean movement of the
Earth, as 10,000,000,000 is to the new measurement. And so the
business can be carried on with only five cube roots, by comparing those
single cube roots with the one number of the Earth.

Accordingly, it is apparent in the last column what the numbers turn out
to be whereby the converging intervals of two planets are expressed. All
of them approach very near to those intervals, which I found from
Brahe's observations. In Mercury alone is there some small difference.
For astronomy is seen to give the following intervals to it: 470, 388, 306,
all shorter. It seems that the reason for the dissonance may be referred
either to the fewness of the observations or to the magnitude of the
eccentricity. (See Chapter 3). But I hurry on to the end of the calculation.
For now it is easy to compare the ratio of the spheres of the figures with
the ratio of the converging intervals.
[320] For if the semidiameter of the sphere circumscribed around the
figure


103

That is to say, the planes of the cube extend down slightly below the
middle circle of Jupiter; the octahedral planes, not quite to the middle
circle of Mercury; the tetrahedral, slightly below the highest circle of
Mars; the sides of the echinus, not quite to the highest circle of Venus;
but the planes of the dodecahedron fall far short of the aphelial circle of
the Earth; the planes of the icosahedron also fall short of the aphelial
circle of Venus, and approximately proportionally; finally, the square in
the octahedron is quite inept, and not unjustly, for what are plane figures
doing among solids? Accordingly, you see that if the planetary intervals
are deduced from the harmonic ratios of movements hitherto
demonstrated, it is necessary that they turn out as great as these allow,
but not as great as the laws of free inscription prescribed in Proposition
XLV would require: because this κόσμος γεωμέτρικος [geometrical
adornment] of perfect inscription was not fully in accordance with that
other κόϐμον ἁρμόνικον ἐνδεχόμενον [possible harmonic adornment]—
to use the words of Galen, taken from the epigraph to this Book v. So
much was to be demonstrated by the calculation of numbers, for the
elucidation of the prescribed proposition.
I do not hide that if I increase the consonance of the diverging
movements of Venus and Mercury by the private ratio of the movements
of Venus, and, as a consequence, diminish the private ratio of Mercury
by the same, then by this process I produce the following intervals
between Mercury and the sun: 469, 388, 307, which are very precisely
represented by astronomy. But, in the first place, I cannot defend that
diminishing by harmonic reasons. For the aphelial movement of
Mercury will not square with that musical scale, nor in the planets which
are opposite in the world is the planetary principle [ratio] of opposition


104

of all conditions kept. Finally, the mean diurnal movement of Mercury
becomes too great, and thereby the periodic time, which is the most
certain fact in all astronomy, is shortened too much. And so I stay within
the harmonic polity here employed and confirmed throughout the whole
of Chapter 9. But none the less with this example I call you all forth, as
many of you as have happened to read this book and are steeped in the
mathematical disciplines and the knowledge of highest philosophy: work
hard and either pluck up one of the consonances applied everywhere,
interchange it with some other, and test whether or not you will come so
near to the astronomy posited in Chapter 4, or else try by reasons
whether or not you can build with the celestial movements something
better and more expedient and destroy in part or in whole the layout
applied by me. But let whatever pertains to the glory of Our Lord and
Founder be equally permissible to you by way of this book, and up to this
very hour I myself have taken the liberty of everywhere changing those
things which I was able to discover on earlier days and which were the
conceptions of a sluggish care or hurrying ardour.
[321] XLIX. ENVOI. It was good that in the genesis of the intervals the
solid figures should yield to the harmonic ratios, and the major
consonances of two planets to the universal consonances of all, in so far
as this was necessary.
With good fortune we have arrived at 49, the square of 7; so that this
may come as a kind of Sabbath, since the six solid eights of discourse
concerning the construction of the heavens has gone before. Moreover, I
have rightly made an envoi which could be placed first among the
axioms: because God also, enjoying the works of His creation, "saw all
things which He had made, and behold! they were very good."
There are two branches to the envoi: First, there is a demonstration
concerning consonances in general, as follows: For where there is choice
among different things which are not of equal weight, there the more
excellent are to be put first and the more vile are to be detracted from, in
so far as that is necessary, as the very word ὁ κόϐμος, which
signifies adornment, seems to argue. But inasmuch as life is more
excellent than the body, the form than the material, by so much does
harmonic adornment excel the geometrical.


105

For as life perfects the bodies of animate things, because they have been
born for the exercise of life—as follows from the archetype of the world,
which is the divine essence—so movement measures the regions
assigned to the planets, each that of its own planet: because that region
was assigned to the planet in order that it should move. But the five
regular solids, by their very name, pertain to the intervals of the regions
and to the number of them and the bodies; but the consonances to the
movements. Again, as matter is diffuse and indefinite of itself, the form
definite, unified, and determinant of the material, so too there are an
infinite number of geometric ratios, but few consonances. For although
among the geometrical ratios there are definite degrees of
determinations, formation, and restriction, and no more than three can
exist from the ascription of spheres to the regular solids; but
nevertheless an accident common to all the rest follows upon even these
geometrical ratios: an infinite possible section of magnitudes is
presupposed, which those ratios whose terms are mutually
incommensurable somehow involve in actuality too. But the harmonic
ratios are all rational, the terms of all are commensurable and are taken
from a definite and finite species of plane figures. But infinity of section
represents the material, while commensurability or rationality of terms
represents the form. Accordingly, as material desires the form, as the
rough-hewn stone, of a just magnitude indeed, the form of a human
body, so the geometric ratios of figures desire the consonances—not in
order to fashion and form those consonances, but because this material
squares better with this form, this quantity of stone with this statue, even
this ratio of regular solids with this consonance—therefore in order so
that they are fashioned and formed more fully, the material by its form,
the stone by the chisel into the form of an animate being; but the ratio of
the spheres of the figure by its own, i.e., the near and fitting, consonance.
The things which have been said up to now will become clearer from the
history of my discoveries. Since I had fallen into this speculation twentyfour years ago, I first inquired whether the single planetary spheres are
equal distances apart from one another (for the spheres are apart in
Copernicus, and do not touch one another), that is to say, I recognized
nothing more beautiful than the ratio of equality. But this ratio is
without head or tail: for this material equality furnished no definite
number of mobile bodies, no definite magnitude for the intervals.


106

Accordingly, I meditated upon the similarity of the intervals to the
spheres, i.e., upon the proportionality. But the same complaint followed.
For although to be sure, intervals which were altogether unequal were
produced between the spheres, yet they were not unequally equal, as
Copernicus wishes, and neither the magnitude of the ratio nor the
number of the spheres was given. I passed on to the regular plane
figures: [322] intervals were formed from them by the ascription of
circles. I came to the five regular solids: here both the number of the
bodies and approximately the true magnitude of the intervals was
disclosed, in such fashion that I summoned to the perfection of
astronomy the discrepancies remaining over and above. Astronomy was
perfect these twenty years; and behold! there was still a discrepancy
between the intervals and the regular solids, and the reasons for the
distribution of unequal eccentricities among the planets were not
disclosed. That is to say, in this house the world, I was asking not only
why stones of a more elegant form but also what form would fit the
stones, in my ignorance that the Sculptor had fashioned them in the very
articulate image of an animated body. So, gradually, especially during
these last three years, I came to the consonances and abandoned the
regular solids in respect to minima, both because the consonances stood
on the side of the form which the finishing touch would give, and the
regular solids, on that of the material—which in the world is the number
of bodies and the rough-hewn amplitude of the intervals—and also
because the consonances gave the eccentricities, which the regular solids
did not even promise—that is to say, the consonances made the nose,
eyes, and remaining limbs a part of the statue, for which the regular
solids had prescribed merely the outward magnitude of the rough-hewn
mass.
Wherefore, just as neither the bodies of animate beings are made nor
blocks of stone are usually made after the pure rule of some geometrical
figure, but something is taken away from the outward spherical figure,
however elegant it maybe (although the just magnitude of the bulk
remains), so that the body may be able to get the organs necessary for
life, and the stone the image of the animate being; so too as the ratio
which the regular solids had been going to prescribe for the planetary
spheres is inferior and looks only towards the body and material, it has
to yield to the consonances, in so far as that was necessary in order for


107

the consonances to be able to stand closely by and adorn the movement
of the globes.
The other branch of the envoi, which concerns universal consonances,
has a proof closely related to the first. (As a matter of fact, it was in part
assumed above, in XVIII, among the Axioms.) For the finishing touch of
perfection, as it were, is due rather to that which perfects the world
more; and conversely that thing which occupies a second position is to
be detracted from, if either is to be detracted from. But the universal
harmony of all perfects the world more than the single twin consonances
of different neighbouring twos. For harmony is a certain ratio of unity;
accordingly the planets are more united, if they all are in concord
together in one harmony, than if each two concord separately in two
consonances. Wherefore, in the conflict of both, either one of the two
single consonances of two planets was due to yield, so that the universal
harmonies of all could stand. But the greater consonances, those of the
diverging movements, were due to yield rather than the lesser, those of
the converging movements. For if the divergent movements diverge,
then they look not towards the planets of the given pair but towards
other neighbouring planets, and if the converging movements converge,
then the movements of one planet are converging toward the movement
of the other, conversely: for example, in the pair Jupiter and Mars the
aphelial movement of Jupiter verges toward Saturn, the perihelial of
Mars towards the Earth: but the perihelial movement of Jupiter verges
toward Mars, the aphelial of Mars toward Jupiter. Accordingly the
consonance of the converging movements is more proper to Jupiter and
Mars; the consonance of the diverging movements is somehow more
foreign to Jupiter and Mars. But the ratio of union which brings together
neighbouring planets by twos and twos is less disturbed if the
consonance which is more foreign and more removed from them should
be adjusted than if the private ratio should be, viz., the one which exists
between the more neighbouring movements of neighbouring planets.
None the less this adjustment was not very great. For the proportionality
has been found in which may stand the universal consonances of all the
planets may exist (and these in two distinct modes), and in which (with a
certain latitude of tuning merely equal to a comma) may also be
embraced the single consonances of two neighbouring planets; the
consonances of the converging movements in four pairs, perfect, of the


108

aphelial movements in one pair, of the perihelial movements in two
pairs, likewise perfect; the consonances of the diverging movements in
four pairs, these, however, within the difference of one diesis (the very
small interval by which the human voice [323] in figured song nearly
always errs; the single consonance of Jupiter and Mars, this between the
diesis and the semitone. Accordingly it is apparent that this mutual
yielding is everywhere very good.
Accordingly let this do for our envoi concerning the work of God the
Creator. It now remains that at last, with my eyes and hands removed
from the tablet of demonstrations and lifted up towards the heavens, I
should pray, devout and supplicating, to the Father of lights: O Thou
Who dost by the light of nature promote in us the desire for the light of
grace, that by its means Thou mayest transport us into the light of
glory, I give thanks to Thee, O Lord Creator, Who hast delighted me
with Thy makings and in the works of Thy hands have I exulted.
Behold! now, I have completed the work of my profession, having
employed as much power of mind as Thou didst give to me; to the men
who are going to read those demonstrations I have made manifest the
glory of Thy works, as much of its infinity as the narrows of my
intellect could apprehend. My mind has been given over to
philosophizing most correctly: if there is anything unworthy of Thy
designs brought forth by me—a worm born and nourished in a
wallowing place of sins—breathe into me also that which Thou dost
wish men to know, that I may make the correction: If I have been
allured into rashness by the wonderful beauty of Thy works, or if I have
loved my own glory among men, while I am advancing in the work
destined for Thy glory, be gentle and merciful and pardon me; and
finally deign graciously to effect that these demonstrations give way to
Thy glory and the salvation of souls and nowhere be an obstacle to that.


109

10. EPILOGUE CONCERNING THE SUN, BY WAY
OF CONJECTURE
See Kepler's commentary on this epilogue in the Epitome, page 850-51.

From the celestial music to the hearer, from the Muses to Apollo the
leader of the Dance, from the six planets revolving and making
consonances to the Sun at the centre of all the circuits, immovable in
place but rotating into itself. For although the harmony is most absolute
between the extreme planetary movements, not with respect to the true
speeds through the ether but with respect to the angles which are formed
by joining with the centre of the sun the termini of the diurnal arcs of the
planetary orbits; while the harmony does not adorn the termini, i.e., the
single movements, in so far as they are considered in themselves but only
in so far as by being taken together and compared with one another, they
become the object of some mind; and although no object is ordained in
vain, without the existence of some thing which may be moved by it,
while those angles seem to presuppose some action similar to our
eyesight or at least to that sense-perception whereby, in Book IV, the
sublunary nature perceived the angles of rays formed by the planets on
the Earth: still it is not easy for dwellers on the Earth to conjecture what
sort of sight is present in the sun, what eyes there are, or what other
instinct there is for perceiving those angles even without eyes and for
evaluating the harmonies of the movements entering into the
antechamber of the mind by whatever doorway, and finally what mind
there is in the sun. None the less, however those things may be, this
composition of the six primary spheres around the sun, cherishing it
with their perpetual revolutions and as it were adoring it (just as,
separately, four moons accompany the globe of Jupiter, two Saturn, but a
single moon by its circuit encompasses, cherishes, fosters the Earth and
us its inhabitants, and ministers to us) and this special business of the
harmonies, which is a most clear footprint of the highest providence over
solar affairs, now being added to that consideration, [324] wrings from
me the following confession: not only does light go out from the sun into
the whole world, as from the focus or eye of the world, as life and heat
from the heart, as every movement from the King and mover, but


110

conversely also by royal law these returns, so to speak, of every lovely
harmony are collected in the sun from every province in the world, nay,
the forms of movements by twos flow together and are bound into one
harmony by the work of some mind, and are as it were coined money
from silver and gold bullion; finally, the curia, palace, and praetorium or
throne-room of the whole realm of nature are in the sun, whatsoever
chancellors, palatines, prefects the Creator has given to nature: for them,
whether created immediately from the beginning or to be transported
hither at some time, has He made ready those seats. For even this
terrestrial adornment, with respect to its principal part, for quite a long
while lacked the contemplators and enjoyers, for whom however it had
been appointed; and those seats were empty. Accordingly the reflection
struck my mind, what did the ancient Pythagoreans in Aristotle mean,
who used to call the centre of the world (which they referred to as the
"fire" but understood by that the sun) "the watchtower of Jupiter," Διος
φυλακὴν; what, likewise, was the ancient interpreter pondering in his
mind when he rendered the verse of the Psalm as: "He has placed His
tabernacle in the sun."
But also I have recently fallen upon the hymn of Proclus the Platonic
philosopher (of whom there has been much mention in the preceding
books), which was composed to the Sun and filled full with venerable
mysteries, if you excise that one κλῦθ (hear me) from it; although the
ancient interpreter already cited has explained this to some extent, viz.,
in invoking the sun, he understands Him Who has placed His tabernacle
in the sun. For Proclus lived at a time in which it was a crime, for which
the rulers of the world and the people itself inflicted all punishments, to
profess Jesus of Nazareth, God Our Savior, and to contemn the gods of
the pagan poets (under Constantine, Maxentius, and Julian the
Apostate). Accordingly Proclus, who from his Platonic philosophy
indeed, by the natural light of the mind, had caught a distant glimpse of
the Son of God, that true light which lighteth every man coming into this
world, and who already knew that divinity must never be sought with a
superstitious mob in sensible things, nevertheless preferred to seem to
look for God in the sun rather than in Christ a sensible man, in order
that at the same time he might both deceive the pagans by honoring
verbally the Titan of the poets and devote himself to his philosophy, by
drawing away both the pagans and the Christians from sensible beings,


111

the pagans from the visible sun, the Christians from the Son of Mary,
because, trusting too much to the natural light of reason, he spit out the
mystery of the Incarnation; and finally that at the same time he might
take over from them and adopt into his own philosophy whatever the
Christians had which was most divine and especially consonant with
Platonic philosophy. 18 And so the accusation of the teaching of the
Gospel concerning Christ is laid against this hymn of Proclus, in its own
matters: let that Titan keep as his private possessions χρῦσα ἡνία [golden
reins] and ταμιεῖυν φαοῦς, μεσσατὶην, αἰθερος ἓδρην, κοδμοῡ κραδιαῖον
ἐριφεγγεᾲ κυκλὸν [a treasury of light, a seat at the midpart of the ether, a
radiant circle at the heart of the world], which visible aspect Copernicus
too bestows upon him; let him even keep his παλιννοστοὺς διφρείς
[cyclical chariot-drivings], although according to the ancient
Pythagoreans he does not possess them but in their place τὸ κέντρον,
Διὸς φυλακήν [the centre, the watchtower of Zeus]—which doctrine,
misshapen by the forgetfulness of ages, as by a flood, was not recognized
by their follower Proclus; let him also keep his γενεθλὴν Βλαστησασαν
[offspring born] of himself, and whatever else is of nature; in turn, let the
philosophy of Proclus yield to Christian doctrines, [325] let the sensible
sun yield to the Son of Mary, the Son of God, Whom Proclus addresses
under the name of the Titan, ζωαρκεὸς, ὢ ἂνα, πηγὴς αὐτὸς ἔχων κλήδα
[O lord, who dost hold the key of the life-supporting spring], and that
πᾴντα τεῆς ἔπλήσας ἐλερσινοοῖο προνόιης [thou didst fulfill all things
with thy mind-awakening foresight], and that immense power over the
μοιρὰων [fates], and things which were read of in no philosophy before
the promulgation of the Gospel 19 , the demons dreading him as their
threatening scourge, the demons lying in ambush for souls, ὂφρα
ὐφιτενοῦς λαθοῖντο πατρὸς περιφέγγεος αὐλής [in order that they might
escape the notice of the light-filled hall of the lofty father]; and who
except the Word of the Father is that εἰκὼν παγγεντετᾴο θεοῦ, οὖ
φᾴεντος ἀπ᾽ ἀῤῥητου γενετῆρος παύσατο στοιχεῖων ο̃ρυμᾴγδος ἐπ
ἀλληλοῖσιν ἰὀντων [image of the all-begetting father, upon whose
18 It was the judgment of the ancients concerning his book Metroace that in it he set forth, not without

divine rapture, his universal doctrine concerning God; and by the frequent tears of the author
apparent in it all suspicion was removed from the hearers. None the less this same man wrote against
the Christians eighteen epichiremata, to which John Philoponus opposed himself, reproaching Proclus
with ignorance of Greek thought, which none the less lie had undertaken to defend. That is to say,
Proclus concealed those things which did not make for his own philosophy.
19 Nevertheless in Suidas some similar things are attributed to ancient Orpheus, nearly equal to
Moses, as if his pupil; see too the hymns of Orpheus, on which Proclus wrote commentaries.


112

manifestation from an ineffable mother the sin of the elements changing
into one another ceased], according to the following: The Earth was
unwrought and a chaotic mass, and darkness was upon the face of the
abyss, and God divided the light from the darkness, the waters from the
waters, the sea from the dry land; and: all things were made by the very
Word. Who except Jesus of Nazareth the Son of God, ψυχῶν ἀναγωγεύς
[the shepherd of souls], to whom ἱκεσιὴ πολυδὰκρους [the prayer of a
tearful suppliant] is to be offered, in order that He cleanse us from sins
and wash us of the filth τῆς γενεθλὴς [of generation]—as if Proclus
acknowledged the forms of original sin—and guard us from punishment
and evil, πρηυνὼν θόον ὀμμα δικῆς [by making mild the quick eye of
justice], namely, the wrath of the Father? And the other things we read
of, which are as it were taken from the hymn of Zacharias (or,
accordingly, was that hymn a part of the Metroace?) Αχλυν ἀποσκεδὰσας
ὀλεσὶμβροτον ὶολοχεύτον [dispersing the poisonous, man-destroying
mist], viz., in order that He may give to souls living in darkness and the
shadows of death the φάος ἁγνο̃ν [holy light] and ὂλβο̃ν ἀστυφελικτὸν
ἀπ᾽ ἐυσεβίνἐρατείης [unshaken happiness from lovely piety]; for that is
to serve God in holiness and justice all our days. Accordingly, let us
separate out these and similar things and restore them to the doctrine of
the Catholic Church to which they belong. But let us see what the
principal reason is why there has been mention made of the hymn. For
this same sun which ὕψοθεν ἁρμνίης ῥῦμα πλοῦσιον ἐξοτεύει [sluices the
rich flow of harmony from on high]—so too Orpheus κόσμου τὸν
ἐναρμόνιον δρόμον ἕλκων [making move the harmonious course of the
world]—the same, concerning whose stock Phoebus about to rise κιθαρῇ
ὑπὸ θέσκελα μελπῶν εὐνάξει μεγὰ κῦμα βαρυφλσισβοῖο γενεθλής [sings
marvellous things on his lyre and lulls to sleep the heavy-sounding surge
of generation] and in whose dance Paean is the partner, πλήσας
ἁρμονὶης παναπήμονος εὔρεα κο̃σμν [striking the wide sweep of
innocent harmony]—him, I say, does Proclus at once salute in the first
verse of the hymn as πῦρος νοεροῦ βασιλέα [king of intellectual fire]. By
that commencement, at the same time, he indicates what the
Pythagoreans understood by the word of fire (so that it is surprising that
the pupil should disagree with the masters in the position of the centre)
and at the same time he transfers his whole hymn from the body of the
sun and its quality and light, which are sensibles, to the intelligibles, and
he has assigned to that πῦρ νοερὸς [intellectual fire] of his—perhaps the


113

artisan fire of the Stoics—to that created God of Plato, that chief or selfruling mind, a royal throne in the solar body, confounding into one the
creature and Him through Whom all things have been created. But we
Christians, who have been taught to make better distinctions, know that
this eternal and untreated "Word," Which was "with God" and Which is
contained by no abode, although He is within all things, excluded by
none, although He is outside of all things, took up into unity of person
flesh out of the womb of the most glorious Virgin Mary, and, when the
ministry of His flesh was finished, occupied as His royal abode the
heavens, wherein by a certain excellence over and above the other parts
of the world, viz., through His glory and majesty, His celestial Father too
is recognized to dwell, and has also promised to His faithful, mansions in
that house of His Father: as for the remainder concerning that abode, we
believe it superfluous to inquire into it too curiously or to forbid the
senses or natural reasons to investigate that which the eye has not seen
nor the ear heard and into which the heart of man has not ascended; but
we duly subordinate the created mind—of whatsoever excellence it may
be—to its Creator, and we introduce neither God-intelligences with
Aristotle and the pagan philosophers nor armies of innumerable
planetary spirits with the Magi, nor do we propose that they are either to
be adored or summoned to intercourse with us by theurgic superstitions,
for we have a careful fear of that; but we freely inquire by natural reasons
what sort of thing each mind is, especially if in the heart of the world
[326] there is any mind bound rather closely to the nature of things and
performing the function of the soul of the world—or if also some
intelligent creatures, of a nature different from human perchance do
inhabit or will inhabit the globe thus animated (see my book on the New
Star, Chapter 24, "On the Soul of the World and Some of Its Functions").
But if it is permissible, using the thread of analogy as a guide, to traverse
the labyrinths of the mysteries of nature, not ineptly, I think, will
someone have argued as follows:
The relation of the six spheres to their common centre, thereby the
centre of the whole world, is also the same as that of διανοὶα [discussive
intellection] to νοῦς [intuitive intellection], according as these faculties
are distinguished by Aristotle, Plato, Proclus, and the rest; and the
relation of the single planets' revolutions in place around the sun to the
ἀμετᾴθεδον [unvarying] rotation of the sun in the central space of the


114

whole system (concerning which the sun-spots are evidence; this has
been demonstrated in the Commentaries on the Movement of Mars) is
the same as the relation of τὸ διανοητικὸν to τὸ νοερὸν, that of the
manifold discourses of ratiocination to the most simple intellection of
the mind. For as the sun rotating into itself moves all the planets by
means of the form emitted from itself, so too—as the philosophers
teach—mind, by understanding itself and in itself all things, stirs up
ratiocinations, and by dispersing and unrolling its simplicity into them,
makes everything to be understood. And the movements of the planets
around the sun at their centre and the discourses of ratiocinations are so
interwoven and bound together that, unless the Earth, our domicile,
measured out the annual circle, midway between the other spheres—
changing from place to place, from station to station—never would
human ratiocination have worked its way to the true intervals of the
planets and to the other things dependent from them, never would it
have constituted astronomy. (See the Optical Part of Astronomy,
Chapter 9.)
On the other hand, in a beautiful correspondence, simplicity of
intellection follows upon the stillness of the sun at the centre of the
world, in that hitherto we have always worked under the assumption that
those solar harmonies of movements are defined neither by the diversity
of regions nor by the amplitude of the expanses of the world.
As a matter of fact, if any mind observes from the sun those harmonies,
that mind is without the assistance afforded by the movement and
diverse stations of his abode, by means of which it may string together
ratiocinations and discourse necessary for measuring out the planetary
intervals. Accordingly, it compares the diurnal movements of each
planet, not as they are in their own orbits but as they pass through the
angles at the centre of the sun. And so if it has knowledge of the
magnitude of the spheres, this knowledge must be present in it a priori,
without any toil of ratiocination: but to what extent that is true of human
minds and of sublunary nature has been made clear above, from Plato
and Proclus.
Under these circumstances, it will not have been surprising if anyone
who has been thoroughly warmed by taking a fairly liberal draft from
that bowl of Pythagoras which Proclus gives to drink from in the very


115

first verse of the hymn, and who has been made drowsy by the very sweet
harmony of the dance of the planets begins to dream (by telling a story
he may imitate Plato's Atlantis and, by dreaming, Cicero's Scipio):
throughout the remaining globes, which follow after from place to place,
there have been disseminated discursive or ratiocinative faculties,
whereof that one ought assuredly to be judged the most excellent and
absolute which is in the middle position among those globes, viz., in
man's earth, while there dwells in the sun simple intellect, πῦρ νοερὸν, or
νοῦς, the source, whatsoever it may be, of every harmony.
For if it was Tycho Brahe's opinion concerning that bare wilderness of
globes that it does not exist fruitlessly in the world but is filled with
inhabitants: with how much greater probability shall we make a
conjecture as to God's works and designs even for the other globes, from
that variety which we discern in this globe of the Earth. For He Who
created the species which should inhabit the waters, beneath which
however there is no room for the air [327] which living things draw in;
Who sent birds supported on wings into the wilderness of the air; Who
gave white bears and white wolves to the snowy regions of the North,
and as food for the bears the whale, and for the wolves, birds' eggs; Who
gave lions to the deserts of burning Libya and camels to the wide-spread
plains of Syria, and to the lions an endurance of hunger, and to the
camels an endurance of thirst: did He use up every art in the globe of the
Earth so that He was unable, every goodness so that he did not wish, to
adorn the other globes too with their fitting creatures, as either the long
or short revolutions, or the nearness or removal of the sun, or the variety
of eccentricities or the shine or darkness of the bodies, or the properties
of the figures wherewith any region is supported persuaded?
Behold, as the generations of animals in this terrestrial globe have an
image of the male in the dodecahedron, of the female in the
icosahedron—whereof the dodecahedron rests on the terrestrial sphere
from the outside and the icosahedron from the inside: what will we
suppose the remaining globes to have, from the remaining figures? For
whose good do four moons encircle Jupiter, two Saturn, as does this our
moon this our domicile? But in the same way we shall ratiocinate
concerning the globe of the sun also, and we shall as it were incorporate
conjectures drawn from the harmonies, et cetera—which are weighty of
themselves—with other conjectures which are more on the side of the


116

bodily, more suited for the apprehension of the vulgar. Is that globe
empty and the others full, if everything else is in due correspondence? If
as the Earth breathes forth clouds, so the sun black smoke?
If as the Earth is moistened and grows under showers, so the sun shines
with those combusted spots, while clear flame-lets sparkle in its all fiery
body. For whose use is all this equipment, if the globe is empty? Indeed,
do not the senses themselves cry out that fiery bodies dwell here which
are receptive of simple intellects, and that truly the sun is, if not the king,
at least the queen πῦρος νοεροῦ [of intellectual fire]?
Purposely I break off the dream and the very vast speculation, merely
crying out with the royal Psalmist:
Great is our Lord and great His virtue and of His wisdom there is no
number: praise Him, ye heavens, praise Him, ye sun, moon, and planets,
use every sense for perceiving, every tongue for declaring your Creator.
Praise Him, ye celestial harmonies, praise Him, ye judges of the
harmonies uncovered (and you before all, old happy Mastlin, for you
used to animate these cares with words of hope): and thou my soul,
praise the Lord thy Creator, as long as I shall be: for out of Him and
through Him and in Him are all things, καὶ τἀ αἰσθητὰ καὶ τὰ νοερὰ [both
the sensible and the intelligible]; for both whose whereof we are utterly
ignorant and those which we know are the least part of them; because
there is still more beyond. To Him be praise, honour, and glory, world
without end. Amen.
THE END
This work was completed on the 17th or 27th day of May, 1618; but Book V was
reread (while the type was being set) on the 9th or 19th of February, 1619. At Linz,
the capital of Austria—above the Enns.


\end{document}
