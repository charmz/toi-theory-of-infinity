\documentclass[11pt]{article}
\usepackage{geometry}
\geometry{margin=1in}
\setlength{\parskip}{0.8em}
\setlength{\parindent}{0pt}
\usepackage{hyperref}
\usepackage{amsmath,amssymb}
\begin{document}
\section*{Formal Axiomatic Framework of Aristotle's \emph{Metaphysics}}

\subsection*{Introduction}
This document follows the guidelines of \texttt{gptlatex.txt} to summarize Aristotle's \emph{Metaphysics}. It gathers definitions and axioms in a concise formal style and includes a few propositions derived from them.

\subsection*{Definitions}
\begin{enumerate}
  \item \textbf{Being (on).} Whatever exists in its various modes. Aristotle distinguishes being by categories such as substance, quantity, quality, relation, place, time, position, state, action, and affection.
  \item \textbf{Substance (ousia).} That which exists in itself and serves as the primary subject of predication. Individual substances are concrete beings; forms and matter are aspects of substance.
  \item \textbf{Matter (hyle).} The potential component of a substance---the underlying stuff which, when combined with form, constitutes a particular being.
  \item \textbf{Form (eidos).} The actuality or defining essence of a substance which gives matter its determinate character.
  \item \textbf{Potentiality (dynamis).} The capacity of matter to receive form or change. Potentiality becomes actuality through the agency of some cause.
  \item \textbf{Actuality (energeia).} The realized state of a substance when its potentiality is fully expressed.
  \item \textbf{Cause (aitia).} An explanatory factor. Aristotle recognizes four causes: material, formal, efficient, and final.
  \item \textbf{Unmoved Mover.} A purely actual substance that initiates motion without itself changing. It is the ultimate final cause of all motion.
\end{enumerate}

\subsection*{Axioms}
\begin{enumerate}
  \item \textbf{Hierarchy of Causes.} Every change or existence of a substance can be explained by four causes---material, formal, efficient, and final.
  \item \textbf{Primacy of Substance.} Being is said primarily of individual substances. Other categories depend on substances for their existence.
  \item \textbf{Principle of Non-Contradiction.} A statement and its negation cannot both be true at the same time and in the same respect.
  \item \textbf{Potentiality Requires Actuality.} Whatever is in potentiality is brought into actuality only by something that is already actual.
  \item \textbf{Composition of Matter and Form.} Every sensible substance is a compound of matter and form; neither exists independently in the physical realm.
  \item \textbf{Existence of an Unmoved Mover.} There exists a first, unmoved mover that is pure actuality and serves as the ultimate final cause of all motion.
  \item \textbf{Teleology of Nature.} Natural substances act toward ends; final causes guide the development and motions of things.
\end{enumerate}

\subsection*{Selected Propositions}
\begin{enumerate}
  \item \textbf{Substances as Subjects.} (from Axioms~1,~2,~5) Each sensible entity is a unified subject composed of matter and form, underlying all its properties.
  \item \textbf{Causal Chains Terminate.} (from Axioms~4 and~6) Regress in efficient causes must end with the Unmoved Mover, ensuring a coherent account of motion.
  \item \textbf{Unity of Science.} The investigation of being qua being forms a single science because all categories relate back to substance and its causes.
\end{enumerate}

\subsection*{Notes on Consistency}
Following \texttt{gptmeta.txt}, these axioms can be interpreted in a classical first-order framework. No contradictions arise if we treat substances as elements of a domain, causes as relations, and the Unmoved Mover as a distinguished element that remains unchanged. The axioms align with Aristotle's intent and provide a basis for deriving traditional metaphysical conclusions.

\end{document}
