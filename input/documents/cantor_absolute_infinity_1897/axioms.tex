\documentclass[11pt]{article}
\usepackage{geometry}
\geometry{margin=1in}
\setlength{\parskip}{0.8em}
\setlength{\parindent}{0pt}
\usepackage{hyperref}
\usepackage{amsthm, amsmath,amssymb}

% Definitions
\newtheorem{definition}{Definition}[section]
\newtheorem{axiom}{Axiom}[section]
\newtheorem{theorem}{Theorem}[section]


\begin{document}
\section*{Cantor's Theory of Transfinite Numbers and the Absolute Infinite (1897) -- Formal Axiomatic Framework}

\subsection*{Introduction}
This document formalizes Georg Cantor's 1897 exposition of transfinite numbers and the Absolute Infinite.  The formulation follows the policy in \texttt{gptlatex.txt}: all explicit definitions, axioms, and theorems from the source text are stated in a clear logical structure.  Historical remarks appear in italics.

\subsection*{Definitions}
\begin{enumerate}
  \item \textbf{Set or Aggregate.} A \emph{set} (aggregate) is any well-defined collection of distinct elements. Two sets are equal iff they have exactly the same elements.\label{def:set}
  \item \textbf{Equivalence and Power of Sets.} Two sets $A$ and $B$ are \emph{equivalent} when a one-to-one correspondence exists between them. The \emph{power} (cardinality) $|A|$ measures the size of $A$, with $|A|=|B|$ iff $A$ and $B$ are equivalent.\label{def:power}
  \item \textbf{Transfinite Numbers -- Cardinals and Ordinals.} A \emph{transfinite number} exceeds all finite numbers. A \emph{cardinal} measures the size of an infinite set; an \emph{ordinal} represents the order type of a well-ordered set.\label{def:transfinite}
  \item \textbf{Number-Classes and Alephs.} Ordinals fall into ascending \emph{number-classes}; each class begins with a new transfinite cardinal $\aleph_{\mu}$.\label{def:numberclass}
  \item \textbf{Absolute Infinite.} The \emph{Absolute Infinite} $\Omega$ denotes an infinity beyond the entire transfinite sequence; it is not a set but a metaphysical ideal.\label{def:absolute}
\end{enumerate}

\subsection*{Axioms}
\begin{enumerate}
  \item \textbf{Existence of Sets and Extensionality.} There exists an infinite set (for example $\mathbb{N}$). Sets are determined solely by their elements: if $\forall x\,(x\in A \Leftrightarrow x\in B)$ then $A=B$.\label{ax:existence}
  \item \textbf{Equinumerosity and Cardinal Comparison.} Every set has a definite cardinal number, and any two sets can be compared in size: for sets $A,B$ exactly one of $|A|<|B|$, $|A|=|B|$, or $|A|>|B|$ holds.\label{ax:compare}
  \item \textbf{Generation of Transfinite Ordinals.} For every ordinal $\alpha$ there exists a successor $\alpha^+$, and every increasing sequence of ordinals has a limit ordinal beyond it.\label{ax:generate}
  \item \textbf{Well-Ordering and Ordinal Assignment.} Every set can be well-ordered and is therefore order-isomorphic to a unique ordinal representing its enumeration type.\label{ax:wellorder}
  \item \textbf{Transfinite Arithmetic.} Ordinal arithmetic (addition, multiplication, exponentiation) and cardinal arithmetic extend consistently to transfinite numbers.\label{ax:arithmetic}
  \item \textbf{Hierarchy of Infinities.} Transfinite numbers form an endless hierarchy of number-classes indexed by the alephs $\aleph_0,\aleph_1,\ldots$ with no maximal class.\label{ax:hierarchy}
  \item \textbf{Absolute Infinity Axiom.} No universal set or set of all ordinals exists; such totalities are treated as proper classes transcending the universe of sets.\label{ax:absolute}
\end{enumerate}

\subsection*{Theorems}
\begin{enumerate}
  \item \textbf{Countable Sets and $\aleph_0$.} From Definitions~\ref{def:set}--\ref{def:transfinite} and Axioms~\ref{ax:existence},\ref{ax:arithmetic} we conclude that $\mathbb{N}$ has the smallest infinite cardinality $\aleph_0$, and any countable union of countable sets is countable.\label{thm:countable}
  \item \textbf{Uncountability of the Continuum.} Using Axioms~\ref{ax:compare} and~\ref{ax:arithmetic}, the real numbers $\mathbb{R}$ have cardinality $2^{\aleph_0}>\aleph_0$ and are therefore uncountable.\label{thm:continuum}
  \item \textbf{Cantor's Power-Set Theorem.} (from Axiom~\ref{ax:arithmetic}) For any set $A$, the power-set $\mathcal{P}(A)$ has strictly larger cardinality than $A$: $|\mathcal{P}(A)|>|A|$.\label{thm:powerset}
  \item \textbf{No Maximum Transfinite Number.} From Axioms~\ref{ax:generate} and~\ref{ax:hierarchy}, there is no largest ordinal or cardinal; for every $\alpha$ a larger $\alpha^+$ exists, and for each cardinal $\kappa$ a larger one follows.\label{thm:nomax}
  \item \textbf{Ordinal Arithmetic Properties.} Based on Axiom~\ref{ax:arithmetic}, transfinite ordinal arithmetic is associative and distributive but not commutative; for infinite cardinals the arithmetic is commutative.\label{thm:arith}
\end{enumerate}

\subsection*{Commentary}
\textit{Cantor's axioms align with modern set theory (ZFC) augmented by proper classes. The Absolute Infinite serves as a philosophical horizon rather than an element of the formal universe.}

%%%%%%%%%%%%%%%%%%%%%%%%%%%%%%%%%%%%%%%%%%%%%%%%%%%%%%%%%%%%%%%%%%%%%%%%
%  axioms.tex
%  Formal reconstruction of Cantor’s 1895–1897 memoirs
%  (See gptmeta.txt §§2-11 and gptlatex.txt “Extraction / Preservation”.)
%%%%%%%%%%%%%%%%%%%%%%%%%%%%%%%%%%%%%%%%%%%%%%%%%%%%%%%%%%%%%%%%%%%%%%%%

\section*{Introduction / Scope}
Cantor’s two memoirs (1895, 1897) lay the groundwork for modern set
theory by (i) defining sets via one-to-one correspondence, (ii) introducing
the concepts of \emph{cardinal} and \emph{ordinal} number, and
(iii) proving basic laws of transfinite arithmetic.  The purpose of this
document is to capture **every formal claim** required for that
foundation in a compact axiomatic scheme suitable for mechanised
reasoning. :contentReference[oaicite:0]{index=0}

\bigskip
\noindent\textbf{Logical background.}
The framework is classical first–order logic with equality and a single
primitive binary relation~$x\in y$.  Quantifiers range over a universe~$V$
of objects.  We treat proper classes informally (italic commentary) and
take \emph{set} as an undefined primitive. 

%%%%%%%%%%%%%%%%%%%%%%%%%%%%%%%%%%%%%%%%%%%%%%%%%%%%%%%%%%%%%%%%%%%%%%%%
\section*{Definitions}
\begin{description}
\item[\textbf{Definition 1 (Set).}]
A \emph{set} is any collection of objects which is \emph{well-defined} in
Cantor’s sense: for each object it is unambiguously determined whether
it belongs to the collection or not. :contentReference[oaicite:2]{index=2}

\item[\textbf{Definition 2 (Equipotence / $\sim$).}]
Two sets~$A,B$ are \emph{equipotent}, written $A\sim B$, if there exists
a bijection $f:A\to B$.  Equipotence is an equivalence relation. :contentReference[oaicite:3]{index=3}

\item[\textbf{Definition 3 (Cardinal Number).}]
The \emph{cardinal} of a set~$A$, written $\lvert A\rvert$, is the
equivalence class of~$A$ under~$\sim$.  Cardinals are written with
Fraktur letters~$\aleph_\alpha,\mathfrak{c},\ldots$ as in Cantor.:contentReference[oaicite:4]{index=4}

\item[\textbf{Definition 4 (Finite, Infinite).}]
A set $A$ is \emph{finite} if~$A\sim n$ for some natural~$n$; otherwise
$A$ is \emph{infinite}.  The first transfinite cardinal is
$\aleph_0=\lvert\mathbb N\rvert$. :contentReference[oaicite:5]{index=5}

\item[\textbf{Definition 5 (Ordinal, $<$, $\omega$).}]
A set $\alpha$ is an \emph{ordinal} if it is transitive
($x\in\alpha\Rightarrow x\subseteq\alpha$) and well-ordered by~$\in$.
The least infinite ordinal is $\omega=\{0,1,2,\dots\}$. :contentReference[oaicite:6]{index=6}

\item[\textbf{Definition 6 (Set Operations).}]
For cardinals $\kappa,\lambda$ define
\[
\kappa+\lambda=\lvert A\cup B\rvert,\quad
\kappa\cdot\lambda=\lvert A\times B\rvert,\quad
\kappa^{\lambda}=\lvert B^{A}\rvert
\]
where $\lvert A\rvert=\kappa$, $\lvert B\rvert=\lambda$ and $A,B$ are
chosen disjoint. :contentReference[oaicite:7]{index=7}

\end{description}

%%%%%%%%%%%%%%%%%%%%%%%%%%%%%%%%%%%%%%%%%%%%%%%%%%%%%%%%%%%%%%%%%%%%%%%%
\section*{Axioms}
\begin{description}

\item[\textbf{Axiom 1 (Extensionality).}]
Two sets with exactly the same elements are identical.  (Standard
background axiom.)

\item[\textbf{Axiom 2 (Equipotence ⇔ Cardinal Equality).}]
$A\sim B \;\iff\; \lvert A\rvert=\lvert B\rvert$.  *Connects Definition 2
to Definition 3.* :contentReference[oaicite:8]{index=8}

\item[\textbf{Axiom 3 (Cantor–Bernstein).}]
If $A\preccurlyeq B$ and $B\preccurlyeq A$ (embeddings exist) then
$A\sim B$.  *Key comparability principle for cardinals.* :contentReference[oaicite:9]{index=9}

\item[\textbf{Axiom 4 (Well-Ordering Principle).}]
Every set is equipotent to some ordinal, hence admits a well-order.
(Cantor’s “linear order theorem”.) :contentReference[oaicite:10]{index=10}

\item[\textbf{Axiom 5 (Transfinite Induction).}]
For any class $C$ of ordinals, if $0\in C$ and
$\forall\alpha\,\bigl(\alpha\in C\Rightarrow\alpha+1\in C\bigr)$ and
$\forall\lambda\bigl(\text{$\lambda$ limit}\wedge
\forall\beta<\lambda(\beta\in C)\Rightarrow\lambda\in C\bigr)$ then
$C$ contains all ordinals. :contentReference[oaicite:11]{index=11}

\item[\textbf{Axiom 6 (Cardinal Exponentiation Bound).}]
For every cardinal $\kappa$ we have $\kappa<2^\kappa$.
(Cantor’s theorem.) :contentReference[oaicite:12]{index=12}

\end{description}

%%%%%%%%%%%%%%%%%%%%%%%%%%%%%%%%%%%%%%%%%%%%%%%%%%%%%%%%%%%%%%%%%%%%%%%%
\section*{Derived Propositions}
\begin{description}

\item[\textbf{Theorem 1 (Existence of Higher Alephs).}]
For every ordinal $\alpha$ there exists a cardinal
$\aleph_{\alpha+1}$ strictly greater than $\aleph_\alpha$.
\emph{Proof sketch.} Define $\aleph_{\alpha+1}$ as the least cardinal
greater than $\aleph_\alpha$ using Axioms 3 and 6.  (Uses 4 & 6.) 

\item[\textbf{Theorem 2 (Arithmetic of $\aleph_0$).}]
$\aleph_0+\aleph_0=\aleph_0$ and $\aleph_0\cdot\aleph_0=\aleph_0$.
\emph{Proof sketch.} Construct explicit bijections; rely on
Axiom 3. 

\item[\textbf{Theorem 3 (Continuum Cardinality).}]
$\lvert\mathbb R\rvert=2^{\aleph_0}$.
\emph{Proof.} Binary‐sequence encoding of reals gives the required
bijection.  (Depends on Defs 3,6.) 

\end{description}

%%%%%%%%%%%%%%%%%%%%%%%%%%%%%%%%%%%%%%%%%%%%%%%%%%%%%%%%%%%%%%%%%%%%%%%%
\section*{Discussion / Consistency}
The above axioms reproduce Cantor’s core results while remaining
compatible with standard ZF set theory (Extensionality, Well-Ordering,
etc.). Axioms 3–6 are specialised forms of accepted principles; no
known contradiction arises.  Cantor’s philosophical remarks on the
“Absolute Infinite” are \emph{not} formalised here but flagged as
historical context only. :contentReference[oaicite:13]{index=13}

%%%%%%%%%%%%%%%%%%%%%%%%%%%%%%%%%%%%%%%%%%%%%%%%%%%%%%%%%%%%%%%%%%%%%%%%
\section*{Traceability Checklist}
\begin{itemize}
  \item Equipotence and cardinal equality (Axiom 2) – Article I §1. :contentReference[oaicite:14]{index=14}
  \item Cantor–Bernstein (Axiom 3) – Article I §5. :contentReference[oaicite:15]{index=15}
  \item Well-ordering theorem (Axiom 4) – Article II §3. :contentReference[oaicite:16]{index=16}
  \item Transfinite induction (Axiom 5) – Article II §4. :contentReference[oaicite:17]{index=17}
  \item Power-set bound (Axiom 6) – Article I §7. :contentReference[oaicite:18]{index=18}
\end{itemize}

%%%%%%%%%%%%%%%%%%%%%%%%%%%%%%%%%%%%%%%%%%%%%%%%%%%%%%%%%%%%%%%%%%%%%%%%
% End of axioms.tex
%%%%%%%%%%%%%%%%%%%%%%%%%%%%%%%%%%%%%%%%%%%%%%%%%%%%%%%%%%%%%%%%%%%%%%%%

\section{Definitions}

\begin{definition}[Set]
A \textbf{set} is a well-defined collection of distinct elements which can be considered as a whole.
\end{definition}

\begin{definition}[Cardinal Number]
A \textbf{cardinal number} is the number representing the size of a set, abstracting the nature of the elements.
\end{definition}

\begin{definition}[Ordinal Number]
An \textbf{ordinal number} is a generalization of natural numbers used to describe the order type of a well-ordered set.
\end{definition}

\begin{definition}[Infinite Set]
An \textbf{infinite set} is a set which can be put into a one-to-one correspondence with a proper subset of itself.
\end{definition}

\begin{definition}[Enumerable Set]
A set is \textbf{enumerable} (countable) if it can be put into a one-to-one correspondence with the set of natural numbers.
\end{definition}

\section{Axioms}

\begin{axiom}[Existence of Sets]
There exist objects called sets that are collections of distinct elements.
\end{axiom}

\begin{axiom}[Principle of Enumeration]
Two sets have the same cardinal number if there exists a one-to-one correspondence between their elements.
\end{axiom}

\begin{axiom}[Infinite Sets and Proper Subsets]
An infinite set has the same cardinal number as one of its proper subsets.
\end{axiom}

\begin{axiom}[Well-ordering Principle]
Every set can be well-ordered, meaning every non-empty subset has a least element under this ordering.
\end{axiom}

\section{Theorems}

\begin{theorem}[Non-Enumerability of the Continuum]
The set of real numbers between any two distinct points is not enumerable.

\textit{(Follows from Axioms 2 and 3)}
\end{theorem}

\begin{theorem}[Existence of Transcendental Numbers]
Within every interval of real numbers, there are infinitely many transcendental numbers.

\textit{(Consequence of Theorem 4.1)}
\end{theorem}

\begin{theorem}[Hierarchy of Infinite Sets]
There are infinite sets with strictly greater cardinal numbers than enumerable sets.

\textit{(Follows directly from Definitions 4, 5, and Axiom 3)}
\end{theorem}

\begin{theorem}[Enumeration of Algebraic Numbers]
The set of algebraic numbers is enumerable.

\textit{(Follows from Axiom 2 and explicit enumeration method)}
\end{theorem}


\end{document}
