\documentclass[11pt]{article}
\usepackage{geometry}
\geometry{margin=1in}
\setlength{\parskip}{0.8em}
\setlength{\parindent}{0pt}
\usepackage{hyperref}
\usepackage{amsmath,amssymb}
\begin{document}
\section*{Cantor's Theory of Transfinite Numbers and the Absolute Infinite (1897) -- Formal Axiomatic Framework}

\subsection*{Introduction}
This document formalizes Georg Cantor's 1897 exposition of transfinite numbers and the Absolute Infinite.  The formulation follows the policy in \texttt{gptlatex.txt}: all explicit definitions, axioms, and theorems from the source text are stated in a clear logical structure.  Historical remarks appear in italics.

\subsection*{Definitions}
\begin{enumerate}
  \item \textbf{Set or Aggregate.} A \emph{set} (aggregate) is any well-defined collection of distinct elements. Two sets are equal iff they have exactly the same elements.\label{def:set}
  \item \textbf{Equivalence and Power of Sets.} Two sets $A$ and $B$ are \emph{equivalent} when a one-to-one correspondence exists between them. The \emph{power} (cardinality) $|A|$ measures the size of $A$, with $|A|=|B|$ iff $A$ and $B$ are equivalent.\label{def:power}
  \item \textbf{Transfinite Numbers -- Cardinals and Ordinals.} A \emph{transfinite number} exceeds all finite numbers. A \emph{cardinal} measures the size of an infinite set; an \emph{ordinal} represents the order type of a well-ordered set.\label{def:transfinite}
  \item \textbf{Number-Classes and Alephs.} Ordinals fall into ascending \emph{number-classes}; each class begins with a new transfinite cardinal $\aleph_{\mu}$.\label{def:numberclass}
  \item \textbf{Absolute Infinite.} The \emph{Absolute Infinite} $\Omega$ denotes an infinity beyond the entire transfinite sequence; it is not a set but a metaphysical ideal.\label{def:absolute}
\end{enumerate}

\subsection*{Axioms}
\begin{enumerate}
  \item \textbf{Existence of Sets and Extensionality.} There exists an infinite set (for example $\mathbb{N}$). Sets are determined solely by their elements: if $\forall x\,(x\in A \Leftrightarrow x\in B)$ then $A=B$.\label{ax:existence}
  \item \textbf{Equinumerosity and Cardinal Comparison.} Every set has a definite cardinal number, and any two sets can be compared in size: for sets $A,B$ exactly one of $|A|<|B|$, $|A|=|B|$, or $|A|>|B|$ holds.\label{ax:compare}
  \item \textbf{Generation of Transfinite Ordinals.} For every ordinal $\alpha$ there exists a successor $\alpha^+$, and every increasing sequence of ordinals has a limit ordinal beyond it.\label{ax:generate}
  \item \textbf{Well-Ordering and Ordinal Assignment.} Every set can be well-ordered and is therefore order-isomorphic to a unique ordinal representing its enumeration type.\label{ax:wellorder}
  \item \textbf{Transfinite Arithmetic.} Ordinal arithmetic (addition, multiplication, exponentiation) and cardinal arithmetic extend consistently to transfinite numbers.\label{ax:arithmetic}
  \item \textbf{Hierarchy of Infinities.} Transfinite numbers form an endless hierarchy of number-classes indexed by the alephs $\aleph_0,\aleph_1,\ldots$ with no maximal class.\label{ax:hierarchy}
  \item \textbf{Absolute Infinity Axiom.} No universal set or set of all ordinals exists; such totalities are treated as proper classes transcending the universe of sets.\label{ax:absolute}
\end{enumerate}

\subsection*{Theorems}
\begin{enumerate}
  \item \textbf{Countable Sets and $\aleph_0$.} From Definitions~\ref{def:set}--\ref{def:transfinite} and Axioms~\ref{ax:existence},\ref{ax:arithmetic} we conclude that $\mathbb{N}$ has the smallest infinite cardinality $\aleph_0$, and any countable union of countable sets is countable.\label{thm:countable}
  \item \textbf{Uncountability of the Continuum.} Using Axioms~\ref{ax:compare} and~\ref{ax:arithmetic}, the real numbers $\mathbb{R}$ have cardinality $2^{\aleph_0}>\aleph_0$ and are therefore uncountable.\label{thm:continuum}
  \item \textbf{Cantor's Power-Set Theorem.} (from Axiom~\ref{ax:arithmetic}) For any set $A$, the power-set $\mathcal{P}(A)$ has strictly larger cardinality than $A$: $|\mathcal{P}(A)|>|A|$.\label{thm:powerset}
  \item \textbf{No Maximum Transfinite Number.} From Axioms~\ref{ax:generate} and~\ref{ax:hierarchy}, there is no largest ordinal or cardinal; for every $\alpha$ a larger $\alpha^+$ exists, and for each cardinal $\kappa$ a larger one follows.\label{thm:nomax}
  \item \textbf{Ordinal Arithmetic Properties.} Based on Axiom~\ref{ax:arithmetic}, transfinite ordinal arithmetic is associative and distributive but not commutative; for infinite cardinals the arithmetic is commutative.\label{thm:arith}
\end{enumerate}

\subsection*{Commentary}
\textit{Cantor's axioms align with modern set theory (ZFC) augmented by proper classes. The Absolute Infinite serves as a philosophical horizon rather than an element of the formal universe.}

\end{document}
