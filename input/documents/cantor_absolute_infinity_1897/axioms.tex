\section*{Cantor's Theory of Transfinite Numbers and the Absolute Infinite (1897) -- Formal Axiomatic Framework}

\subsection*{Introduction}
This framework summarizes Cantor's 1897 exposition using modern notation while retaining his philosophical context. Definitions and axioms reflect Cantor's terminology; theorems follow from these assumptions.

\subsection*{Definitions}
\begin{definition}[Set or Aggregate]
A \emph{set} (aggregate) is any well-defined collection of distinct elements. Two sets are equal iff they share exactly the same elements (Extensionality).
\end{definition}
%ID: Cantor.1897.Def1

\begin{definition}[Equivalence and Power of Sets]
Two sets $A$ and $B$ are \emph{equivalent} when there exists a one-to-one correspondence between their elements. The \emph{power} (cardinal number) $|A|$ of a set $A$ characterizes its size so that $|A|=|B|$ iff $A$ and $B$ are equivalent.
\end{definition}
%ID: Cantor.1897.Def2

\begin{definition}[Transfinite Numbers -- Cardinals and Ordinals]
A \emph{transfinite number} exceeds all finite numbers. A \emph{cardinal} measures the size of an infinite set, while an \emph{ordinal} describes the order type of a well-ordered set.
\end{definition}
%ID: Cantor.1897.Def3

\begin{definition}[Number-Classes and Alephs]
Ordinals are grouped into ascending \emph{number-classes}; each class corresponds to a new transfinite cardinal $\aleph_\mu$.
\end{definition}
%ID: Cantor.1897.Def4

\begin{definition}[Absolute Infinite]
The \emph{Absolute Infinite} $\Omega$ denotes an infinity beyond the entire transfinite sequence; it transcends all sets.
\end{definition}
%ID: Cantor.1897.Def5

\subsection*{Axioms}
\begin{axiom}[Existence of Sets and Extensionality]
There exists an infinite set (e.g. $\mathbb{N}$). Sets are determined solely by their elements: if $\forall x\,(x\in A\Leftrightarrow x\in B)$, then $A=B$.
\end{axiom}
%ID: Cantor.1897.Axiom1

\begin{axiom}[Equinumerosity and Cardinal Comparison]
Every set has a definite cardinal number, and any two sets can be compared in size: for sets $A,B$ exactly one of $|A|<|B|$, $|A|=|B|$, or $|A|>|B|$ holds.
\end{axiom}
%ID: Cantor.1897.Axiom2

\begin{axiom}[Generation of Transfinite Ordinals]
For every ordinal $\alpha$ there exists a successor $\alpha^+$, and every increasing sequence of ordinals has a limit ordinal beyond it.
\end{axiom}
%ID: Cantor.1897.Axiom3

\begin{axiom}[Well-Ordering and Ordinal Assignment]
Every set can be well-ordered and is therefore order-isomorphic to a unique ordinal representing its enumeration type.
\end{axiom}
%ID: Cantor.1897.Axiom4

\begin{axiom}[Transfinite Arithmetic]
Ordinal arithmetic (addition, multiplication, exponentiation) and cardinal arithmetic extend consistently to transfinite numbers.
\end{axiom}
%ID: Cantor.1897.Axiom5

\begin{axiom}[Hierarchy of Infinities]
Transfinite numbers form an endless hierarchy of number-classes indexed by the alephs $\aleph_0,\aleph_1,\ldots$ with no maximal class.
\end{axiom}
%ID: Cantor.1897.Axiom6

\begin{axiom}[Absolute Infinity Axiom]
No universal set or set of all ordinals exists; such totalities are treated as proper classes transcending the universe of sets.
\end{axiom}
%ID: Cantor.1897.Axiom7

\subsection*{Theorems}
\begin{theorem}[Countable Sets and $\aleph_0$]
The natural numbers $\mathbb{N}$ have the smallest infinite cardinality $\aleph_0$. Any countable union of countable sets is countable.
\end{theorem}
%ID: Cantor.1897.Thm1

\begin{theorem}[Uncountability of the Continuum]
The real numbers $\mathbb{R}$ have cardinality $2^{\aleph_0}$, strictly greater than $\aleph_0$; hence $\mathbb{R}$ is uncountable.
\end{theorem}
%ID: Cantor.1897.Thm2

\begin{theorem}[Cantor's Power-Set Theorem]
For any set $A$, the power-set $\mathcal{P}(A)$ has strictly larger cardinality than $A$: $|\mathcal{P}(A)|>|A|$.
\end{theorem}
%ID: Cantor.1897.Thm3

\begin{theorem}[No Maximum Transfinite Number]
There is no largest ordinal or cardinal. For every ordinal $\alpha$ there exists a larger ordinal $\alpha^+$, and for each cardinal $\kappa$ there is a larger cardinal.
\end{theorem}
%ID: Cantor.1897.Thm4

\begin{theorem}[Ordinal Arithmetic Properties]
Transfinite ordinal arithmetic is associative and distributive but not commutative; cardinal arithmetic for infinite cardinals is commutative.
\end{theorem}
%ID: Cantor.1897.Thm5

\subsection*{Commentary}
Cantor's axioms align with modern set theory (ZFC) augmented by proper classes. The Absolute Infinite represents a metaphysical ideal rather than a set.
