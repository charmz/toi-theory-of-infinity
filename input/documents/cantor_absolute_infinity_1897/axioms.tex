\begin{axiom}[Transfinite Unboundedness]
\label{cantor1897:axiom1}
For every number or set that is infinite (transfinite), there always exists a larger transfinite number beyond it. In other words, no highest or “ultimate” transfinite number can ever be attained. \textit{(This axiom formalizes Cantor’s insight that the creation of higher infinities never ends—given any infinite quantity, one can always find a larger one.)}
\end{axiom}
%ID: Cantor.1897.Axiom1

\begin{axiom}[Absolute Infinity Unattainable]
\label{cantor1897:axiom2}
The \emph{Absolute Infinity} $\Omega$ cannot be reached or even approximated by any sequence of increasing transfinite numbers. $\Omega$ does not exist as a set or completed totality within the mathematical universe; it can only be \emph{acknowledged} as an ideal concept, never fully \emph{comprehended} or concretely realized. \textit{(This axiom captures Cantor’s assertion that no matter how far one progresses in the transfinite hierarchy, one is still no closer to grasping the Absolute Infinity. $\Omega$ is not a regular mathematical object but an idea—the “true infinite” recognized only in principle, not in constructive reality.)}
\end{axiom}
%ID: Cantor.1897.Axiom2

\begin{axiom}[Inexhaustibility of the Transfinite]
\label{cantor1897:axiom3}
Given any transfinite number $\alpha$, the collection of all numbers beyond $\alpha$ is just as “absolutely infinite” as the collection of all numbers from the beginning. In other words, even after advancing to some higher infinity $\alpha$, there remain infinitely many further transfinite numbers undiminished in their total potency (no portion of the sequence exhausts the infinite progression). \textit{(This axiom emphasizes that the transfinite sequence never loses its endless character: after any point $\alpha$, the “remaining” infinity of numbers is equally as large as the whole, reflecting Cantor’s idea that every segment of the ordinals or cardinals beyond a given point has the same unlimited richness as the full universe of discourse.)}
\end{axiom}
%ID: Cantor.1897.Axiom3

\begin{axiom}[Absolute Unity and Divine Infinity]
\label{cantor1897:axiom4}
There exists one unique \emph{Absolutely Infinite} entity $\Omega$ which transcends all relative (created) infinities. This $\Omega$ is an actual infinite of the highest order: it admits no limitation or “determination” and is not comparable to any transfinite number. In Cantor’s view, $\Omega$ exists only in the realm of the divine (the “true infinite” realized solely \emph{in Deo}), never as a finite or mathematical object. \textit{(This axiom formalizes Cantor’s identification of the absolutely infinite with a unique, Godlike infinity beyond all set-theoretic hierarchies. It asserts that all other infinities are subordinate or relative, whereas $\Omega$ alone is the all-encompassing infinite, not subject to the constraints of the created mathematical universe.)}
\end{axiom}
%ID: Cantor.1897.Axiom4