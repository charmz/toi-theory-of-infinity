\documentclass[11pt]{article}
\usepackage{geometry}
\geometry{margin=1in}
\setlength{\parskip}{0.8em}
\setlength{\parindent}{0pt}
\usepackage{hyperref}
\usepackage{amsmath,amssymb}
\begin{document}
\section*{Formal Axiomatization of Einstein's Relativity}

\subsection*{Introduction}
This document presents a rigorous axiomatic framework for Albert Einstein's\textit{Relativity: The Special and General Theory} (1916). Every definition and axiom is drawn from the source text and organized systematically. The goal is a minimal, logically consistent structure faithful to Einstein's exposition.

\subsection*{Definitions}
\begin{enumerate}
  \item \textbf{Reference Frame.} A coordinate system or body of reference used to measure positions and motions.
  \begin{itemize}
    \item \textbf{Inertial Frame.} Moves at constant velocity without rotation; Newton's first law holds.
    \item \textbf{Non-Inertial Frame.} Accelerated or rotating relative to an inertial frame; fictitious forces appear unless attributed to gravity.
  \end{itemize}
  \item \textbf{Event.} A point $(t,x,y,z)$ in space--time with no extent.
  \item \textbf{Speed of Light $c$.} A universal constant, the same in all inertial frames.
  \item \textbf{Simultaneity.} Two events are simultaneous in frame $K$ if they share the same $t$ coordinate in $K$; simultaneity depends on the frame.
  \item \textbf{Lorentz Transformation.} Linear mapping between inertial frames that preserves the space--time interval.
  \item \textbf{Gravitational Field.} A manifestation of space--time curvature giving bodies the same acceleration.
  \item \textbf{Inertial vs. Gravitational Mass.} Empirically equal measures of inertia and gravitational interaction.
  \item \textbf{Space--Time Continuum.} Four--dimensional manifold; flat (Minkowski) when gravity is negligible and curved otherwise.
  \item \textbf{Geodesic.} Path of a free object in space--time; locally inertial and determined by the metric.
\end{enumerate}

\subsection*{Axioms}
\begin{enumerate}
  \item \textbf{Principle of Relativity (Inertial Frames).} The laws of nature are identical in all inertial reference frames.
  \item \textbf{Invariant Speed of Light.} Light in vacuum propagates with the same speed $c$ in every inertial frame.
  \item \textbf{Equivalence Principle.} A uniform gravitational field is locally indistinguishable from an accelerated frame.
  \item \textbf{General Covariance.} All reference frames are equivalent for expressing physical laws, provided gravitational fields are included when necessary.
  \item \textbf{Space--Time Geometry and Gravitation.} Gravity corresponds to curvature of a four--dimensional space--time equipped with a metric tensor $g_{\mu\nu}$.
  \item \textbf{Correspondence Principle.} Relativistic results reduce to classical physics in the limits of low velocity or weak gravitation.
\end{enumerate}

\subsection*{Derived Theorems and Propositions}
\begin{enumerate}
  \item \textbf{Lorentz Transformation and Relativistic Kinematics.} (from Axioms~1 and~2) The coordinate change between inertial frames is given by the Lorentz transformation
  \[t' = \gamma(t-\tfrac{v}{c^2}x),\quad x' = \gamma(x-vt),\quad y'=y,\quad z'=z,\]
  where $\gamma=1/\sqrt{1-v^2/c^2}$. Consequences include relativity of simultaneity, time dilation and length contraction.
  \item \textbf{Mass--Energy Equivalence.} (from Axioms~1 and~2) Energy $E$ and mass $m$ are related by $E=mc^2$; a change $\Delta E$ implies $\Delta m=\Delta E/c^2$.
  \item \textbf{Gravitational Time Dilation and Light Bending.} (from Axioms~3--5) Time runs slower in stronger gravitational potential and light rays follow curved paths near massive bodies.
  \item \textbf{Finite yet Unbounded Universe.} (from Axiom~5) Space can be finite in volume but without boundary if the global geometry is curved.
\end{enumerate}

\subsection*{Consistency and Interpretation}
The axioms above are mutually compatible and recover classical physics in the appropriate limit. They interpret gravity geometrically and extend the relativity principle to all frames. This framework parallels the methodology used in the Theory of Infinity draft, emphasizing explicit definitions and minimal axioms.

\end{document}
