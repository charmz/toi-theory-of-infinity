\documentclass[11pt]{article}

% Page geometry
\usepackage{geometry}
\geometry{margin=1in}
\setlength{\parskip}{0.8em}
\setlength{\parindent}{0pt}

\usepackage[dvipsnames]{xcolor}
\usepackage[colorlinks=true, linkcolor=blue!60!black, citecolor=blue!60!black, urlcolor=blue!60!black]{hyperref}

% Mathematics and graphics
\usepackage{fontspec}
\usepackage{xeCJK}
\usepackage{graphicx}
\usepackage{caption}
\usepackage{amsmath,amssymb,amsthm}

\usepackage{polyglossia}
\setdefaultlanguage{english}
\setotherlanguage[variant=ancient]{greek}
\setotherlanguage{hebrew}
\setotherlanguage{chinese}

\newfontfamily\hebrewfont{Ezra SIL}
\setCJKmainfont[
  UprightFont    = *-Regular ,
  BoldFont       = *-Bold    ,
  ItalicFont     = *-Regular ,
  BoldItalicFont = *-Bold    ,
  AutoFakeSlant  = true
]{Noto Serif CJK SC}

\newfontfamily\greekfont{Libertinus Serif}[Script=Greek]
\setmainfont{Libertinus Serif}
\captionsetup[figure]{labelfont=bf, font=small, margin=10pt}

% Lists
\usepackage{enumitem}
\setlist[itemize]{left=0pt, label=--, itemsep=0.5em}
\setlist[enumerate]{left=0pt, itemsep=0.5em}

% Custom commands
\newtheorem{concept}{Concept}
\newcommand{\symtry}{\mathbin{/}}
\newcommand{\goldenset}{\varnothing}
\newcommand{\knotinfinity}{\textnormal{0}}

\title{\Huge\bfseries THEORY OF INFINITY\\[0.5em] \Large Formal Axiomatic Framework}
\author{Converted from\texttt{ draft.txt }}
\date{\today}

\begin{document}

\maketitle

\section*{Introduction}
The \emph{Theory of Infinity} (TOI) postulates a universal domain of discourse $\infty$ alongside a fundamental operator of symmetry $\symtry$ and a hierarchy of self-contained contexts.  This document casts the core principles of TOI into a formal style, addressing concerns of rigor and demonstrating internal consistency.

\section*{Definitions}
\begin{itemize}
  \item \textbf{Universe $\infty$.}  A distinguished class containing every mathematical object of discourse.  Formally every $x$ is in $\infty$, while $\infty$ itself is not an element of any other collection.
  \item \textbf{Symmetry $\symtry$.}  A class of transformations on $\infty$ forming a group.  Two elements $x,y$ are equivalent ($x\sim y$) if some transformation sends $x$ to $y$.
  \item \textbf{Knot Infinity $\knotinfinity$.}  A chosen element of $\infty$ serving as an anchor for a symmetry orbit.
  \item \textbf{Golden Set $\goldenset$.}  The orbit of $\knotinfinity$ under all symmetries: $\goldenset=\{x\in\infty\mid x\sim \knotinfinity\}$.  It is a self--contained context closed under the allowed transformations.
\end{itemize}

\section*{Axioms of TOI}
\begin{enumerate}
  \item \textbf{Infinity as Universal Context.}  Every object lies in $\infty$, but $\infty$ is not itself a set.  This treats $\infty$ like a proper class and avoids a naive universal set paradox.
  \item \textbf{Symmetry as Fundamental Operator.}  There exists a group $\symtry$ acting on $\infty$ whose elements preserve structure.  Symmetry partitions $\infty$ into equivalence classes (orbits).
  \item \textbf{Existence of a Knot and Golden Set.}  There is at least one element $\knotinfinity\in \infty$ whose orbit $\goldenset$ under $\symtry$ is nontrivial and closed under the induced transformations.
\end{enumerate}

\section*{Key Consequences and Theorems}
\begin{concept}[Partition of $\infty$]
The relation $\sim$ partitions $\infty$ into disjoint orbits.  Each orbit is a Golden Set associated with some Knot Infinity.
\end{concept}

\begin{concept}[Hierarchy]
Golden Sets can contain further knots and corresponding golden subsets.  Iterating this process yields an infinite hierarchy of nested contexts.
\end{concept}

\section*{Relation to ZFC and Standard Set Theory}
\begin{itemize}
  \item \textbf{Universal Domain vs.~ZFC.}  ZFC lacks a set of all sets; TOI introduces $\infty$ as a class analogous to the universe $V$ in class theories.
  \item \textbf{Symmetry Axiom.}  TOI postulates $\symtry$ as primitive, enriching ordinary set theory with a structural group action.
  \item \textbf{Hierarchy.}  TOI's layered golden sets parallel the cumulative hierarchy of ZFC while emphasizing symmetry orbits rather than power sets.
\end{itemize}

\section*{Conclusion}
The axioms above formalize TOI within a first--order set theoretic language augmented by symmetry.  By interpreting $\infty$ as a proper class and by structuring contexts through symmetry orbits, TOI maintains logical consistency while offering a novel perspective on infinity.

\section*{Sources}
\begin{itemize}
  \item Excerpts from \emph{Theory of Infinity} documentation.
  \item Classical set theory references on Russell's paradox and proper classes.
  \item Parallels to Tarski's hierarchy in theories of truth.
\end{itemize}

\end{document}

